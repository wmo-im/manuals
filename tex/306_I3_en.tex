Manual on Codes

International Codesport

Volume~I.3

Annex II to the WMO Technical Regulations

Part~D -- Representations derived from data models

Manual on Codes

International Codes

Volume~I.3

Annex II to the WMO Technical Regulations

Part~D -- Representations derived from data models

PUBLICATION REVISION TRACK RECORD

\begin{longtable}[]{@{}lllll@{}}
\toprule
Date & \vtop{\hbox{\strut Part/chapter/}\hbox{\strut section}} & Purpose of amendment & Proposed by & Approved by\tabularnewline
\midrule
\endhead
\begin{minipage}[t]{0.17\columnwidth}\raggedright
17/05/ 2017\strut
\end{minipage} & \begin{minipage}[t]{0.17\columnwidth}\raggedright
(i) Introduction and General provisions\\
(pp. xi-xxii)

(ii) Pages 1 to 178\strut
\end{minipage} & \begin{minipage}[t]{0.17\columnwidth}\raggedright
(i) Alignment of the procedures for amendments to Manuals and Guides under the responsibility of the Commission for Basic Systems, and aligment with the new structure of the \emph{Techniacl Regulations} (WMO-No.~49

(ii) Introduce IWXXM 2.1, METCE 1.2, COLLECT 1.2, TimeseriesML, WaterML2 and code tables supporting WIGOS metadata\strut
\end{minipage} & \begin{minipage}[t]{0.17\columnwidth}\raggedright
(i) CBS Management Group

(ii) Decision 8\\
(CBS-16) and Recommendations~7, 8, 9, 11, 12 and 13 (CBS-16)\strut
\end{minipage} & \begin{minipage}[t]{0.17\columnwidth}\raggedright
(i) Resolution~12 (EC-68) and Resolution~20 (EC-69)

(ii) Resolutions\\
9, 10, 11 and 12 (EC-69)\strut
\end{minipage}\tabularnewline
& & & &\tabularnewline
& & & &\tabularnewline
& & & &\tabularnewline
& & & &\tabularnewline
& & & &\tabularnewline
& & & &\tabularnewline
& & & &\tabularnewline
& & & &\tabularnewline
& & & &\tabularnewline
& & & &\tabularnewline
& & & &\tabularnewline
& & & &\tabularnewline
& & & &\tabularnewline
& & & &\tabularnewline
& & & &\tabularnewline
\bottomrule
\end{longtable}

INTRODUCTION

Volume~I of the \emph{Manual on Codes} contains WMO international codes for meteorological data and other geophysical data relating to meteorology; it constitutes Annex~II to the \emph{Technical Regulations} (WMO-No.~49) and has therefore the status of a Technical Regulation. It is issued in three volumes: Volume~I.1, containing Part~A; Volume~I.2, containing Part~B and Part~C; and Volume~I.3 containing Part~D.

Coded messages are used for the international exchange of meteorological information comprising observational data provided by the World Weather Watch (WWW) Global Observing System and processed data provided by the WWW Global Data-processing and Forecasting System. Coded messages are also used for the international exchange of observed and processed data required in specific applications of meteorology to various human activities and for exchanges of information related to meteorology.

The codes are composed of a set of CODE FORMS and BINARY CODES made up of SYMBOLIC LETTERS (or groups of letters) representing meteorological or, as the case may be, other geophysical elements. In messages, these symbolic letters (or groups of letters) are transcribed into figures indicating the value or the state of the elements described. SPECIFICATIONS have been defined for the various symbolic letters to permit their transcription into figures. In some cases, the specification of the symbolic letter is sufficient to permit a direct transcription into figures. In other cases, it requires the use of CODE FIGURES, the specifications of which are given in CODE TABLES. Furthermore, a certain number of SYMBOLIC WORDS and SYMBOLIC FIGURE GROUPS have been developed for use as code names, code words, symbolic prefixes or indicator groups.

Rules concerning the selection of code forms to be exchanged for international purposes, and the selection of their symbolic words, figure groups and letters, are laid down in the \emph{Technical Regulations} (WMO-No.~49), Volume I, Part II, section~2 (2015 edition, updated in 2016). These code forms are contained in Volume I of the \emph{Manual on Codes}, issued as Volume~I.1 -- Part~A, Volume~I.2 -- Part~B and Part~C, and Volume~I.3 -- Part~D.

Apart from these international codes, several sets of \emph{regional codes} exist which are intended only for exchanges within a given WMO Region. These codes are contained in Volume~II of the \emph{Manual on Codes}, which also contains descriptions of the following:

-- Regional coding procedures for the use of international code forms;

-- National coding practices in the use of international or regional codes of which the WMO Secretariat has been informed;

-- National code forms.

A number of special codes that are used in messages exchanged over the WWW Global Telecommunication System circuits, and which comprise ice and satellite ephemeris codes, are included in Volume~II as an appendix.

\hypertarget{volume-i.1}{%
\section{VOLUME I.1:}\label{volume-i.1}}

\textbf{Part A -- Alphanumeric Codes} consists of five sections. The standard coding procedures are distinguished by the use of the term ``shall'' in the English text, and by suitable equivalent terms in the French, Russian and Spanish texts. Where national practices do not conform with these regulations, Members concerned shall formally notify the Secretary-General of WMO for the benefit of other Members.

\hypertarget{volume-i.2}{%
\section{VOLUME I.2:}\label{volume-i.2}}

Note: This volume is designated in its entirety as Technical Specifications.

\textbf{Part B -- Binary Codes} consists of the list of binary codes with their specifications and associated code tables.

\textbf{Part C -- Common Features to Binary and Alphanumeric Codes} consists of table-driven alphanumeric codes and of common code tables to binary and alphanumeric codes.

\hypertarget{volume-i.3}{%
\section{VOLUME I.3:}\label{volume-i.3}}

Note: This volume is designated in its entirety as Technical Specifications.

\textbf{Part D -- Representations derived from data models} consists of the specification of the list of standard representations derived from data models, including those using extensible markup language (XML), with their specifications and associated code tables.

This is the first edition of Volume~I.3 of the \emph{Manual on Codes} and introduces the use of XML.

GENERAL PROVISIONS

1. The \emph{Technical Regulations} (WMO-No.~49) of the World Meteorological Organization are presented in three volumes:

Volume I -- General meteorological standards and recommended practices\\
Volume II -- Meteorological service for international air navigation\\
Volume III -- Hydrology

Purpose of the Technical Regulations

2. The Technical Regulations are determined by the World Meteorological Congress in accordance with Article~8~(d) of the Convention.

3. These Regulations are designed:

(a) To facilitate cooperation in meteorology and hydrology among Members;

(b) To meet, in the most effective manner, specific needs in the various fields of application of meteorology and operational hydrology in the international sphere;

(c) To ensure adequate uniformity and standardization in the practices and procedures employed in achieving (a) and (b) above.

Types of Regulations

4. The Technical Regulations comprise \emph{standard} practices and procedures and \emph{recommended} practices and procedures.

5. The definitions of these two types of Regulations are as follows:

The \emph{standard} practices and procedures:

(a) Shall be the practices and procedures that Members are required to follow or implement;

(b) Shall have the status of requirements in a technical resolution in respect of which Article~9~(b) of the Convention is applicable;

(c) Shall invariably be distinguished by the use of the term \emph{shall} in the English text, and by suitable equivalent terms in the Arabic, Chinese, French, Russian and Spanish texts.

The \emph{recommended} practices and procedures:

(a) Shall be the practices and procedures with which Members are urged to comply;

(b) Shall have the status of recommendations to Members, to which Article~9~(b) of the Convention shall not be applied;

(c) Shall be distinguished by the use of the term \emph{should} in the English text (except where otherwise provided by decision of Congress) and by suitable equivalent terms in the Arabic, Chinese, French, Russian and Spanish texts.

6. In accordance with the above definitions, Members shall do their utmost to implement the \emph{standard} practices and procedures. In accordance with Article~9~(b) of the Convention and in conformity with Regulation~128 of the General Regulations, Members shall formally notify the Secretary-General, in writing, of their intention to apply the \emph{standard} practices and procedures of the Technical Regulations, except those for which they have lodged a specific deviation. Members shall also inform the Secretary-General, at least three months in advance, of any change in the degree of their implementation of a \emph{standard} practice or procedure as previously notified and the effective date of the change.

7. Members are urged to comply with \emph{recommended} practices and procedures, but it is not necessary to notify the Secretary-General of non-observance except with regard to practices and procedures contained in Volume~II.

8. In order to clarify the status of the various Regulations, the \emph{standard} practices and procedures are distinguished from the \emph{recommended} practices and procedures by a difference in typographical practice, as indicated in the editorial note.

Status of annexes and appendices

9. The following annexes to the \emph{Technical Regulations} (Volumes~I to III), also called Manuals, are published separately and contain regulatory material having the status of \emph{standard} and/or \emph{recommended} practices and procedures:

I \emph{International Cloud Atlas} (WMO-No.~407) -- Manual on the Observation of Clouds and Other Meteors, sections 1, 2.1.1, 2.1.4, 2.1.5, 2.2.2, 1 to 4 in 2.3.1 to 2.3.10 (for example, 2.3.1.1, 2.3.1.2, etc.), 2.8.2, 2.8.3, 2.8.5, 3.1 and the definitions (in grey-shaded boxes) of~3.2;

II \emph{Manual on Codes} (WMO-No.~306), Volume~I;

III \emph{Manual on the Global Telecommunication System} (WMO-No.~386);

IV \emph{Manual on the Global Data-processing and Forecasting System} (WMO-No.~485);

V \emph{Manual on the Global Observing System} (WMO-No.~544), Volume~I;

VI \emph{Manual on Marine Meteorological Services} (WMO-No.~558), Volume~I;

VII \emph{Manual on the WMO Information System} (WMO-No.~1060);

VIII \emph{Manual on the WMO Integrated Global Observing System} (WMO-No.~1160).

These annexes (Manuals) are established by decision of Congress and are intended to facilitate the application of Technical Regulations to specific fields. Annexes may contain both \emph{standard} and \emph{recommended} practices and procedures.

10. Texts called appendices, appearing in the \emph{Technical Regulations} or in an annex to the \emph{Technical Regulations}, have the same status as the Regulations to which they refer.

Status of notes and attachments

11. Certain notes (preceded by the indication ``Note'') are included in the \emph{Technical Regulations} for explanatory purposes; they may, for instance, refer to relevant WMO Guides and publications. These notes do not have the status of Technical Regulations.

12. The \emph{Technical Regulations} may also include attachments, which usually contain detailed guidelines related to \emph{standard} and \emph{recommended} practices and procedures. Attachments, however, do not have regulatory status.

Updating of the \emph{\textbf{Technical Regulations}} and their annexes (Manuals)

13. The \emph{Technical Regulations} are updated, as necessary, in the light of developments in meteorology and hydrology and related techniques, and in the application of meteorology and operational hydrology. Certain principles previously agreed upon by Congress and applied in the selection of material for inclusion in the Technical Regulations are reproduced below. These principles provide guidance for constituent bodies, in particular technical commissions, when dealing with matters pertaining to the Technical Regulations:

(a) Technical commissions should not recommend that a Regulation be a standard practice unless it is supported by a strong majority;

(b) Technical Regulations should contain appropriate instructions to Members regarding implementation of the provision in question;

(c) No major changes should be made to the Technical Regulations without consulting the appropriate technical commissions;

(d) Any amendments to the Technical Regulations submitted by Members or by constituent bodies should be communicated to all Members at least three months before they are submitted to Congress.

14. Amendments to the \emph{Technical Regulations} -- as a rule -- are approved by Congress.

15. If a recommendation for an amendment is made by a session of the appropriate technical commission and if the new regulation needs to be implemented before the next session of Congress, the Executive Council may, on behalf of the Organization, approve the amendment in accordance with Article~14~(c) of the Convention. Amendments to annexes to the \emph{Technical Regulations} proposed by the appropriate technical commissions are normally approved by the Executive Council.

16. If a recommendation for an amendment is made by the appropriate technical commission and the implementation of the new regulation is urgent, the President of the Organization may, on behalf of the Executive Council, take action as provided by Regulation~9~(5) of the General Regulations.

Note: A simple (fast-track) procedure may be used for amendments to technical specifications in Annexes II (\emph{Manual on Codes} (WMO-No. 306)), III (\emph{Manual on the Global Telecommunication System} (WMO-No. 386)), IV (\emph{Manual on the Global Data-processing and Forecasting System} (WMO-No. 485)), V (\emph{Manual on the Global Observing System} (WMO-No. 544)), VII (\emph{Manual on the WMO Information System} (WMO-No. 1060)) and VIII (\emph{Manual on the WMO Integrated Global Observing System} (WMO-No. 1160)). Application of the simple (fast-track) procedure is defined in the appendix to these General Provisions.

17. After each session of Congress (every four years), a new edition of the \emph{Technical Regulations}, including the amendments approved by Congress, is issued. With regard to the amendments between sessions of Congress, Volumes~I and III of the \emph{Technical Regulations} are updated, as necessary, upon approval of changes thereto by the Executive Council. The \emph{Technical Regulations} updated as a result of an approved amendment by the Executive Council are considered a new update of the current edition. The material in Volume~II is prepared by the World Meteorological Organization and the International Civil Aviation Organization working in close cooperation, in accordance with the Working Arrangements agreed by these Organizations. In order to ensure consistency between Volume~II and Annex~3 to the Convention on International Civil Aviation -- \emph{Meteorological Service for International Air Navigation}, the issuance of amendments to Volume~II is synchronized with the respective amendments to Annex~3 by the International Civil Aviation Organization.

Note: Editions are identified by the year of the respective session of Congress whereas updates are identified by the year of approval by the Executive Council, for example ``Updated in~2012''.

WMO Guides

18. In addition to the \emph{Technical Regulations}, appropriate Guides are published by the Organization. They describe practices, procedures and specifications which Members are invited to follow or implement in establishing and conducting their arrangements for compliance with the Technical Regulations, and in otherwise developing meteorological and hydrological services in their respective countries. The Guides are updated, as necessary, in the light of scientific and technological developments in hydrometeorology, climatology and their applications. The technical commissions are responsible for the selection of material to be included in the Guides. These Guides and their subsequent amendments shall be considered by the Executive Council.

APPENDIX. PROCEDURES FOR AMENDING WMO MANUALS AND GUIDES THAT ARE THE RESPONSIBILITY OF THE COMMISSION FOR BASIC SYSTEMS

\hypertarget{designation-of-responsible-committees}{%
\section{1. DESIGNATION OF RESPONSIBLE COMMITTEES}\label{designation-of-responsible-committees}}

The Commission for Basic Systems (CBS) shall, for each Manual and Guide, designate one of its Open Programme Area Groups (OPAGs) as being responsible for that Manual and its associated technical guides. The Open Programme Area Group may choose to designate one of its Expert Teams as the designated committee for managing changes to all or part of that Manual; if no Expert Team is designated, the Implementation Coordination Team for the OPAG takes on the role of the designated committee.

\hypertarget{general-validation-and-implementation-procedures}{%
\section{2. GENERAL VALIDATION AND IMPLEMENTATION PROCEDURES}\label{general-validation-and-implementation-procedures}}

\hypertarget{proposal-of-amendments}{%
\subsection{2.1 Proposal of amendments}\label{proposal-of-amendments}}

Amendments to a Manual or a Guide managed by CBS shall be proposed in writing to the Secretariat. The proposal shall specify the needs, purposes and requirements and include information on a contact point for technical matters.

\hypertarget{drafting-recommendation}{%
\subsection{2.2 Drafting recommendation}\label{drafting-recommendation}}

The designated committee for the relevant part of a Manual or a Guide, supported by the Secretariat, shall validate the stated requirement (unless it is consequential to an amendment to the WMO Technical Regulations) and develop a draft recommendation to respond to the requirement, as appropriate.

\hypertarget{procedures-for-approval}{%
\subsection{2.3 Procedures for approval}\label{procedures-for-approval}}

After a draft recommendation of the designated committee is validated in accordance with the procedure given in section~7 below, depending on the type of amendments, the designated committee should select one of the following procedures for the approval of the amendments:

(a) Simple (fast-track) procedure (see section~3 below);

(b) Standard (adoption of amendments between CBS sessions) procedure (see section~4 below);

(c) Complex (adoption of amendments during CBS sessions) procedure (see section~5 below).

\hypertarget{date-of-implementation}{%
\subsection{2.4 Date of implementation}\label{date-of-implementation}}

The designated committee should define an implementation date in order to give WMO Members sufficient time to implement the amendments after the date of notification. For procedures other than the simple (fast-track) one, if the time between the date of notification and implementation date is less than six months, the designated committee shall document the reasons for its decision.

\hypertarget{urgent-introduction}{%
\subsection{2.5 Urgent introduction}\label{urgent-introduction}}

Regardless of the above procedures, as an exceptional measure, the following procedure accommodates urgent user needs to introduce elements in lists of technical details, or to correct errors:

(a) A draft recommendation developed by the designated committee shall be validated according to the steps defined in section~7 below;

(b) The draft recommendation for pre-operational use of a list entry, which can be used in operational data and products, shall be approved by the chairperson of the designated committee and the chairperson of the responsible OPAG, and the president of CBS. A listing of pre-operational list entries is kept online on the WMO web server;

(c) Pre-operational list entries shall then be submitted for approval by one of the procedures in 2.3 above for operational use;

(d) Any version numbers associated with the technical implementation should be incremented at the least significant level.

\hypertarget{issuing-updated-version}{%
\subsection{2.6 Issuing updated version}\label{issuing-updated-version}}

Once amendments to a Manual or a Guide are adopted, an updated version of the relevant part of the Manual shall be issued in the languages agreed for its publication. The Secretariat shall inform all Members of the availability of a new updated version of that part at the date of notification mentioned in 2.4 above. If amendments are not incorporated into the published text of the relevant Manual or Guide at the time of the amendment, there should be a mechanism to publish the amendments at the time of their implementation and to retain a permanent record of the sequence of amendments.

\hypertarget{simple-fast-track-procedure}{%
\section{3. SIMPLE (FAST-TRACK) PROCEDURE}\label{simple-fast-track-procedure}}

\hypertarget{scope}{%
\subsection{3.1 Scope}\label{scope}}

The simple (fast-track) procedure shall be used only for changes to components of the Manual that have been designated and marked as ``technical specifications to which the simple (fast-track) procedure for the approval of amendments may be applied''.

Note: An example would be the addition of code list items in the \emph{Manual on Codes} (WMO-No.~306).

\hypertarget{endorsement}{%
\subsection{3.2 Endorsement}\label{endorsement}}

Draft recommendations developed by the responsible committee, including a date for implementation of the amendments, shall be submitted to the chairperson of the relevant OPAG for endorsement.

\hypertarget{approval}{%
\subsection{3.3 Approval}\label{approval}}

\hypertarget{minor-adjustments}{%
\subsubsection{3.3.1 Minor adjustments}\label{minor-adjustments}}

Correcting typographical errors in descriptive text is considered a minor adjustment, and will be done by the Secretariat in consultation with the president of CBS. See Figure~1.

Figure 1. Adoption of amendments to a Manual by minor adjustment

\hypertarget{other-types-of-amendments}{%
\subsubsection{3.3.2 Other types of amendments}\label{other-types-of-amendments}}

For other types of amendments, the English version of the draft recommendation, including a date of implementation, should be distributed to the focal points for matters concerning the relevant Manual for comments, with a deadline of two months for the reply. It should then be submitted to the president of CBS for consultation with presidents of technical commissions affected by the change. If endorsed by the president of CBS, the change should be passed to the President of WMO for consideration and adoption on behalf of the Executive Council (EC).

\hypertarget{frequency}{%
\subsubsection{3.3.3 Frequency }\label{frequency}}

The implementation of amendments approved through the simple (fast-track) procedure can be twice a year in May and November. See Figure~2.

The implementation of amendments approved through the simple (fast-track) procedure can be twice a year in May and November. See Figure 2.

Figure 2. Adoption of amendments to a Manual be simple (fast-track) procedure.

\hypertarget{standard-adoption-of-amendments-between-cbs-sessions-procedure}{%
\section{4. STANDARD (ADOPTION OF AMENDMENTS BETWEEN CBS SESSIONS) PROCEDURE}\label{standard-adoption-of-amendments-between-cbs-sessions-procedure}}

\hypertarget{scope-1}{%
\subsection{4.1 Scope}\label{scope-1}}

The standard (adoption of amendments between CBS sessions) procedure shall be used for changes that have an operational impact on those Members who do not wish to exploit the change, but that have only minor financial impact, or that are required to implement changes in the \emph{Technical Regulations} (WMO-No.~49), Volume~II -- Meteorological Service for International Air Navigation.

\hypertarget{approval-of-draft-recommendations}{%
\subsection{4.2 Approval of draft recommendations}\label{approval-of-draft-recommendations}}

For the direct adoption of amendments between CBS sessions, the draft recommendation developed by the designated committee, including a date of implementation of the amendments, shall be submitted to the chairperson of the responsible OPAG and president and vice-president of CBS for approval. The president of CBS shall consult with the presidents of technical commissions affected by the change. In the case of recommendations in response to changes in the \emph{Technical Regulations} (WMO-No.~49), Volume~II -- Meteorological Service for International Air Navigation, the president of CBS shall consult with the president of the Commission for Aeronautical Meteorology.

\hypertarget{circulation-to-members}{%
\subsection{4.3 Circulation to Members}\label{circulation-to-members}}

Upon approval of the president of CBS, the Secretariat sends the recommendation to all Members, in the languages in which the Manual is published, including a date of implementation of the amendments, for comments to be submitted within two months following the dispatch of the amendments. If the recommendation is sent to Members via electronic mail, there shall be public announcement of the amendment process including dates, for example by WMO Operational Newsletter on the WMO website, to ensure all relevant Members are informed.

\hypertarget{agreement}{%
\subsection{4.4 Agreement}\label{agreement}}

Those Members not having replied within the two months following the dispatch of the amendments are implicitly considered as having agreed with the amendments.

\hypertarget{coordination}{%
\subsection{4.5 Coordination}\label{coordination}}

Members are invited to designate a focal point responsible to discuss any comments/disagreements with the designated committee. If the discussion between the designated committee and the focal point cannot result in an agreement on a specific amendment by a Member, this amendment will be reconsidered by the designated committee. If a Member cannot agree that the financial or operational impact is minor, the redrafted amendment shall be approved by the complex (adoption of amendments during CBS sessions) procedure described in section 5 below.

\hypertarget{notification}{%
\subsection{4.6 Notification}\label{notification}}

Once amendments are agreed by Members, and after consultation with the chairperson of the responsible OPAG, the vice-president of CBS and the president of CBS (who should consult with presidents of other commissions affected by the change), the Secretariat notifies at the same time the Members and the members of the Executive Council of the approved amendments and of the date of their implementation. See Figure~3.

Figure 3. Adoption of amendments between CBS sessions

\hypertarget{complex-adoption-of-amendments-during-cbs-sessions-procedure}{%
\section{5. COMPLEX (ADOPTION OF AMENDMENTS DURING CBS SESSIONS) PROCEDURE}\label{complex-adoption-of-amendments-during-cbs-sessions-procedure}}

\hypertarget{scope-2}{%
\subsection{5.1 Scope}\label{scope-2}}

The complex (adoption of amendments during CBS sessions) procedure shall be used for changes for which the simple (fast-track) procedure or standard (adoption of amendments between CBS sessions) procedure cannot be applied.

\hypertarget{procedure}{%
\subsection{5.2 Procedure}\label{procedure}}

For the adoption of amendments during CBS sessions, the designated committee submits its recommendation, including a date of implementation of the amendments, to the Implementation Coordination Team of the responsible Open Programme Area Group. The recommendation is then passed to the presidents of technical commissions affected by the change for consultation, and to a CBS session that shall be invited to consider comments submitted by presidents of technical commissions. The document for the CBS session shall be distributed not later than 45~days before the opening of the session. Following the CBS session, the recommendation shall then be submitted to a session of the Executive Council for decision. See Figure~4.

\hypertarget{procedure-for-the-correction-of-existing-manual-contents}{%
\section{6. PROCEDURE FOR THE CORRECTION OF EXISTING MANUAL CONTENTS}\label{procedure-for-the-correction-of-existing-manual-contents}}

\hypertarget{correcting-errors-in-items-within-manuals}{%
\subsection{6.1 Correcting errors in items within Manuals }\label{correcting-errors-in-items-within-manuals}}

Where a minor error in the specification of an item that defines elements within a Manual is found, for example, a typing error or an incomplete definition, the item shall be amended and re-published. Any version numbers associated with items edited as a result of the change should be incremented at their lowest level of significance. If, however, the change has an impact on the meaning of the item, then a new item should be created and the existing (erroneous) item marked as deprecated. This situation is considered a minor adjustment according to 3.3.1 above.

Note: An example of an item for which this type of change applies is a code list entry for the Table Driven Code Forms or WMO Core Metadata Profile, in which the description contains typographical errors that can be corrected without changing the meaning of the description.

Figure 4. Adoption of amendments during CBS sessions

\hypertarget{correcting-an-error-in-the-specification-of-how-conformance-with-the-requirements-of-the-manual-can-be-checked}{%
\subsection{6.2 Correcting an error in the specification of how conformance with the requirements of the Manual can be checked}\label{correcting-an-error-in-the-specification-of-how-conformance-with-the-requirements-of-the-manual-can-be-checked}}

If an erroneous specification of a conformance-checking rule is found, the preferred approach is to add a new specification using the simple (fast-track) procedure or standard (adoption of amendments between CBS sessions) procedure. The new conformance-checking rule should be used instead of the old. An appropriate explanation shall be added to the description of the conformance-checking rule to clarify the practice along with the date of the change.

Note: An example of such a change would be correcting a conformance-checking rule in the WMO Core Metadata Profile.

\hypertarget{submission-of-corrections-to-errors}{%
\subsection{6.3 Submission of corrections to errors}\label{submission-of-corrections-to-errors}}

Such changes shall be submitted through the simple (fast-track) procedure.

\hypertarget{validation-procedure}{%
\section{7. VALIDATION PROCEDURE}\label{validation-procedure}}

\hypertarget{documentation-of-need-and-purpose}{%
\subsection{7.1 Documentation of need and purpose}\label{documentation-of-need-and-purpose}}

The need for, and the purpose of, the proposal for changes should be documented.

\hypertarget{documentation-of-result}{%
\subsection{7.2 Documentation of result}\label{documentation-of-result}}

This documentation shall include the results of validation testing of the proposal as described in 7.3 below.

\hypertarget{testing-with-relevant-applications}{%
\subsection{7.3 Testing with relevant applications}\label{testing-with-relevant-applications}}

For changes that have an impact on automated processing systems, the extent of the testing required before validation should be decided by the designated committee on a case-by-case basis, depending on the nature of the change. Changes involving a relatively high risk and/or impact on the systems should be tested by the use of at least two independently developed tool sets and two independent centres. In that case, results should be made available to the designated committee with a view to verifying the technical specifications.

DEFINITIONS

\textbf{Actual time of observation.}

(1) In the case of a surface synoptic observation, the time at which the barometer is read.

(2) In the case of upper-air observations, the time at which the balloon, parachute or rocket is actually released.

All-components schema document. An XML schema document that includes, either directly, or indirectly, all the components defined and declared in a namespace.

Alpine glow. Pink or yellow colouring assumed by mountain tops opposite the Sun when it is only just below the horizon before it rises and after it sets. This phenomenon vanishes after a brief interval of blue colouring, when the Earth's shadow reaches these summits.

Anomalous propagation. Propagation of radio energy in abnormal conditions of vertical distribution of refractive index, in association with abnormal distribution of atmospheric temperature and humidity. Use of the term is mainly confined to conditions in which abnormally large distances of propagation are attained.

Application schema. A conceptual schema for data required by one or more applications. (Source: International Organization for Standardization (ISO) 19101:2002, definition~4.2)

Atmospheric -- Sferic. Electromagnetic wave resulting from an electric discharge (lightning) in the atmosphere.

Automatic station. Meteorological station at which instruments make and transmit observations, the conversion to code form for international exchange being made either directly or at an editing station.

Aviation routine weather report. A statement of the observed meteorological conditions related to a specified time and location, issued on a routine basis for use in international air navigation.

BUFR -- Binary universal form for the representation of meteorological data. BUFR is the name of a binary code for the exchange and storage of data.

BUFR message. A single complete BUFR entity.

Category. The lists of sequence descriptors tabulated in BUFR or CREX Table~D are categorized according to their application; categories are provided for non-meteorological sequences, for various types of meteorological sequences, and for sequences which define reports, or major subsets of reports.

Class. A set of elements tabulated together in BUFR/CREX Table~B.

Condensation trails (contrails). Clouds which form in the wake of an aircraft when the atmosphere at flying level is sufficiently cold and humid.

Coordinate class. Classes~0--9 inclusive in BUFR/CREX Table~B define elements which assist in the definition of elements from subsequent classes; each of these classes is referred to as a coordinate class.

CREX -- Character form for the representation and exchange of data. CREX is the name of a table-driven alphanumeric code for the exchange and storage of data.

Data description operator. Operators which define replication or the operations listed in BUFR or CREX Table~C.

Data entity. A single data item.

Data subset. A set of data corresponding to the data description in a BUFR or CREX message; for observational data, a data subset usually corresponds to one observation.

Day darkness. Sky covered with clouds with very strong optical thickness (dark clouds) having a threatening appearance.

Descriptor. An entity entered within the Data description section to describe or define data; a descriptor may take the form of an element descriptor, a replication operator, an operator descriptor, or a sequence descriptor.

Dry thunderstorm. A thunderstorm without precipitation reaching the ground (distinct from a nearby thunderstorm with precipitation reaching the ground but not at the station at the time of observation).

Dust wall or sand wall. Front of a duststorm or sandstorm, having the appearance of a gigantic high wall which moves more or less rapidly.

Element descriptor. A descriptor containing a code figure reference to BUFR/CREX Table~B; the referenced entry defines an element, together with the units, scale factor, reference value and data width to be used to represent that element as data.

Equatorial regions. For the purpose of the analysis codes, the region between 30 °N and 30~°S latitudes.

Extensible markup language (XML). A markup language that defines a set of rules for encoding documents in a format that is both human-readable and machine-readable. It is defined in the World Wide Web Consortium (W3C) \href{http://www.w3.org/TR/REC-xml/}{XML 1.0 Specification}.

Geography markup language (GML). An XML encoding in compliance with ISO~19118 for the transport and storage of geographic information modelled in accordance with the conceptual modelling framework used in the ISO~19100 series of International Standards and including both the spatial and non-spatial properties of geographic features.

Geometric altitude. Vertical distance (Z) of a level, a point or an object considered as a point, measured from mean sea level.

Geopotential. That potential with which the Earth's gravitational field is associated. It is equivalent to the potential energy of unit mass relative to a standard level (mean sea level by convention) and is numerically equal to the work which would be done against gravity in raising the unit mass from sea level to the level at which the mass is located.

Geopotential \emph{ϕ} at geometric height \emph{z} is given by

where \emph{g} is the acceleration of gravity.

Geopotential height. Height of a point in the atmosphere expressed in units (geopotential metres) proportional to the geopotential at that height. Geopotential height expressed in geopotential metres is approximately equal to \emph{g}/9.8 times the geometric height expressed in (geometric) metres, \emph{g} being the local acceleration of gravity.

GML application schema. An application schema written in XML schema in accordance with the rules specified in ISO~19136:2007.

GML document. An XML document with a root element that is one of the XML elements AbstractFeature, Dictionary or TopoComplex specified in the GML schema or any element of a substitution group of any of these XML elements.

GML schema. The XML schema components in the XML namespace \url{http://www.opengis.net/gml/3.2} as specified in ISO 19136:2007.

Haboob. A strong wind and duststorm or sandstorm in northern and central Sudan. Its average duration is three hours; the average maximum wind velocity is over 15 m s\textsuperscript{--1}. The dust or sand forms a dense whirling wall which may be 1~000~m high; it is often preceded by isolated dust whirls. Haboobs usually occur after a few days of rising temperature and falling pressure.

\textbf{Ice crust (ice slick).}

(1) A type of snow crust; a layer of ice, thicker than a film crust, upon a snow surface. It is formed by the freezing of melt water or rainwater which has flowed into it.

(2) See \emph{Ice rind}.

Ice rind. A thin but hard layer of sea ice, river ice or lake ice. Apparently this term is used in at least two ways: (a) for a new encrustation upon old ice; and (b) for a single layer of ice usually found in bays and fjords where freshwater freezes on top of slightly colder sea water.

Instrumental wave data. Data on measured characteristics relating to period and height of the wave motion of the sea surface.

Inversion (layer). Atmospheric layer, horizontal or approximately so, in which the temperature increases with increasing height.

Isothermal layer. Atmospheric layer through which there is no change of temperature with height.

Jet stream. Flat tubular current of air, quasi-horizontal, whose axis is along a line of maximum speed and which is characterized not only by great speeds but also by strong transverse gradients of speed.

Line squall. Squall which occurs along a squall line.

Lithometeor. Meteor consisting of an ensemble of particles most of which are solid and non-aqueous. The particles are more or less suspended in the air, or lifted by the wind from the ground.

Mountain waves. Oscillatory motions of the atmosphere induced by flow over a mountain; such waves are formed over and to the lee of the mountain or mountain chain.

Namespace. A collection of names, identified by a uniform resource identifier reference, which are used in XML documents as element names and attribute names.

Normals. Period averages computed for over a uniform and relatively long period comprising at least three consecutive 10-year periods.

Obscured sky. Occasions of hydrometeors or lithometeors which are so dense as to make it impossible to tell whether there is cloud above or not.

Ocean weather station. A station aboard a suitably equipped and staffed ship that endeavours to remain at a fixed sea position and that makes and reports surface and upper-air observations and may also make and report subsurface observations.

Operator descriptor. A descriptor containing a code figure reference to BUFR or CREX Table~C, together with data to be used as an operand.

Past weather. Predominant characteristic of weather which had existed at the station during a given period of time.

Persistent condensation trail. Long-lived condensation trails which have spread to form clouds having the appearance of cirrus or patches of cirrocumulus or cirrostratus. It is sometimes impossible to distinguish such clouds from other cirrus, cirrocumulus or cirrostratus.

Present weather. Weather existing at the time of observation, or under certain conditions, during the hour preceding the time of observation.

Prevailing visibility. The greatest visibility value, observed in accordance with the definition of "visibility", which is reached within at least half the horizon circle or within at least half of the surface of the aerodrome. These areas could comprise contiguous or non-contiguous sectors.

Note: This value may be assessed by human observation and/or instrumented systems. When instruments are installed, they are used to obtain the best estimate of the prevailing visibility.

Purple light. Glow with a hue varying between pink and red, which is to be seen in the direction of the Sun before it rises and after it sets and is about 3° to 6° below the horizon. It takes the form of a segment of a more or less large luminous disc which appears above the horizon.

Reference value. All data are represented within a BUFR or CREX message by positive integers; to enable negative values to be represented, suitable negative base values are specified as reference values. The true value is obtained by addition of the reference value and the data as represented.

Replication descriptor. A special descriptor is reserved to define the replication operation; it is used to enable a given number of subsequent descriptors to be replicated a given number of times.

Root element. Each XML document has exactly one root element. This element, also known as the document element, encloses all the other elements and is therefore the sole parent element to all the other elements. The root element provides the starting point for processing the document.

Runway visual range. The range over which the pilot of an aircraft on the centre line of the runway can see the runway markings or the lights delineating the runway or identifying its centre line.

Schematron. A definition language for making assertions about patterns found in XML documents, differing in basic concept from other schema languages in that it is not based on grammars but on finding patterns in the parsed document.

Sea station. An observing station situated at sea. Sea stations include ships, ocean weather stations and stations on fixed or drifting platforms (rigs, platforms, lightships and buoys).

Section. A logical subdivision of a BUFR or CREX message, to aid description and definition.

Sequence descriptor. A descriptor used as a code figure to reference a single entry in BUFR or CREX Table~D; the referenced entry contains a list of descriptors to be substituted for the sequence descriptor.

Severe line squall. Severe squall which occurs along squall line (see \emph{Line squall}).

Snow haze. A suspension in the air of numerous minute snow particles, considerably reducing the visibility at the Earth's surface (visibility in snow haze often decreases to 50~m). Snow haze is observed most frequently in Arctic regions, before or after a snowstorm.

Squall. Atmospheric phenomenon characterized by a very large variation of wind speed: it begins suddenly, has a duration of the order of minutes and decreases rather suddenly in speed. It is often accompanied by a shower or thunderstorm.

Squall line. Fictitious moving line, sometimes of considerable extent, along which squall phenomena occur.

Sun pillar. Pillar of white light, which may or may not be continuous, which may be observed vertically above or below the sun. Sun pillars are most frequently observed near sunrise or sunset; they may extend to about 20° above the Sun, and generally end in a point. When a sun pillar appears together with a well-developed parhelic circle, a sun cross may appear at their intersection.

Synoptic hour. Hour, expressed in terms of universal time coordinated (UTC), at which, by international agreement, meteorological observations are made simultaneously throughout the globe.

Synoptic observation. A surface or upper-air observation made at standard time.

Synoptic surface observation. Synoptic observation, other than an upper-air observation, made by an observer or an automatic weather station on the Earth's surface.

Template. Description of the standardized layout of a set of data entities.

Tropical (Tropic). Pertaining to that region of the Earth's surface lying between the Tropic of Cancer and Tropic of Capricorn at 23° 30´ N and S, respectively.

Tropical cyclone. Cyclone of tropical origin of small diameter (some hundreds of kilometres) with minimum surface pressure in some cases less than 900~hPa, very violent winds and torrential rain; sometimes accompanied by thunderstorms. It usually contains a central region, known as the ``eye'' of the storm, with a diameter of the order of some tens of kilometres, and with light winds and more or less lightly clouded sky.

Tropical revolving storm. Tropical cyclone.

\textbf{Tropopause.}

(1) Upper limit of the troposphere. By convention, the ``first tropopause'' is defined as the lowest level at which the lapse rate decreases to 2~°C~km\textsuperscript{--1} or less, provided also the average lapse rate between this level and all higher levels within 2~km does not exceed 2~°C~km\textsuperscript{--1}.

(2) If, above the first tropopause, the average lapse rate between any level and all higher levels within 1~km exceeds 3~°C~km\textsuperscript{--1}, then a ``second tropopause'' is defined by the same criterion as under (1). This second tropopause may be either within or above the 1-km layer.

Twilight glow. See \emph{Purple light}.

Twilight glow in the mountains (Alpenglühen). See \emph{Alpine glow}.

Uniform resource identifier (URI). A compact sequence of characters that identifies an abstract~or physical resource. URI syntax is defined in the Internet Engineering Task Force (\href{http://www.ietf.org/rfc/rfc3986.txt}{IETF) RFC 3986}.

Unit of geopotential (\emph{H\textsubscript{m'}}). 1 standard geopotential metre = 0.980 665 dynamic metre

\begin{longtable}[]{@{}llll@{}}
\toprule
where & \emph{g(z)} & = & acceleration of gravity, in m s\textsuperscript{--2}, as a function of geometric height;\tabularnewline
\midrule
\endhead
& \emph{z} & = & geometric height, in metres;\tabularnewline
& \emph{H\textsubscript{m'}} & = & geopotential, in geopotential metres.\tabularnewline
\bottomrule
\end{longtable}

Vertical visibility. Maximum distance at which an observer can see and identify an object on the same vertical as himself, above or below

Visibility (for aeronautical purposes). Visibility for aeronautical purposes is the greater of:

(a) The greatest distance at which a black object of suitable dimensions, situated near the ground, can be seen and recognized when observed against a bright background;

(b) The greatest distance at which lights in the vicinity of 1~000~candelas can be seen and identified against an unlit background.

Note: The two distances have different values in air of a given extinction coefficient, and the latter (b) varies with the background illumination. The former (a) is represented by the meteorological optical range (MOR).

Whiteout. Uniformly white appearance of the landscape when the ground is snow covered and the sky is uniformly covered with clouds. An atmospheric optical phenomenon of the polar regions in which the observer appears to be engulfed in a uniformly white glow. Neither shadows, horizon, nor clouds are discernible; sense of depth and orientation are lost; only very dark, nearby objects can be seen. Whiteout occurs over an unbroken snow cover and beneath a uniformly overcast sky, when, with the aid of the snowblink effect, the light from the sky is about equal to that from the snow surface. Blowing snow may be an additional cause. The phenomenon is experienced in the air as well as on the ground.

Wind (mean wind, spot wind). Air motion relative to the Earth's surface. Unless it is otherwise specified, only the horizontal component is considered.

(1) \emph{Mean wind}: For the purpose of upper air reports from aircraft, mean wind is derived from the drift of the aircraft when flying from one fixed point to another or obtained by flying on a circuit around a fixed observed point and an immediate wind deduced from the drift of the aircraft.

(2) \emph{Spot wind}: For the purpose of upper-air reports from aircraft, the wind velocity, observed or predicted, for a specified location, height and time.

XML attribute. A start tag delimiting an XML element may contain one or more attributes. Attributes are Name-Value pairs, with the Name in each pair referred to as the attribute name and the Value (the text between the quote delimiters, that is, ' or ") as the attribute value. The order of attribute specifications in a start-tag or empty-element tag is not significant.

XML document. A structured document conforming to the rules specified in Extensible Markup Language (XML)~1.0 (Second Edition).

XML element. Each XML document contains one or more elements, the boundaries of which are either delimited by start-tags and end-tags, or, for empty elements, by an empty-element tag. Each element has a type, identified by name, sometimes called its generic identifier (GI), and may have a set of attribute specifications. An XML element may contain other XML elements, XML attributes or character data.

XML schema. A definition language offering facilities for describing the structure and constraining the contents of XML documents. The set of definitions for describing a particular XML document structure and associated constraints is referred to as an XML schema document.

XML schema document (XSD). An XML document containing XML schema component definitions and declarations.

Zodiacal light. White or yellowish light which spreads out, in the night sky, more or less along the zodiac from the horizon on the side on which the Sun is hidden. It is observed when the sky is sufficiently dark and the atmosphere sufficiently clear.

FM SYSTEM OF NUMBERING XML MARKUP LANGUAGE APPLICATION SCHEMAS

Each XML application schema bears a number, preceded by the letters FM. This number is followed by a numeral to identify the session of the Commission for Basic Systems (CBS) that either approved the XML application schema as a new one or made the latest amendment to its previous version. An XML application schema approved or amended by correspondence after a CBS session receives the number of that session.

Furthermore, an indicator term is used to designate the XML representation colloquially and is therefore called a "code name".

Notes on nomenclature:

(a) Changes and augmentations to the structure of the XML data representation shall be identified as different ``editions''. Each edition of the XML code is allocated a unique namespace. To distinguish between editions, namespaces include EITHER a year field, denoting the year in which those changes and augmentations were begun, OR a version number. For example, FM~202-16 METCE-XML has the namespace \url{http://def.wmo.int/metce/2013} (initial year of work 2013) whilst FM~205-16 IWXXM-XML has the namespace \url{http://icao.int/iwxxm/2.1} (version number~2.1).

(b) Changes to the content of any of the supporting tables are backward compatible. Terms used within the supporting tables may be deprecated; they will not be deleted. Once changes to the supporting tables are approved, a snapshot containing all the supporting tables required for XML codes will be published. Each snapshot is referred to as a ``table version''. The current table version for XML codes is Version 1.

(c) Backward-compatible changes, including addition of new elements or attributes to supporting tables, do not require a new edition of the XML code.

(d) Further XML code editions and table versions may be generated independently of one another in the future as requirements dictate.

The following table lists XML application schemas included within the FM numbering system, together with the corresponding code names and their reference list of approval decisions of the World Meteorological Congress, the Executive Council or CBS.

FM System of extensible markup language representations

\begin{longtable}[]{@{}ll@{}}
\toprule
\begin{minipage}[b]{0.47\columnwidth}\raggedright
\textbf{FM~201-15 Ext.\\
COLLECT‑XML}\strut
\end{minipage} & \begin{minipage}[b]{0.47\columnwidth}\raggedright
Collection of reports that use the same XML application schemas.

Resolution 32 (Cg-17)\strut
\end{minipage}\tabularnewline
\midrule
\endhead
\begin{minipage}[t]{0.47\columnwidth}\raggedright
\textbf{FM~201-16\\
COLLECT‑XML}\strut
\end{minipage} & \begin{minipage}[t]{0.47\columnwidth}\raggedright
Collection of reports that use the same XML application schemas.

Resolution 9 (EC-69)\strut
\end{minipage}\tabularnewline
\begin{minipage}[t]{0.47\columnwidth}\raggedright
\textbf{FM~202-15 Ext.\\
METCE‑XML}\strut
\end{minipage} & \begin{minipage}[t]{0.47\columnwidth}\raggedright
Foundation Meteorological Information. \emph{Modèle pour l'échange des informations sur le temps, le climat et l'eau} (Model for the Exchange of Information on Weather, Climate and Water). A set of foundation building blocks to support application schemas in the domains of interest to WMO, notably the weather, climate, hydrology, oceanography and space weather disciplines.

Resolution 32 (Cg-17)\strut
\end{minipage}\tabularnewline
\begin{minipage}[t]{0.47\columnwidth}\raggedright
\textbf{FM~202-16\\
METCE‑XML}\strut
\end{minipage} & \begin{minipage}[t]{0.47\columnwidth}\raggedright
Foundation Meteorological Information. \emph{Modèle pour l'échange des informations sur le temps, le climat et l'eau} (Model for the Exchange of Information on Weather, Climate and Water). A set of foundation building blocks to support application schemas in the domains of interest to WMO, notably the weather, climate, hydrology, oceanography and space weather disciplines.

Resolution 9 (EC-69)\strut
\end{minipage}\tabularnewline
\begin{minipage}[t]{0.47\columnwidth}\raggedright
\textbf{FM~203-15~Ext. OPM‑XML}\strut
\end{minipage} & \begin{minipage}[t]{0.47\columnwidth}\raggedright
Observable Property Model. Based on work by the Open Geospatial Consortium (OGC) Sensor Web Enablement Domain Working Group, this allows observable properties (also known as quantity kinds) to be aggregated into groups, and for any qualification or constraint relating to those observable properties to be described explicitly.

Resolution 32 (Cg-17)\strut
\end{minipage}\tabularnewline
\begin{minipage}[t]{0.47\columnwidth}\raggedright
\textbf{FM~204-15~Ext. SAF‑XML}\strut
\end{minipage} & \begin{minipage}[t]{0.47\columnwidth}\raggedright
Simple Aeronautical Features. Allows items such as airports or runways to be described to the level of detail required for reporting weather information for international civil aviation purposes.

Resolution 32 (Cg-17)\strut
\end{minipage}\tabularnewline
\begin{minipage}[t]{0.47\columnwidth}\raggedright
\textbf{FM~205-15~Ext. IWXXM‑XML}\strut
\end{minipage} & \begin{minipage}[t]{0.47\columnwidth}\raggedright
ICAO Meteorological Information Exchange Model. Defines the reports required by the International Civil Aviation Organization (ICAO) -- with information content equivalent to that in the alphanumeric METAR/SPECI, TAF and SIGMET code forms -- that are built from the components of the packages managed by WMO.

Resolution 32 (Cg-17)\strut
\end{minipage}\tabularnewline
\begin{minipage}[t]{0.47\columnwidth}\raggedright
\textbf{FM~205-16 IWXXM‑XML}\strut
\end{minipage} & \begin{minipage}[t]{0.47\columnwidth}\raggedright
ICAO Meteorological Information Exchange Model. Defines the reports required by the International Civil Aviation Organization (ICAO) -- with information content equivalent to that in the alphanumeric METAR/SPECI, TAF, SIGMET, AIRMET, Tropical Cyclone Advisory, and Volcanic Ash Advisory code forms -- that are built from the components of the packages managed by WMO.

Resolution 9 (EC-69)\strut
\end{minipage}\tabularnewline
\begin{minipage}[t]{0.47\columnwidth}\raggedright
\textbf{FM~221-16\\
TSML‑XML}\strut
\end{minipage} & \begin{minipage}[t]{0.47\columnwidth}\raggedright
Representation of information as time series.

Resolution 9 (EC‑69)\strut
\end{minipage}\tabularnewline
\begin{minipage}[t]{0.47\columnwidth}\raggedright
\textbf{FM~231-16\\
WMLTS‑XML}\strut
\end{minipage} & \begin{minipage}[t]{0.47\columnwidth}\raggedright
Hydrological Time Series. Allows a monotonic series of observations over time to be described to the level of detail required for accurate representation as time series, with specific consideration for hydrological data.

Resolution 11 (EC-69)\strut
\end{minipage}\tabularnewline
\begin{minipage}[t]{0.47\columnwidth}\raggedright
\textbf{FM~232-16\\
WMLRGS‑XML}\strut
\end{minipage} & \begin{minipage}[t]{0.47\columnwidth}\raggedright
Ratings, Gaugings and Sections. Allows the description of the process and conversions used to determine hydrological observations such as river discharge.

Resolution 11 (EC-69)\strut
\end{minipage}\tabularnewline
\begin{minipage}[t]{0.47\columnwidth}\raggedright
\textbf{FM 241-16\\
WMDR-XML}\strut
\end{minipage} & \begin{minipage}[t]{0.47\columnwidth}\raggedright
WMO Integrated Global Observing System (WIGOS) metadata representation. Allows WIGOS metadata to be exchanged.

The code tables supporting WIGOS metadata are included in this code form.

The code tables were approved by Resolution 10 (EC-69).\strut
\end{minipage}\tabularnewline
\bottomrule
\end{longtable}

1. Representation of information in extensible markup language

1.1 XML documents shall be well-formed with respect to XML~1.0 {[}Extensible Markup Language (XML) 1.0 (Second Edition){]}.

Notes:

1. The XML implementation specified in this Manual is described using the XML schema definition language (XSD) {[}XML Schema Part~1: Structures (Second Edition), XML Schema Part~2: Datatypes (Second Edition){]} and Schematron {[}ISO/IEC~19757-3:2006, Information technology -- Document Schema Definition Languages (DSDL) -- Part~3: Rule-based validation -- Schematron{]}.

2. Within this Manual, XPath {[}XML Path Language (XPath) 2.0 (Second Edition){]} is used to refer to particular elements and attributes within an XML document.

1.2 XML documents conforming to XML schema that have been allocated an FM identifier within this Manual shall, in addition to conforming to the Regulation for the specified code form, conform to the requirements specified in clause 2.4 of ISO~19136:2007 {[}ISO~19136:2007, Geographic information -- Geography Markup Language (GML){]}.

Notes:

1. XML schemas defined in this Manual are conformant to the encoding rules specified in ISO~19136:2007 and are categorized as ``GML application schema''. Similarly, XML documents conforming to requirements from ISO~19136:2007 are termed ``GML documents''.

2. Conformance tests for GML documents are provided in ISO~19136:2007, Annex~A, A.3 -- Abstract test suite for GML documents.

3. The Content-Type {[}IETF RFC 2387 MIME Multipart/Related Content-type{]} for GML documents is ``application/gml+xml''.

1.3 Information exchanged in XML using the WMO Information System (WIS) shall conform to publicly available GML application schemas.

1.4 Information that is exchanged as XML using WIS and that is capable of being represented according to the GML application schemas defined in this Manual should conform to the GML application schemas defined within this Manual.

1.5 Creators of GML documents conforming to the GML application schemas defined in this Manual shall ensure that their GML documents are valid with respect to the associated XML schema documents (XSD).

1.6 Creators of GML documents conforming to the GML application schemas defined in this Manual shall ensure that their GML documents validate against the associated Schematron schema(s) that test conformance with the specified GML application schema.

Note: It is not necessary for recipients to validate each document.

1.7 All date-time elements shall be encoded using ISO~8601 extended time format {[}ISO~8601:2004, Data elements and interchange formats -- Information interchange -- Representation of dates and times{]}.

1.8 The value of each time element shall include a time zone definition according to the ISO~8601 standard. The time zone provided should be universal time coordinated (UTC).

Note: A time zone is specified using a signed four-digit character or a ``Z'' to represent Zulu or UTC according to the following regular expression: (Z\textbar{[}+-{]}HH:MM).

1.9 All units of measure shall use the appropriate code from the Unified Code for Units of Measure (UCUM) code system. The unit of measure shall be identified by encoding the UCUM code in the ``uom'' attribute of the gml:MeasureType. Where no UCUM code is provided for the unit of measure, the unit of measure should be identified using a URI that resolves to an online definition that is recognized by some level of authority.

Notes:

1. The UCUM base codes are available in XML form at \url{http://unitsofmeasure.org/ucum-essence.xml}.

2. A list of units of measure appropriate to the weather, water and climate domains are provided at \url{http://codes.wmo.int/common/unit}. Each unit of measurement listed therein has a URI identifier.

1.10 Where an xlink:href attribute is used to reference a resource from within an XML document, the xlink:title attribute should not be used to provide a textual description of that resource.

2. Unique identifiers to identify code table items and definitions

2.1 The GML application schemas defined in this Manual make extensive use of externally managed codes and vocabulary items, with the majority drawn from code tables or code lists in Volumes~I.1 and I.2.

2.2 Code or vocabulary items are referenced from within XML documents using the xlink:href attribute {[}XML Linking Language (XLink) Version~1.1{]}.

2.3 The target code table or vocabulary from which codes or vocabulary items shall, should or may be drawn is defined within the GML application schema using the //annotation/appinfo/vocabulary element within the XML type definition.

2.4 The level of validation applied when assessing membership of codes or vocabulary items within the target code table or vocabulary is defined within the GML application schema using the //annotation/appinfo/extensibility element within the XML type definition. The interpretation of extensibility is as follows:

(a) \textless extensibility\textgreater{} ``none'' indicates that codes or vocabulary items shall be drawn from the target code table or vocabulary;

(b) \textless extensibility\textgreater{} ``narrower'' indicates that codes or vocabulary items shall be drawn from the target code table or vocabulary, or that the code or vocabulary item used shall be derived from another term within the target code table or vocabulary using a more refined, or narrower, definition;

(c) \textless extensibility\textgreater{} ``any'' indicates that codes or vocabulary items may be drawn from the target code table, code list or vocabulary or any other code table or vocabulary deemed appropriate by the author.

2.5 Code or vocabulary items referenced from within GML documents should have an available online definition and have been recognized by some level of authority.

2.6 Each code list managed by WMO in support of XML application schemas shall have a unique identifier of the form: http://codes.wmo.int/\textless identifier\textgreater.

Notes:

1. The recommended practice for selecting \textless identifier\textgreater{} is to base it on the WMO number of the publication defining the appropriate regulation, and the table within that publication. An example of a unique identifier is \url{http://codes.wmo.int/306/4678/BLSN}.

2. WMO provides a web service that makes the unique references ``resolvable''. This means that if a unique identifier, such as \url{http://codes.wmo.int/306/4678/BLSN}, is entered as a URL into a browser, the definition of the item corresponding to the unique reference is displayed.

3. Tables and code lists supporting the WMO Logical Data Model

3.1 Application of code tables and code lists

Regulations specified in code tables or code lists in Volumes~I.1 and I.2 shall apply to the corresponding entries in code tables used within the GML application schemas defined in this Manual.

3.2 Nil reasons

3.2.1 Nil-reason terms from Code table~D-1 shall, where permitted within the GML application schemas defined in this Manual, be used to provide an explanation for recording a missing (or void) value within a GML document.

Notes:

1. Code table~D-1 is described in Appendix~A.

2. Code table~D-1 is published online at \url{http://codes.wmo.int/common/nil}.

3.2.2 Each nil-reason term is identified with a URI {[}IETF RFC~3986 Uniform Resource Identifier (URI): Generic Syntax{]}. The URI shall comprise the ``Code-space'' column concatenated with the ``Notation'' column of Code table~D-1.

3.3 Physical quantities

3.3.1 Terms from Code table~D-2 shall be used within the GML application schemas defined in this Manual to describe physical quantity kinds.

3.3.2 Each physical quantity kind is identified with a URI {[}IETF RFC~3986 Uniform Resource Identifier (URI): Generic Syntax{]}. The URI shall comprise the path \url{http://codes.wmo.int/common/quantity-kind} concatenated with the value listed in the ``Notation'' column of Code table~D-2.

4. References

4.1 Normative references

-- Extensible Markup Language (XML) 1.0 (Second Edition), W3C Recommendation (6~October~2000)

-- XML Schema Part~1: Structures (Second Edition), W3C Recommendation (28~October~2004)

-- XML Schema Part~2: Datatypes (Second Edition), W3C Recommendation (28~October~2004)

-- Namespaces in XML 1.0 (Third Edition), W3C Recommendation (8~December~2009)

-- XML Linking Language (XLink) Version 1.1, W3C Recommendation (6~May~2010)

-- ISO/IEC~19757-3:2006, Information technology -- Document Schema Definition Languages (DSDL) -- Part~3: Rule-based validation -- Schematron

-- ISO~8601:2004, Data elements and interchange formats -- Information interchange -- Representation of dates and times

-- ISO~19136:2007, Geographic information -- Geography Markup Language (GML)

-- ISO/TS~19139:2007, Geographic information -- Metadata -- XML schema implementation

-- OGC/IS~08-094r1 SWE Common Data Model Encoding Standard 2.0

-- OGC/SAP~09-146r2 GML Application Schema -- Coverages 1.0.1

-- OGC/IS~10-025r1 Observations and Measurements 2.0 -- XML Implementation

-- OGC/IS 10-126r4 WaterML 2.0: Part 1 -- Timeseries

-- OGC/IS 15-018r2 OGC WaterML2.0: Part 2 -- Ratings, Gaugings and Sections

-- OGC/IS 15-042r3 TimeseriesML 1.0 -- XML Encoding of the Timeseries Profile of Observations and Measurements

-- OGC/IS 15-043r3 Timeseries Profile of Observations and Measurements

4.2 Informative references

-- XML Path Language (XPath) 2.0 (Second Edition), W3C Recommendation (14~December~2010; correction 3~January~2011)

-- ISO~19103:2005 Geographic information -- Conceptual schema language

-- ISO~19109:2005 Geographic information -- Rules for application schema

-- ISO~19123:2005 Geographic information -- Schema for coverage geometry and functions

-- ISO~19156:2011 Geographic information -- Observations and measurements

-- IETF RFC~2387 MIME Multipart/Related Content-type (August~1998)

-- IETF RFC~3986 Uniform Resource Identifier (URI): Generic Syntax (January~2005)

FM~201: COLLECTION OF REPORTS

FM~201-15 EXT. COLLECT-XML Collection of reports

201-15-Ext.1 Scope

COLLECT-XML shall be used to represent a collection of GML feature instances of the same type of meteorological information. The intent is to allow XML encoded meteorological information to be packaged in a way that emulates the existing data distribution practices used within the Global Telecommunication System and aeronautical fixed service (AFS).

Notes:

1. The collection of meteorological information is often referred to as a bulletin.

2. XML encodings of meteorological information are defined in this Manual; for example, FM~205-15 EXT. IWXXM‑XML.

3. Aggregation of meteorological information in the form of meteorological bulletins usually takes place at a station or centre originating or compiling the bulletin, as agreed internationally. A meteorological bulletin may have one or more instances of meteorological information. If meteorological reports of routine messages are not available during compilation, a NIL report of that station should be included in the published contents of the bulletin.

The requirements classes defined in COLLECT-XML are listed in Table~201-15-Ext.1.

Table~201-15-Ext.1. Requirements classes defined in COLLECT-XML

\begin{longtable}[]{@{}ll@{}}
\toprule
Requirements classes &\tabularnewline
\midrule
\endhead
Requirements class & \url{http://def.wmo.int/collect/2014/req/xsd-meteorological-bulletin}, 201-15-Ext.3\tabularnewline
\bottomrule
\end{longtable}

201-15-Ext.2 XML schema for COLLECT-XML

Representations of information in COLLECT-XML shall declare the XML namespaces listed in Table~201-15-Ext.2 and Table~201-15-Ext.3.

Notes:

1. Additional namespace declarations may be required depending on the XML elements used within COLLECT-XML. In particular, the meteorological information included within the bulletin is likely to imply specific requirements regarding namespace declaration.

2. Schematron schemas providing additional constraints are provided as an external file to the XSD defining COLLECT-XML. The canonical location of this file is \url{http://schemas.wmo.int/rule/1.1/collect.sch}.

Table~201-15-Ext.2. XML namespaces defined for COLLECT-XML

\begin{longtable}[]{@{}lll@{}}
\toprule
XML namespace & Default namespace prefix & Canonical location of all-components schema document\tabularnewline
\midrule
\endhead
\url{http://def.wmo.int/collect/2014} & collect & \url{http://schemas.wmo.int/collect/1.1/collect.xsd}\tabularnewline
\bottomrule
\end{longtable}

Table~201-15-Ext.3. External XML namespaces used in COLLECT-XML

\begin{longtable}[]{@{}llll@{}}
\toprule
Standard & XML namespace & Default namespace prefix & Canonical location of all-components schema document\tabularnewline
\midrule
\endhead
XML schema & \url{http://www.w3.org/2001/XMLSchema} & xs &\tabularnewline
Schematron & \url{http://purl.oclc.org/dsdl/schematron} & sch &\tabularnewline
XSLT v2 & \url{http://www.w3.org/1999/XSL/Transform} & xsl &\tabularnewline
XML Linking Language & \url{http://www.w3.org/1999/xlink} & xlink & \url{http://www.w3.org/1999/xlink.xsd}\tabularnewline
ISO 19136:2007 GML & \url{http://www.opengis.net/gml/3.2} & gml & \url{http://schemas.opengis.net/gml/3.2.1/gml.xsd}\tabularnewline
\bottomrule
\end{longtable}

201-15-Ext.3 Requirements class: Meteorological bulletin

201-15-Ext.3.1 This requirements class is used to describe the collection of GML feature instances of meteorological information.

201-15-Ext.3.2 XML elements describing a meteorological bulletin shall conform to all requirements specified in Table~201-15-Ext.4.

201-15-Ext.3.3 XML elements describing a meteorological bulletin shall conform to all requirements of all relevant dependencies specified in Table~201-15-Ext.4.

Table~201-15-Ext.4. Requirements class xsd-meteorological-bulletin

\begin{longtable}[]{@{}ll@{}}
\toprule
Requirements class &\tabularnewline
\midrule
\endhead
\url{http://def.wmo.int/collect/2014/req/xsd-meteorological-bulletin} &\tabularnewline
Target type & Data instance\tabularnewline
Name & Meteorological bulletin\tabularnewline
\begin{minipage}[t]{0.47\columnwidth}\raggedright
Requirement\strut
\end{minipage} & \begin{minipage}[t]{0.47\columnwidth}\raggedright
\url{http://def.wmo.int/collect/2014/req/xsd-meteorological-bulletin/valid}

The content model of this element shall have a value that matches the content model of collect:MeteorologicalBulletin.\strut
\end{minipage}\tabularnewline
\begin{minipage}[t]{0.47\columnwidth}\raggedright
Requirement\strut
\end{minipage} & \begin{minipage}[t]{0.47\columnwidth}\raggedright
\url{http://def.wmo.int/collect/2014/req/xsd-meteorological-bulletin/bulletin-identifier}

The value of XML element collect:MeteorologicalBulletin/bulletinIdentifier shall conform to the general file-naming convention described in the \emph{Manual on the Global Telecommunication System} (WMO-No.~386), Attachment II-15.\strut
\end{minipage}\tabularnewline
\begin{minipage}[t]{0.47\columnwidth}\raggedright
Requirement\strut
\end{minipage} & \begin{minipage}[t]{0.47\columnwidth}\raggedright
\url{http://def.wmo.int/collect/2014/req/xsd-meteorological-bulletin/meteorological-information}

The XML element collect:MeteorologicalBulletin shall contain one or more child elements collect:MeteorologicalBulletin/collect:meteorologicalInformation, each of which shall contain one and only one child element expressing a report of meteorological information.\strut
\end{minipage}\tabularnewline
\begin{minipage}[t]{0.47\columnwidth}\raggedright
Requirement\strut
\end{minipage} & \begin{minipage}[t]{0.47\columnwidth}\raggedright
\url{http://def.wmo.int/collect/2014/req/xsd-meteorological-bulletin/consistent-meteorological-information-type}

An instance of collect:MeteorologicalBulletin shall contain only one type of meteorological information reports. All child elements of XML element collect:MeteorologicalBulletin/collect:meteorologicalInformation shall be of the same type, and hence have the same qualified name.\strut
\end{minipage}\tabularnewline
\bottomrule
\end{longtable}

Notes:

1. In the context of the file-naming convention, abbreviated headings are described in the \emph{Manual on the Global Telecommunication System} (WMO-No.~386), Part~II, 2.3.2.

2. Meteorological information reports include METAR, SPECI, TAF and SIGMET -- represented using XML elements iwxxm:METAR, iwxxm:SPECI, iwxxm:TAF and iwxxm:SIGMET.

3. The qualified name of a METAR is iwxxm:METAR, which is of type iwxxm:METARType.

FM~201-16 COLLECT-XML Collection of reports

201-16.1 Scope

COLLECT-XML shall be used to represent a collection of GML feature instances of the same type of meteorological information. The intent is to allow XML encoded meteorological information to be packaged in a way that emulates the existing data distribution practices used within the Global Telecommunication System and aeronautical fixed service (AFS).

Notes:

1. The collection of meteorological information is often referred to as a bulletin.

2. XML encodings of meteorological information are defined in this Manual; for example, FM~205-15~EXT. IWXXM‑XML.

3. Aggregation of meteorological information in the form of meteorological bulletins usually takes place at a station or centre originating or compiling the bulletin, as agreed internationally. A meteorological bulletin may have one or more instances of meteorological information. If meteorological reports of routine messages are not available during compilation, a NIL report of that station should be included in the published contents of the bulletin.

The requirements classes defined in COLLECT-XML are listed in Table~201-16.1.

Table~201-16.1. Requirements classes defined in COLLECT-XML

\begin{longtable}[]{@{}ll@{}}
\toprule
Requirements class &\tabularnewline
\midrule
\endhead
Requirement & \url{http://def.wmo.int/collect/2014/req/xsd-meteorological-bulletin}, 201-16.3\tabularnewline
\bottomrule
\end{longtable}

201-16.2 XML schema for COLLECT-XML

Representations of information in COLLECT-XML shall declare the XML namespaces listed in Table~201-16.2 and Table~201-16.3.

Notes:

1. Additional namespace declarations may be required depending on the XML elements used within COLLECT-XML. In particular, the meteorological information included within the bulletin is likely to imply specific requirements regarding namespace declaration.

2. Schematron schemas providing additional constraints are provided as an external file to the XSD defining COLLECT-XML. The canonical location of this file is \url{http://schemas.wmo.int/collect/1.2/rule/collect.sch}.

Table~201-16.2. XML namespaces defined for COLLECT-XML

\begin{longtable}[]{@{}lll@{}}
\toprule
XML namespace & Default namespace prefix & Canonical location of all-components schema document\tabularnewline
\midrule
\endhead
\url{http://def.wmo.int/collect/2014} & collect & \url{http://schemas.wmo.int/collect/1.2/collect.xsd}\tabularnewline
\bottomrule
\end{longtable}

Table~201-16.3. External XML namespaces used in COLLECT-XML

\begin{longtable}[]{@{}llll@{}}
\toprule
Standard & XML namespace & Default namespace prefix & Canonical location of all-components schema document\tabularnewline
\midrule
\endhead
XML schema & \url{http://www.w3.org/2001/XMLSchema} & xs &\tabularnewline
Schematron & \url{http://purl.oclc.org/dsdl/schematron} & sch &\tabularnewline
XSLT v2 & \url{http://www.w3.org/1999/XSL/Transform} & xsl &\tabularnewline
XML Linking Language & \url{http://www.w3.org/1999/xlink} & xlink & \url{http://www.w3.org/1999/xlink.xsd}\tabularnewline
ISO 19136:2007 GML & \url{http://www.opengis.net/gml/3.2} & gml & \href{http://schemas.opengis.net/\%20gml/3.2.1/gml.xsd}{http://schemas.opengis.net/ gml/3.2.1/gml.xsd}\tabularnewline
\bottomrule
\end{longtable}

201-16.3 Requirements class: Meteorological bulletin

201-16.3.1 This requirements class is used to describe the collection of GML feature instances of meteorological information.

201-16.3.2 XML elements describing a meteorological bulletin shall conform to all requirements specified in Table~201-16.4.

201-16.3.3 XML elements describing a meteorological bulletin shall conform to all requirements of all relevant dependencies specified in Table~201-16.4.

Table~201-16.4. Requirements class xsd-meteorological-bulletin

\begin{longtable}[]{@{}ll@{}}
\toprule
Requirements class &\tabularnewline
\midrule
\endhead
\url{http://def.wmo.int/collect/2014/req/xsd-meteorological-bulletin} &\tabularnewline
Target type & Data instance\tabularnewline
Name & Meteorological bulletin\tabularnewline
\begin{minipage}[t]{0.47\columnwidth}\raggedright
Requirement\strut
\end{minipage} & \begin{minipage}[t]{0.47\columnwidth}\raggedright
\url{http://def.wmo.int/collect/2014/req/xsd-meteorological-bulletin/valid}

The content model of this element shall have a value that matches the content model of collect:MeteorologicalBulletin.\strut
\end{minipage}\tabularnewline
\begin{minipage}[t]{0.47\columnwidth}\raggedright
Requirement\strut
\end{minipage} & \begin{minipage}[t]{0.47\columnwidth}\raggedright
\url{http://def.wmo.int/collect/2014/req/xsd-meteorological-bulletin/bulletin-identifier}

The value of XML element collect:MeteorologicalBulletin/bulletinIdentifier shall conform to the general file-naming convention described in the \emph{Manual on the Global Telecommunication System} (WMO-No.~386), Attachment~II-15.\strut
\end{minipage}\tabularnewline
\begin{minipage}[t]{0.47\columnwidth}\raggedright
Requirement\strut
\end{minipage} & \begin{minipage}[t]{0.47\columnwidth}\raggedright
\url{http://def.wmo.int/collect/2014/req/xsd-meteorological-bulletin/meteorological-information}

The XML element collect:MeteorologicalBulletin shall contain one or more child elements collect:MeteorologicalBulletin/collect:meteorologicalInformation, each of which shall contain one and only one child element expressing a report of meteorological information.\strut
\end{minipage}\tabularnewline
\begin{minipage}[t]{0.47\columnwidth}\raggedright
Requirement\strut
\end{minipage} & \begin{minipage}[t]{0.47\columnwidth}\raggedright
\url{http://def.wmo.int/collect/2014/req/xsd-meteorological-bulletin/consistent-meteorological-information-type}

An instance of collect:MeteorologicalBulletin shall contain only one type of meteorological information reports. All child elements of XML element collect:MeteorologicalBulletin/collect:meteorologicalInformation shall be of the same type, and hence have the same qualified name.\strut
\end{minipage}\tabularnewline
\bottomrule
\end{longtable}

Notes:

1. In the context of the file-naming convention, abbreviated headings are described in the Manual on the Global Telecommunication System (WMO-No.~386), Part~II, 2.3.2.

2. Meteorological information reports include METAR, SPECI, TAF, SIGMET, AIRMET, Tropical Cyclone Advisory and Volcanic Ash Advisory -- represented using XML elements iwxxm:METAR, iwxxm:SPECI, iwxxm:TAF, iwxxm:SIGMET, iwxxm:AIRMET, iwxxm:TropicalCycloneAdvisory and iwxxm:VolcanicAshAdvisory.

3. The qualified name of a METAR is iwxxm:METAR, which is of type iwxxm:METARType.

FM~202: METCE

FM~202-15 EXT. METCE-XML FOUNDATION METEOROLOGICAL INFORMATION

202-15-Ext.1 Scope

METCE-XML shall be used for the exchange in XML of meteorological information conforming to the \emph{Modèle pour l'échange des informations sur le temps, le climat et l'eau} (METCE) application schema. METCE-XML may be used directly to encode meteorological information or incorporated as components within other XML encodings.

Note: The METCE application schema is described in the \emph{Guidelines on Data Modelling for WMO Codes} (available in English only from \url{http://wis.wmo.int/metce-uml}).

The requirements classes defined in METCE-XML are listed in Table~202-15-Ext.1.

Table~202-15-Ext.1. Requirements classes defined in METCE-XML

\begin{longtable}[]{@{}ll@{}}
\toprule
Requirements classes &\tabularnewline
\midrule
\endhead
Requirements class & \vtop{\hbox{\strut \url{http://def.wmo.int/metce/2013/req/xsd-complex-sampling-measurement},}\hbox{\strut 202-15-Ext.4}}\tabularnewline
Requirements class & \vtop{\hbox{\strut \url{http://def.wmo.int/metce/2013/req/xsd-sampling-coverage-measurement},}\hbox{\strut 202-15-Ext.5}}\tabularnewline
Requirements class & \url{http://def.wmo.int/metce/2013/req/xsd-sampling-observation}, 202-15-Ext.6\tabularnewline
Requirements class & \url{http://def.wmo.int/metce/2013/req/xsd-volcano}, 202-15-Ext.7\tabularnewline
Requirements class & \url{http://def.wmo.int/metce/2013/req/xsd-erupting-volcano}, 202-15-Ext.8\tabularnewline
Requirements class & \url{http://def.wmo.int/metce/2013/req/xsd-tropical-cyclone}, 202-15-Ext.9\tabularnewline
Requirements class & \url{http://def.wmo.int/metce/2013/req/xsd-process}, 202-15-Ext.10\tabularnewline
Requirements class & \url{http://def.wmo.int/metce/2013/req/xsd-measurement-context}, 202-15-Ext.11\tabularnewline
\bottomrule
\end{longtable}

202-15-Ext.2 XML schema for METCE-XML

Representations of information in METCE-XML shall declare the XML namespaces listed in Table~202-15-Ext.2 and Table~202-15-Ext.3.

Notes:

1. Additional namespace declarations may be required depending on the XML elements used within METCE‑XML.

2. The XML schema is packaged in three XML schema documents (XSD) describing one XML namespace: \url{http://def.wmo.int/metce/2013}.

3. Schematron schemas providing additional constraints are embedded within the XSD defining METCE-XML.

Table~202-15-Ext.2. XML namespaces defined for METCE-XML

\begin{longtable}[]{@{}lll@{}}
\toprule
XML namespace & Default namespace prefix & Canonical location of all-components schema document\tabularnewline
\midrule
\endhead
\url{http://def.wmo.int/metce/2013} & metce & \href{http://schemas.wmo.int/metce/1.0/metce.xsd}{http://schemas.wmo.int/metce/1.1/metce.xsd}\tabularnewline
\bottomrule
\end{longtable}

Table~202-15-Ext.3. External XML namespaces used in METCE-XML

\begin{longtable}[]{@{}llll@{}}
\toprule
Standard & XML namespace & Default namespace prefix & Canonical location of all-components schema document\tabularnewline
\midrule
\endhead
XML schema & \url{http://www.w3.org/2001/XMLSchema} & xs &\tabularnewline
Schematron & \url{http://purl.oclc.org/dsdl/schematron} & sch &\tabularnewline
XSLT v2 & \url{http://www.w3.org/1999/XSL/Transform} & xsl &\tabularnewline
XML Linking Language & \url{http://www.w3.org/1999/xlink} & xlink & \url{http://www.w3.org/1999/xlink.xsd}\tabularnewline
ISO 19136:2007 GML & \url{http://www.opengis.net/gml/3.2} & gml & \url{http://schemas.opengis.net/gml/3.2.1/gml.xsd}\tabularnewline
ISO/TS 19139:2007 metadata XML implementation & \url{http://www.isotc211.org/2005/gmd} & gmd & \url{http://standards.iso.org/ittf/PubliclyAvailableStandards/ISO_19139_Schemas/gmd/gmd.xsd}\tabularnewline
OGC OMXML & \url{http://www.opengis.net/om/2.0} & om & \url{http://schemas.opengis.net/om/2.0/observation.xsd}\tabularnewline
OGC OMXML & \url{http://www.opengis.net/samplingSpatial/2.0} & sams & \url{http://schemas.opengis.net/samplingSpatial/2.0/spatialSamplingFeature.xsd}\tabularnewline
FM 203-15 Ext. OPM-XML & \url{http://def.wmo.int/opm/2013} & opm & \url{http://schemas.wmo.int/opm/1.1/opm.xsd}\tabularnewline
\bottomrule
\end{longtable}

202-15-Ext.3 Virtual typing

In accordance with OMXML (clause~7.2), the specialization of OM\_Observation is provided through Schematron restriction. The om:type element shall be used to specify the type of OM\_Observation that is being encoded using the URI for the corresponding observation type listed in Code table~D-3.

Notes:

1. Code table~D-3 is described in Appendix~A.

2. Code table~D-3 is published online at \url{http://codes.wmo.int/common/observation-type/METCE/2013}.

3. The URI for each observation type is composed by appending the \emph{notation} to the \emph{code-space}. As an example, the URI of ComplexSamplingMeasurement is \url{http://codes.wmo.int/common/observation-type/METCE/2013/ComplexSamplingMeasurement}.

4. Each URI will resolve to provide further information about the associated observation type.

5. The terms ``observation'' and ``measurement'' evoke a particular concept to meteorologists (for example, the measurement of a physical phenomenon using an instrument or sensor). As defined in ISO~19156:2011, Geographic information -- Observations and measurements, an instance of OM\_Observation is defined as an ``estimate of the value of some property of some feature of interest using a specified procedure''. OM\_Measurement is clearly applicable to the measurement of some physical property values using an instrument or sensor but is equally applicable to the numerical simulation of physical property values using a computational model (for example, a forecast or reanalysis).

202-15-Ext.4 Requirements class: Complex sampling measurement

202-15-Ext.4.1 This requirements class restricts the content model for the XML element om:OM\_Observation such that the ``result'' of the observation is a set of values relating to a specified location and time instant or duration, the ``feature of interest'' is a representative subset of the atmosphere or body of water and so forth based on a predetermined sampling regime and the ``procedure'' provides the set of information as specified by WMO.

Note: ComplexSamplingMeasurement (a subclass of OM\_ComplexObservation) is intended for use where the observation event is concerned with the evaluation of multiple measurands at a specified location and time instant or duration. OM\_ComplexObservation is used because the ``result'' of this class of observations is a group of measures, provided as a Record (as defined in ISO~19103:2005, Geographic information -- Conceptual schema language).

202-15-Ext.4.2 Instances of om:OM\_Observation with element om:type specifying \url{http://codes.wmo.int/common/observation-type/METCE/2013/ComplexSamplingMeasurement} shall conform to all requirements specified in Table~202-15-Ext.4.

202-15-Ext.4.3 Instances of om:OM\_Observation with element om:type specifying \url{http://codes.wmo.int/common/observation-type/METCE/2013/ComplexSamplingMeasurement} shall conform to all requirements of all relevant dependencies specified in Table~202-15-Ext.4.

Note: XML implementation of metce:ComplexSamplingMeasurement is dependent on:

-- OMXML {[}OGC/IS 10-025r1 Observations and Measurements 2.0 -- XML Implementation{]};

-- SWE~Common~2.0 {[}OGC/IS 08-094r1 SWE Common Data Model Encoding Standard 2.0{]}.

Table~202-15-Ext.4. Requirements class xsd-complex-sampling-measurement

\begin{longtable}[]{@{}ll@{}}
\toprule
Requirements class &\tabularnewline
\midrule
\endhead
\url{http://def.wmo.int/metce/2013/req/xsd-complex-sampling-measurement} &\tabularnewline
Target type & Data instance\tabularnewline
Name & Complex sampling measurement\tabularnewline
Dependency & \url{http://www.opengis.net/spec/OMXML/2.0/req/observation}, OMXML clause~7.3\tabularnewline
Dependency & \url{http://www.opengis.net/spec/OMXML/2.0/req/complexObservation}, OMXML clause~7.10\tabularnewline
Dependency & \url{http://www.opengis.net/spec/OMXML/2.0/req/sampling}, OMXML clause~7.14\tabularnewline
Dependency & \url{http://www.opengis.net/spec/OMXML/2.0/req/spatialSampling}, OMXML clause~7.15\tabularnewline
Dependency & \url{http://www.opengis.net/spec/SWE/2.0/req/xsd-simple-components}, SWE~Common~2.0 clause~8.1\tabularnewline
Dependency & \url{http://www.opengis.net/spec/SWE/2.0/req/xsd-record-components}, SWE~Common~2.0 clause~8.2\tabularnewline
Dependency & \url{http://www.opengis.net/spec/SWE/2.0/req/xsd-simple-encodings}, SWE~Common~2.0 clause~8.5\tabularnewline
Dependency & \url{http://www.opengis.net/spec/SWE/2.0/req/general-encoding-rules}, SWE~Common~2.0 clause~9.1\tabularnewline
Dependency & \url{http://www.opengis.net/spec/SWE/2.0/req/text-encoding-rules}, SWE~Common~2.0 clause~9.2\tabularnewline
Dependency & \url{http://www.opengis.net/spec/SWE/2.0/req/xml-encoding-rules}, SWE~Common~2.0 clause~9.3\tabularnewline
\begin{minipage}[t]{0.47\columnwidth}\raggedright
Requirement\strut
\end{minipage} & \begin{minipage}[t]{0.47\columnwidth}\raggedright
\url{http://def.wmo.int/metce/2013/req/xsd-complex-sampling-measurement/xmlns-declaration-swe}

The OGC SWE~Common~2.0 namespace \url{http://www.opengis.net/swe/2.0} shall be declared within the XML document.\strut
\end{minipage}\tabularnewline
\begin{minipage}[t]{0.47\columnwidth}\raggedright
Requirement\strut
\end{minipage} & \begin{minipage}[t]{0.47\columnwidth}\raggedright
\url{http://def.wmo.int/metce/2013/req/xsd-complex-sampling-measurement/procedure-metce-process}

The XML element om:procedure shall contain a child element metce:Process or any element of a substitution group of metce:Process.\strut
\end{minipage}\tabularnewline
Recommendation & The default namespace prefix used for \url{http://www.opengis.net/swe/2.0} should be ``swe''.\tabularnewline
\bottomrule
\end{longtable}

Notes:

1. Dependency \url{http://www.opengis.net/spec/OMXML/2.0/req/observation} has associated conformance class \url{http://www.opengis.net/spec/OMXML/2.0/conf/observation} (OMXML clause~A.1).

2. Dependency \url{http://www.opengis.net/spec/OMXML/2.0/req/complexObservation} has associated conformance class \url{http://www.opengis.net/spec/OMXML/2.0/conf/complexObservation} (OMXML clause~A.8).

3. Dependency \url{http://www.opengis.net/spec/OMXML/2.0/req/sampling} has associated conformance class \url{http://www.opengis.net/spec/OMXML/2.0/conf/sampling} (OMXML clause~A.12).

4. Dependency \url{http://www.opengis.net/spec/OMXML/2.0/req/spatialSampling} has associated conformance class \url{http://www.opengis.net/spec/OMXML/2.0/conf/spatialSampling} (OMXML clause~A.13).

5. Dependency \url{http://www.opengis.net/spec/SWE/2.0/req/xsd-simple-components} has associated conformance class \url{http://www.opengis.net/spec/SWE/2.0/conf/xsd-simple-components} (SWE~Common~2.0 clause~A.8).

6. Dependency \url{http://www.opengis.net/spec/SWE/2.0/req/xsd-record-components} has associated conformance class \url{http://www.opengis.net/spec/SWE/2.0/conf/xsd-record-components} (SWE~Common~2.0 clause~A.9).

7. Dependency \url{http://www.opengis.net/spec/SWE/2.0/req/xsd-simple-encodings} has associated conformance class \url{http://www.opengis.net/spec/SWE/2.0/conf/xsd-simple-encodings} (SWE~Common~2.0 clause~A.12).

8. Dependency \url{http://www.opengis.net/spec/SWE/2.0/req/general-encoding-rules} has associated conformance class \url{http://www.opengis.net/spec/SWE/2.0/conf/general-encoding-rules} (SWE~Common~2.0 clause~A.14).

9. Dependency \url{http://www.opengis.net/spec/SWE/2.0/req/text-encoding-rules} has associated conformance class \url{http://www.opengis.net/spec/SWE/2.0/conf/text-encoding-rules} (SWE~Common~2.0 clause~A.15).

10. Dependency \url{http://www.opengis.net/spec/SWE/2.0/req/xml-encoding-rules} has associated conformance class \url{http://www.opengis.net/spec/SWE/2.0/conf/xml-encoding-rules} (SWE~Common~2.0 clause~A.16).

11. The canonical schema location for OGC SWE~Common~2.0 (\url{http://www.opengis.net/swe/2.0}) is \url{http://schemas.opengis.net/sweCommon/2.0/swe.xsd}.

202-15-Ext.5 Requirements class: Sampling coverage measurement

202-15-Ext.5.1 This requirements class restricts the content model for the XML element om:OM\_Observation such that the ``result'' of the observation is a set of values describing the variation of properties with space and/or time, the ``feature of interest'' is a representative subset of the atmosphere or body of water and so forth based on a predetermined sampling regime and the ``procedure'' provides the set of information as specified by WMO.

Notes:

1. SamplingCoverageMeasurement (a subclass of OM\_DiscreteCoverageObservation) is intended for use where the observation event is concerned with the evaluation of measurands that vary with respect to space and/or time. OM\_DiscreteCoverageObservation is used because the ``result'' of this class of observations is a discrete coverage (as defined in ISO~19123:2005 Geographic information -- Schema for coverage geometry and functions).

2. SamplingCoverageMeasurement is based on the informative SamplingCoverageObservation specialization of OM\_Observation outlined in ISO~19156:2011, clause~D.3.4. Within METCE, additional restrictions are applied to the ``procedure''. Furthermore, the name is changed from SamplingCoverageObservation to SamplingCoverageMeasurement in an attempt to disambiguate the two classes and to mitigate confusion arising from use of the term observation.

202-15-Ext.5.2 Instances of om:OM\_Observation with element om:type specifying \url{http://codes.wmo.int/common/observation-type/METCE/2013/SamplingCoverageMeasurement} shall conform to all requirements specified in Table~202-15-Ext.5.

202-15-Ext.5.3 Instances of om:OM\_Observation with element om:type specifying \url{http://codes.wmo.int/common/observation-type/METCE/2013/SamplingCoverageMeasurement} shall conform to all requirements of all relevant dependencies specified in Table~202-15-Ext.5.

Note: XML implementation of metce:ComplexSamplingMeasurement is dependent on:

-- OMXML {[}OGC/IS 10-025r1 Observations and Measurements 2.0 -- XML Implementation{]};

-- SWE~Common~2.0 {[}OGC/IS 08-094r1 SWE Common Data Model Encoding Standard 2.0{]};

-- GMLCOV 1.0 {[}OGC/SAP 09-146r2 GML Application Schema -- Coverages 1.0.1{]}.

Table~202-15-Ext.5. Requirements class xsd-sampling-coverage-measurement

\begin{longtable}[]{@{}ll@{}}
\toprule
Requirements class &\tabularnewline
\midrule
\endhead
\url{http://def.wmo.int/metce/2013/req/xsd-sampling-coverage-measurement} &\tabularnewline
Target type & Data instance\tabularnewline
Name & Sampling coverage measurement\tabularnewline
Dependency & \url{http://www.opengis.net/spec/OMXML/2.0/req/observation}, OMXML clause~7.3\tabularnewline
Dependency & \url{http://www.opengis.net/spec/OMXML/2.0/req/sampling}, OMXML clause~7.14\tabularnewline
Dependency & \url{http://www.opengis.net/spec/OMXML/2.0/req/spatialSampling}, OMXML clause~7.15\tabularnewline
Dependency & \url{http://www.opengis.net/spec/SWE/2.0/req/xsd-simple-components}, SWE~Common~2.0 clause~8.1\tabularnewline
Dependency & \url{http://www.opengis.net/spec/SWE/2.0/req/xsd-record-components}, SWE~Common~2.0 clause~8.2\tabularnewline
Dependency & \url{http://www.opengis.net/spec/SWE/2.0/req/xsd-block-components}, SWE~Common~2.0 clause~8.4\tabularnewline
Dependency & \url{http://www.opengis.net/spec/SWE/2.0/req/xsd-simple-encodings}, SWE~Common~2.0 clause~8.5\tabularnewline
Dependency & \url{http://www.opengis.net/spec/SWE/2.0/req/general-encoding-rules}, SWE~Common~2.0 clause~9.1\tabularnewline
Dependency & \url{http://www.opengis.net/spec/SWE/2.0/req/text-encoding-rules}, SWE~Common~2.0 clause~9.2\tabularnewline
Dependency & \url{http://www.opengis.net/spec/SWE/2.0/req/xml-encoding-rules}, SWE~Common~2.0 clause~9.3\tabularnewline
Dependency & \url{http://www.opengis.net/spec/gmlcov/1.0/req/gml-coverage}, GMLCOV 1.0 clause~6\tabularnewline
Dependency & \url{http://www.opengis.net/doc/gml/gmlcov/1.0.1}, GMLCOV 1.0.1 clause~7\tabularnewline
\begin{minipage}[t]{0.47\columnwidth}\raggedright
Requirement\strut
\end{minipage} & \begin{minipage}[t]{0.47\columnwidth}\raggedright
\url{http://def.wmo.int/metce/2013/req/xsd-sampling-coverage-measurement/xmlns-declaration-swe}

The OGC SWE~Common~2.0 namespace \url{http://www.opengis.net/swe/2.0} shall be declared within the XML document.\strut
\end{minipage}\tabularnewline
\begin{minipage}[t]{0.47\columnwidth}\raggedright
Requirement\strut
\end{minipage} & \begin{minipage}[t]{0.47\columnwidth}\raggedright
\url{http://def.wmo.int/metce/2013/req/xsd-sampling-coverage-measurement/xmlns-declaration-gmlcov}

The OGC GMLCOV 1.0 namespace \url{http://www.opengis.net/gmlcov/1.0} shall be declared within the XML document.\strut
\end{minipage}\tabularnewline
\begin{minipage}[t]{0.47\columnwidth}\raggedright
Requirement\strut
\end{minipage} & \begin{minipage}[t]{0.47\columnwidth}\raggedright
\url{http://def.wmo.int/metce/2013/req/xsd-sampling-coverage-measurement/result-discrete-or-grid-coverage}

The XML element om:result shall contain a child element gml:DiscreteCoverage (or any element of a substitution group of gml:DiscreteCoverage), gml:GridCoverage, gml:RectifiedGridCoverage or gml:ReferenceableGridCoverage.\strut
\end{minipage}\tabularnewline
\begin{minipage}[t]{0.47\columnwidth}\raggedright
Requirement\strut
\end{minipage} & \begin{minipage}[t]{0.47\columnwidth}\raggedright
\url{http://def.wmo.int/metce/2013/req/xsd-sampling-coverage-measurement/result-coverage-gml-encoding}

The child element of om:result shall be represented in GML as defined in GMLCOV 1.0.1 clause~7. Multipart~representation and special format representation shall not be used.\strut
\end{minipage}\tabularnewline
\begin{minipage}[t]{0.47\columnwidth}\raggedright
Requirement\strut
\end{minipage} & \begin{minipage}[t]{0.47\columnwidth}\raggedright
\url{http://def.wmo.int/metce/2013/req/xsd-sampling-coverage-measurement/procedure-metce-process}

The XML element om:procedure shall contain a child element metce:Process or any element of a substitution group of metce:Process.\strut
\end{minipage}\tabularnewline
Recommendation & The default namespace prefix used for \url{http://www.opengis.net/swe/2.0} should be ``swe''.\tabularnewline
Recommendation & The default namespace prefix used for \url{http://www.opengis.net/gmlcov/1.0} should be ``gmlcov''.\tabularnewline
\bottomrule
\end{longtable}

Notes:

1. Dependency \url{http://www.opengis.net/spec/OMXML/2.0/req/observation} has associated conformance class \url{http://www.opengis.net/spec/OMXML/2.0/conf/observation} (OMXML clause~A.1).

2. Dependency \url{http://www.opengis.net/spec/OMXML/2.0/req/sampling} has associated conformance class \url{http://www.opengis.net/spec/OMXML/2.0/conf/sampling} (OMXML clause~A.12).

3. Dependency \url{http://www.opengis.net/spec/OMXML/2.0/req/spatialSampling} has associated conformance class \url{http://www.opengis.net/spec/OMXML/2.0/conf/spatialSampling} (OMXML clause~A.13).

4. Dependency \url{http://www.opengis.net/spec/SWE/2.0/req/xsd-simple-components} has associated conformance class \url{http://www.opengis.net/spec/SWE/2.0/conf/xsd-simple-components} (SWE~Common~2.0 clause~A.8).

5. Dependency \url{http://www.opengis.net/spec/SWE/2.0/req/xsd-record-components} has associated conformance class \url{http://www.opengis.net/spec/SWE/2.0/conf/xsd-record-components} (SWE~Common~2.0 clause~A.9).

6. Dependency \url{http://www.opengis.net/spec/SWE/2.0/req/xsd-block-components} has associated conformance class \url{http://www.opengis.net/spec/SWE/2.0/conf/xsd-block-components} (SWE~Common~2.0 clause~A.11).

7. Dependency \url{http://www.opengis.net/spec/SWE/2.0/req/xsd-simple-encodings} has associated conformance class \url{http://www.opengis.net/spec/SWE/2.0/conf/xsd-simple-encodings} (SWE~Common~2.0 clause~A.12).

8. Dependency \url{http://www.opengis.net/spec/SWE/2.0/req/general-encoding-rules} has associated conformance class \url{http://www.opengis.net/spec/SWE/2.0/conf/general-encoding-rules} (SWE~Common~2.0 clause~A.14).

9. Dependency \url{http://www.opengis.net/spec/SWE/2.0/req/text-encoding-rules} has associated conformance class \url{http://www.opengis.net/spec/SWE/2.0/conf/text-encoding-rules} (SWE~Common~2.0 clause~A.15).

10. Dependency \url{http://www.opengis.net/spec/SWE/2.0/req/xml-encoding-rules} has associated conformance class \url{http://www.opengis.net/spec/SWE/2.0/conf/xml-encoding-rules} (SWE~Common~2.0 clause~A.16).

11. The canonical schema location for OGC SWE~Common~2.0 (\url{http://www.opengis.net/swe/2.0}) is \url{http://schemas.opengis.net/sweCommon/2.0/swe.xsd}.

12. Dependency \url{http://www.opengis.net/spec/gmlcov/1.0/req/gml-coverage} has associated conformance class (GMLCOV~1.0 clause~A.1)

13. Dependency \url{http://www.opengis.net/spec/gmlcov/1.0/req/gml-coverage} has associated conformance class (GMLCOV~1.0 clause~A.2)

14. The canonical schema location for OGC GMLCOV~1.0 (\url{http://www.opengis.net/gmlcov/1.0}) is \url{http://schemas.opengis.net/gmlcov/1.0/gmlcovAll.xsd}.

202-15-Ext.6 Requirements class: Sampling observation

202-15-Ext.6.1 This requirements class restricts the content model for the XML element om:OM\_Observation such that the ``feature of interest'' is a representative subset of the atmosphere or body of water and so forth based on a predetermined sampling regime and the ``procedure'' provides the set of information as specified by WMO.

Note: SamplingObservation (a subclass of OM\_Observation) is the most flexible of the three observation specializations defined in METCE, as it adds no additional constraints on the type of ``result''.

202-15-Ext.6.2 Where the semantics of one's application are appropriate, ComplexSamplingMeasurement or SamplingCoverageMeasurement should be used in preference to SamplingObservation, as it is anticipated that software applications will be more readily able to parse and process data conforming to the former two observation types due to their more structured ``result'' types.

202-15-Ext.6.3 Instances of om:OM\_Observation with element om:type specifying \url{http://codes.wmo.int/common/observation-type/METCE/2013/SamplingObservation} shall conform to all requirements specified in Table~202-15-Ext.6.

202-15-Ext.6.4 Instances of om:OM\_Observation with element om:type specifying \url{http://codes.wmo.int/common/observation-type/METCE/2013/SamplingObservation} shall conform to all requirements of all relevant dependencies specified in Table~202-15-Ext.6.

Note: XML implementation of metce:SamplingObservation is dependent on:

-- OMXML {[}OGC/IS 10-025r1 Observations and Measurements 2.0 -- XML Implementation{]}.

Table~202-15-Ext.6. Requirements class xsd-sampling-observation

\begin{longtable}[]{@{}ll@{}}
\toprule
Requirements class &\tabularnewline
\midrule
\endhead
\url{http://def.wmo.int/metce/2013/req/xsd-sampling-observation} &\tabularnewline
Target type & Data instance\tabularnewline
Name & Sampling observation\tabularnewline
Dependency & \url{http://www.opengis.net/spec/OMXML/2.0/req/observation}, OMXML clause~7.3\tabularnewline
Dependency & \url{http://www.opengis.net/spec/OMXML/2.0/req/sampling}, OMXML clause~7.14\tabularnewline
Dependency & \url{http://www.opengis.net/spec/OMXML/2.0/req/spatialSampling}, OMXML clause~7.15\tabularnewline
\begin{minipage}[t]{0.47\columnwidth}\raggedright
Requirement\strut
\end{minipage} & \begin{minipage}[t]{0.47\columnwidth}\raggedright
\url{http://def.wmo.int/metce/2013/req/xsd-sampling-observation/procedure-metce-process}

The XML element om:procedure shall contain a child element metce:Process or any element of a substitution group of metce:Process.\strut
\end{minipage}\tabularnewline
\bottomrule
\end{longtable}

Notes:

1. Dependency \url{http://www.opengis.net/spec/OMXML/2.0/req/observation} has associated conformance class \url{http://www.opengis.net/spec/OMXML/2.0/conf/observation} (OMXML clause~A.1).

2. Dependency \url{http://www.opengis.net/spec/OMXML/2.0/req/sampling} has associated conformance class \url{http://www.opengis.net/spec/OMXML/2.0/conf/sampling} (OMXML clause~A.12).

3. Dependency \url{http://www.opengis.net/spec/OMXML/2.0/req/spatialSampling} has associated conformance class \url{http://www.opengis.net/spec/OMXML/2.0/conf/spatialSampling} (OMXML clause~A.13).

202-15-Ext.7 Requirements class: Volcano

202-15-Ext.7.1 This requirements class is used to describe the representation of a volcano.~The class is targeted at providing a basic description of the volcano as a meteorological phenomenon.

Note: Representations providing more detailed information may be used if required.

202-15-Ext.7.2 XML elements describing volcanoes shall conform to all requirements specified in Table~202-15-Ext.7.

202-15-Ext.7.3 XML elements describing volcanoes shall conform to all requirements of all relevant dependencies specified in Table~202-15-Ext.7.

Table~202-15-Ext.7. Requirements class xsd-volcano

\begin{longtable}[]{@{}ll@{}}
\toprule
Requirements class &\tabularnewline
\midrule
\endhead
\url{http://def.wmo.int/metce/2013/req/xsd-volcano} &\tabularnewline
Target type & Data instance\tabularnewline
Name & Volcano\tabularnewline
\begin{minipage}[t]{0.47\columnwidth}\raggedright
Requirement\strut
\end{minipage} & \begin{minipage}[t]{0.47\columnwidth}\raggedright
\url{http://def.wmo.int/metce/2013/req/xsd-volcano/valid}

The content model of this element shall have a value that matches the content model of metce:Volcano.\strut
\end{minipage}\tabularnewline
\begin{minipage}[t]{0.47\columnwidth}\raggedright
Requirement\strut
\end{minipage} & \begin{minipage}[t]{0.47\columnwidth}\raggedright
\url{http://def.wmo.int/metce/2013/req/xsd-volcano/name}

The XML element metce:name shall provide an authoritative name for the given volcano as a literal character string.\strut
\end{minipage}\tabularnewline
\begin{minipage}[t]{0.47\columnwidth}\raggedright
Requirement\strut
\end{minipage} & \begin{minipage}[t]{0.47\columnwidth}\raggedright
\url{http://def.wmo.int/metce/2013/req/xsd-volcano/position}

The XML element metce:position shall contain a valid child element gml:Point that provides the reference location of the volcano in question.\strut
\end{minipage}\tabularnewline
\begin{minipage}[t]{0.47\columnwidth}\raggedright
Recommendation\strut
\end{minipage} & \begin{minipage}[t]{0.47\columnwidth}\raggedright
\url{http://def.wmo.int/metce/2013/req/xsd-volcano/name-as-block-caps}

The authoritative name for the given volcano should be expressed in block capitals.\strut
\end{minipage}\tabularnewline
\bottomrule
\end{longtable}

Note: The Global Volcanism Program provides an online, searchable catalogue of volcanoes that may assist in identifying the authoritative name for a given volcano.~The catalogue is accessed at \url{http://wis.wmo.int/volcano}. No guarantee is made regarding the availability of this catalogue service.

202-15-Ext.8 Requirements class: Erupting volcano

202-15-Ext.8.1 This requirements class is used to describe the representation of a currently erupting, or recently erupted, volcano that is the source of volcanic ash or other significant meteorological phenomena described in weather reports.

Note: Representations providing more detailed information may be used if required.

202-15-Ext.8.2 XML elements describing volcanoes where the date of a particular eruption is deemed important shall conform to all requirements specified in Table~202-15-Ext.8.

202-15-Ext.8.3 XML elements describing volcanoes where the date of a particular eruption is deemed important shall conform to all requirements of all relevant dependencies specified in Table~202-15-Ext.8.

Table~202-15-Ext.8. Requirements class xsd-erupting-volcano

\begin{longtable}[]{@{}ll@{}}
\toprule
Requirements class &\tabularnewline
\midrule
\endhead
\url{http://def.wmo.int/metce/2013/req/xsd-erupting-volcano} &\tabularnewline
Target type & Data instance\tabularnewline
Name & Erupting volcano\tabularnewline
Dependency & \url{http://def.wmo.int/metce/2013/req/xsd-volcano}\tabularnewline
\begin{minipage}[t]{0.47\columnwidth}\raggedright
Requirement\strut
\end{minipage} & \begin{minipage}[t]{0.47\columnwidth}\raggedright
\url{http://def.wmo.int/metce/2013/req/xsd-erupting-volcano/valid}

The content model of this element shall have a value that matches the content model of metce:EruptingVolcano.\strut
\end{minipage}\tabularnewline
\begin{minipage}[t]{0.47\columnwidth}\raggedright
Requirement\strut
\end{minipage} & \begin{minipage}[t]{0.47\columnwidth}\raggedright
\url{http://def.wmo.int/metce/2013/req/xsd-erupting-volcano/eruption-date}

The XML element metce:eruptionDate shall provide the date at which the current or recent eruption began expressed in ISO 8601 date-time format.\strut
\end{minipage}\tabularnewline
\bottomrule
\end{longtable}

202-15-Ext.9 Requirements class: Tropical cyclone

202-15-Ext.9.1 This requirements class is used to describe the representation of a tropical cyclone.

Note: In this release of METCE-XML, the information expressed about a tropical cyclone is limited to the cyclone's name. Representations providing more detailed information may be used if required.

202-15-Ext.9.2 XML elements describing tropical cyclones shall conform to all requirements specified in Table~202-15-Ext.9.

202-15-Ext.9.3 XML elements describing tropical cyclones shall conform to all requirements of all relevant dependencies specified in Table~202-15-Ext.9.

Table~202-15-Ext.9. Requirements class xsd-tropical-cyclone

\begin{longtable}[]{@{}ll@{}}
\toprule
Requirements class &\tabularnewline
\midrule
\endhead
\url{http://def.wmo.int/metce/2013/req/xsd-tropical-cyclone} &\tabularnewline
Target type & Data instance\tabularnewline
Name & Tropical cyclone\tabularnewline
\begin{minipage}[t]{0.47\columnwidth}\raggedright
Requirement\strut
\end{minipage} & \begin{minipage}[t]{0.47\columnwidth}\raggedright
\url{http://def.wmo.int/metce/2013/req/xsd-tropical-cyclone/valid}

The content model of this element shall have a value that matches the content model of metce:TropicalCyclone.\strut
\end{minipage}\tabularnewline
\begin{minipage}[t]{0.47\columnwidth}\raggedright
Requirement\strut
\end{minipage} & \begin{minipage}[t]{0.47\columnwidth}\raggedright
\url{http://def.wmo.int/metce/2013/req/xsd-tropical-cyclone/name}

The XML element metce:name shall provide an authoritative name for the given tropical cyclone as a literal character string.\strut
\end{minipage}\tabularnewline
\begin{minipage}[t]{0.47\columnwidth}\raggedright
Recommendation\strut
\end{minipage} & \begin{minipage}[t]{0.47\columnwidth}\raggedright
\url{http://def.wmo.int/metce/2013/req/xsd-tropical-cyclone/name-as-block-caps}

The authoritative name for the given tropical cyclone should be expressed in block capitals.\strut
\end{minipage}\tabularnewline
\bottomrule
\end{longtable}

202-15-Ext.10 Requirements class: Process

202-15-Ext.10.1 This requirements class is used to describe the procedures involved in generating an observation or measurement.

Notes:

1. Process provides a concrete implementation of the abstract OM\_Process class.

2. An instance of Process is often an instrument or sensor (perhaps even a sensor in a given calibrated state), but it may be a human observer executing a set of repeatable instructions, a simulator or a process algorithm.

3. Process is intended to allow the provision of reference(s) to supporting documentation (for example, online documentation describing the procedure in detail) plus the resolution (for example, the smallest quantity being measured that causes a perceptible change in the corresponding indication) and measuring interval (for example, the range of values for a given quantity kind that an instrument or sensor can detect under the defined conditions) for each physical quantity kind observed.

4. Process is targeted at providing a basic process description; representations providing more detailed information may be used if required.

202-15-Ext.10.2 An instance of Process should provide sufficient information for one to interpret the result of an observation.

Note: The recalibration of a sensor such as an anemometer, or modifying its height above local ground, is likely to affect the values that that sensor records; a new instance of Process may be required to express such changes enabling accurate interpretation of the observation result.

202-15-Ext.10.3 XML elements describing procedures relating to observations or measurements shall conform to all requirements specified in Table~202-15-Ext.10.

202-15-Ext.10.4 XML elements describing procedures relating to observations or measurements shall conform to all requirements of all relevant dependencies specified in Table~202-15-Ext.10.

Table~202-15-Ext.10. Requirements class xsd-process

\begin{longtable}[]{@{}ll@{}}
\toprule
Requirements class &\tabularnewline
\midrule
\endhead
\url{http://def.wmo.int/metce/2013/req/xsd-process} &\tabularnewline
Target type & Data instance\tabularnewline
Name & Process\tabularnewline
\begin{minipage}[t]{0.47\columnwidth}\raggedright
Requirement\strut
\end{minipage} & \begin{minipage}[t]{0.47\columnwidth}\raggedright
\url{http://def.wmo.int/metce/2013/req/xsd-process/valid}

The content model of this element shall have a value that matches the content model of metce:Process.\strut
\end{minipage}\tabularnewline
\begin{minipage}[t]{0.47\columnwidth}\raggedright
Recommendation\strut
\end{minipage} & \begin{minipage}[t]{0.47\columnwidth}\raggedright
\url{http://def.wmo.int/metce/2013/req/xsd-process/description}

A description of the procedure, or citation to some well-known description of the procedure, should be provided using the gml:description element.\strut
\end{minipage}\tabularnewline
\begin{minipage}[t]{0.47\columnwidth}\raggedright
Recommendation\strut
\end{minipage} & \begin{minipage}[t]{0.47\columnwidth}\raggedright
\url{http://def.wmo.int/metce/2013/req/xsd-process/documentation-reference}

Where more information about the procedure is accessible online, a reference to that information should be provided using the xlink:href attribute of the metce:documentationRef element to indicate the URL of the online documentation.\strut
\end{minipage}\tabularnewline
\begin{minipage}[t]{0.47\columnwidth}\raggedright
Recommendation\strut
\end{minipage} & \begin{minipage}[t]{0.47\columnwidth}\raggedright
\url{http://def.wmo.int/metce/2013/req/xsd-process/configuration}

Where more information about the procedure configuration is available, such as details of a sensor's calibration or deployment environment, that information should be included in the procedure description. For each configuration item, an XML element metce:parameter should be provided, each with a child element om:NamedValue. XML element //metce:parameter/om:NamedValue/om:name should indicate the meaning of the parameter. The parameter name should be taken from a well-governed source. Furthermore, to avoid ambiguity, there should be no more than one parameter with the same name within a given procedure description. XML element //metce:parameter/om:NamedValue/om:value provides the value of the parameter using any suitable concrete type.\strut
\end{minipage}\tabularnewline
\begin{minipage}[t]{0.47\columnwidth}\raggedright
Recommendation\strut
\end{minipage} & \begin{minipage}[t]{0.47\columnwidth}\raggedright
\url{http://def.wmo.int/metce/2013/req/xsd-process/measurement-context}

Where additional information about the quantity kind(s) observed or measured by the procedure, such as qualification or constraint of the quantity kind, or details of the resolution and/or range with which the procedure is able to measure a given quantity kind is available, that information should be included in the procedure description. For each quantity kind for which additional information is to be provided, an XML element metce:context should be provided, each with a child element metce:MeasurementContext (or any element of a substitution group of metce:MeasurementContext).\strut
\end{minipage}\tabularnewline
\bottomrule
\end{longtable}

Note: Where a procedure is common for many observations, the metce:Process describing that procedure may be published online at an accessible location and referenced from each observation using xlink:href to indicate the URL.

202-15-Ext.11 Requirements class: Measurement context

202-15-Ext.11.1 This requirements class is used to describe the additional context that may be provided for a quantity kind measured by a given procedure.

Note: The measurement context allows the resolution scale (for example, the smallest change in quantity being measured that causes a perceptible change in the corresponding indication) and/or the measuring interval (for example, the range of values that can be measured) to be defined for a given quantity kind for the associated procedure. For example, it is possible to state that a given procedure, say a thermometer, is able to measure air temperature to a resolution of 0.5 degree Celsius in the range --30 degrees Celsius to +50 degrees Celsius.

202-15-Ext.11.2 XML elements describing procedures relating to observations or measurements shall conform to all requirements specified in Table~202-15-Ext.11.

202-15-Ext.11.3 XML elements describing procedures relating to observations or measurements shall conform to all requirements of all relevant dependencies specified in Table~202-15-Ext.11.

Table~202-15-Ext.11. Requirements class xsd-measurement-context

\begin{longtable}[]{@{}ll@{}}
\toprule
Requirements class &\tabularnewline
\midrule
\endhead
\url{http://def.wmo.int/metce/2013/req/xsd-measurement-context} &\tabularnewline
Target type & Data instance\tabularnewline
Name & Measurement context\tabularnewline
Dependency & \url{http://def.wmo.int/opm/2013/req/xsd-observable-property}, 203-15-Ext.3\tabularnewline
\begin{minipage}[t]{0.47\columnwidth}\raggedright
Requirement\strut
\end{minipage} & \begin{minipage}[t]{0.47\columnwidth}\raggedright
\url{http://def.wmo.int/metce/2013/req/xsd-measurement-context/valid}

The content model of this element shall have a value that matches the content model of metce:MeasurementContext.\strut
\end{minipage}\tabularnewline
\begin{minipage}[t]{0.47\columnwidth}\raggedright
Requirement\strut
\end{minipage} & \begin{minipage}[t]{0.47\columnwidth}\raggedright
\url{http://def.wmo.int/metce/2013/req/xsd-measurement-context/measurand}

The quantity kind to which this element applies shall be specified via the XML element metce:measurand.\strut
\end{minipage}\tabularnewline
\begin{minipage}[t]{0.47\columnwidth}\raggedright
Requirement\strut
\end{minipage} & \begin{minipage}[t]{0.47\columnwidth}\raggedright
\url{http://def.wmo.int/metce/2013/req/xsd-measurement-context/unit-of-measure-consistent}

The unit of measurement specified in XML element metce:unitOfMeasure shall be consistent with the unit of measurement used to express resolution scale and/or measuring interval.\strut
\end{minipage}\tabularnewline
\begin{minipage}[t]{0.47\columnwidth}\raggedright
Requirement\strut
\end{minipage} & \begin{minipage}[t]{0.47\columnwidth}\raggedright
\url{http://def.wmo.int/metce/2013/req/xsd-measurement-context/unit-of-measure-provision}

Where either XML elements metce:resolutionScale or metce:measuringInterval or both are present, XML element metce:unitOfMeasure shall be provided.\strut
\end{minipage}\tabularnewline
\begin{minipage}[t]{0.47\columnwidth}\raggedright
Requirement\strut
\end{minipage} & \begin{minipage}[t]{0.47\columnwidth}\raggedright
\url{http://def.wmo.int/metce/2013/req/xsd-measurement-context/measuring-interval-range-bounds-order}

Where the measuring interval is specified, the lower limit of the interval, expressed via XML element //metce:measuringInterval/metce:RangeBounds/metce:rangeStart, shall be less than the upper limit of the interval, expressed via XML element //metce:measuringInterval/metce:RangeBounds/metce:rangeEnd.\strut
\end{minipage}\tabularnewline
\bottomrule
\end{longtable}

Notes:

1. The XML element metce:measurand may reference a quantity kind provided by some authority using xlink:href to indicate the URI of the quantity kind, or provide a child element opm:ObservableProperty (or element within the substitution group of opm:ObservableProperty). The latter case may be useful where additional qualification or constraint relating to the quantity kind needs to be embedded within the GML document as XML element //om:OM\_Observation/om:observedProperty only permits the observed quantity kind to be expressed by reference using xlink:href.

2. Units of measurement are specified in accordance with 1.9 above.

FM~202-16 METCE-XML FOUNDATION METEOROLOGICAL INFORMATION

202-16.1 Scope

METCE-XML shall be used for the exchange in XML of meteorological information conforming to the \emph{Modèle pour l'échange des informations sur le temps, le climat et l'eau} (METCE) application schema. METCE-XML may be used directly to encode meteorological information or incorporated as components within other XML encodings.

Note: The METCE application schema is described in the \emph{Guidelines on Data Modelling for WMO Codes} (available in English only from \url{http://wis.wmo.int/metce-uml}).

The requirements classes defined in METCE-XML are listed in Table~202-16.1.

Table~202-16.1. Requirements classes defined in METCE-XML

\begin{longtable}[]{@{}ll@{}}
\toprule
Requirements classes &\tabularnewline
\midrule
\endhead
Requirements class & \url{http://def.wmo.int/metce/2013/req/xsd-volcano}, 202-16.7\tabularnewline
Requirements class & \url{http://def.wmo.int/metce/2013/req/xsd-erupting-volcano}, 202-16.8\tabularnewline
Requirements class & \url{http://def.wmo.int/metce/2013/req/xsd-tropical-cyclone}, 202-16.9\tabularnewline
Requirements class & \url{http://def.wmo.int/metce/2013/req/xsd-process}, 202-16.10\tabularnewline
Requirements class & \url{http://def.wmo.int/metce/2013/req/xsd-measurement-context}, 202-16.11\tabularnewline
Requirements class & \href{http://def.wmo.int/metce/2013/req/xsd-observation}{http://def.wmo.int/metce/2013/req/xsd-observation,} 202-16.12\tabularnewline
Requirements class & \href{http://def.wmo.int/metce/2013/req/xsd-measurement}{http://def.wmo.int/metce/2013/req/xsd-measurement,} 202-16.13\tabularnewline
Requirements class & \href{http://def.wmo.int/metce/2013/req/xsd-category-observation}{http://def.wmo.int/metce/2013/req/xsd-category-observation,} 202-16.14\tabularnewline
Requirements class & \href{http://def.wmo.int/metce/2013/req/xsd-count-observation}{http://def.wmo.int/metce/2013/req/xsd-count-observation,} 202-16.15\tabularnewline
Requirements class & \href{http://def.wmo.int/metce/2013/req/xsd-truth-observation}{http://def.wmo.int/metce/2013/req/xsd-truth-observation,} 202-16.16\tabularnewline
Requirements class & \href{http://def.wmo.int/metce/2013/req/xsd-geometry-observation}{http://def.wmo.int/metce/2013/req/xsd-geometry-observation,} 202-16.17\tabularnewline
Requirements class & \href{http://def.wmo.int/metce/2013/req/xsd-temporal-observation}{http://def.wmo.int/metce/2013/req/xsd-temporal-observation,} 202-16.18\tabularnewline
Requirements class & \href{http://def.wmo.int/metce/2013/req/xsd-complex-observation}{http://def.wmo.int/metce/2013/req/xsd-complex-observation,} 202-16.19\tabularnewline
Requirements class & \vtop{\hbox{\strut \href{http://def.wmo.int/metce/2013/req/xsd-discrete-coverage-observation}{http://def.wmo.int/metce/2013/req/xsd-discrete-coverage-observation,}}\hbox{\strut 202-16.20}}\tabularnewline
\bottomrule
\end{longtable}

202-16.2 XML schema for METCE-XML

Representations of information in METCE-XML shall declare the XML namespaces listed in Table~202-16.2 and Table~202-16.3.

Notes:

1. Additional namespace declarations may be required depending on the XML elements used within METCE‑XML.

2. The XML schema is packaged in three XML schema documents (XSD) describing one XML namespace: \url{http://def.wmo.int/metce/2013}.

3. Schematron schemas providing additional constraints are embedded within the XSD defining METCE-XML.

Table~202-16.2. XML namespaces defined for METCE-XML

\begin{longtable}[]{@{}lll@{}}
\toprule
XML namespace & Default namespace prefix & Canonical location of all-components schema document\tabularnewline
\midrule
\endhead
\url{http://def.wmo.int/metce/2013} & metce & \url{http://schemas.wmo.int/metce/1.2/metce.xsd}\tabularnewline
\bottomrule
\end{longtable}

Table~202-16.3. External XML namespaces used in METCE-XML

\begin{longtable}[]{@{}llll@{}}
\toprule
Standard & XML namespace & Default namespace prefix & Canonical location of all-components schema document\tabularnewline
\midrule
\endhead
XML schema & \url{http://www.w3.org/2001/XMLSchema} & xs &\tabularnewline
Schematron & \url{http://purl.oclc.org/dsdl/schematron} & sch &\tabularnewline
XSLT v2 & \url{http://www.w3.org/1999/XSL/Transform} & xsl &\tabularnewline
XML Linking Language & \url{http://www.w3.org/1999/xlink} & xlink & \url{http://www.w3.org/1999/xlink.xsd}\tabularnewline
ISO 19136:2007 GML & \url{http://www.opengis.net/gml/3.2} & gml & \url{http://schemas.opengis.net/gml/3.2.1/gml.xsd}\tabularnewline
ISO/TS 19139:2007 metadata XML implementation & \url{http://www.isotc211.org/2005/gmd} & gmd & \url{http://standards.iso.org/ittf/PubliclyAvailableStandards/ISO_19139_Schemas/gmd/gmd.xsd}\tabularnewline
OGC OMXML & \url{http://www.opengis.net/om/2.0} & om & \url{http://schemas.opengis.net/om/2.0/observation.xsd}\tabularnewline
FM 203-15 Ext. OPM-XML & \url{http://def.wmo.int/opm/2013} & opm & \url{http://schemas.wmo.int/opm/1.1/opm.xsd}\tabularnewline
\bottomrule
\end{longtable}

202-16.3 Virtual typing

In accordance with OMXML (clause~7.2), the specialization of OM\_Observation is provided through Schematron restriction. The om:type element shall be used to specify the type of OM\_Observation that is being encoded using the URI for the corresponding observation type listed in Code table~D-3.

Notes:

1. Code table~D-3 is described in Appendix~A.

2. Code table~D-3 is published online at \url{http://codes.wmo.int/common/observation-type/METCE/2013}.

3. The URI for each observation type is composed by appending the \emph{notation} to the \emph{code-space}. As an example, the URI of DiscreteTimeSeriesObservation is \url{http://codes.wmo.int/common/observation-type/METCE/2013/DiscreteTimeSeriesObservation}.

4. Each URI will resolve to provide further information about the associated observation type.

5. The terms ``observation'' and ``measurement'' evoke a particular concept to meteorologists (for example, the measurement of a physical phenomenon using an instrument or sensor). As defined in ISO~19156:2011, Geographic information -- Observations and measurements, an instance of OM\_Observation is defined as an ``estimate of the value of some property of some feature of interest using a specified procedure''. OM\_Measurement is clearly applicable to the measurement of some physical property values using an instrument or sensor but is equally applicable to the numerical simulation of physical property values using a computational model (for example, a forecast or reanalysis).

202-16.4 Not used

202-16.5 Not used

202-16.6 Not used

202-16.7 Requirements class: Volcano

202-16.7.1 This requirements class is used to describe the representation of a volcano. The class is targeted at providing a basic description of the volcano as a meteorological phenomenon.

Note: Representations providing more detailed information may be used if required.

202-16.7.2 XML elements describing volcanoes shall conform to all requirements specified in Table~202-16.4.

202-16.7.3 XML elements describing volcanoes shall conform to all requirements of all relevant dependencies specified in Table~202-16.4.

Table~202-16.4. Requirements class xsd-volcano

\begin{longtable}[]{@{}ll@{}}
\toprule
Requirements class &\tabularnewline
\midrule
\endhead
\url{http://def.wmo.int/metce/2013/req/xsd-volcano} &\tabularnewline
Target type & Data instance\tabularnewline
Name & Volcano\tabularnewline
\begin{minipage}[t]{0.47\columnwidth}\raggedright
Requirement\strut
\end{minipage} & \begin{minipage}[t]{0.47\columnwidth}\raggedright
\url{http://def.wmo.int/metce/2013/req/xsd-volcano/valid}

The content model of this element shall have a value that matches the content model of metce:Volcano.\strut
\end{minipage}\tabularnewline
\begin{minipage}[t]{0.47\columnwidth}\raggedright
Requirement\strut
\end{minipage} & \begin{minipage}[t]{0.47\columnwidth}\raggedright
\url{http://def.wmo.int/metce/2013/req/xsd-volcano/name}

The XML element metce:name shall provide an authoritative name for the given volcano as a literal character string.\strut
\end{minipage}\tabularnewline
\begin{minipage}[t]{0.47\columnwidth}\raggedright
Requirement\strut
\end{minipage} & \begin{minipage}[t]{0.47\columnwidth}\raggedright
\url{http://def.wmo.int/metce/2013/req/xsd-volcano/position}

The XML element metce:position shall contain a valid child element gml:Point that provides the reference location of the volcano in question.\strut
\end{minipage}\tabularnewline
\begin{minipage}[t]{0.47\columnwidth}\raggedright
Recommendation\strut
\end{minipage} & \begin{minipage}[t]{0.47\columnwidth}\raggedright
\url{http://def.wmo.int/metce/2013/req/xsd-volcano/name-as-block-caps}

The authoritative name for the given volcano should be expressed in block capitals.\strut
\end{minipage}\tabularnewline
\bottomrule
\end{longtable}

Note: The Global Volcanism Program provides an online, searchable catalogue of volcanoes that may assist in identifying the authoritative name for a given volcano.~The catalogue is accessed at \url{http://wis.wmo.int/volcano}. No guarantee is made regarding the availability of this catalogue service.

202-16.8 Requirements class: Erupting volcano

202-16.8.1 This requirements class is used to describe the representation of a currently erupting, or recently erupted, volcano that is the source of volcanic ash or other significant meteorological phenomena described in weather reports.

Note: Representations providing more detailed information may be used if required.

202-16.8.2 XML elements describing volcanoes where the date of a particular eruption is deemed important shall conform to all requirements specified in Table~202-16.5.

202-16.8.3 XML elements describing volcanoes where the date of a particular eruption is deemed important shall conform to all requirements of all relevant dependencies specified in Table~202-16.5.

Table~202-16.5. Requirements class xsd-erupting-volcano

\begin{longtable}[]{@{}ll@{}}
\toprule
Requirements class &\tabularnewline
\midrule
\endhead
\url{http://def.wmo.int/metce/2013/req/xsd-erupting-volcano} &\tabularnewline
Target type & Data instance\tabularnewline
Name & Erupting volcano\tabularnewline
Dependency & \url{http://def.wmo.int/metce/2013/req/xsd-volcano}\tabularnewline
\begin{minipage}[t]{0.47\columnwidth}\raggedright
Requirement\strut
\end{minipage} & \begin{minipage}[t]{0.47\columnwidth}\raggedright
\url{http://def.wmo.int/metce/2013/req/xsd-erupting-volcano/valid}

The content model of this element shall have a value that matches the content model of metce:EruptingVolcano.\strut
\end{minipage}\tabularnewline
\begin{minipage}[t]{0.47\columnwidth}\raggedright
Requirement\strut
\end{minipage} & \begin{minipage}[t]{0.47\columnwidth}\raggedright
\url{http://def.wmo.int/metce/2013/req/xsd-erupting-volcano/eruption-date}

The XML element metce:eruptionDate shall provide the date at which the current or recent eruption began expressed in ISO 8601 date-time format.\strut
\end{minipage}\tabularnewline
\bottomrule
\end{longtable}

202-16.9 Requirements class: Tropical cyclone

202-16.9.1 This requirements class is used to describe the representation of a tropical cyclone.

Note: In this release of METCE-XML, the information expressed about a tropical cyclone is limited to the cyclone's name. Representations providing more detailed information may be used if required.

202-16.9.2 XML elements describing tropical cyclones shall conform to all requirements specified in Table~202-16.6.

202-16.9.3 XML elements describing tropical cyclones shall conform to all requirements of all relevant dependencies specified in Table~202-16.6.

Table~202-16.6. Requirements class xsd-tropical-cyclone

\begin{longtable}[]{@{}ll@{}}
\toprule
Requirements class &\tabularnewline
\midrule
\endhead
\url{http://def.wmo.int/metce/2013/req/xsd-tropical-cyclone} &\tabularnewline
Target type & Data instance\tabularnewline
Name & Tropical cyclone\tabularnewline
\begin{minipage}[t]{0.47\columnwidth}\raggedright
Requirement\strut
\end{minipage} & \begin{minipage}[t]{0.47\columnwidth}\raggedright
\url{http://def.wmo.int/metce/2013/req/xsd-tropical-cyclone/valid}

The content model of this element shall have a value that matches the content model of metce:TropicalCyclone.\strut
\end{minipage}\tabularnewline
\begin{minipage}[t]{0.47\columnwidth}\raggedright
Requirement\strut
\end{minipage} & \begin{minipage}[t]{0.47\columnwidth}\raggedright
\url{http://def.wmo.int/metce/2013/req/xsd-tropical-cyclone/name}

The XML element metce:name shall provide an authoritative name for the given tropical cyclone as a literal character string.\strut
\end{minipage}\tabularnewline
\begin{minipage}[t]{0.47\columnwidth}\raggedright
Recommendation\strut
\end{minipage} & \begin{minipage}[t]{0.47\columnwidth}\raggedright
\url{http://def.wmo.int/metce/2013/req/xsd-tropical-cyclone/name-as-block-caps}

The authoritative name for the given tropical cyclone should be expressed in block capitals.\strut
\end{minipage}\tabularnewline
\bottomrule
\end{longtable}

202-16.10 Requirements class: Process

202-16.10.1 This requirements class is used to describe the procedures involved in generating an observation or measurement.

Notes:

1. Process provides a concrete implementation of the abstract OM\_Process class.

2. An instance of Process is often an instrument or sensor (perhaps even a sensor in a given calibrated state), but it may be a human observer executing a set of repeatable instructions, a simulator or a process algorithm.

3. Process is intended to allow the provision of reference(s) to supporting documentation (for example, online documentation describing the procedure in detail) plus the resolution (for example, the smallest quantity being measured that causes a perceptible change in the corresponding indication) and measuring interval (for example, the range of values for a given quantity kind that an instrument or sensor can detect under the defined conditions) for each physical quantity kind observed.

4. Process is targeted at providing a basic process description; representations providing more detailed information may be used if required.

202-16.10.2 An instance of Process should provide sufficient information for one to interpret the result of an observation.

Note: The recalibration of a sensor such as an anemometer, or modifying its height above local ground, is likely to affect the values that that sensor records; a new instance of Process may be required to express such changes enabling accurate interpretation of the observation result.

202-16.10.3 XML elements describing procedures relating to observations or measurements shall conform to all requirements specified in Table~202-16.7.

202-16.10.4 XML elements describing procedures relating to observations or measurements shall conform to all requirements of all relevant dependencies specified in Table~202-16.7.

Table~202-16.7. Requirements class xsd-process

\begin{longtable}[]{@{}ll@{}}
\toprule
Requirements class &\tabularnewline
\midrule
\endhead
\url{http://def.wmo.int/metce/2013/req/xsd-process} &\tabularnewline
Target type & Data instance\tabularnewline
Name & Process\tabularnewline
\begin{minipage}[t]{0.47\columnwidth}\raggedright
Requirement\strut
\end{minipage} & \begin{minipage}[t]{0.47\columnwidth}\raggedright
\url{http://def.wmo.int/metce/2013/req/xsd-process/valid}

The content model of this element shall have a value that matches the content model of metce:Process.\strut
\end{minipage}\tabularnewline
\begin{minipage}[t]{0.47\columnwidth}\raggedright
Recommendation\strut
\end{minipage} & \begin{minipage}[t]{0.47\columnwidth}\raggedright
\url{http://def.wmo.int/metce/2013/req/xsd-process/description}

A description of the procedure, or citation to some well-known description of the procedure, should be provided using the gml:description element.\strut
\end{minipage}\tabularnewline
\begin{minipage}[t]{0.47\columnwidth}\raggedright
Recommendation\strut
\end{minipage} & \begin{minipage}[t]{0.47\columnwidth}\raggedright
\url{http://def.wmo.int/metce/2013/req/xsd-process/documentation-reference}

Where more information about the procedure is accessible online, a reference to that information should be provided using the xlink:href attribute of the metce:documentationRef element to indicate the URL of the online documentation.\strut
\end{minipage}\tabularnewline
\begin{minipage}[t]{0.47\columnwidth}\raggedright
Recommendation\strut
\end{minipage} & \begin{minipage}[t]{0.47\columnwidth}\raggedright
\url{http://def.wmo.int/metce/2013/req/xsd-process/configuration}

Where more information about the procedure configuration is available, such as details of a sensor's calibration or deployment environment, that information should be included in the procedure description. For each configuration item, an XML element metce:parameter should be provided, each with a child element om:NamedValue. XML element //metce:parameter/om:NamedValue/om:name should indicate the meaning of the parameter. The parameter name should be taken from a well-governed source. Furthermore, to avoid ambiguity, there should be no more than one parameter with the same name within a given procedure description. XML element //metce:parameter/om:NamedValue/om:value provides the value of the parameter using any suitable concrete type.\strut
\end{minipage}\tabularnewline
\begin{minipage}[t]{0.47\columnwidth}\raggedright
Recommendation\strut
\end{minipage} & \begin{minipage}[t]{0.47\columnwidth}\raggedright
\url{http://def.wmo.int/metce/2013/req/xsd-process/measurement-context}

Where additional information about the quantity kind(s) observed or measured by the procedure, such as qualification or constraint of the quantity kind, or details of the resolution and/or range with which the procedure is able to measure a given quantity kind is available, that information should be included in the procedure description. For each quantity kind for which additional information is to be provided, an XML element metce:context should be provided, each with a child element metce:MeasurementContext (or any element of a substitution group of metce:MeasurementContext).\strut
\end{minipage}\tabularnewline
\bottomrule
\end{longtable}

Note: Where a procedure is common for many observations, the metce:Process describing that procedure may be published online at an accessible location and referenced from each observation using xlink:href to indicate the URL.

202-16.11 Requirements class: Measurement context

202-16.11.1 This requirements class is used to describe the additional context that may be provided for a quantity kind measured by a given procedure.

Note: The measurement context allows the resolution scale (for example, the smallest change in quantity being measured that causes a perceptible change in the corresponding indication) and/or the measuring interval (for example, the range of values that can be measured) to be defined for a given quantity kind for the associated procedure. For example, it is possible to state that a given procedure, say a thermometer, is able to measure air temperature to a resolution of 0.5 degree Celsius in the range --30 degrees Celsius to +50 degrees Celsius.

202-16.11.2 XML elements describing procedures relating to observations or measurements shall conform to all requirements specified in Table~202-16.8.

202-16.11.3 XML elements describing procedures relating to observations or measurements shall conform to all requirements of all relevant dependencies specified in Table~202-16.8.

Table~202-16.8. Requirements class xsd-measurement-context

\begin{longtable}[]{@{}ll@{}}
\toprule
Requirements class &\tabularnewline
\midrule
\endhead
\url{http://def.wmo.int/metce/2013/req/xsd-measurement-context} &\tabularnewline
Target type & Data instance\tabularnewline
Name & Measurement context\tabularnewline
Dependency & \url{http://def.wmo.int/opm/2013/req/xsd-observable-property}, 203-15-Ext.3\tabularnewline
\begin{minipage}[t]{0.47\columnwidth}\raggedright
Requirement\strut
\end{minipage} & \begin{minipage}[t]{0.47\columnwidth}\raggedright
\url{http://def.wmo.int/metce/2013/req/xsd-measurement-context/valid}

The content model of this element shall have a value that matches the content model of metce:MeasurementContext.\strut
\end{minipage}\tabularnewline
\begin{minipage}[t]{0.47\columnwidth}\raggedright
Requirement\strut
\end{minipage} & \begin{minipage}[t]{0.47\columnwidth}\raggedright
\url{http://def.wmo.int/metce/2013/req/xsd-measurement-context/measurand}

The quantity kind to which this element applies shall be specified via the XML element metce:measurand.\strut
\end{minipage}\tabularnewline
\begin{minipage}[t]{0.47\columnwidth}\raggedright
Requirement\strut
\end{minipage} & \begin{minipage}[t]{0.47\columnwidth}\raggedright
\url{http://def.wmo.int/metce/2013/req/xsd-measurement-context/unit-of-measure-consistent}

The unit of measurement specified in XML element metce:unitOfMeasure shall be consistent with the unit of measurement used to express resolution scale and/or measuring interval.\strut
\end{minipage}\tabularnewline
\begin{minipage}[t]{0.47\columnwidth}\raggedright
Requirement\strut
\end{minipage} & \begin{minipage}[t]{0.47\columnwidth}\raggedright
\url{http://def.wmo.int/metce/2013/req/xsd-measurement-context/unit-of-measure-provision}

Where either XML elements metce:resolutionScale or metce:measuringInterval or both are present, XML element metce:unitOfMeasure shall be provided.\strut
\end{minipage}\tabularnewline
\begin{minipage}[t]{0.47\columnwidth}\raggedright
Requirement\strut
\end{minipage} & \begin{minipage}[t]{0.47\columnwidth}\raggedright
\url{http://def.wmo.int/metce/2013/req/xsd-measurement-context/measuring-interval-range-bounds-order}

Where the measuring interval is specified, the lower limit of the interval, expressed via XML element //metce:measuringInterval/metce:RangeBounds/metce:rangeStart, shall be less than the upper limit of the interval, expressed via XML element //metce:measuringInterval/metce:RangeBounds/metce:rangeEnd.\strut
\end{minipage}\tabularnewline
\bottomrule
\end{longtable}

Notes:

1. The XML element metce:measurand may reference a quantity kind provided by some authority using xlink:href to indicate the URI of the quantity kind, or provide a child element opm:ObservableProperty (or element within the substitution group of opm:ObservableProperty). The latter case may be useful where additional qualification or constraint relating to the quantity kind needs to be embedded within the GML document as XML element //om:OM\_Observation/om:observedProperty only permits the observed quantity kind to be expressed by reference using xlink:href.

2. Units of measurement are specified in accordance with 1.9 above.

202-16.12 Requirements class: Generic observation data

202-16.12.1 This requirements class mandates use of the XML element om:OM\_Observation for encoding observation data.

Note: The term ````observation'' evokes a particular concept to meteorologists: the measurement of physical phenomena using a human observer, instrument or sensor. However, OM\_Observation, as defined in ISO 19156:2011, provides a generalized pattern for describing an event that estimates the value(s) of a property for a subject using a specified process. These semantics are equally applicable to describe forecasts and other numerical simulations.

202-16.12.2 Instances of om:OM\_Observation with element om:type specifying \url{http://www.opengis.net/def/observationType/OGC-OM/2.0/OM_Observation} or where element om:type is not specified shall conform to all requirements specified in Table~202-16.9.

202-16.12.3 Instances of om:OM\_Observation with element om:type specifying \url{http://www.opengis.net/def/observationType/OGC-OM/2.0/OM_Observation} or where element\\
om:type is not specified shall conform to all requirements of all relevant dependencies specified in Table~202-16.9.

Note: XML implementation of OM\_Observation is dependent on:

-- GML {[}ISO 19136:2007 Geographic information -- Geography Markup Language{]};

-- OMXML {[}OGC/IS 10-025r1 Observations and Measurements 2.0 -- XML Implementation{]}.

Table~202-16.9. Requirements class xsd-observation

\begin{longtable}[]{@{}ll@{}}
\toprule
Requirements class &\tabularnewline
\midrule
\endhead
\url{http://def.wmo.int/metce/2013/req/xsd-observation} &\tabularnewline
Target type & Data instance\tabularnewline
Name & Observation\tabularnewline
Dependency & \href{http://www.opengis.net/spec/OMXML/2.0/req/observation}{http://www.opengis.net/spec/OMXML/2.0/req/observation,} OMXML clause 7.3\tabularnewline
\bottomrule
\end{longtable}

Note: Dependency \url{http://www.opengis.net/spec/OMXML/2.0/req/observation} has associated conformance class \url{http://www.opengis.net/spec/OMXML/2.0/conf/observation} (OMXML clause~A.1).

202-16.13 Requirements class: Measurement data

202-16.13.1 This requirements class restricts the content model for the XML element\\
om:OM\_Observation such that the ``result'' of the observation is a scaled number. This ``result'' is represented in XML as a measure from the GML.

Notes:

1. OM\_Measurement, as defined in ISO 19156:2011, is intended for use where the observation event is concerned with measurement of a single property at a specified location and time instant or duration. The ``result'' of this class of observations is a single scaled value with associated unit of measurement.

2. A description of the XML implementation of measures (gml:MeasureType) is provided in ISO 19136:2007, clause~16.3.

202-16.13.2 Instances of om:OM\_Observation with element om:type specifying \url{http://www.opengis.net/def/observationType/OGC-OM/2.0/OM_Measurement} shall conform to all requirements specified in Table~202-16.10.

202-16.13.3 Instances of om:OM\_Observation with element om:type specifying \url{http://www.opengis.net/def/observationType/OGC-OM/2.0/OM_Measurement} shall conform to all requirements of all relevant dependencies specified in Table~202-16.10.

Note: XML implementation of OM\_Measurement is dependent on:

-- GML {[}ISO 19136:2007 Geographic information -- Geography Markup Language{]};

-- OMXML {[}OGC/IS 10-025r1 Observations and Measurements 2.0 -- XML Implementation{]}.

Table~202-16.10. Requirements class xsd-measurement

\begin{longtable}[]{@{}ll@{}}
\toprule
Requirements class &\tabularnewline
\midrule
\endhead
\url{http://def.wmo.int/metce/2013/req/xsd-measurement} &\tabularnewline
Target type & Data instance\tabularnewline
Name & Measurement\tabularnewline
Dependency & \href{http://www.opengis.net/spec/OMXML/2.0/req/measurement}{http://www.opengis.net/spec/OMXML/2.0/req/measurement,} OMXML clause 7.4\tabularnewline
\bottomrule
\end{longtable}

Note: Dependency \url{http://www.opengis.net/spec/OMXML/2.0/req/measurement} has associated conformance class \url{http://www.opengis.net/spec/OMXML/2.0/conf/measurement} (OMXML clause A.2).

202-16.14 Requirements class: Category observation data

202-16.14.1 This requirements class restricts the content model for the XML element\\
om:OM\_Observation such that the ``result'' of the observation is a term from a controlled vocabulary or ontology.

Note: A description of the XML implementation of terms from a controlled vocabulary (gml:ReferenceType) is provided in ISO 19136:2007, clause 7.2.3.

202-16.14.2 Instances of om:OM\_Observation with element om:type specifying \url{http://www.opengis.net/def/observationType/OGC-OM/2.0/OM_CategoryObservation} shall conform to all requirements specified in Table 202-16.11.

202-16.14.3 Instances of om:OM\_Observation with element om:type specifying \url{http://www.opengis.net/def/observationType/OGC-OM/2.0/OM_CategoryObservation} shall conform to all requirements of all relevant dependencies specified in Table~202-16.11.

Note: XML implementation of OM\_CategoryObservation is dependent on:

-- GML {[}ISO 19136:2007 Geographic information -- Geography Markup Language{]};

-- OMXML {[}OGC/IS 10-025r1 Observations and Measurements 2.0 -- XML Implementation{]}.

Table~202-16.11. Requirements class xsd-category-observation

\begin{longtable}[]{@{}ll@{}}
\toprule
Requirements class &\tabularnewline
\midrule
\endhead
\url{http://def.wmo.int/metce/2013/req/xsd-category-observation} &\tabularnewline
Target type & Data instance\tabularnewline
Name & Category observation\tabularnewline
Dependency & \href{http://www.opengis.net/spec/OMXML/2.0/req/categoryObservation}{http://www.opengis.net/spec/OMXML/2.0/req/categoryObservation,} OMXML clause~7.5\tabularnewline
\bottomrule
\end{longtable}

Note: Dependency \url{http://www.opengis.net/spec/OMXML/2.0/req/categoryObservation} has associated conformance class \url{http://www.opengis.net/spec/OMXML/2.0/conf/categoryObservation} (OMXML clause A.3).

202-16.15 Requirements class: Count observation data

202-16.15.1 This requirements class restricts the content model for the XML element\\
om:OM\_Observation such that the ``result'' of the observation is an integer.

Note: The XML implementation of integer values is provided in XML Schema Part 2: Datatypes, clause 3.3.13.

202-16.15.2 Instances of om:OM\_Observation with element om:type specifying \url{http://www.opengis.net/def/observationType/OGC-OM/2.0/OM_CountObservation} shall conform to all requirements specified in Table~202-16.12.

202-16.15.3 Instances of om:OM\_Observation with element om:type specifying \url{http://www.opengis.net/def/observationType/OGC-OM/2.0/OM_CountObservation} shall conform to all requirements of all relevant dependencies specified in Table~202-16.12.

Note: XML implementation of OM\_CountObservation is dependent on:

-- GML {[}ISO 19136:2007 Geographic information -- Geography Markup Language{]};

-- OMXML {[}OGC/IS 10-025r1 Observations and Measurements 2.0 -- XML Implementation{]}.

Table~202-16.12. Requirements class xsd-count-observation

\begin{longtable}[]{@{}ll@{}}
\toprule
Requirements class &\tabularnewline
\midrule
\endhead
\url{http://def.wmo.int/metce/2013/req/xsd-count-observation} &\tabularnewline
Target type & Data instance\tabularnewline
Name & Count observation\tabularnewline
Dependency & \href{http://www.opengis.net/spec/OMXML/2.0/req/countObservation}{http://www.opengis.net/spec/OMXML/2.0/req/countObservation,} OMXML clause~7.6\tabularnewline
\bottomrule
\end{longtable}

Note: Dependency \url{http://www.opengis.net/spec/OMXML/2.0/req/countObservation} has associated conformance class \url{http://www.opengis.net/spec/OMXML/2.0/conf/countObservation} (OMXML clause~A.4).

202-16.16 Requirements class: Truth observation data

202-16.16.1 This requirements class restricts the content model for the XML element\\
om:OM\_Observation such that the ``result'' of the observation is a Boolean that can be used to express ``true'' or ``false''.

Note: The XML implementation of Boolean values is provided in XML Schema Part 2: Datatypes, clause 3.2.2.

202-16.16.2 Instances of om:OM\_Observation with element om:type specifying \url{http://www.opengis.net/def/observationType/OGC-OM/2.0/OM_TruthObservation} shall conform to all requirements specified in Table~202-16.13.

202-16.16.3 Instances of om:OM\_Observation with element om:type specifying \url{http://www.opengis.net/def/observationType/OGC-OM/2.0/OM_TruthObservation} shall conform to all requirements of all relevant dependencies specified in Table~202-16.13.

Note: XML implementation of OM\_TruthObservation is dependent on:

-- GML {[}ISO 19136:2007 Geographic information -- Geography Markup Language{]};

-- OMXML {[}OGC/IS 10-025r1 Observations and Measurements 2.0 -- XML Implementation{]}.

Table~202-16.13. Requirements class xsd-truth-observation

\begin{longtable}[]{@{}ll@{}}
\toprule
Requirements class &\tabularnewline
\midrule
\endhead
\url{http://def.wmo.int/metce/2013/req/xsd-truth-observation} &\tabularnewline
Target type & Data instance\tabularnewline
Name & Truth observation\tabularnewline
Dependency & \href{http://www.opengis.net/spec/OMXML/2.0/req/truthObservation}{http://www.opengis.net/spec/OMXML/2.0/req/truthObservation,} OMXML clause~7.7\tabularnewline
\bottomrule
\end{longtable}

Note: Dependency \url{http://www.opengis.net/spec/OMXML/2.0/req/truthObservation} has associated conformance class \url{http://www.opengis.net/spec/OMXML/2.0/conf/truthObservation} (OMXML clause~A.5).

202-16.17 Requirements class: Geometry observation data

202-16.17.1 This requirements class restricts the content model for the XML element\\
om:OM\_Observation such that the ``result'' of the observation is a geometry object.

Note: The XML implementation of geometry objects (geometric primitives) is provided in ISO 19136:2007, clause~10.

202-16.17.2 Instances of om:OM\_Observation with element om:type specifying \url{http://www.opengis.net/def/observationType/OGC-OM/2.0/OM_GeometryObservation} shall conform to all requirements specified in Table~202-16.14.

202-16.17.3 Instances of om:OM\_Observation with element om:type specifying \url{http://www.opengis.net/def/observationType/OGC-OM/2.0/OM_GeometryObservation} shall conform to all requirements of all relevant dependencies specified in Table~202-16.14.

Note: XML implementation of OM\_GeometryObservation is dependent on:

-- GML {[}ISO 19136:2007 Geographic information -- Geography Markup Language{]};

-- OMXML {[}OGC/IS 10-025r1 Observations and Measurements 2.0 -- XML Implementation{]}.

Table~202-16.14. Requirements class xsd-geometry-observation

\begin{longtable}[]{@{}ll@{}}
\toprule
Requirements class &\tabularnewline
\midrule
\endhead
\url{http://def.wmo.int/metce/2013/req/xsd-geometry-observation} &\tabularnewline
Target type & Data instance\tabularnewline
Name & Geometry observation\tabularnewline
Dependency & \href{http://www.opengis.net/spec/OMXML/2.0/req/geometryObservation}{http://www.opengis.net/spec/OMXML/2.0/req/geometryObservation,} OMXML clause 7.8\tabularnewline
\bottomrule
\end{longtable}

Note: Dependency \url{http://www.opengis.net/spec/OMXML/2.0/req/geometryObservation} has associated conformance class \url{http://www.opengis.net/spec/OMXML/2.0/conf/geometryObservation} (OMXML clause A.6).

202-16.18 Requirements class: Temporal observation data

202-16.18.1 This requirements class restricts the content model for the XML element\\
om:OM\_Observation such that the ``result'' of the observation is a temporal object.

Note: The XML implementation of temporal objects is provided in ISO 19136:2007, clause 14.

202-16.18.2 Instances of om:OM\_Observation with element om:type specifying \url{http://www.opengis.net/def/observationType/OGC-OM/2.0/OM_TemporalObservation} shall conform to all requirements specified in Table~202-16.15.

202-16.18.3 Instances of om:OM\_Observation with element om:type specifying \url{http://www.opengis.net/def/observationType/OGC-OM/2.0/OM_TemporalObservation} shall conform to all requirements of all relevant dependencies specified in Table~202-16.15.

Note: XML implementation of OM\_TemporalObservation is dependent on:

-- GML {[}ISO 19136:2007 Geographic information -- Geography Markup Language{]};

-- OMXML {[}OGC/IS 10-025r1 Observations and Measurements 2.0 -- XML Implementation{]}.

Table~202-16.15. Requirements class xsd-temporal-observation

\begin{longtable}[]{@{}ll@{}}
\toprule
Requirements class &\tabularnewline
\midrule
\endhead
\url{http://def.wmo.int/metce/2013/req/xsd-temporal-observation} &\tabularnewline
Target type & Data instance\tabularnewline
Name & Temporal observation\tabularnewline
Dependency & \href{http://www.opengis.net/spec/OMXML/2.0/req/temporalObservation}{http://www.opengis.net/spec/OMXML/2.0/req/temporalObservation,} OMXML clause 7.9\tabularnewline
\bottomrule
\end{longtable}

Note: Dependency \url{http://www.opengis.net/spec/OMXML/2.0/req/temporalObservation} has associated conformance class \url{http://www.opengis.net/spec/OMXML/2.0/conf/temporalObservation} (OMXML clause A.7).

202-16.19 Requirements class: Complex observation data

202-16.19.1 This requirements class restricts the content model for the XML element\\
om:OM\_Observation such that the ``result'' of the observation is a set of values relating to a specified location and time instant or duration. This ``result'' is represented in XML as a simple data record or vector from the SWE Common Data Model.

Notes:

1. OM\_ComplexObservation, as defined in ISO 19156:2011, is intended for use where the observation event is concerned with the evaluation of multiple properties at a specified location and time instant or duration. The ``result'' of this class of observations is a group of values, provided as a Record (as defined in ISO 19103:2005, Geographic information -- Conceptual schema language).

2. A description of data records, vectors and their XML implementation is provided in SWE Common 2.0, clauses 7.3 and 8.2.

202-16.19.2 Instances of om:OM\_Observation with element om:type specifying \url{http://www.opengis.net/def/observationType/OGC-OM/2.0/OM_ComplexObservation} shall conform to all requirements specified in Table~202-16.16.

202-16.19.3 Instances of om:OM\_Observation with element om:type specifying \url{http://www.opengis.net/def/observationType/OGC-OM/2.0/OM_ComplexObservation} shall conform to all requirements of all relevant dependencies specified in Table~202-16.16

Note: XML implementation of OM\_ComplexObservation is dependent on:

-- GML {[}ISO 19136:2007 Geographic information -- Geography Markup Language{]};

-- OMXML {[}OGC/IS 10-025r1 Observations and Measurements 2.0 -- XML Implementation{]};

-- SWE Common 2.0 {[}OGC/IS 08-094r1 SWE Common Data Model Encoding Standard 2.0{]}.

Table~202-16.16. Requirements class xsd-complex-observation

\begin{longtable}[]{@{}ll@{}}
\toprule
Requirements class &\tabularnewline
\midrule
\endhead
\url{http://def.wmo.int/metce/2013/req/xsd-complex-observation} &\tabularnewline
Target type & Data instance\tabularnewline
Name & Complex observation\tabularnewline
Dependency & \href{http://www.opengis.net/spec/OMXML/2.0/req/complexObservation}{http://www.opengis.net/spec/OMXML/2.0/req/complexObservation,} OMXML clause 7.10\tabularnewline
\begin{minipage}[t]{0.47\columnwidth}\raggedright
Requirement\strut
\end{minipage} & \begin{minipage}[t]{0.47\columnwidth}\raggedright
\url{http://def.wmo.int/metce/2013/req/xsd-complex-observation/xmlns-declaration-swe}

The OGC SWE Common 2.0 namespace \url{http://www.opengis.net/swe/2.0} shall be declared within the XML document.\strut
\end{minipage}\tabularnewline
\begin{minipage}[t]{0.47\columnwidth}\raggedright
Recommendation\strut
\end{minipage} & \begin{minipage}[t]{0.47\columnwidth}\raggedright
\url{http://def.wmo.int/metce/2013/req/xsd-complex-observation/xmlns-prefix-swe}

The default namespace prefix used for \url{http://www.opengis.net/swe/2.0} should be ``swe''.\strut
\end{minipage}\tabularnewline
\bottomrule
\end{longtable}

Notes:

1. Dependency \url{http://www.opengis.net/spec/OMXML/2.0/req/complexObservation} has associated conformance class \url{http://www.opengis.net/spec/OMXML/2.0/conf/complexObservation} (OMXML clause A.8).

2. The canonical schema location for OGC SWE Common 2.0 (\url{http://www.opengis.net/swe/2.0}) is \url{http://schemas.opengis.net/sweCommon/2.0/swe.xsd}.

202-16.20 Requirements class: Discrete coverage observation data

202-16.20.1 This requirements class restricts the content model for the XML element\\
om:OM\_Observation such that the ``result'' of the observation is a set of values describing the variation of properties with space and/or time.

Note: OM\_DiscreteCoverageObservation, as defined in ISO 19156:2011, is intended for use where the observation event is concerned with the evaluation of properties that vary with respect to space and/or time. The ``result'' of this class of observations is a discrete coverage (as defined in ISO 19123:2005 Geographic information -- Schema for coverage geometry and functions).

202-16.20.2 Instances of om:OM\_Observation with element om:type specifying \url{http://www.opengis.net/def/observationType/OGC-OM/2.0/OM_DiscreteCoverageObservation} shall conform to all requirements specified in Table~202-16.17.

202-16.20.3 Instances of om:OM\_Observation with element om:type specifying \url{http://www.opengis.net/def/observationType/OGC-OM/2.0/OM_DiscreteCoverageObservation} shall conform to all requirements of all relevant dependencies specified in Table~202-16.17.

Note: XML implementation of OM\_DiscreteCoverageObservation is dependent on:

-- GML {[}ISO 19136:2007 Geographic information -- Geography Markup Language{]};

-- OMXML {[}OGC/IS 10-025r1 Observations and Measurements 2.0 -- XML Implementation{]};

-- SWE Common 2.0 {[}OGC/IS 08-094r1 SWE Common Data Model Encoding Standard 2.0{]};

-- GMLCOV 1.0 {[}OGC/SAP 09-146r2 GML Application Schema -- Coverages 1.0.1{]}.

Table~202-16.17. Requirements class xsd-discrete-coverage-observation

\begin{longtable}[]{@{}ll@{}}
\toprule
Requirements class &\tabularnewline
\midrule
\endhead
\url{http://def.wmo.int/metce/2013/req/xsd-discrete-coverage-observation} &\tabularnewline
Target type & Data instance\tabularnewline
Name & Discrete coverage observation\tabularnewline
Dependency & \href{http://www.opengis.net/spec/OMXML/2.0/req/observation}{http://www.opengis.net/spec/OMXML/2.0/req/observation,} OMXML clause~7.3\tabularnewline
Dependency & \href{http://www.opengis.net/spec/SWE/2.0/req/xsd-simple-components}{http://www.opengis.net/spec/SWE/2.0/req/xsd-simple-components,} SWE Common 2.0 clause~8.1\tabularnewline
Dependency & \href{http://www.opengis.net/spec/SWE/2.0/req/xsd-record-components}{http://www.opengis.net/spec/SWE/2.0/req/xsd-record-components,} SWE Common 2.0 clause~8.2\tabularnewline
Dependency & \href{http://www.opengis.net/spec/SWE/2.0/req/xsd-block-components}{http://www.opengis.net/spec/SWE/2.0/req/xsd-block-components,} SWE Common 2.0 clause~8.4\tabularnewline
Dependency & \href{http://www.opengis.net/spec/SWE/2.0/req/xsd-simple-encodings}{http://www.opengis.net/spec/SWE/2.0/req/xsd-simple-encodings,} SWE Common 2.0 clause~8.5\tabularnewline
Dependency & \href{http://www.opengis.net/spec/gmlcov/1.0/req/gml-coverage}{http://www.opengis.net/spec/gmlcov/1.0/req/gml-coverage,} GMLCOV 1.0 clause~6\tabularnewline
Dependency & \url{http://www.opengis.net/doc/gml/gmlcov/1.0.1}, GMLCOV 1.0.1 clause~7\tabularnewline
\begin{minipage}[t]{0.47\columnwidth}\raggedright
Requirement\strut
\end{minipage} & \begin{minipage}[t]{0.47\columnwidth}\raggedright
\url{http://def.wmo.int/metce/2013/req/xsd-discrete-coverage-observation/xmlns-declaration-swe}

The OGC SWE Common 2.0 namespace \url{http://www.opengis.net/swe/2.0} shall be declared within the XML document.\strut
\end{minipage}\tabularnewline
\begin{minipage}[t]{0.47\columnwidth}\raggedright
Requirement\strut
\end{minipage} & \begin{minipage}[t]{0.47\columnwidth}\raggedright
\url{http://def.wmo.int/metce/2013/req/xsd-discrete-coverage-observation/xmlns-declaration-gmlcov}

The OGC GMLCOV 1.0 namespace \url{http://www.opengis.net/gmlcov/1.0} shall be declared within the XML document.\strut
\end{minipage}\tabularnewline
\begin{minipage}[t]{0.47\columnwidth}\raggedright
Requirement\strut
\end{minipage} & \begin{minipage}[t]{0.47\columnwidth}\raggedright
\url{http://def.wmo.int/metce/2013/req/xsd-discrete-coverage-observation/result-discrete-or-grid-coverage}

The XML element om:result shall contain a child element gml:DiscreteCoverage (or any element of a substitution group of gml:DiscreteCoverage), gml:GridCoverage, gml:RectifiedGridCoverage or gml:ReferenceableGridCoverage.\strut
\end{minipage}\tabularnewline
\begin{minipage}[t]{0.47\columnwidth}\raggedright
Requirement\strut
\end{minipage} & \begin{minipage}[t]{0.47\columnwidth}\raggedright
\url{http://def.wmo.int/metce/2013/req/xsd-discrete-coverage-observation/result-coverage-gml-encoding}

The child element of om:result shall be represented in GML as defined in GMLCOV 1.0.1 clause 7. MultiPart representation and special format representation shall not be used.\strut
\end{minipage}\tabularnewline
\begin{minipage}[t]{0.47\columnwidth}\raggedright
Recommendation\strut
\end{minipage} & \begin{minipage}[t]{0.47\columnwidth}\raggedright
\url{http://def.wmo.int/metce/2013/req/xsd-discrete-coverage-observation/xmlns-prefix-swe}

The default namespace prefix used for \url{http://www.opengis.net/swe/2.0} should be ``swe''.\strut
\end{minipage}\tabularnewline
\begin{minipage}[t]{0.47\columnwidth}\raggedright
Recommendation\strut
\end{minipage} & \begin{minipage}[t]{0.47\columnwidth}\raggedright
\url{http://def.wmo.int/metce/2013/req/xsd-discrete-coverage-observation/xmlns-prefix-gmlcov}

The default namespace prefix used for \url{http://www.opengis.net/gmlcov/1.0} should be ``gmlcov''.\strut
\end{minipage}\tabularnewline
\bottomrule
\end{longtable}

Notes:

1. Dependency \url{http://www.opengis.net/spec/OMXML/2.0/req/observation} has associated conformance class \url{http://www.opengis.net/spec/OMXML/2.0/conf/observation} (OMXML clause~A.1).

2. Dependency \url{http://www.opengis.net/spec/SWE/2.0/req/xsd-simple-components} has associated conformance class \url{http://www.opengis.net/spec/SWE/2.0/conf/xsd-simple-components} (SWE Common 2.0 clause~A.8).

3. Dependency \url{http://www.opengis.net/spec/SWE/2.0/req/xsd-record-components} has associated conformance class \url{http://www.opengis.net/spec/SWE/2.0/conf/xsd-record-components} (SWE Common 2.0 clause A.9).

4. Dependency \url{http://www.opengis.net/spec/SWE/2.0/req/xsd-block-components} has associated conformance class \url{http://www.opengis.net/spec/SWE/2.0/conf/xsd-block-components} (SWE Common 2.0 clause A.11).

5. Dependency \url{http://www.opengis.net/spec/SWE/2.0/req/xsd-simple-encodings} has associated conformance class \url{http://www.opengis.net/spec/SWE/2.0/conf/xsd-simple-encodings} (SWE Common 2.0 clause A.12).

6. The canonical schema location for OGC SWE Common 2.0 (\url{http://www.opengis.net/swe/2.0}) is \href{http://schemas.opengis.net/sweCommon/2.0/swe.xsd}{http://schemas.opengis.net/sweCommon/2.0/swe.xsd.}

7. Dependency \url{http://www.opengis.net/spec/gmlcov/1.0/req/gml-coverage} has associated conformance class (GMLCOV 1.0 clause A.1)

8. Dependency \url{http://www.opengis.net/spec/gmlcov/1.0/req/gml-coverage} has associated conformance class (GMLCOV 1.0 clause A.2)

9. The canonical schema location for OGC GMLCOV 1.0 (\url{http://www.opengis.net/gmlcov/1.0}) is \href{http://schemas.opengis.net/gmlcov/1.0/gmlcovAll.xsd}{http://schemas.opengis.net/gmlcov/1.0/gmlcovAll.xsd.}

FM~203: OPM

FM~203-15 Ext. OPM-XML Observable property Model

203-15-Ext.1 Scope

OPM-XML shall be used to represent complex observable properties (also known as quantity kinds) where individual observable properties are aggregated into groups or where the qualification and/or constraint applied to an observable property needs to be explicitly described.

Notes:

1. An ``observable property'' is a physical property that can be observed; typically this will be a quantitative property, such as dewpoint temperature.

2. OPM, the Observable Property Model, was developed to a draft level by the Open Geospatial Consortium Sensor Web Enablement Domain Working Group and was re-used in the INSPIRE Generic Conceptual Model. It is published by WMO to ensure that there is a stable definition of its XML schemas.

3. The Observable Property Model application schema is described in the \emph{Guidelines on Data Modelling for WMO Codes} (available in English only from \url{http://wis.wmo.int/metce-uml}).

The requirements classes defined in OPM-XML are listed in Table~203-15-Ext.1.

Table~203-15-Ext.1. Requirements classes defined in OPM-XML

\begin{longtable}[]{@{}ll@{}}
\toprule
Requirements classes &\tabularnewline
\midrule
\endhead
Requirements class & \url{http://def.wmo.int/opm/2013/req/xsd-observable-property}, 203-15-Ext.3\tabularnewline
Requirements class & \vtop{\hbox{\strut \url{http://def.wmo.int/opm/2013/req/xsd-composite-observable-property},}\hbox{\strut 203-15-Ext.4}}\tabularnewline
Requirements class & \vtop{\hbox{\strut \url{http://def.wmo.int/opm/2013/req/xsd-qualified-observable-property},}\hbox{\strut 203-15-Ext.5}}\tabularnewline
Requirements class & \url{http://def.wmo.int/opm/2013/req/xsd-statistical-qualifier}, 203-15-Ext.6\tabularnewline
Requirements class & \url{http://def.wmo.int/opm/2013/req/xsd-constraint}, 203-15-Ext.7\tabularnewline
Requirements class & \url{http://def.wmo.int/opm/2013/req/xsd-category-constraint}, 203-15-Ext.8\tabularnewline
Requirements class & \url{http://def.wmo.int/opm/2013/req/xsd-scalar-constraint}, 203-15-Ext.9\tabularnewline
Requirements class & \url{http://def.wmo.int/opm/2013/req/xsd-range-constraint}, 203-15-Ext.10\tabularnewline
\bottomrule
\end{longtable}

203-15-Ext.2 XML schema for OPM-XML

Representations of information in OPM-XML shall declare the XML namespaces listed in Table~203-15-Ext.2 and Table~203-15-Ext.3.

Notes:

1. Additional namespace declarations may be required depending on the XML elements used within OPM-XML.

2. The XML schema is packaged in two XML schema documents (XSD) describing one XML namespace: \url{http://def.wmo.int/opm/2013}.

3. Schematron schemas providing additional constraints are embedded within the XSD defining OPM-XML.

Table~203-15-Ext.2. XML namespaces defined for OPM-XML

\begin{longtable}[]{@{}lll@{}}
\toprule
XML namespace & Default namespace prefix & Canonical location of all-components schema document\tabularnewline
\midrule
\endhead
\url{http://def.wmo.int/opm/2013} & opm & \url{http://schemas.wmo.int/opm/1.1/opm.xsd}\tabularnewline
\bottomrule
\end{longtable}

Table~203-15-Ext.3. External XML namespaces used in OPM-XML

\begin{longtable}[]{@{}llll@{}}
\toprule
Standard & XML namespace & Default namespace prefix & Canonical location of all-components schema document\tabularnewline
\midrule
\endhead
XML schema & \url{http://www.w3.org/2001/XMLSchema} & xs &\tabularnewline
Schematron & \url{http://purl.oclc.org/dsdl/schematron} & sch &\tabularnewline
XSLT v2 & \url{http://www.w3.org/1999/XSL/Transform} & xsl &\tabularnewline
XML Linking Language & \url{http://www.w3.org/1999/xlink} & xlink & \url{http://www.w3.org/1999/xlink.xsd}\tabularnewline
ISO 19136:2007 GML & \url{http://www.opengis.net/gml/3.2} & gml & \url{http://schemas.opengis.net/gml/3.2.1/gml.xsd}\tabularnewline
\bottomrule
\end{longtable}

203-15-Ext.3 Requirements class: Observable property

203-15-Ext.3.1 This requirements class is used to describe the representation of an observable property.

Note: Representations providing more detailed information, such as composite observable property and qualified observable property (see 203-15-Ext.4 and 203-15-Ext.5), may be used if required.

203-15-Ext.3.2 XML elements describing observable properties shall conform to all requirements specified in Table~203-15-Ext.4.

203-15-Ext.3.3 XML elements describing observable properties shall conform to all requirements of all relevant dependencies specified in Table~203-15-Ext.4.

Table~203-15-Ext.4. Requirements class xsd-observable-property

\begin{longtable}[]{@{}ll@{}}
\toprule
Requirements class &\tabularnewline
\midrule
\endhead
\url{http://def.wmo.int/opm/2013/req/xsd-observable-property} &\tabularnewline
Target type & Data instance\tabularnewline
Name & Observable property\tabularnewline
\begin{minipage}[t]{0.47\columnwidth}\raggedright
Requirement\strut
\end{minipage} & \begin{minipage}[t]{0.47\columnwidth}\raggedright
\url{http://def.wmo.int/opm/2013/req/xsd-observable-property/valid}

The content model of this element shall have a value that matches the content model of opm:AbstractObservableProperty.\strut
\end{minipage}\tabularnewline
\begin{minipage}[t]{0.47\columnwidth}\raggedright
Recommendation\strut
\end{minipage} & \begin{minipage}[t]{0.47\columnwidth}\raggedright
\href{http://def.wmo.int/opm/2013/req/xsd-observable-property/valid}{http://def.wmo.int/opm/2013/req/xsd-observable-property/label}

The primary human-readable label for the observable property should be specified using the opm:label XML element.\strut
\end{minipage}\tabularnewline
\bottomrule
\end{longtable}

Notes:

1. Alternative human-readable labels may be specified using one or more instances of the XML element opm:altLabel.

2. The XML element opm:notation may be used to specify a notation or code-value that is used to identify the observable property within a given context (for example, providing a local identifier).

203-15-Ext.4 Requirements class: Composite observable property

203-15-Ext.4.1 This requirements class is used to describe the representation of an aggregate set of observable properties.

203-15-Ext.4.2 XML elements describing composite observable properties shall conform to all requirements specified in Table~203-15-Ext.5.

203-15-Ext.4.3 XML elements describing composite observable properties shall conform to all requirements of all relevant dependencies specified in Table~203-15-Ext.5.

Table~203-15-Ext.5. Requirements class xsd-composite-observable-property

\begin{longtable}[]{@{}ll@{}}
\toprule
Requirements class &\tabularnewline
\midrule
\endhead
\url{http://def.wmo.int/opm/2013/req/xsd-composite-observable-property} &\tabularnewline
Target type & Data instance\tabularnewline
Name & Composite observable property\tabularnewline
Dependency & \url{http://def.wmo.int/opm/2013/req/xsd-observable-property}, 203-15-Ext.3\tabularnewline
\begin{minipage}[t]{0.47\columnwidth}\raggedright
Requirement\strut
\end{minipage} & \begin{minipage}[t]{0.47\columnwidth}\raggedright
\url{http://def.wmo.int/opm/2013/req/xsd-composite-observable-property/valid}

The content model of this element shall have a value that matches the content model of opm:CompositeObservableProperty.\strut
\end{minipage}\tabularnewline
\begin{minipage}[t]{0.47\columnwidth}\raggedright
Requirement\strut
\end{minipage} & \begin{minipage}[t]{0.47\columnwidth}\raggedright
\url{http://def.wmo.int/opm/2013/req/xsd-composite-observable-property/set}

A composite observable property shall contain a minimum of two child observable properties.\strut
\end{minipage}\tabularnewline
\begin{minipage}[t]{0.47\columnwidth}\raggedright
Requirement\strut
\end{minipage} & \begin{minipage}[t]{0.47\columnwidth}\raggedright
\url{http://def.wmo.int/opm/2013/req/xsd-composite-observable-property/child-property}

For each child observable property within the composite observable property, the XML element //opm:CompositeObservableProperty/opm:property shall either contain a valid child element in the substitution group of opm:AbstractObservableProperty or provide a reference to the definition of the child observable property using the xlink:href attribute to indicate the URL where a description is located.\strut
\end{minipage}\tabularnewline
\begin{minipage}[t]{0.47\columnwidth}\raggedright
Requirement\strut
\end{minipage} & \begin{minipage}[t]{0.47\columnwidth}\raggedright
\url{http://def.wmo.int/opm/2013/req/xsd-composite-observable-property/count}

The XML attribute //opm:CompositeObservableProperty/@count shall specify the number of child observable properties from which the composite observable property is comprised.\strut
\end{minipage}\tabularnewline
\bottomrule
\end{longtable}

Note: A child observable property specified within a composite observable property instance may itself be a composite observable property, thus allowing arbitrarily complex nesting of sets of observable properties.

203-15-Ext.5 Requirements class: Qualified observable property

203-15-Ext.5.1 This requirements class is used to describe the representation of an observable property subject to additional qualification or constraint.

Note: The observable property to which the additional qualification or constraint is applied is known as the base property.

203-15-Ext.5.2 XML elements describing qualified observable properties shall conform to all requirements specified in Table~203-15-Ext.6.

203-15-Ext.5.3 XML elements describing qualified observable properties shall conform to all requirements of all relevant dependencies specified in Table~203-15-Ext.6.

Table~203-15-Ext.6. Requirements class xsd-qualified-observable-property

\begin{longtable}[]{@{}ll@{}}
\toprule
Requirements class &\tabularnewline
\midrule
\endhead
\url{http://def.wmo.int/opm/2013/req/xsd-qualified-observable-property} &\tabularnewline
Target type & Data instance\tabularnewline
Name & Qualified observable property\tabularnewline
Dependency & \url{http://def.wmo.int/opm/2013/req/xsd-observable-property}, 203-15-Ext.3\tabularnewline
\begin{minipage}[t]{0.47\columnwidth}\raggedright
Requirement\strut
\end{minipage} & \begin{minipage}[t]{0.47\columnwidth}\raggedright
\url{http://def.wmo.int/opm/2013/req/xsd-qualified-observable-property/valid}

The content model of this element shall have a value that matches the content model of opm:QualifiedObservableProperty.\strut
\end{minipage}\tabularnewline
\begin{minipage}[t]{0.47\columnwidth}\raggedright
Requirement\strut
\end{minipage} & \begin{minipage}[t]{0.47\columnwidth}\raggedright
\url{http://def.wmo.int/opm/2013/req/xsd-qualified-observable-property/base-property}

The XML element //opm:QualifiedObservableProperty/opm:baseProperty shall either contain a valid child element opm:ObservableProperty (or element within the substitution group of opm:ObservableProperty) that describes the base property or provide a reference to the definition of the base property using the xlink:href attribute to indicate the URL where a description is located.\strut
\end{minipage}\tabularnewline
\begin{minipage}[t]{0.47\columnwidth}\raggedright
Requirement\strut
\end{minipage} & \begin{minipage}[t]{0.47\columnwidth}\raggedright
\url{http://def.wmo.int/opm/2013/req/xsd-qualified-observable-property/specified-unit-of-measure}

If the base property is qualified such that values of the qualified observable property are always provided using a given unit of measurement, the XML attribute //opm:QualifiedObservableProperty/opm:unitOfMeasure/@uom shall be used to specify that unit of measurement.\strut
\end{minipage}\tabularnewline
\begin{minipage}[t]{0.47\columnwidth}\raggedright
Requirement\strut
\end{minipage} & \begin{minipage}[t]{0.47\columnwidth}\raggedright
\url{http://def.wmo.int/opm/2013/req/xsd-qualified-observable-property/valid-unit-of-measure}

If specified, the unit of measurement referenced via XML attribute //opm:QualifiedObservableProperty/opm:unitOfMeasure/@uom shall be appropriate for the base property.\strut
\end{minipage}\tabularnewline
\begin{minipage}[t]{0.47\columnwidth}\raggedright
Requirement\strut
\end{minipage} & \begin{minipage}[t]{0.47\columnwidth}\raggedright
\url{http://def.wmo.int/opm/2013/req/xsd-qualified-observable-property/qualifier}

If specified, the XML element //opm:QualifiedObservableProperty/opm:qualifier shall contain a valid child element opm:StatisticalQualifier that provides details of any statistical qualification applied to the base property.\strut
\end{minipage}\tabularnewline
\begin{minipage}[t]{0.47\columnwidth}\raggedright
Requirement\strut
\end{minipage} & \begin{minipage}[t]{0.47\columnwidth}\raggedright
\url{http://def.wmo.int/opm/2013/req/xsd-qualified-observable-property/constraint}

If specified, the XML element //opm:QualifiedObservableProperty/opm:constraint shall contain a valid child element opm:Constraint, or element in the substitution group of opm:Constraint, that provides details of any constraint applied to the base property.\strut
\end{minipage}\tabularnewline
\begin{minipage}[t]{0.47\columnwidth}\raggedright
Recommendation\strut
\end{minipage} & \begin{minipage}[t]{0.47\columnwidth}\raggedright
\url{http://def.wmo.int/opm/2013/req/xsd-qualified-observable-property/minimal-qualification}

At least one of the XML elements //opm:QualifiedObservableProperty/opm:unitOfMeasure, //opm:QualifiedObservableProperty/opm:qualifier or //opm:QualifiedObservableProperty/opm:constraint should be included within a qualified observable property.\strut
\end{minipage}\tabularnewline
\bottomrule
\end{longtable}

Note: Units of measurement are specified in accordance with 1.9 above.

203-15-Ext.6 Requirements class: Statistical qualifier

203-15-Ext.6.1 This requirements class is used to describe the representation of statistical qualifiers applied to an observable property.

Note: Typically, statistical qualification is based on some geometric or temporal aggregation using a given statistical function, for example, the maximum temperature in a 24-hour duration.

203-15-Ext.6.2 XML elements describing statistical qualification of observable properties shall conform to all requirements specified in Table~203-15-Ext.7.

203-15-Ext.6.3 XML elements describing statistical qualification of observable properties shall conform to all requirements of all relevant dependencies specified in Table~203-15-Ext.7.

Table~203-15-Ext.7. Requirements class xsd-statistical-qualifier

\begin{longtable}[]{@{}ll@{}}
\toprule
Requirements class &\tabularnewline
\midrule
\endhead
\url{http://def.wmo.int/opm/2013/req/xsd-statistical-qualifier} &\tabularnewline
Target type & Data instance\tabularnewline
Name & Statistical qualifier\tabularnewline
\begin{minipage}[t]{0.47\columnwidth}\raggedright
Requirement\strut
\end{minipage} & \begin{minipage}[t]{0.47\columnwidth}\raggedright
\url{http://def.wmo.int/opm/2013/req/xsd-statistical-qualifier/valid}

The content model of this element shall have a value that matches the content model of opm:StatisticalQualifier.\strut
\end{minipage}\tabularnewline
\begin{minipage}[t]{0.47\columnwidth}\raggedright
Requirement\strut
\end{minipage} & \begin{minipage}[t]{0.47\columnwidth}\raggedright
\url{http://def.wmo.int/opm/2013/req/xsd-statistical-qualifier/statistical-function-code}

The XML element //opm:StatisticalQualifier/opm:statisticalFunction shall reference the function used in the statistical qualification using the xlink:href attribute to specify the URI used to identify the target statistical function.\strut
\end{minipage}\tabularnewline
\begin{minipage}[t]{0.47\columnwidth}\raggedright
Requirement\strut
\end{minipage} & \begin{minipage}[t]{0.47\columnwidth}\raggedright
\url{http://def.wmo.int/opm/2013/req/xsd-statistical-qualifier/single-qualification-domain}

One, and only one, of XML elements //opmStatisticalQualifier/opm:aggregationArea, //opmStatisticalQualifier/opm:aggregationLength, //opmStatisticalQualifier/opm:aggregationTimePeriod, //opmStatisticalQualifier/opm:aggregationVolume~and\\
//opmStatisticalQualifier/opm:otherAggregation shall be included within a statistical qualification.\strut
\end{minipage}\tabularnewline
\begin{minipage}[t]{0.47\columnwidth}\raggedright
Recommendation\strut
\end{minipage} & \begin{minipage}[t]{0.47\columnwidth}\raggedright
\url{http://def.wmo.int/opm/2013/req/xsd-statistical-qualifier/description}

A textual description of the statistical qualification applied to the observable property should be provided using the //opmStatisticalQualifier/opm:description XML element.\strut
\end{minipage}\tabularnewline
\begin{minipage}[t]{0.47\columnwidth}\raggedright
Recommendation\strut
\end{minipage} & \begin{minipage}[t]{0.47\columnwidth}\raggedright
\url{http://def.wmo.int/opm/2013/req/xsd-statistical-qualifier/statistical-function-code-online-definition}

The URI used to identify the statistical function should have an available online definition and have been recognized by some level of authority.\strut
\end{minipage}\tabularnewline
\begin{minipage}[t]{0.47\columnwidth}\raggedright
Recommendation\strut
\end{minipage} & \begin{minipage}[t]{0.47\columnwidth}\raggedright
\url{http://def.wmo.int/opm/2013/req/xsd-statistical-qualifier/qualification-domain-type}

Where the statistical qualification domain relates to geometric area, geometric length, time period or geometric volume~then XML elements\\
//opmStatisticalQualifier/opm:aggregationArea, //opmStatisticalQualifier/opm:aggregationLength, //opmStatisticalQualifier/opm:aggregationTimePeriod or\\
//opmStatisticalQualifier/opm:aggregationVolume~should be used in preference to XML element //opmStatisticalQualifier/opm:otherAggregation to describe the statistical qualification domain.\strut
\end{minipage}\tabularnewline
\bottomrule
\end{longtable}

Notes:

1. Groups of statistical qualifiers may be applied to a given base property; in such a case the order of the statistical qualification is important. For example, mean daily maximum temperature over a one-month period comprises two statistical operations with respect to the base property ``air temperature'' -- a maximum over a 24-hour duration followed by a mean over a one-month duration. A collection of statistical qualifiers can be linked using the XML element //opmStatisticalQualifier/opm:derivedFrom to establish an ordered set.

2. Terms from Volume~I.2, FM 92 GRIB, Code table~4.10: Type of statistical processing, may be used to describe the statistical function. An alternative source of statistical function codes is provided in Volume~I.2, FM 94 BUFR, Code table~0~08~023: First-order statistics. For convenience, these code tables have been published online at \url{http://codes.wmo.int/grib2/codeflag/4.10} and \url{http://codes.wmo.int/bufr4/codeflag/0-08-023}, respectively.

203-15-Ext.7 Requirements class: Constraint

203-15-Ext.7.1 This requirements class is used to describe the representation of constraints applied to an observable property.

Notes:

1. The observable property that is used to constrain the base property is known as the constraining property. For example, the observed property ``radiance'' may be constrained such that one is concerned only with the radiance between wavelengths of 50 nm to 100 nm -- in which case the constraining property is ``wavelength''.

2. Representations providing more detailed information, such as scalar constraint, range constraint or category constraint, may be used if required.

203-15-Ext.7.2 XML elements describing the constraint of observable properties shall conform to all requirements specified in Table~203-15-Ext.8.

203-15-Ext.7.3 XML elements describing constraint of observable properties shall conform to all requirements of all relevant dependencies specified in Table~203-15-Ext.8.

Table~203-15-Ext.8. Requirements class xsd-constraint

\begin{longtable}[]{@{}ll@{}}
\toprule
Requirements class &\tabularnewline
\midrule
\endhead
\url{http://def.wmo.int/opm/2013/req/xsd-constraint} &\tabularnewline
Target type & Data instance\tabularnewline
Name & Constraint\tabularnewline
\begin{minipage}[t]{0.47\columnwidth}\raggedright
Requirement\strut
\end{minipage} & \begin{minipage}[t]{0.47\columnwidth}\raggedright
\url{http://def.wmo.int/opm/2013/req/xsd-constraint/valid}

The content model of this element shall have a value that matches the content model of opm:Constraint.\strut
\end{minipage}\tabularnewline
\begin{minipage}[t]{0.47\columnwidth}\raggedright
Requirement\strut
\end{minipage} & \begin{minipage}[t]{0.47\columnwidth}\raggedright
\url{http://def.wmo.int/opm/2013/req/xsd-constraint/constraint-property}

The XML element opm:constraintProperty shall either contain a valid child element opm:ObservableProperty (or element within the substitution group of opm:ObservableProperty) that describes the constraining property or provide a reference to the definition of the constraining property using the xlink:href attribute to indicate the URL where a description is located.\strut
\end{minipage}\tabularnewline
\begin{minipage}[t]{0.47\columnwidth}\raggedright
Recommendation\strut
\end{minipage} & \begin{minipage}[t]{0.47\columnwidth}\raggedright
\url{http://def.wmo.int/opm/2013/req/xsd-constraint/description}

A textual description of the constraint applied to the observable property should be provided using the opm:description XML element.\strut
\end{minipage}\tabularnewline
\bottomrule
\end{longtable}

203-15-Ext.8 Requirements class: Category constraint

203-15-Ext.8.1 This requirements class is used to describe the representation of category-based constraints applied to an observable property.

Note: For example, where one is interested only in the cloud-base height of convective clouds, the base property is ``cloud-base height'', the constraining property is ``cloud type'' and the values of the category constraint element list the particular cloud types of interest (for example, cumulonimbus and towering cumulus.).

203-15-Ext.8.2 XML elements describing a category-based constraint of observable properties shall conform to all requirements specified in Table~203-15-Ext.9.

203-15-Ext.8.3 XML elements describing a category-based constraint of observable properties shall conform to all requirements of all relevant dependencies specified in Table~203-15-Ext.9.

Table~203-15-Ext.9. Requirements class xsd-category-constraint

\begin{longtable}[]{@{}ll@{}}
\toprule
Requirements class &\tabularnewline
\midrule
\endhead
\url{http://def.wmo.int/opm/2013/req/xsd-category-constraint} &\tabularnewline
Target type & Data instance\tabularnewline
Name & Category constraint\tabularnewline
Dependency & \url{http://def.wmo.int/opm/2013/req/xsd-constraint}, 203-15-Ext.7\tabularnewline
\begin{minipage}[t]{0.47\columnwidth}\raggedright
Requirement\strut
\end{minipage} & \begin{minipage}[t]{0.47\columnwidth}\raggedright
\url{http://def.wmo.int/opm/2013/req/xsd-category-constraint/valid}

The content model of this element shall have a value that matches the content model of opm:CategoryConstraint.\strut
\end{minipage}\tabularnewline
\begin{minipage}[t]{0.47\columnwidth}\raggedright
Requirement\strut
\end{minipage} & \begin{minipage}[t]{0.47\columnwidth}\raggedright
\url{http://def.wmo.int/opm/2013/req/xsd-category-constraint/category-member}

One or more instances of the XML element //opm:CategoryConstraint/opm:value shall be used to specify the category members relevant to this constraint.\strut
\end{minipage}\tabularnewline
\begin{minipage}[t]{0.47\columnwidth}\raggedright
Requirement\strut
\end{minipage} & \begin{minipage}[t]{0.47\columnwidth}\raggedright
\url{http://def.wmo.int/opm/2013/req/xsd-category-constraint/category-member-appropriate-to-constraining-property}

Each of the category members defined using XML element //opm:CategoryConstraint/opm:value shall be appropriate to the constraining property.\strut
\end{minipage}\tabularnewline
\begin{minipage}[t]{0.47\columnwidth}\raggedright
Recommendation\strut
\end{minipage} & \begin{minipage}[t]{0.47\columnwidth}\raggedright
\url{http://def.wmo.int/opm/2013/req/xsd-category-constraint/category-value-code-space}

The XML attribute //opm:CategoryConstraint/opm:value/@gml:codeSpace should be provided when specifying each category member.\strut
\end{minipage}\tabularnewline
\begin{minipage}[t]{0.47\columnwidth}\raggedright
Recommendation\strut
\end{minipage} & \begin{minipage}[t]{0.47\columnwidth}\raggedright
\url{http://def.wmo.int/opm/2013/req/xsd-category-constraint/category-value-online-definition}

Appending the content of the XML element //opm:CategoryConstraint/opm:value to the content of XML attribute //opm:CategoryConstraint/opm:value/@gml:codeSpace should create a URI that resolves to an online definition that is recognized by some authority.\strut
\end{minipage}\tabularnewline
\bottomrule
\end{longtable}

203-15-Ext.9 Requirements class: Scalar constraint

203-15-Ext.9.1 This requirements class is used to describe the representation of scalar constraints applied to an observable property.

Note: For example, the base property ``air temperature'' may be constrained such that one is concerned only with air temperature at 1.2~metres above the local ground level (for example, a screen temperature); height above local ground level is the constraining property.

203-15-Ext.9.2 XML elements describing the scalar constraint of observable properties shall conform to all requirements specified in Table~203-15-Ext.10.

203-15-Ext.9.3 XML elements describing the scalar constraint of observable properties shall conform to all requirements of all relevant dependencies specified in Table~203-15-Ext.10.

Table~203-15-Ext.10. Requirements class xsd-scalar-constraint

\begin{longtable}[]{@{}ll@{}}
\toprule
Requirements class &\tabularnewline
\midrule
\endhead
\url{http://def.wmo.int/opm/2013/req/xsd-scalar-constraint} &\tabularnewline
Target type & Data instance\tabularnewline
Name & Scalar constraint\tabularnewline
Dependency & \url{http://def.wmo.int/opm/2013/req/xsd-constraint}, 203-15-Ext.7\tabularnewline
\begin{minipage}[t]{0.47\columnwidth}\raggedright
Requirement\strut
\end{minipage} & \begin{minipage}[t]{0.47\columnwidth}\raggedright
\url{http://def.wmo.int/opm/2013/req/xsd-scalar-constraint/valid}

The content model of this element shall have a value that matches the content model of opm:ScalarConstraint.\strut
\end{minipage}\tabularnewline
\begin{minipage}[t]{0.47\columnwidth}\raggedright
Requirement\strut
\end{minipage} & \begin{minipage}[t]{0.47\columnwidth}\raggedright
\url{http://def.wmo.int/opm/2013/req/xsd-scalar-constraint/comparison-operator}

The XML attribute //opm:ScalarConstraint/@comparisonOperator shall specify the mathematical operator relating the scalar constraint to the supplied numeric value.\strut
\end{minipage}\tabularnewline
\begin{minipage}[t]{0.47\columnwidth}\raggedright
Requirement\strut
\end{minipage} & \begin{minipage}[t]{0.47\columnwidth}\raggedright
\url{http://def.wmo.int/opm/2013/req/xsd-scalar-constraint/comparison-operator-enumeration}

The value of XML attribute //opm:ScalarConstraint/@comparisonOperator shall be one of the enumeration: ``ne'' (not equal to), ``lt'' (less than), ``le'' (less than or equal to), ``eq'' (equal to), ``ge'' (greater than or equal to) or ``gt'' (greater than).\strut
\end{minipage}\tabularnewline
\begin{minipage}[t]{0.47\columnwidth}\raggedright
Requirement\strut
\end{minipage} & \begin{minipage}[t]{0.47\columnwidth}\raggedright
\url{http://def.wmo.int/opm/2013/req/xsd-scalar-constraint/unit-of-measure}

Unless the constraining property is dimensionless, a unit of measurement shall be indicated that is appropriate for the constraining property via XML attribute //opm:ScalarConstraint/opm:unitOfMeasure/@uom.\strut
\end{minipage}\tabularnewline
\bottomrule
\end{longtable}

Note: Units of measurement are specified in accordance with 1.9 above.

203-15-Ext.10 Requirements class: Range constraint

203-15-Ext.10.1 This requirements class is used to describe the representation of constraints applied to an observable property according to a range of values.

Note: For example, the base property ``radiance'' may be constrained such that one is only concerned with radiance between wavelengths of 50~nm and 100~nm -- ``wavelength'' is the constraining property and is limited to the range 50~nm to 100~nm.

203-15-Ext.10.2 XML elements describing the range constraint of observable properties shall conform to all requirements specified in Table~203-15-Ext.11.

203-15-Ext.10.3 XML elements describing the range constraint of observable properties shall conform to all requirements of all relevant dependencies specified in Table~203-15-Ext.11.

Table~203-15-Ext.11. Requirements class xsd-range-constraint

\begin{longtable}[]{@{}ll@{}}
\toprule
Requirements class &\tabularnewline
\midrule
\endhead
\url{http://def.wmo.int/opm/2013/req/xsd-range-constraint} &\tabularnewline
Target type & Data instance\tabularnewline
Name & Range constraint\tabularnewline
Dependency & \url{http://def.wmo.int/opm/2013/req/xsd-constraint}, 203-15-Ext.7\tabularnewline
\begin{minipage}[t]{0.47\columnwidth}\raggedright
Requirement\strut
\end{minipage} & \begin{minipage}[t]{0.47\columnwidth}\raggedright
\url{http://def.wmo.int/opm/2013/req/xsd-range-constraint/valid}

The content model of this element shall have a value that matches the content model of opm:RangeConstraint.\strut
\end{minipage}\tabularnewline
\begin{minipage}[t]{0.47\columnwidth}\raggedright
Requirement\strut
\end{minipage} & \begin{minipage}[t]{0.47\columnwidth}\raggedright
\url{http://def.wmo.int/opm/2013/req/xsd-range-constraint/unit-of-measure}

Unless the constraining property is dimensionless, a unit of measurement shall be indicated that is appropriate for the constraining property via XML attribute\\
//opm:RangeConstraint/opm:unitOfMeasure/@uom.\strut
\end{minipage}\tabularnewline
\begin{minipage}[t]{0.47\columnwidth}\raggedright
Requirement\strut
\end{minipage} & \begin{minipage}[t]{0.47\columnwidth}\raggedright
\url{http://def.wmo.int/opm/2013/req/xsd-range-constraint/value}

The XML element //opm:RangeConstraint/opm:value shall contain a valid child element opm:RangeBounds wherein the start and end values of the constraining property range are defined.\strut
\end{minipage}\tabularnewline
\begin{minipage}[t]{0.47\columnwidth}\raggedright
Requirement\strut
\end{minipage} & \begin{minipage}[t]{0.47\columnwidth}\raggedright
\url{http://def.wmo.int/opm/2013/req/xsd-range-constraint/valid-range}

The numeric value of XML element //opm:RangeConstraint/opm:value/opm:RangeBounds/rangeStart shall be less than the numeric value of XML element\\
//opm:RangeConstraint/opm:value/opm:RangeBounds/rangeEnd.\strut
\end{minipage}\tabularnewline
\begin{minipage}[t]{0.47\columnwidth}\raggedright
Requirement\strut
\end{minipage} & \begin{minipage}[t]{0.47\columnwidth}\raggedright
\url{http://def.wmo.int/opm/2013/req/xsd-range-constraint/start-comparison}

The XML attribute //opm:RangeConstraint/opm:value/opm:RangeBounds/@startComparison shall specify the mathematical operator relating the range constraint to the supplied numeric value at the lower limit of the range.\strut
\end{minipage}\tabularnewline
\begin{minipage}[t]{0.47\columnwidth}\raggedright
Requirement\strut
\end{minipage} & \begin{minipage}[t]{0.47\columnwidth}\raggedright
\url{http://def.wmo.int/opm/2013/req/xsd-range-constraint/end-comparison}

The XML attribute //opm:RangeConstraint/opm:value/opm:RangeBounds/@endComparison shall specify the mathematical operator relating the range constraint to the supplied numeric value at the upper limit of the range.\strut
\end{minipage}\tabularnewline
\begin{minipage}[t]{0.47\columnwidth}\raggedright
Requirement\strut
\end{minipage} & \begin{minipage}[t]{0.47\columnwidth}\raggedright
\url{http://def.wmo.int/opm/2013/req/xsd-range-constraint/comparison-operator-enumeration}

The value of XML attributes //opm:RangeConstraint/opm:value/opm:RangeBounds/@startComparison and //opm:RangeConstraint/opm:value/opm:RangeBounds/@endComparison shall be one of the enumeration: ``ne'' (not equal to), ``lt'' (less than), ``le'' (less than or equal to), ``eq'' (equal to), ``ge'' (greater than or equal to) or ``gt'' (greater than).\strut
\end{minipage}\tabularnewline
\bottomrule
\end{longtable}

Note: Units of measurement are specified in accordance with 1.9 above.

FM~204: SAF

FM~204-15 Ext. SAF-XML Simple Aeronautical Features

204-15-Ext.1 Scope

SAF-XML shall be used to represent features relating to the provision of meteorological services for aviation, such as aerodromes, runways, air traffic management units and flight information regions.

Notes:

1. The entities provided in SAF-XML are intended to describe only the level of detail required for reporting meteorological information for international civil aviation purposes. Representations providing more detailed information may be used if required.

2. SAF-XML is used in FM 205-15 EXT. IWXXM (IWXXM 1.1). FM 205-16 IWXXM (IWXXM 2.1) does not require the use of SAF-XML.

The requirements classes defined in SAF-XML are listed in Table~204-15-Ext.1.

Table~204-15-Ext.1. Requirements classes defined in SAF-XML

\begin{longtable}[]{@{}ll@{}}
\toprule
Requirements classes &\tabularnewline
\midrule
\endhead
Requirements class & \url{http://icao.int/saf/1.1/req/xsd-unique-identification}, 204-15-Ext.3\tabularnewline
Requirements class & \url{http://icao.int/saf/1.1/req/xsd-aerodrome}, 204-15-Ext.4\tabularnewline
Requirements class & \url{http://icao.int/saf/1.1/req/xsd-runway}, 204-15-Ext.5\tabularnewline
Requirements class & \url{http://icao.int/saf/1.1/req/xsd-runway-direction}, 204-15-Ext.6\tabularnewline
Requirements class & \url{http://icao.int/saf/1.1/req/xsd-aeronautical-service-provision-units}, 204-15-Ext.7\tabularnewline
Requirements class & \url{http://icao.int/saf/1.1/req/xsd-airspace-volume}, 204-15-Ext.8\tabularnewline
Requirements class & \url{http://icao.int/saf/1.1/req/xsd-airspace}, 204-15-Ext.9\tabularnewline
\bottomrule
\end{longtable}

204-15-Ext.2 XML schema for SAF-XML

Representations of information in SAF-XML shall declare XML namespaces listed in Table 204-15-Ext.2 and Table~204-15-Ext.3.

Notes:

1. Additional namespace declarations may be required depending on the XML elements used within SAF-XML.

2. The XML schema is packaged into four XML schema documents (XSD) describing one XML namespace: \url{http://icao.int/saf/1.1}.

3. Schematron schemas providing additional constraints are embedded within the XSD defining SAF-XML.

Table~204-15-Ext.2. XML namespaces defined for SAF-XML

\begin{longtable}[]{@{}lll@{}}
\toprule
XML namespace & Default namespace prefix & Canonical location of all-components schema document\tabularnewline
\midrule
\endhead
\url{http://icao.int/saf/1.1} & saf & \url{http://schemas.wmo.int/saf/1.1/saf.xsd}\tabularnewline
\bottomrule
\end{longtable}

Table~204-15-Ext.3. External XML namespaces used in SAF-XML

\begin{longtable}[]{@{}llll@{}}
\toprule
Standard & XML namespace & Default namespace prefix & Canonical location of all-components schema document\tabularnewline
\midrule
\endhead
XML schema & \url{http://www.w3.org/2001/XMLSchema} & xs &\tabularnewline
Schematron & \url{http://purl.oclc.org/dsdl/schematron} & sch &\tabularnewline
XSLT v2 & \url{http://www.w3.org/1999/XSL/Transform} & xsl &\tabularnewline
XML Linking Language & \url{http://www.w3.org/1999/xlink} & xlink & \url{http://www.w3.org/1999/xlink.xsd}\tabularnewline
ISO 19136:2007 GML & \url{http://www.opengis.net/gml/3.2} & gml & \url{http://schemas.opengis.net/gml/3.2.1/gml.xsd}\tabularnewline
\bottomrule
\end{longtable}

204-15-Ext.3 Requirements class: Unique identification

204-15-Ext.3.1 This requirements class is used to describe how the information representations of aeronautical features are identified.

Notes:

1. Examples of aeronautical features include aerodromes, air traffic management units and flight information regions.

2. To achieve consistency with the Aeronautical Information Exchange Model (AIXM 5), the method of identification defined therein is adopted here. More details may be found in the document ``AIXM 5 Feature Identification and Reference''.

3. The identifier does not identify the real-world aeronautical feature itself; rather, it identifies the information representation about a given aeronautical feature. The originator of information about a given aeronautical feature uniquely identifies the information record it maintains within its data management systems about a given real-world aeronautical feature. The same identifier is then re-used in downstream information systems when referring to that information record. This ensures that all parties can be confident that they are working with the same information about a given aeronautical feature. Thus, if multiple systems use the same identifier for an aeronautical feature, this indicates (i)~the data are from the same source, or (ii) there are processes in place to ensure the consistency of data between those systems.

204-15-Ext.3.2 XML elements describing aeronautical features shall conform to all requirements specified in Table~204-15-Ext.4.

204-15-Ext.3.3 XML elements describing aeronautical features shall conform to all requirements of all relevant dependencies specified in Table~204-15-Ext.4.

Table~204-15-Ext.4. Requirements class xsd-unique-identification

\begin{longtable}[]{@{}ll@{}}
\toprule
Requirements class &\tabularnewline
\midrule
\endhead
\url{http://icao.int/saf/1.1/req/xsd-unique-identification} &\tabularnewline
Target type & Data instance\tabularnewline
Name & Unique identification\tabularnewline
\begin{minipage}[t]{0.47\columnwidth}\raggedright
Requirement\strut
\end{minipage} & \begin{minipage}[t]{0.47\columnwidth}\raggedright
\url{http://icao.int/saf/1.1/req/xsd-unique-identification/uniqueness}

An identifier scheme shall be used for assigning identity to information records that describe real-world aeronautical features, which ensures that there is a reasonable confidence that a given identifier will never be unintentionally used by anyone for anything else.

Different versions of the information record describing a given aeronautical feature shall be assigned different identifiers.\strut
\end{minipage}\tabularnewline
\begin{minipage}[t]{0.47\columnwidth}\raggedright
Requirement\strut
\end{minipage} & \begin{minipage}[t]{0.47\columnwidth}\raggedright
\url{http://icao.int/saf/1.1/req/xsd-unique-identification/gml-identifier}

The identifier for the information record describing a real-world aeronautical feature shall be specified using the XML element //gml:identifier.\strut
\end{minipage}\tabularnewline
\begin{minipage}[t]{0.47\columnwidth}\raggedright
Recommendation\strut
\end{minipage} & \begin{minipage}[t]{0.47\columnwidth}\raggedright
\url{http://icao.int/saf/1.1/req/xsd-unique-identification/uuid}

The identifier scheme that should be used for assigning identity to information records that describe real-world aeronautical features is universally unique identifier (UUID) version 4 based on random number generation.

The corresponding value of XML attribute //gml:identifier/@codeSpace associated with the use of UUID is "urn:uuid:".\strut
\end{minipage}\tabularnewline
\bottomrule
\end{longtable}

Note: UUID generators are widely available; for example, refer to the UUID generator of the International Telecommunication Union, which can be accessed at \url{http://www.itu.int/ITU-T/asn1/cgi-bin/uuid_generate}. No guarantee is made regarding the availability of this UUID generation service.

204-15-Ext.4 Requirements class: Aerodrome

204-15-Ext.4.1 This requirements class is used to describe the representation of an aerodrome. The class is targeted at providing a basic description of the aerodrome required for reporting meteorological information for international civil aviation purposes.

Notes:

1. Representations providing more detailed information may be used if required.

2. An aerodrome is a defined area on land or water (including any buildings, installations and equipment) intended to be used either wholly or in part~for the arrival, departure and surface movement of aircraft or helicopters.

204-15-Ext.4.2 XML elements describing aerodromes shall conform to all requirements specified in Table~204-15-Ext.5.

204-15-Ext.4.3 XML elements describing aerodromes shall conform to all requirements of all relevant dependencies specified in Table~204-15-Ext.5.

Table~204-15-Ext.5. Requirements class xsd-aerodrome

\begin{longtable}[]{@{}ll@{}}
\toprule
Requirements class &\tabularnewline
\midrule
\endhead
\url{http://icao.int/saf/1.1/req/xsd-aerodrome} &\tabularnewline
Target type & Data instance\tabularnewline
Name & Aerodrome\tabularnewline
Dependency & \url{http://icao.int/saf/1.1/req/xsd-unique-identification}, 204-15-Ext.3\tabularnewline
\begin{minipage}[t]{0.47\columnwidth}\raggedright
Requirement\strut
\end{minipage} & \begin{minipage}[t]{0.47\columnwidth}\raggedright
\url{http://icao.int/saf/1.1/req/xsd-aerodrome/valid}

The content model of this element shall have a value that matches the content model of saf:Aerodrome.\strut
\end{minipage}\tabularnewline
\begin{minipage}[t]{0.47\columnwidth}\raggedright
Requirement\strut
\end{minipage} & \begin{minipage}[t]{0.47\columnwidth}\raggedright
\url{http://icao.int/saf/1.1/req/xsd-aerodrome/icao-location-indicator}

If the aerodrome has a four-letter ICAO location indicator, this shall be specified using the XML element //saf:Aerodrome/saf:locationIndicatorICAO.\strut
\end{minipage}\tabularnewline
\begin{minipage}[t]{0.47\columnwidth}\raggedright
Requirement\strut
\end{minipage} & \begin{minipage}[t]{0.47\columnwidth}\raggedright
\url{http://icao.int/saf/1.1/req/xsd-aerodrome/iata-deisgnator}

If the aerodrome has a three-letter International Air Transport Association (IATA) designator, this shall be specified using the XML element //saf:Aerodrome/saf:designatorIATA.\strut
\end{minipage}\tabularnewline
\begin{minipage}[t]{0.47\columnwidth}\raggedright
Recommendation\strut
\end{minipage} & \begin{minipage}[t]{0.47\columnwidth}\raggedright
\url{http://icao.int/saf/1.1/req/xsd-aerodrome/designator}

The XML element //saf:Aerodrome/saf:designator should be used to specify the designator code for the aerodrome.

If the aerodrome has a four-letter ICAO location indicator, this should be used as the designator code.

If the aerodrome does not have a four-letter ICAO location indicator but does have a three-letter IATA code, this should be used as the designator code.

Alternatively, an artificially generated code should be used.\strut
\end{minipage}\tabularnewline
\begin{minipage}[t]{0.47\columnwidth}\raggedright
Recommendation\strut
\end{minipage} & \begin{minipage}[t]{0.47\columnwidth}\raggedright
\url{http://icao.int/saf/1.1/req/xsd-aerodrome/name}

The XML element //saf:Aerodrome/saf:name should be used to specify the primary official name of the aerodrome as designated by the appropriate authority.

The name should be provided in block capitals.\strut
\end{minipage}\tabularnewline
\begin{minipage}[t]{0.47\columnwidth}\raggedright
Recommendation\strut
\end{minipage} & \begin{minipage}[t]{0.47\columnwidth}\raggedright
\url{http://icao.int/saf/1.1/req/xsd-aerodrome/field-elevation}

The XML element //saf:Aerodrome/saf:fieldElevation should be used to specify the vertical distance above mean sea level of the highest point of the landing area.\strut
\end{minipage}\tabularnewline
\begin{minipage}[t]{0.47\columnwidth}\raggedright
Recommendation\strut
\end{minipage} & \begin{minipage}[t]{0.47\columnwidth}\raggedright
\url{http://icao.int/saf/1.1/req/xsd-aerodrome/field-elevation-unit-of-measure}

If specified, the vertical distance above mean sea level of the highest point of the landing area (field elevation) should be expressed in metres using the XML attribute //saf:Aerodrome/saf:fieldElevation/@uom with value ``m''.\strut
\end{minipage}\tabularnewline
\begin{minipage}[t]{0.47\columnwidth}\raggedright
Recommendation\strut
\end{minipage} & \begin{minipage}[t]{0.47\columnwidth}\raggedright
\url{http://icao.int/saf/1.1/req/xsd-aerodrome/aerodrome-reference-point}

The XML element //saf:Aerodrome/saf:ARP should be used to specify location of the aerodrome reference point.

Coordinate reference system EPSG 4326 should be used to report the location in latitude and longitude.

Coordinate reference system EPSG 4979 should be used to report the location in latitude, longitude and altitude.\strut
\end{minipage}\tabularnewline
\bottomrule
\end{longtable}

Notes:

1. ICAO designators are listed in \emph{Location Indicators} (ICAO Doc 7910).

2. Units of measurement are specified in accordance with 1.9 above.

204-15-Ext.5 Requirements class: Runway

204-15-Ext.5.1 This requirements class is used to describe the representation of a runway. The class is targeted at providing a basic description of the runway required for reporting meteorological information for international civil aviation purposes.

Notes:

1. Representations providing more detailed information may be used if required.

2. A runway is a defined rectangular area on a land aerodrome prepared for the landing and take-off of aircraft. This includes the concept of final approach and take-off area (FATO) for helicopters.

204-15-Ext.5.2 XML elements describing a runway shall conform to all requirements specified in Table~204-15-Ext.6.

204-15-Ext.5.3 XML elements describing a runway shall conform to all requirements of all relevant dependencies specified in Table~204-15-Ext.6.

Table~204-15-Ext.6. Requirements class xsd-runway

\begin{longtable}[]{@{}ll@{}}
\toprule
Requirements class &\tabularnewline
\midrule
\endhead
\url{http://icao.int/saf/1.1/req/xsd-runway} &\tabularnewline
Target type & Data instance\tabularnewline
Name & Runway\tabularnewline
Dependency & \url{http://icao.int/saf/1.1/req/xsd-unique-identification}, 204-15-Ext.3\tabularnewline
Dependency & \url{http://icao.int/saf/1.1/req/xsd-aerodrome}, 204-15-Ext.4\tabularnewline
\begin{minipage}[t]{0.47\columnwidth}\raggedright
Requirement\strut
\end{minipage} & \begin{minipage}[t]{0.47\columnwidth}\raggedright
\url{http://icao.int/saf/1.1/req/xsd-runway/valid}

The content model of this element shall have a value that matches the content model of saf:Runway.\strut
\end{minipage}\tabularnewline
\begin{minipage}[t]{0.47\columnwidth}\raggedright
Recommendation\strut
\end{minipage} & \begin{minipage}[t]{0.47\columnwidth}\raggedright
\url{http://icao.int/saf/1.1/req/xsd-runway/associated-aerodrome}

The XML element //saf:Runway/saf:associatedAirportHeliport should be used to indicate the aerodrome at which the runway is situated, using a value that matches the content model of saf:Aerodrome.\strut
\end{minipage}\tabularnewline
\begin{minipage}[t]{0.47\columnwidth}\raggedright
Recommendation\strut
\end{minipage} & \begin{minipage}[t]{0.47\columnwidth}\raggedright
\url{http://icao.int/saf/1.1/req/xsd-runway/designator}

Where an aerodrome has more than one runway, the XML element //saf:Runway/saf:designator should be used to specify the unique identifier for the runway within the aerodrome.\strut
\end{minipage}\tabularnewline
\bottomrule
\end{longtable}

204-15-Ext.6 Requirements class: Runway direction

204-15-Ext.6.1 This requirements class is used to describe the representation of one of the two landing and take-off directions of a runway.

Note: Representations providing more detailed information may be used if required.

204-15-Ext.6.2 XML elements describing runway direction shall conform to all requirements specified in Table~204-15-Ext.7.

204-15-Ext.6.3 XML elements describing runway direction shall conform to all requirements of all relevant dependencies specified in Table~204-15-Ext.7.

Table~204-15-Ext.7. Requirements class xsd-runway-direction

\begin{longtable}[]{@{}ll@{}}
\toprule
Requirements class &\tabularnewline
\midrule
\endhead
\url{http://icao.int/saf/1.1/req/xsd-runway-direction} &\tabularnewline
Target type & Data instance\tabularnewline
Name & Runway direction\tabularnewline
Dependency & \url{http://icao.int/saf/1.1/req/xsd-unique-identification}, 204-15-Ext.3\tabularnewline
Dependency & \url{http://icao.int/saf/1.1/req/xsd-runway}, 204-15-Ext.5\tabularnewline
\begin{minipage}[t]{0.47\columnwidth}\raggedright
Requirement\strut
\end{minipage} & \begin{minipage}[t]{0.47\columnwidth}\raggedright
\url{http://icao.int/saf/1.1/req/xsd-runway-direction/valid}

The content model of this element shall have a value that matches the content model of saf:RunwayDirection.\strut
\end{minipage}\tabularnewline
\begin{minipage}[t]{0.47\columnwidth}\raggedright
Recommendation\strut
\end{minipage} & \begin{minipage}[t]{0.47\columnwidth}\raggedright
\url{http://icao.int/saf/1.1/req/xsd-runway-direction/used-runway}

The XML element //saf:RunwayDirection/saf:usedRunway should be used to indicate the associated runway using a value that matches the content model of saf:Runway.\strut
\end{minipage}\tabularnewline
\begin{minipage}[t]{0.47\columnwidth}\raggedright
Recommendation\strut
\end{minipage} & \begin{minipage}[t]{0.47\columnwidth}\raggedright
\url{http://icao.int/saf/1.1/req/xsd-runway-direction/designator}

The textual designator for the landing and take-off direction of the associated runway should be specified using the XML element //saf:RunwayDirection/saf:designator.\strut
\end{minipage}\tabularnewline
\begin{minipage}[t]{0.47\columnwidth}\raggedright
Recommendation\strut
\end{minipage} & \begin{minipage}[t]{0.47\columnwidth}\raggedright
\url{http://icao.int/saf/1.1/req/xsd-runway-direction/true-bearing}

The measured angle between true north and the landing and take-off direction should be specified using the XML element //saf:RunwayDirection/saf:trueBearing.\strut
\end{minipage}\tabularnewline
\begin{minipage}[t]{0.47\columnwidth}\raggedright
Recommendation\strut
\end{minipage} & \begin{minipage}[t]{0.47\columnwidth}\raggedright
\url{http://icao.int/saf/1.1/req/xsd-runway-direction/true-bearing-unit-of-measure}

The measured angle between true north and the landing and take-off direction should be expressed in degrees using the XML attribute //saf:RunwayDirection/saf:trueBearing/@uom with value ``deg''.\strut
\end{minipage}\tabularnewline
\begin{minipage}[t]{0.47\columnwidth}\raggedright
Recommendation\strut
\end{minipage} & \begin{minipage}[t]{0.47\columnwidth}\raggedright
\url{http://icao.int/saf/1.1/req/xsd-runway-direction/elevation}

The vertical distance above mean sea level of the highest point of the runway touchdown zone should be specified using the XML element //saf:RunwayDirection/saf:elevationTDZ.\strut
\end{minipage}\tabularnewline
\begin{minipage}[t]{0.47\columnwidth}\raggedright
Recommendation\strut
\end{minipage} & \begin{minipage}[t]{0.47\columnwidth}\raggedright
\url{http://icao.int/saf/1.1/req/xsd-runway-direction/elevation-unit-of-measure}

If specified, the vertical distance above mean sea level of the highest point of the runway touchdown zone should be expressed in metres using the XML attribute\\
//saf:RunwayDirection/saf:elevationTDZ/@uom with value ``m''.\strut
\end{minipage}\tabularnewline
\bottomrule
\end{longtable}

Notes:

1. Examples of runway direction designators include ``27'', ``35L'' and ``01R''.

2. The true north is the north point at which the meridian lines meet.

3. Units of measurement are specified in accordance with 1.9 above.

204-15-Ext.7 Requirements class: Aeronautical service provision units

204-15-Ext.7.1 This requirements class is used to describe the representation of aeronautical service provision units.

Notes:

1. Aeronautical service provision units include air traffic services reporting office (ARO), air traffic control centre (ATCC), air traffic services unit (ATSU), flight information centre (FIC) and meteorological watch office (MWO).

2. Representations providing more detailed information may be used if required.

204-15-Ext.7.2 XML elements describing aeronautical service provision units shall conform to all requirements specified in Table~204-15-Ext.8.

204-15-Ext.7.3 XML elements describing aeronautical service provision units shall conform to all requirements of all relevant dependencies specified in Table~204-15-Ext.8.

Table~204-15-Ext.8. Requirements class xsd-aeronautical-service-provision-units

\begin{longtable}[]{@{}ll@{}}
\toprule
Requirements class &\tabularnewline
\midrule
\endhead
\url{http://icao.int/saf/1.1/req/xsd-aeronautical-service-provision-units} &\tabularnewline
Target type & Data instance\tabularnewline
Name & Aeronautical service provision units\tabularnewline
Dependency & \url{http://icao.int/saf/1.1/req/xsd-unique-identification}, 204-15-Ext.3\tabularnewline
\begin{minipage}[t]{0.47\columnwidth}\raggedright
Requirement\strut
\end{minipage} & \begin{minipage}[t]{0.47\columnwidth}\raggedright
\url{http://icao.int/saf/1.1/req/xsd-aeronautical-service-provision-units/valid}

The content model of this element shall have a value that matches the content model of saf:Unit.\strut
\end{minipage}\tabularnewline
\begin{minipage}[t]{0.47\columnwidth}\raggedright
Requirement\strut
\end{minipage} & \begin{minipage}[t]{0.47\columnwidth}\raggedright
\url{http://icao.int/saf/1.1/req/xsd-aeronautical-service-provision-units/unit-type-enumeration}

If specified, the value of XML element //saf:Unit/saf:type shall be one of the enumeration: ``ARO'' (Air traffic services Reporting Office), ``ATCC'' (Air Traffic Control Centre), ``ATSU'' (Air Traffic Services Unit), ``FIC'' (Flight Information Centre) or ``MWO'' (Meteorological Watch Office).\strut
\end{minipage}\tabularnewline
\begin{minipage}[t]{0.47\columnwidth}\raggedright
Recommendation\strut
\end{minipage} & \begin{minipage}[t]{0.47\columnwidth}\raggedright
\url{http://icao.int/saf/1.1/req/xsd-aeronautical-service-provision-units/name}

The XML element //saf:Unit/saf:name should be used to specify the primary official name of the aeronautical service provision unit as designated by the appropriate authority.

The name should be provided in block capitals.\strut
\end{minipage}\tabularnewline
\begin{minipage}[t]{0.47\columnwidth}\raggedright
Recommendation\strut
\end{minipage} & \begin{minipage}[t]{0.47\columnwidth}\raggedright
\url{http://icao.int/saf/1.1/req/xsd-aeronautical-service-provision-units/type}

The type of aeronautical service provision unit should be indicated using the XML element //saf:Unit/saf:type.\strut
\end{minipage}\tabularnewline
\begin{minipage}[t]{0.47\columnwidth}\raggedright
Recommendation\strut
\end{minipage} & \begin{minipage}[t]{0.47\columnwidth}\raggedright
\url{http://icao.int/saf/1.1/req/xsd-aeronautical-service-provision-units/designator}

The coded designator used to identify the aeronautical service provision unit should be indicated using the XML element //saf:Unit/saf:designator.\strut
\end{minipage}\tabularnewline
\bottomrule
\end{longtable}

Notes:

1. Coded designators for aeronautical service provision units are specified in \emph{Location Indicators} (ICAO Doc~7910).

2. The location of the aeronautical service provision unit, expressed as a reference point, may be specified using XML element //saf:Unit/saf:position.

204-15-Ext.8 Requirements class: Airspace volume

204-15-Ext.8.1 This requirements class is used to describe the geometric representation of a three-dimensional airspace volume.

Notes:

1. Representations providing more detailed information may be used if required.

2. The three-dimensional region of space is specified as a two-dimensional horizontal region with bounded vertical extent.

204-15-Ext.8.2 XML elements describing an airspace volume~shall conform to all requirements specified in Table~204-15-Ext.9.

204-15-Ext.8.3 XML elements describing an airspace volume shall conform to all requirements of all relevant dependencies specified in Table~204-15-Ext.9.

Table~204-15-Ext.9. Requirements class xsd-airspace-volume

\begin{longtable}[]{@{}ll@{}}
\toprule
Requirements class &\tabularnewline
\midrule
\endhead
\url{http://icao.int/saf/1.1/req/xsd-airspace-volume} &\tabularnewline
Target type & Data instance\tabularnewline
Name & Airspace volume\tabularnewline
\begin{minipage}[t]{0.47\columnwidth}\raggedright
Requirement\strut
\end{minipage} & \begin{minipage}[t]{0.47\columnwidth}\raggedright
\url{http://icao.int/saf/1.1/req/xsd-airspace-volume/valid}

The content model of this element shall have a value that matches the content model of saf:AirspaceVolume.\strut
\end{minipage}\tabularnewline
\begin{minipage}[t]{0.47\columnwidth}\raggedright
Requirement\strut
\end{minipage} & \begin{minipage}[t]{0.47\columnwidth}\raggedright
\url{http://icao.int/saf/1.1/req/xsd-airspace-volume/upper-limit}

If the upper limit of the vertical extent of the airspace is specified (using XML element //saf:AirspaceVolume/saf:upperLimit), then the XML element //saf:AirspaceVolume/saf:upperLimitReference shall be used to specify the associated vertical reference system.\strut
\end{minipage}\tabularnewline
\begin{minipage}[t]{0.47\columnwidth}\raggedright
Requirement\strut
\end{minipage} & \begin{minipage}[t]{0.47\columnwidth}\raggedright
\url{http://icao.int/saf/1.1/req/xsd-airspace-volume/lower-limit}

If the lower limit of the vertical extent of the airspace is specified (using XML element //saf:AirspaceVolume/saf:lowerLimit), then the XML element //saf:AirspaceVolume/saf:lowerLimitReference shall be used to specify the associated vertical reference system.\strut
\end{minipage}\tabularnewline
\begin{minipage}[t]{0.47\columnwidth}\raggedright
Requirement\strut
\end{minipage} & \begin{minipage}[t]{0.47\columnwidth}\raggedright
\url{http://icao.int/saf/1.1/req/xsd-airspace-volume/limit-type}

The values of XML elements //saf:AirspaceVolume/saf:upperLimitReference and\\
//saf:AirspaceVolume/saf:lowerLimitReference specifying a vertical reference system shall be one of the enumeration: ``SFC'' (distance measured from the surface of the earth), ``MSL'' (distance measured from mean sea level), ``W84'' (distance measured from the WGS84 ellipsoid) or ``STD'' (distance measured with an altimeter set to the standard atmosphere).\strut
\end{minipage}\tabularnewline
\begin{minipage}[t]{0.47\columnwidth}\raggedright
Recommendation\strut
\end{minipage} & \begin{minipage}[t]{0.47\columnwidth}\raggedright
\url{http://icao.int/saf/1.1/req/xsd-airspace-volume/horizontal-projection}

The XML element //saf:Airspace/saf:horizontalProjection should be used to describe the geometry of the horizontal extent of the airspace volume.\strut
\end{minipage}\tabularnewline
\bottomrule
\end{longtable}

Notes:

1. Omission of the upper limit of the vertical extent (the airspace ceiling) indicates that the airspace extends upward to, or beyond, the limit of aeronautical operations, whilst omission of the lower limit of the vertical extent (the airspace floor) indicates that the airspace extends to the land/sea surface.

2. Distance measured from mean sea level is equivalent to ``altitude''.

204-15-Ext.9 Requirements class: Airspace

204-15-Ext.9.1 This requirements class is used to describe the representation of airspaces.

Notes:

1. An airspace is a defined three-dimensional region of space relevant to air traffic. Airspace types include flight information region (FIR), upper flight information region (UIR) and controlled airspace (CTA).

2. Representations providing more detailed information may be used if required.

204-15-Ext.9.2 XML elements describing airspaces shall conform to all requirements specified in Table~204-15-Ext.10.

204-15-Ext.9.3 XML elements describing airspaces shall conform to all requirements of all relevant dependencies specified in Table~204-15-Ext.10.

Table~204-15-Ext.10. Requirements class xsd-airspace

\begin{longtable}[]{@{}ll@{}}
\toprule
Requirements class &\tabularnewline
\midrule
\endhead
\url{http://icao.int/saf/1.1/req/xsd-airspace} &\tabularnewline
Target type & Data instance\tabularnewline
Name & Airspace\tabularnewline
Dependency & \url{http://icao.int/saf/1.1/req/xsd-unique-identification}, 204-15-Ext.3\tabularnewline
Dependency & \url{http://icao.int/saf/1.1/req/xsd-airspace-volume}, 204-15-Ext.8\tabularnewline
\begin{minipage}[t]{0.47\columnwidth}\raggedright
Requirement\strut
\end{minipage} & \begin{minipage}[t]{0.47\columnwidth}\raggedright
\url{http://icao.int/saf/1.1/req/xsd-airspace/valid}

The content model of this element shall have a value that matches the content model of saf:Airspace.\strut
\end{minipage}\tabularnewline
\begin{minipage}[t]{0.47\columnwidth}\raggedright
Requirement\strut
\end{minipage} & \begin{minipage}[t]{0.47\columnwidth}\raggedright
\url{http://icao.int/saf/1.1/req/xsd-airspace/icao-designator-indication}

If the coded designator used to identify the airspace is an ICAO recognized designator, then the XML element //saf:Airspace/saf:designatorICAO shall have the value ``true''.\strut
\end{minipage}\tabularnewline
\begin{minipage}[t]{0.47\columnwidth}\raggedright
Requirement\strut
\end{minipage} & \begin{minipage}[t]{0.47\columnwidth}\raggedright
\url{http://icao.int/saf/1.1/req/xsd-airspace/airspace-type-enumeration}

If specified, the value of XML element //saf:Airspace/saf:type shall be one of the enumeration: ``FIR'' (Flight Information Region), ``UIR'' (Upper Flight Information Region), ``FIR\_UIR'' (Flight Information Region or Upper Flight Information Region) or ``CTA'' (Controlled Airspace).\strut
\end{minipage}\tabularnewline
\begin{minipage}[t]{0.47\columnwidth}\raggedright
Recommendation\strut
\end{minipage} & \begin{minipage}[t]{0.47\columnwidth}\raggedright
\url{http://icao.int/saf/1.1/req/xsd-airspace/type}

The type of airspace should be indicated using the XML element //saf:Airspace/saf:type.\strut
\end{minipage}\tabularnewline
\begin{minipage}[t]{0.47\columnwidth}\raggedright
Recommendation\strut
\end{minipage} & \begin{minipage}[t]{0.47\columnwidth}\raggedright
\url{http://icao.int/saf/1.1/req/xsd-airspace/designator}

The coded designator used to identify the airspace should be indicated using the XML element //saf:Airspace/saf:designator.\strut
\end{minipage}\tabularnewline
\begin{minipage}[t]{0.47\columnwidth}\raggedright
Recommendation\strut
\end{minipage} & \begin{minipage}[t]{0.47\columnwidth}\raggedright
\url{http://icao.int/saf/1.1/req/xsd-airspace/name}

The XML element //saf:Airspace/saf:name should be used to specify the official name of the airspace as designated by the appropriate authority.

The name should be provided in block capitals.\strut
\end{minipage}\tabularnewline
\begin{minipage}[t]{0.47\columnwidth}\raggedright
Recommendation\strut
\end{minipage} & \begin{minipage}[t]{0.47\columnwidth}\raggedright
\url{http://icao.int/saf/1.1/req/xsd-airspace/geometry-component}

The XML element //saf:Airspace/saf:geometryComponent should be used to describe the geometric volume~of the airspace using a value that matches the content model of saf:AirspaceVolume.\strut
\end{minipage}\tabularnewline
\bottomrule
\end{longtable}

Notes:

1. ICAO designators are listed in \emph{Location Indicators} (ICAO Doc~7910).

2. An airspace may comprise multiple geometry elements.

FM~205: IWXXM

FM~205-15 Ext. IWXXM-XML ICAO Meteorological informAtion Exchange Model

205-15-Ext.1 Scope

205-15-Ext.1.1 IWXXM-XML shall be used to represent observations and forecasts, and reports thereof, for international civil aviation, as specified by the \emph{Technical Regulations} (WMO-No.~49), Volume~II -- Meteorological Service for International Air Navigation.

205-15-Ext.1.2 IWXXM-XML includes provision for aerodrome routine meteorological reports (METAR), aerodrome special meteorological reports (SPECI), aerodrome forecast (TAF) reports and SIGMET information.

Note: SIGMET information is information issued by a meteorological watch office concerning the occurrence or expected occurrence of specified en-route weather phenomena that may affect the safety of aircraft operations.

205-15-Ext.1.3 The requirements classes defined in IWXXM-XML are listed in Table~205-15-Ext.1.

Table~205-15-Ext.1. Requirements classes defined in IWXXM-XML

\begin{longtable}[]{@{}ll@{}}
\toprule
Requirements classes &\tabularnewline
\midrule
\endhead
Requirements class & \url{http://icao.int/iwxxm/1.1/req/xsd-cloud-layer}, 205-15-Ext.4\tabularnewline
Requirements class & \url{http://icao.int/iwxxm/1.1/req/xsd-aerodrome-cloud-forecast}, 205-15-Ext.5\tabularnewline
Requirements class & \url{http://icao.int/iwxxm/1.1/req/xsd-aerodrome-runway-state}, 205-15-Ext.6\tabularnewline
Requirements class & \url{http://icao.int/iwxxm/1.1/req/xsd-aerodrome-wind-shear}, 205-15-Ext.7\tabularnewline
Requirements class & \url{http://icao.int/iwxxm/1.1/req/xsd-aerodrome-observed-clouds}, 205-15-Ext.8\tabularnewline
Requirements class & \url{http://icao.int/iwxxm/1.1/req/xsd-aerodrome-runway-visual-range}, 205-15-Ext.9\tabularnewline
Requirements class & \url{http://icao.int/iwxxm/1.1/req/xsd-aerodrome-sea-state}, 205-15-Ext.10\tabularnewline
Requirements class & \url{http://icao.int/iwxxm/1.1/req/xsd-aerodrome-horizontal-visibility}, 205-15-Ext.11\tabularnewline
Requirements class & \url{http://icao.int/iwxxm/1.1/req/xsd-aerodrome-surface-wind}, 205-15-Ext.12\tabularnewline
Requirements class & \url{http://icao.int/iwxxm/1.1/req/xsd-meteorological-aerodrome-observation-record}, 205-15-Ext.13\tabularnewline
Requirements class & \vtop{\hbox{\strut \url{http://icao.int/iwxxm/1.1/req/xsd-meteorological-aerodrome-observation},}\hbox{\strut 205-15-Ext.14}}\tabularnewline
Requirements class & \vtop{\hbox{\strut \url{http://icao.int/iwxxm/1.1/req/xsd-aerodrome-surface-wind-trend-forecast},}\hbox{\strut 205-15-Ext.15}}\tabularnewline
Requirements class & \url{http://icao.int/iwxxm/1.1/req/xsd-meteorological-aerodrome-trend-forecast-record}, 205-15-Ext.16\tabularnewline
Requirements class & \vtop{\hbox{\strut \url{http://icao.int/iwxxm/1.1/req/xsd-meteorological-aerodrome-trend-forecast},}\hbox{\strut 205-15-Ext.17}}\tabularnewline
Requirements class & \url{http://icao.int/iwxxm/1.1/req/xsd-meteorological-aerodrome-observation-report}, 205-15-Ext.18\tabularnewline
Requirements class & \url{http://icao.int/iwxxm/1.1/req/xsd-metar}, 205-15-Ext.19\tabularnewline
Requirements class & \url{http://icao.int/iwxxm/1.1/req/xsd-speci}, 205-15-Ext.20\tabularnewline
Requirements class & \vtop{\hbox{\strut \url{http://icao.int/iwxxm/1.1/req/xsd-aerodrome-surface-wind-forecast},}\hbox{\strut 205-15-Ext.21}}\tabularnewline
Requirements class & \vtop{\hbox{\strut \url{http://icao.int/iwxxm/1.1/req/xsd-aerodrome-air-temperature-forecast},}\hbox{\strut 205-15-Ext.22}}\tabularnewline
Requirements class & \vtop{\hbox{\strut \url{http://icao.int/iwxxm/1.1/req/xsd-meteorological-aerodrome-forecast-record},}\hbox{\strut 205-15-Ext.23}}\tabularnewline
Requirements class & \vtop{\hbox{\strut \url{http://icao.int/iwxxm/1.1/req/xsd-meteorological-aerodrome-forecast},}\hbox{\strut 205-15-Ext.24}}\tabularnewline
Requirements class & \url{http://icao.int/iwxxm/1.1/req/xsd-taf}, 205-15-Ext.25\tabularnewline
Requirements class & \vtop{\hbox{\strut \url{http://icao.int/iwxxm/1.1/req/xsd-evolving-meteorological-condition},}\hbox{\strut 205-15-Ext.26}}\tabularnewline
Requirements class & \vtop{\hbox{\strut \url{http://icao.int/iwxxm/1.1/req/xsd-sigmet-evolving-condition-analysis},}\hbox{\strut 205-15-Ext.27}}\tabularnewline
Requirements class & \url{http://icao.int/iwxxm/1.1/req/xsd-meteorological-position}, 205-15-Ext.28\tabularnewline
Requirements class & \vtop{\hbox{\strut \url{http://icao.int/iwxxm/1.1/req/xsd-meteorological-position-collection},}\hbox{\strut 205-15-Ext.29}}\tabularnewline
Requirements class & \url{http://icao.int/iwxxm/1.1/req/xsd-sigmet-position-analysis}, 205-15-Ext.30\tabularnewline
Requirements class & \url{http://icao.int/iwxxm/1.1/req/xsd-sigmet}, 205-15-Ext.31\tabularnewline
Requirements class & \url{http://icao.int/iwxxm/1.1/req/xsd-volcanic-ash-sigmet}, 205-15-Ext.32\tabularnewline
Requirements class & \url{http://icao.int/iwxxm/1.1/req/xsd-tropical-cyclone-sigmet}, 205-15-Ext.33\tabularnewline
\bottomrule
\end{longtable}

205-15-Ext.2 XML schema for IWXXM-XML

Representations of information in IWXXM-XML shall declare the XML namespaces listed in Table~205-15-Ext.2 and Table~205-15-Ext.3.

Notes:

1. Additional namespace declarations may be required depending on the XML elements used within IWXXM‑XML.

2. The XML schema is packaged in five XML schema documents (XSD) describing one XML namespace: \url{http://icao.int/iwxxm/1.1}.

3. Schematron schemas providing additional constraints are embedded within the XSD defining IWXXM-XML.

Table~205-15-Ext.2. XML namespaces defined for IWXXM-XML

\begin{longtable}[]{@{}lll@{}}
\toprule
XML namespace & Default namespace prefix & Canonical location of all-components schema document\tabularnewline
\midrule
\endhead
\url{http://icao.int/iwxxm/1.1} & iwxxm & \url{http://schemas.wmo.int/iwxxm/1.1/iwxxm.xsd}\tabularnewline
\bottomrule
\end{longtable}

Table~205-15-Ext.3. External XML namespaces used in IWXXM-XML

\begin{longtable}[]{@{}llll@{}}
\toprule
Standard & XML namespace & Default namespace prefix & Canonical location of all-components schema document\tabularnewline
\midrule
\endhead
XML schema & \url{http://www.w3.org/2001/XMLSchema} & xs &\tabularnewline
Schematron & \url{http://purl.oclc.org/dsdl/schematron} & sch &\tabularnewline
XSLT v2 & \url{http://www.w3.org/1999/XSL/Transform} & xsl &\tabularnewline
XML Linking Language & \url{http://www.w3.org/1999/xlink} & xlink & \url{http://www.w3.org/1999/xlink.xsd}\tabularnewline
ISO 19136:2007 GML & \url{http://www.opengis.net/gml/3.2} & gml & \url{http://schemas.opengis.net/gml/3.2.1/gml.xsd}\tabularnewline
ISO/TS 19139:2007 metadata XML implementation & \url{http://www.isotc211.org/2005/gmd} & gmd & \url{http://standards.iso.org/ittf/PubliclyAvailableStandards/ISO_19139_Schemas/gmd/gmd.xsd}\tabularnewline
OGC OMXML & \url{http://www.opengis.net/om/2.0} & om & \url{http://schemas.opengis.net/om/2.0/observation.xsd}\tabularnewline
OGC OMXML & \url{http://www.opengis.net/sampling/2.0} & sam & \url{http://schemas.opengis.net/sampling/2.0/samplingFeature.xsd}\tabularnewline
OGC OMXML & \url{http://www.opengis.net/samplingSpatial/2.0} & sams & \url{http://schemas.opengis.net/samplingSpatial/2.0/spatialSamplingFeature.xsd}\tabularnewline
FM 202-15 Ext METCE-XML & \url{http://def.wmo.int/metce/2013} & metce & \url{http://schemas.wmo.int/metce/1.1/metce.xsd}\tabularnewline
FM 203-15 Ext. OPM-XML & \url{http://def.wmo.int/opm/2013} & opm & \url{http://schemas.wmo.int/opm/1.1/opm.xsd}\tabularnewline
FM 204-15 Ext SAF-XML & \url{http://icao.int/saf/1.1} & saf & \url{http://schemas.wmo.int/saf/1.1/saf.xsd}\tabularnewline
\bottomrule
\end{longtable}

205-15-Ext.3 Virtual typing

In accordance with OMXML (clause~7.2), the specialization of OM\_Observation is provided through Schematron restriction. The om:type element shall be used to specify the type of OM\_Observation that is being encoded using the URI for the corresponding observation type listed in Code table~D-4.

Notes:

1. Code table~D-4 is described in Appendix~A.

2. Code table~D-4 is published online at \url{http://codes.wmo.int/49-2/observation-type/IWXXM/1.0}.

3. The URI for each observation type is composed by appending the \emph{notation} to the \emph{code-space}. As an example, the URI of MeteorologicalAerodromeForecast is \href{http://codes.wmo.int/49-2/observation-type/IWXXM/1.0/MeteorologicalAerodromeForecast}{http://codes.wmo.int/49-2/observation-type/IWXXM/1.0/\\
MeteorologicalAerodromeForecast}.

4. Each URI will resolve to provide further information about the associated observation type.

205-15-Ext.4 Requirements class: Cloud layer

205-15-Ext.4.1 This requirements class is used to describe the representation of a cloud layer. The class is targeted at providing a basic description of the cloud layer as required for international civil aviation purposes.

Notes:

1. Representations providing more detailed information may be used if required.

2. The requirements for reporting cloud information are specified in the \emph{Technical Regulations} (WMO-No.~49), Volume~II, Part~II, Appendix~3, 4.5 and Appendix~5, 2.2.5 and 1.2.4.

205-15-Ext.4.2 XML elements describing cloud layers shall conform to all requirements specified in Table~205-15-Ext.4.

205-15-Ext.4.3 XML elements describing cloud layers shall conform to all requirements of all relevant dependencies specified in Table~205-15-Ext.4.

Table~205-15-Ext.4. Requirements class: xsd-cloud-layer

\begin{longtable}[]{@{}ll@{}}
\toprule
Requirements class &\tabularnewline
\midrule
\endhead
\url{http://icao.int/iwxxm/1.1/req/xsd-cloud-layer} &\tabularnewline
Target type & Data instance\tabularnewline
Name & Cloud layer\tabularnewline
\begin{minipage}[t]{0.47\columnwidth}\raggedright
Requirement\strut
\end{minipage} & \begin{minipage}[t]{0.47\columnwidth}\raggedright
\url{http://icao.int/iwxxm/1.1/req/xsd-cloud-layer/valid}

The content model of this element shall have a value that matches the content model of iwxxm:CloudLayer.\strut
\end{minipage}\tabularnewline
\begin{minipage}[t]{0.47\columnwidth}\raggedright
Requirement\strut
\end{minipage} & \begin{minipage}[t]{0.47\columnwidth}\raggedright
\url{http://icao.int/iwxxm/1.1/req/xsd-cloud-layer/cloud-amount}

The XML element //iwxxm:CloudLayer/iwxxm:amount shall be used to report an amount of cloud of operational significance.\strut
\end{minipage}\tabularnewline
\begin{minipage}[t]{0.47\columnwidth}\raggedright
Requirement\strut
\end{minipage} & \begin{minipage}[t]{0.47\columnwidth}\raggedright
\url{http://icao.int/iwxxm/1.1/req/xsd-cloud-layer/cloud-amount-code}

If cloud amount is reported, the value of XML attribute //iwxxm:CloudLayer/iwxxm:amount/@xlink:href shall be the URI of the valid term from Code table~D‑8: Cloud amount reported at aerodrome.\strut
\end{minipage}\tabularnewline
\begin{minipage}[t]{0.47\columnwidth}\raggedright
Requirement\strut
\end{minipage} & \begin{minipage}[t]{0.47\columnwidth}\raggedright
\url{http://icao.int/iwxxm/1.1/req/xsd-cloud-layer/cloud-base}

The XML element //iwxxm:CloudLayer/iwxxm:base shall indicate the height of the lowest level in the atmosphere that contains a perceptible quantity of cloud particles or the reason for not reporting the cloud base shall be expressed using the XML attribute //iwxxm:CloudLayer/iwxxm:base/@nilReason to indicate the appropriate nil-reason code.

If a nil-reason code is provided, the XML attributes //iwxxm:CloudLayer/iwxxm:base/@xsi:nil and //iwxxm:CloudLayer/iwxxm:base/@uom shall have the values ``true'' and ``N/A'', respectively.\strut
\end{minipage}\tabularnewline
\begin{minipage}[t]{0.47\columnwidth}\raggedright
Requirement\strut
\end{minipage} & \begin{minipage}[t]{0.47\columnwidth}\raggedright
\url{http://icao.int/iwxxm/1.1/req/xsd-cloud-layer/cloud-base-unit-of-measure}

If the cloud base is reported, then the vertical distance shall be expressed in metres or feet. The unit of measure shall be indicated using the XML attribute //iwxxm:CloudLayer/iwxxm:base/@uom with value ``m'' (metres) or ``{[}ft\_i{]}'' (feet).\strut
\end{minipage}\tabularnewline
\begin{minipage}[t]{0.47\columnwidth}\raggedright
Requirement\strut
\end{minipage} & \begin{minipage}[t]{0.47\columnwidth}\raggedright
\url{http://icao.int/iwxxm/1.1/req/xsd-cloud-layer/cloud-type-code}

If cloud type is reported, the value of XML attribute //iwxxm:CloudLayer/iwxxm:cloudType/@xlink:href shall be the URI of the valid cloud type from Code table~D-9: Significant convective cloud type.\strut
\end{minipage}\tabularnewline
\begin{minipage}[t]{0.47\columnwidth}\raggedright
Recommendation\strut
\end{minipage} & \begin{minipage}[t]{0.47\columnwidth}\raggedright
\url{http://icao.int/iwxxm/1.1/req/xsd-cloud-layer/cloud-type}

If reporting observed cloud, then the XML element //iwxxm:CloudLayer/iwxxm:cloudType should be used to report the most cloud of operational significance type in the layer of cloud.\strut
\end{minipage}\tabularnewline
\begin{minipage}[t]{0.47\columnwidth}\raggedright
Recommendation\strut
\end{minipage} & \begin{minipage}[t]{0.47\columnwidth}\raggedright
\url{http://icao.int/iwxxm/1.1/req/xsd-cloud-layer/nil-significant-cloud}

If no cloud of operational significance is reported, then the value of XML attribute\\
//iwxxm:CloudLayer/iwxxm:amount/@nilReason should be set to \url{http://codes.wmo.int/common/nil/nothingOfOperationalSignificance}.

If reporting observed cloud, then the value of XML attribute //iwxxm:CloudLayer/iwxxm:cloudType/@nilReason should also be set to \url{http://codes.wmo.int/common/nil/nothingOfOperationalSignificance}.\strut
\end{minipage}\tabularnewline
\bottomrule
\end{longtable}

Notes:

1. Cloud of operational significance includes cloud below 1~500~metres or the highest minimum sector altitude, whichever is greater, and cumulonimbus whenever present.

2. Code table~D-1 provides a set of nil-reason codes and is published at \url{http://codes.wmo.int/common/nil}.

3. Units of measurement are specified in accordance with 1.9 above.

4. Code table~D-8 is published online at \url{http://codes.wmo.int/49-2/CloudAmountReportedAtAerodrome}.

5. Code table~D-9 is published online at \url{http://codes.wmo.int/49-2/SigConvectiveCloudType}.

205-15-Ext.5 Requirements class: Aerodrome cloud forecast

205-15-Ext.5.1 This requirements class is used to describe forecast cloud conditions at an aerodrome. The class is targeted at providing a basic description of the forecast cloud conditions as required for civil aviation purposes.

Notes:

1. Representations providing more detailed information may be used if required.

2. The requirements for reporting forecast cloud conditions are specified in the \emph{Technical Regulations} (WMO-No.~49), Volume~II, Part~II, Appendix~5, 2.2.5 and 1.2.4.

205-15-Ext.5.2 XML elements describing forecast cloud conditions shall conform to all requirements specified in Table~205-15-Ext.5.

205-15-Ext.5.3 XML elements describing forecast cloud conditions shall conform to all requirements of all relevant dependencies specified in Table~205-15-Ext.5.

Table~205-15-Ext.5. Requirements class xsd-aerodrome-cloud-forecast

\begin{longtable}[]{@{}ll@{}}
\toprule
Requirements class &\tabularnewline
\midrule
\endhead
\url{http://icao.int/iwxxm/1.1/req/xsd-aerodrome-cloud-forecast} &\tabularnewline
Target type & Data instance\tabularnewline
Name & Aerodrome cloud forecast\tabularnewline
Dependency & \url{http://icao.int/iwxxm/1.1/req/xsd-cloud-layer}, 205-15-Ext.4\tabularnewline
\begin{minipage}[t]{0.47\columnwidth}\raggedright
Requirement\strut
\end{minipage} & \begin{minipage}[t]{0.47\columnwidth}\raggedright
\url{http://icao.int/iwxxm/1.1/req/xsd-aerodrome-cloud-forecast/valid}

The content model of this element shall have a value that matches the content model of iwxxm:AerodromeCloudForecast.\strut
\end{minipage}\tabularnewline
\begin{minipage}[t]{0.47\columnwidth}\raggedright
Requirement\strut
\end{minipage} & \begin{minipage}[t]{0.47\columnwidth}\raggedright
\url{http://icao.int/iwxxm/1.1/req/xsd-aerodrome-cloud-forecast/vertical-visibility}

When cloud of operational significance is forecast, then the XML element //iwxxm:AerodromeCloudForecast/iwxxm:verticalVisibility shall be used to report the vertical visual range.\strut
\end{minipage}\tabularnewline
\begin{minipage}[t]{0.47\columnwidth}\raggedright
Requirement\strut
\end{minipage} & \begin{minipage}[t]{0.47\columnwidth}\raggedright
\url{http://icao.int/iwxxm/1.1/req/xsd-aerodrome-cloud-forecast/vertical-visibility-unit-of-measure}

If the vertical visibility is reported, then the vertical distance shall be expressed in metres or feet. The unit of measure shall be indicated using the XML attribute\\
//iwxxm:AerodromeCloudForecast/iwxxm:verticalVisibility/@uom with value ``m'' (metres) or ``{[}ft\_i{]}'' (feet).\strut
\end{minipage}\tabularnewline
\begin{minipage}[t]{0.47\columnwidth}\raggedright
Requirement\strut
\end{minipage} & \begin{minipage}[t]{0.47\columnwidth}\raggedright
\url{http://icao.int/iwxxm/1.1/req/xsd-aerodrome-cloud-forecast/cloud-layers}

When cloud of operational significance is forecast, then the XML element //iwxxm:AerodromeCloudForecast/iwxxm:layer, containing a valid child element //iwxxm:AerodromeCloudForecast/iwxxm:layer/iwxxm:CloudLayer, shall be used to describe each cloud layer.\strut
\end{minipage}\tabularnewline
\begin{minipage}[t]{0.47\columnwidth}\raggedright
Requirement\strut
\end{minipage} & \begin{minipage}[t]{0.47\columnwidth}\raggedright
\url{http://icao.int/iwxxm/1.1/req/xsd-aerodrome-cloud-forecast/number-of-cloud-layers}

No more than four cloud layers shall be reported. If more than four significant cloud layers are forecast, then the four most significant cloud layers with respect to aviation operations shall be prioritized.\strut
\end{minipage}\tabularnewline
\bottomrule
\end{longtable}

Notes:

1. Cloud of operational significance includes cloud below 1~500 metres or the highest minimum sector altitude, whichever is greater, and cumulonimbus whenever present.

2. Vertical visibility is defined as the vertical visual range into an obscuring medium.

3. Units of measurement are specified in accordance with 1.9 above.

205-15-Ext.6 Requirements class: Aerodrome runway state

205-15-Ext.6.1 This requirements class is used to describe the observed runway state.

Notes:

1. Representations providing more detailed information may be used if required.

2. The requirements for reporting runway state are specified in the \emph{Technical Regulations} (WMO-No.~49), Volume~II, Part~II, Appendix~3, 4.8.1.5.

205-15-Ext.6.2 XML elements describing observed runway state shall conform to all requirements specified in Table~205-15-Ext.6.

205-15-Ext.6.3 XML elements describing observed runway state shall conform to all requirements of all relevant dependencies specified in Table~205-15-Ext.6.

Table~205-15-Ext.6. Requirements class xsd-aerodrome-runway-state

\begin{longtable}[]{@{}ll@{}}
\toprule
Requirements class &\tabularnewline
\midrule
\endhead
\url{http://icao.int/iwxxm/1.1/req/xsd-aerodrome-runway-state} &\tabularnewline
Target type & Data instance\tabularnewline
Name & Aerodrome runway state\tabularnewline
Dependency & \url{http://icao.int/saf/1.1/req/xsd-runway-direction}, 204-15-Ext.6\tabularnewline
\begin{minipage}[t]{0.47\columnwidth}\raggedright
Requirement\strut
\end{minipage} & \begin{minipage}[t]{0.47\columnwidth}\raggedright
\url{http://icao.int/iwxxm/1.1/req/xsd-aerodrome-runway-state/valid}

The content model of this element shall have a value that matches the content model of iwxxm:AerodromeRunwayState.\strut
\end{minipage}\tabularnewline
\begin{minipage}[t]{0.47\columnwidth}\raggedright
Requirement\strut
\end{minipage} & \begin{minipage}[t]{0.47\columnwidth}\raggedright
\url{http://icao.int/iwxxm/1.1/req/xsd-aerodrome-runway-state/applicable-runway}

If XML attribute //iwxxm:AerodromeRunwayState/@allRunways is absent or has value ``false'', then XML element //iwxxm:AerodromeRunwayState/iwxxm:runway, with valid child element //iwxxm:AerodromeRunwayState/iwxxm:runway/saf:RunwayDirection, shall be used to indicate the runway direction to which these conditions apply.\strut
\end{minipage}\tabularnewline
\begin{minipage}[t]{0.47\columnwidth}\raggedright
Requirement\strut
\end{minipage} & \begin{minipage}[t]{0.47\columnwidth}\raggedright
\url{http://icao.int/iwxxm/1.1/req/xsd-aerodrome-runway-state/all-runways}

If XML attribute //iwxxm:AerodromeRunwayState/@allRunways has value ``true'', then XML element //iwxxm:AerodromeRunwayState/iwxxm:runway shall be absent.\strut
\end{minipage}\tabularnewline
\begin{minipage}[t]{0.47\columnwidth}\raggedright
Requirement\strut
\end{minipage} & \begin{minipage}[t]{0.47\columnwidth}\raggedright
\url{http://icao.int/iwxxm/1.1/req/xsd-aerodrome-runway-state/snow-closure}

If the aerodrome is closed due to an extreme deposit of snow, XML attribute\\
//iwxxm:AerodromeRunwayState/@snowClosure shall have the value ``true''.\strut
\end{minipage}\tabularnewline
\begin{minipage}[t]{0.47\columnwidth}\raggedright
Requirement\strut
\end{minipage} & \begin{minipage}[t]{0.47\columnwidth}\raggedright
\url{http://icao.int/iwxxm/1.1/req/xsd-aerodrome-runway-state/cleared}

If the runway has been cleared of meteorological deposits, then XML attribute //iwxxm:AerodromeRunwayState/@cleared shall have the value ``true'' and XML elements //iwxxm:AerodromeRunwayState/iwxxm:depositType, //iwxxm:AerodromeRunwayState/iwxxm:contamination, //iwxxm:AerodromeRunwayState/iwxxm:depthOfDeposit and //iwxxm:AerodromeRunwayState/iwxxm:estimatedSurfaceFriction shall be absent.\strut
\end{minipage}\tabularnewline
\begin{minipage}[t]{0.47\columnwidth}\raggedright
Requirement\strut
\end{minipage} & \begin{minipage}[t]{0.47\columnwidth}\raggedright
\url{http://icao.int/iwxxm/1.1/req/xsd-aerodrome-runway-state/surface-friction-estimate}

If reported, the estimated surface friction shall be stated using the XML element\\
//iwxxm:AerodromeRunwayState/iwxxm:estimatedSurfaceFriction and shall have numeric value greater than 0.0 and less than or equal to 0.9.\strut
\end{minipage}\tabularnewline
\begin{minipage}[t]{0.47\columnwidth}\raggedright
Requirement\strut
\end{minipage} & \begin{minipage}[t]{0.47\columnwidth}\raggedright
\url{http://icao.int/iwxxm/1.1/req/xsd-aerodrome-runway-state/surface-friction-estimate-unit-of-measure}

If reported, the estimated surface friction shall be expressed as a unitless ratio with the value of XML attribute //iwxxm:AerodromeRunwayState/iwxxm:estimatedSurfaceFriction/@uom specified as ``\url{http://www.opengis.net/def/uom/OGC/1.0/unity}''.\strut
\end{minipage}\tabularnewline
\begin{minipage}[t]{0.47\columnwidth}\raggedright
Requirement\strut
\end{minipage} & \begin{minipage}[t]{0.47\columnwidth}\raggedright
\url{http://icao.int/iwxxm/1.1/req/xsd-aerodrome-runway-state/unreliable-surface-friction-estimate}

If the surface friction estimate for the runway is considered to be unreliable, then XML attribute //iwxxm:AerodromeRunwayState/\\
@estimatedSurfaceFrictionUnreliable shall have the value ``true''.\strut
\end{minipage}\tabularnewline
\begin{minipage}[t]{0.47\columnwidth}\raggedright
Requirement\strut
\end{minipage} & \begin{minipage}[t]{0.47\columnwidth}\raggedright
\url{http://icao.int/iwxxm/1.1/req/xsd-aerodrome-runway-state/unreliable-surface-friction-estimate-true}

If XML attribute //iwxxm:AerodromeRunwayState/\\
@estimatedSurfaceFrictionUnreliable has value ``true'', then XML element //iwxxm:AerodromeRunwayState/iwxxm:estimatedSurfaceFriction shall be absent.\strut
\end{minipage}\tabularnewline
\begin{minipage}[t]{0.47\columnwidth}\raggedright
Requirement\strut
\end{minipage} & \begin{minipage}[t]{0.47\columnwidth}\raggedright
\url{http://icao.int/iwxxm/1.1/req/xsd-aerodrome-runway-state/deposit-type-code}

If deposit type is reported, then the value of XML attribute //iwxxm:AerodromeRunwayState/iwxxm:depositType/@xlink:href shall be the URI of the valid term from Volume~I.2, FM 94 BUFR, Code table 0 20 086: Runway deposits.\strut
\end{minipage}\tabularnewline
\begin{minipage}[t]{0.47\columnwidth}\raggedright
Requirement\strut
\end{minipage} & \begin{minipage}[t]{0.47\columnwidth}\raggedright
\url{http://icao.int/iwxxm/1.1/req/xsd-aerodrome-runway-state/contamination-code}

If runway contamination is reported, then the value of XML attribute //iwxxm:AerodromeRunwayState/iwxxm:contamination/@xlink:href shall be the URI of the valid term from Volume~I.2, FM 94 BUFR, Code table 0 20 087: Runway contamination.\strut
\end{minipage}\tabularnewline
\begin{minipage}[t]{0.47\columnwidth}\raggedright
Recommendation\strut
\end{minipage} & \begin{minipage}[t]{0.47\columnwidth}\raggedright
\url{http://icao.int/iwxxm/1.1/req/xsd-aerodrome-runway-state/snow-closure-affects-all-runways}

If XML attribute //iwxxm:AerodromeRunwayState/@snowClosure has value ``true'', then XML //iwxxm:AerodromeRunwayState/@allRunways should also have value ``true''; snow closure affects all runways at an aerodrome.\strut
\end{minipage}\tabularnewline
\begin{minipage}[t]{0.47\columnwidth}\raggedright
Recommendation\strut
\end{minipage} & \begin{minipage}[t]{0.47\columnwidth}\raggedright
\url{http://icao.int/iwxxm/1.1/req/xsd-aerodrome-runway-state/deposit-depth-unit-of-measure}

If reported, the depth of deposit should be expressed in millimetres, with the value of XML attribute //iwxxm:AerodromeRunwayState/iwxxm:depthOfDeposit/@uom specified as ``mm''.\strut
\end{minipage}\tabularnewline
\bottomrule
\end{longtable}

Notes:

1. For convenience, FM~94 BUFR, Code table 0 20 086 from Volume~I.2, is published online at \url{http://codes.wmo.int/bufr4/codeflag/0-20-086}.

2. Runway contamination is expressed as a percentage of the total runway area that is contaminated according to a predefined set of categories: less than 10\%, between 11\% and 25\%, between 25\% and 50\% and more than 50\%. These categories are listed in Volume~I.2, FM~94 BUFR, Code table~0 20 087: Runway contamination. For convenience, this code table is published online at \url{http://codes.wmo.int/bufr4/codeflag/0-20-087}.

3. Units of measurement are specified in accordance with 1.9 above.

205-15-Ext.7 Requirements class: Aerodrome wind shear

205-15-Ext.7.1 This requirements class is used to describe the aerodrome wind shear. The class is targeted at providing a basic description of the wind shear as required for civil aviation purposes -- currently limited to indicating whether a wind shear threshold has been exceeded.

Notes:

1. The information on wind shear includes, but is not necessarily limited to, wind shear of a non-transitory nature such as might be associated with low-level temperature inversions or local topography.

2. Representations providing more detailed information may be used if required.

3. The requirements for reporting aerodrome wind shear are specified in the \emph{Technical Regulations} (WMO-No.~49), Volume~II, Part~II, Appendix~3, 4.8.1.4.

205-15-Ext.7.2 XML elements describing aerodrome wind shear shall conform to all requirements specified in Table~205-15-Ext.7.

205-15-Ext.7.3 XML elements describing aerodrome wind shear shall conform to all requirements of all relevant dependencies specified in Table~205-15-Ext.7.

Table~205-15-Ext.7. Requirements class xsd-aerodrome-wind-shear

\begin{longtable}[]{@{}ll@{}}
\toprule
Requirements class &\tabularnewline
\midrule
\endhead
\url{http://icao.int/iwxxm/1.1/req/xsd-aerodrome-wind-shear} &\tabularnewline
Target type & Data instance\tabularnewline
Name & Aerodrome wind shear\tabularnewline
Dependency & \url{http://icao.int/saf/1.1/req/xsd-runway-direction}, 204-15-Ext.6\tabularnewline
\begin{minipage}[t]{0.47\columnwidth}\raggedright
Requirement\strut
\end{minipage} & \begin{minipage}[t]{0.47\columnwidth}\raggedright
\url{http://icao.int/iwxxm/1.1/req/xsd-aerodrome-wind-shear/valid}

The content model of this element shall have a value that matches the content model of iwxxm:AerodromeWindShear.\strut
\end{minipage}\tabularnewline
\begin{minipage}[t]{0.47\columnwidth}\raggedright
Requirement\strut
\end{minipage} & \begin{minipage}[t]{0.47\columnwidth}\raggedright
\url{http://icao.int/iwxxm/1.1/req/xsd-aerodrome-wind-shear/applicable-runways}

If XML attribute //iwxxm:AerodromeWindShear/@allRunways is absent or has value ``false'', then one or more XML elements //iwxxm:AerodromeWindShear/iwxxm:runway, each with valid child element //iwxxm:AerodromeWindShear/iwxxm:runway/saf:RunwayDirection, shall be used to indicate the set of runway directions to which wind shear conditions apply.\strut
\end{minipage}\tabularnewline
\begin{minipage}[t]{0.47\columnwidth}\raggedright
Requirement\strut
\end{minipage} & \begin{minipage}[t]{0.47\columnwidth}\raggedright
\url{http://icao.int/iwxxm/1.1/req/xsd-aerodrome-wind-shear/all-runways}

If XML attribute //iwxxm:AerodromeWindShear/@allRunways has value ``true'', then XML element //iwxxm:AerodromeWindShear/iwxxm:runway shall be absent.\strut
\end{minipage}\tabularnewline
\bottomrule
\end{longtable}

205-15-Ext.8 Requirements class: Aerodrome observed clouds

205-15-Ext.8.1 This requirements class is used to describe observed cloud conditions at an aerodrome. The class is targeted at providing a basic description of the observed cloud conditions as required for civil aviation purposes.

Notes:

1. Representations providing more detailed information may be used if required.

2. The requirements for reporting observed cloud conditions are specified in the \emph{Technical Regulations} (WMO-No.~49), Volume~II, Part~II, Appendix~3, 4.5.

205-15-Ext.8.2 XML elements describing observed cloud conditions shall conform to all requirements specified in Table~205-15-Ext.8.

205-15-Ext.8.3 XML elements describing observed cloud conditions shall conform to all requirements of all relevant dependencies specified in Table~205-15-Ext.8.

Table~205-15-Ext.8. Requirements class xsd-aerodrome-observed-clouds

\begin{longtable}[]{@{}ll@{}}
\toprule
Requirements class &\tabularnewline
\midrule
\endhead
\url{http://icao.int/iwxxm/1.1/req/xsd-aerodrome-observed-clouds} &\tabularnewline
Target type & Data instance\tabularnewline
Name & Aerodrome observed clouds\tabularnewline
Dependency & \url{http://icao.int/iwxxm/1.1/req/xsd-cloud-layer}, 205-15-Ext.4\tabularnewline
\begin{minipage}[t]{0.47\columnwidth}\raggedright
Requirement\strut
\end{minipage} & \begin{minipage}[t]{0.47\columnwidth}\raggedright
\url{http://icao.int/iwxxm/1.1/req/xsd-aerodrome-observed-clouds/valid}

The content model of this element shall have a value that matches the content model of iwxxm:AerodromeObservedClouds.\strut
\end{minipage}\tabularnewline
\begin{minipage}[t]{0.47\columnwidth}\raggedright
Requirement\strut
\end{minipage} & \begin{minipage}[t]{0.47\columnwidth}\raggedright
\url{http://icao.int/iwxxm/1.1/req/xsd-aerodrome-observed-clouds/amount-and-height-not-detectable-by-auto-system}

When an automatic observing system observes cumulonimbus clouds or towering cumulus clouds but the amount and height cannot be observed, the XML attribute\\
//iwxxm:AerodromeObservedClouds/@amountAndHeightUnobservableByAutoSystem shall have the value set to ``true'' and XML element //iwxxm:AerodromeObservedClouds/iwxxm:layer shall be absent.\strut
\end{minipage}\tabularnewline
\begin{minipage}[t]{0.47\columnwidth}\raggedright
Requirement\strut
\end{minipage} & \begin{minipage}[t]{0.47\columnwidth}\raggedright
\url{http://icao.int/iwxxm/1.1/req/xsd-aerodrome-observed-clouds/either-vertical-visibility-or-cloud-layers}

When vertical visibility is reported, cloud layers shall not be reported.

When cloud layers are reported, vertical visibility shall not be reported.\strut
\end{minipage}\tabularnewline
\begin{minipage}[t]{0.47\columnwidth}\raggedright
Requirement\strut
\end{minipage} & \begin{minipage}[t]{0.47\columnwidth}\raggedright
\url{http://icao.int/iwxxm/1.1/req/xsd-aerodrome-observed-clouds/vertical-visibility}

When cloud of operational significance is observed but the amount and height cannot be observed, then the XML element //iwxxm:AerodromeObservedClouds/iwxxm:verticalVisibility shall be used to report the vertical visibility.\strut
\end{minipage}\tabularnewline
\begin{minipage}[t]{0.47\columnwidth}\raggedright
Requirement\strut
\end{minipage} & \begin{minipage}[t]{0.47\columnwidth}\raggedright
\url{http://icao.int/iwxxm/1.1/req/xsd-aerodrome-observed-clouds/vertical-visibility-unit-of-measure}

If the vertical visibility is reported then the vertical distance shall be expressed in metres or feet. The unit of measure shall be indicated using the XML attribute\\
//iwxxm:AerodromeObservedClouds/iwxxm:verticalVisibility/@uom with value ``m'' (metres) or ``{[}ft\_i{]}'' (feet).\strut
\end{minipage}\tabularnewline
\begin{minipage}[t]{0.47\columnwidth}\raggedright
Requirement\strut
\end{minipage} & \begin{minipage}[t]{0.47\columnwidth}\raggedright
\url{http://icao.int/iwxxm/1.1/req/xsd-aerodrome-observed-clouds/cloud-layers}

When the amount and height of cloud of operational significance are observed, then the XML element //iwxxm:AerodromeObservedClouds/iwxxm:layer, containing a valid child element //iwxxm:AerodromeObservedClouds/iwxxm:layer/iwxxm:CloudLayer, shall be used to describe each cloud layer.\strut
\end{minipage}\tabularnewline
\begin{minipage}[t]{0.47\columnwidth}\raggedright
Requirement\strut
\end{minipage} & \begin{minipage}[t]{0.47\columnwidth}\raggedright
\href{http://icao.int/iwxxm/1.1/req/xsd-aerodrome-cloud-forecast/number-of-cloud-layers}{http://icao.int/iwxxm/1.1/req/xsd-aerodrome-observed-clouds/number-of-cloud-layers}

No more than four cloud layers shall be reported. If more than four significant cloud layers are observed, then the four most significant cloud layers with respect to aviation operations shall be prioritized.\strut
\end{minipage}\tabularnewline
\bottomrule
\end{longtable}

Notes:

1. Cloud of operational significance includes cloud below 1~500 metres or the highest minimum sector altitude, whichever is greater, and cumulonimbus whenever present.

2. Vertical visibility is defined as the vertical visual range into an obscuring medium.

3. Units of measurement are specified in accordance with 1.9 above.

205-15-Ext.9 Requirements class: Aerodrome runway visual range

205-15-Ext.9.1 This requirements class is used to describe runway visual range for a specific runway direction at an aerodrome.

Note: The requirements for reporting runway visual range are specified in the \emph{Technical Regulations} (WMO-No.~49), Volume~II, Part~II, Appendix~3, 4.3.

205-15-Ext.9.2 XML elements describing runway visual range shall conform to all requirements specified in Table~205-15-Ext.9.

205-15-Ext.9.3 XML elements describing runway visual range shall conform to all requirements of all relevant dependencies specified in Table~205-15-Ext.9.

Table~205-15-Ext.9. Requirements class xsd-aerodrome-runway-visual-range

\begin{longtable}[]{@{}ll@{}}
\toprule
Requirements class &\tabularnewline
\midrule
\endhead
\url{http://icao.int/iwxxm/1.1/req/xsd-aerodrome-runway-visual-range} &\tabularnewline
Target type & Data instance\tabularnewline
Name & Aerodrome runway visual range\tabularnewline
Dependency & \url{http://icao.int/saf/1.1/req/xsd-runway-direction}, 204-15-Ext.6\tabularnewline
\begin{minipage}[t]{0.47\columnwidth}\raggedright
Requirement\strut
\end{minipage} & \begin{minipage}[t]{0.47\columnwidth}\raggedright
\url{http://icao.int/iwxxm/1.1/req/xsd-aerodrome-runway-visual-range/valid}

The content model of this element shall have a value that matches the content model of iwxxm:RunwayVisualRange.\strut
\end{minipage}\tabularnewline
\begin{minipage}[t]{0.47\columnwidth}\raggedright
Requirement\strut
\end{minipage} & \begin{minipage}[t]{0.47\columnwidth}\raggedright
\url{http://icao.int/iwxxm/1.1/req/xsd-aerodrome-runway-visual-range/applicable-runway}

The XML element //iwxxm:AerodromeRunwayVisualRange/iwxxm:runway, with valid child element //iwxxm:AerodromeRunwayState/iwxxm:runway/saf:RunwayDirection, shall be used to indicate the runway direction to which these visual range conditions apply.\strut
\end{minipage}\tabularnewline
\begin{minipage}[t]{0.47\columnwidth}\raggedright
Requirement\strut
\end{minipage} & \begin{minipage}[t]{0.47\columnwidth}\raggedright
\url{http://icao.int/iwxxm/1.1/req/xsd-aerodrome-runway-visual-range/mean-rvr}

The XML element //iwxxm:AerodromeRunwayVisualRange/iwxxm:meanRVR shall be used to express the 10-minute average for observed runway visual range or, if a marked discontinuity in visual range occurs during the 10-minute period, the average runway visual range following that marked discontinuity.\strut
\end{minipage}\tabularnewline
\begin{minipage}[t]{0.47\columnwidth}\raggedright
Requirement\strut
\end{minipage} & \begin{minipage}[t]{0.47\columnwidth}\raggedright
\url{http://icao.int/iwxxm/1.1/req/xsd-aerodrome-runway-visual-range/mean-rvr-unit-of-measure}

The mean runway visual range shall be reported in metres. The unit of measure shall be indicated using the XML attribute //iwxxm:AerodromeRunwayVisualRange/iwxxm:meanRVR/@uom with value ``m''.\strut
\end{minipage}\tabularnewline
\begin{minipage}[t]{0.47\columnwidth}\raggedright
Requirement\strut
\end{minipage} & \begin{minipage}[t]{0.47\columnwidth}\raggedright
\url{http://icao.int/iwxxm/1.1/req/xsd-aerodrome-runway-visual-range/mean-rvr-exceeds-2000m}

If the mean runway visual range exceeds 2~000 metres, then the numeric value of XML element //iwxxm:AerodromeRunwayVisualRange/iwxxm:meanRVR shall be set to 2000 and the XML element //iwxxm:AerodromeRunwayVisualRange/iwxxm:meanRVROperator shall have the value ``ABOVE''.\strut
\end{minipage}\tabularnewline
\begin{minipage}[t]{0.47\columnwidth}\raggedright
Requirement\strut
\end{minipage} & \begin{minipage}[t]{0.47\columnwidth}\raggedright
\url{http://icao.int/iwxxm/1.1/req/xsd-aerodrome-runway-visual-range/mean-rvr-comparison-operator}

If present, the value of XML element //iwxxm:AerodromeRunwayVisualRange/iwxxm:meanRVROperator shall be one of the enumeration: ``ABOVE'' or ``BELOW''.\strut
\end{minipage}\tabularnewline
\begin{minipage}[t]{0.47\columnwidth}\raggedright
Requirement\strut
\end{minipage} & \begin{minipage}[t]{0.47\columnwidth}\raggedright
\url{http://icao.int/iwxxm/1.1/req/xsd-aerodrome-runway-visual-range/upward-or-downward-visual-range-tendency}

If the runway visual range values observed in the 10-minute period have shown a distinct tendency, such that the mean during the first 5~minutes varies by 100~metres or more when compared with the second 5~minutes, this shall be indicated using the XML element //iwxxm:AerodromeRunwayVisualRange/iwxxm:pastTendency with value ``UPWARD'' (visual range is increasing) or ``DOWNWARD'' (visual range is decreasing) as appropriate.\strut
\end{minipage}\tabularnewline
\begin{minipage}[t]{0.47\columnwidth}\raggedright
Recommendation\strut
\end{minipage} & \begin{minipage}[t]{0.47\columnwidth}\raggedright
\url{http://icao.int/iwxxm/1.1/req/xsd-aerodrome-runway-visual-range/no-change-in-visual-range-tendency}

If the runway visual range values observed in the 10-minute period have not shown a distinct tendency, this should be indicated using the XML element //iwxxm:AerodromeRunwayVisualRange/iwxxm:pastTendency with value ``NO\_CHANGE''.\strut
\end{minipage}\tabularnewline
\bottomrule
\end{longtable}

Notes:

1. Units of measurement are specified in accordance with 1.9 above.

2. The absence of XML element //iwxxm:AerodromeRunwayVisualRange/iwxxm:meanRVROperator indicates that the mean runway visual range has the numeric value reported.

3. The absence of XML element //iwxxm:AerodromeRunwayVisualRange/iwxxm:pastTendency indicates that no distinct tendency in visual range has been observed.

205-15-Ext.10 Requirements class: Aerodrome sea state

205-15-Ext.10.1 This requirements class is used to describe an aggregated set of sea-state conditions reported at an aerodrome.

Note: The requirements for reporting sea state are specified in the \emph{Technical Regulations} (WMO-No.~49), Volume~II, Part~II, Appendix~3, 4.8.1.5.

205-15-Ext.10.2 XML elements describing sea state shall conform to all requirements specified in Table~205-15-Ext.10.

205-15-Ext.10.3 XML elements describing sea state shall conform to all requirements of all relevant dependencies specified in Table~205-15-Ext.10.

Table~205-15-Ext.10. Requirements class xsd-aerodrome-sea-state

\begin{longtable}[]{@{}ll@{}}
\toprule
Requirements class &\tabularnewline
\midrule
\endhead
\url{http://icao.int/iwxxm/1.1/req/xsd-aerodrome-sea-state} &\tabularnewline
Target type & Data instance\tabularnewline
Name & Aerodrome sea state\tabularnewline
\begin{minipage}[t]{0.47\columnwidth}\raggedright
Requirement\strut
\end{minipage} & \begin{minipage}[t]{0.47\columnwidth}\raggedright
\url{http://icao.int/iwxxm/1.1/req/xsd-aerodrome-sea-state/valid}

The content model of this element shall have a value that matches the content model of iwxxm:AerodromeSeaState.\strut
\end{minipage}\tabularnewline
\begin{minipage}[t]{0.47\columnwidth}\raggedright
Requirement\strut
\end{minipage} & \begin{minipage}[t]{0.47\columnwidth}\raggedright
\url{http://icao.int/iwxxm/1.1/req/xsd-aerodrome-sea-state/sea-surface-temperature}

The sea-surface temperature shall be reported in Celsius (°C) using the XML element //iwxxm:AerdromeSeaState/iwxxm:seaSurfaceTemperature. The value of the associated XML attribute //iwxxm:AerdromeSeaState/iwxxm:seaSurfaceTemperature/@uom shall be ``Cel''.\strut
\end{minipage}\tabularnewline
\begin{minipage}[t]{0.47\columnwidth}\raggedright
Requirement\strut
\end{minipage} & \begin{minipage}[t]{0.47\columnwidth}\raggedright
\url{http://icao.int/iwxxm/1.1/req/xsd-aerodrome-sea-state/either-significant-wave-height-or-sea-state}

When significant wave height is reported, sea state shall not be reported.

When sea state is reported, significant wave height shall not be reported.\strut
\end{minipage}\tabularnewline
\begin{minipage}[t]{0.47\columnwidth}\raggedright
Requirement\strut
\end{minipage} & \begin{minipage}[t]{0.47\columnwidth}\raggedright
\url{http://icao.int/iwxxm/1.1/req/xsd-aerodrome-sea-state/significant-wave-height}

If reported, the observed significant wave height shall be expressed using the XML element //iwxxm:AerodromeSeaState/iwxxm:significantWaveHeight.\strut
\end{minipage}\tabularnewline
\begin{minipage}[t]{0.47\columnwidth}\raggedright
Requirement\strut
\end{minipage} & \begin{minipage}[t]{0.47\columnwidth}\raggedright
\url{http://icao.int/iwxxm/1.1/req/xsd-aerodrome-sea-state/sea-state-code}

If sea state is reported, then the value of XML attribute //iwxxm:AerodromeSeaState/iwxxm:seaState/@xlink:href shall be the URI of the valid term from Volume~I.2, FM 94 BUFR, Code table 0 22 061: State of the sea.\strut
\end{minipage}\tabularnewline
\begin{minipage}[t]{0.47\columnwidth}\raggedright
Recommendation\strut
\end{minipage} & \begin{minipage}[t]{0.47\columnwidth}\raggedright
\url{http://icao.int/iwxxm/1.1/req/xsd-aerodrome-sea-state/significant-wave-height-unit-of-measure}

The significant wave height should be reported in metres. The unit of measure should be indicated using the XML attribute //iwxxm:AerodromeSeaState/iwxxm:significantWaveHeight/@uom with value~``m''.\strut
\end{minipage}\tabularnewline
\bottomrule
\end{longtable}

Notes:

1. Units of measurement are specified in accordance with 1.9 above.

2. The term sea-surface temperature is generally meant to be representative of the upper few metres of the ocean as opposed to the skin temperature.

3. For convenience, FM~94 BUFR, Code table~0~22~061 from Volume~I.2 is published online at \url{http://codes.wmo.int/bufr4/codeflag/0-22-061}.

205-15-Ext.11 Requirements class: Aerodrome horizontal visibility

205-15-Ext.11.1 This requirements class is used to describe the horizontal visibility conditions observed at an aerodrome.

Note: The requirements for reporting horizontal visibility are specified in the \emph{Technical Regulations} (WMO-No.~49), Volume~II, Part~II, Appendix~3, 4.2.

205-15-Ext.11.2 XML elements describing horizontal visibility shall conform to all requirements specified in Table~205-15-Ext.11.

205-15-Ext.11.3 XML elements describing horizontal visibility shall conform to all requirements of all relevant dependencies specified in Table~205-15-Ext.11.

Table~205-15-Ext.11. Requirements class xsd-aerodrome-horizontal-visibility

\begin{longtable}[]{@{}ll@{}}
\toprule
Requirements class &\tabularnewline
\midrule
\endhead
\url{http://icao.int/iwxxm/1.1/req/xsd-aerodrome-horizontal-visibility} &\tabularnewline
Target type & Data instance\tabularnewline
Name & Aerodrome horizontal visibility\tabularnewline
\begin{minipage}[t]{0.47\columnwidth}\raggedright
Requirement\strut
\end{minipage} & \begin{minipage}[t]{0.47\columnwidth}\raggedright
\url{http://icao.int/iwxxm/1.1/req/xsd-aerodrome-horizontal-visibility/valid}

The content model of this element shall have a value that matches the content model of iwxxm:AerodromeHorizontalVisibility.\strut
\end{minipage}\tabularnewline
\begin{minipage}[t]{0.47\columnwidth}\raggedright
Requirement\strut
\end{minipage} & \begin{minipage}[t]{0.47\columnwidth}\raggedright
\url{http://icao.int/iwxxm/1.1/req/xsd-aerodrome-horizontal-visibility/prevailing-visibility}

The prevailing visibility shall be stated using the XML element //iwxxm:AerodromeHorizontalVisibility/iwxxm:prevailingVisibility with the unit of measure metres, indicated using the XML attribute //iwxxm:AerodromeHorizontalVisibility/iwxxm:prevailingVisibility/@uom with value ``m''.\strut
\end{minipage}\tabularnewline
\begin{minipage}[t]{0.47\columnwidth}\raggedright
Requirement\strut
\end{minipage} & \begin{minipage}[t]{0.47\columnwidth}\raggedright
\url{http://icao.int/iwxxm/1.1/req/xsd-aerodrome-horizontal-visibility/prevailing-visibility-exceeds-10000m}

If the prevailing visibility exceeds 10~000 metres, then the numeric value of XML element //iwxxm:AerodromeHorizontalVisibility/iwxxm:prevailingVisibility shall be set to 10000 and the XML element //iwxxm:AerodromeHorizontalVisibility/iwxxm:prevailingVisibilityOperator shall have the value ``ABOVE''.\strut
\end{minipage}\tabularnewline
\begin{minipage}[t]{0.47\columnwidth}\raggedright
Requirement\strut
\end{minipage} & \begin{minipage}[t]{0.47\columnwidth}\raggedright
\url{http://icao.int/iwxxm/1.1/req/xsd-aerodrome-horizontal-visibility/prevailing-visibility-comparison-operator}

If present, the value of XML element //iwxxm:AerodromeHorizontalVisibility/iwxxm:prevailingVisibilityOperator shall be one of the enumeration: ``ABOVE'' or ``BELOW''.\strut
\end{minipage}\tabularnewline
\begin{minipage}[t]{0.47\columnwidth}\raggedright
Requirement\strut
\end{minipage} & \begin{minipage}[t]{0.47\columnwidth}\raggedright
\url{http://icao.int/iwxxm/1.1/req/xsd-aerodrome-horizontal-visibility/minimum-visibility}

If reported, the minimum visibility shall be expressed using XML element //iwxxm:AerodromeHorizontalVisibility/iwxxm:minimumVisibility with the unit of measure metres, indicated using the XML attribute //iwxxm:AerodromeHorizontalVisibility/iwxxm:minimumVisibility/@uom with value ``m''.\strut
\end{minipage}\tabularnewline
\begin{minipage}[t]{0.47\columnwidth}\raggedright
Requirement\strut
\end{minipage} & \begin{minipage}[t]{0.47\columnwidth}\raggedright
\url{http://icao.int/iwxxm/1.1/req/xsd-aerodrome-horizontal-visibility/minimum-visibility-direction}

If reported, the observed angle between true north and the direction of minimum visibility shall be expressed in degrees using XML element //iwxxm:AerodromeHorizontalVisibility/iwxxm:minimumVisibilityDirection, with the unit of measure indicated using the XML attribute //iwxxm:AerodromeHorizontalVisibility/iwxxm:minimumVisibilityDirection/@uom with value ``deg''.\strut
\end{minipage}\tabularnewline
\bottomrule
\end{longtable}

Notes:

1. Units of measurement are specified in accordance with 1.9 above.

2. Visibility for aeronautical purposes is defined as the greater of: (i) the greatest distance at which a black object of suitable dimensions, situated near the ground, can be seen and recognized when observed against a bright background; or (ii) the greatest distance at which lights in the vicinity of 1~000~candelas can be seen and identified against an unlit background.

3. Prevailing visibility is defined as the greatest visibility value observed which is reached within at least half the horizon circle or within at least half of the surface of the aerodrome. These areas could comprise contiguous or non-contiguous sectors.

4. The absence of XML element //iwxxm:AerodromeHorizontalVisibility/iwxxm:prevailingVisibilityOperator indicates that the prevailing visibility has the numeric value reported.

5. The conditions for reporting minimum visibility are that the visibility is not the same in different directions and (i)~when the lowest visibility is different from the prevailing visibility and less than 1~500~metres or less than 50\% of the prevailing visibility and less than 5~000~metres, or (ii) when the visibility is fluctuating rapidly and the prevailing visibility cannot be determined.

6. When reporting minimum visibility, the general direction of the minimum visibility in relation to the aerodrome should be reported unless the visibility is fluctuating rapidly.

7. The true north is the north point at which the meridian lines meet.

205-15-Ext.12 Requirements class: Aerodrome surface wind

205-15-Ext.12.1 This requirements class is used to describe the surface wind conditions observed at an aerodrome.

Note: The requirements for reporting surface wind conditions are specified in the \emph{Technical Regulations} (WMO-No.~49), Volume~II, Part~II, Appendix~3, 4.1.

205-15-Ext.12.2 XML elements describing surface wind conditions shall conform to all requirements specified in Table 205-15-Ext.12.

205-15-Ext.12.3 XML elements describing surface wind conditions shall conform to all requirements of all relevant dependencies specified in Table~205-15-Ext.12.

Table~205-15-Ext.12. Requirements class xsd-aerodrome-surface-wind

\begin{longtable}[]{@{}ll@{}}
\toprule
Requirements class &\tabularnewline
\midrule
\endhead
\url{http://icao.int/iwxxm/1.1/req/xsd-aerodrome-surface-wind} &\tabularnewline
Target type & Data instance\tabularnewline
Name & Aerodrome surface wind\tabularnewline
\begin{minipage}[t]{0.47\columnwidth}\raggedright
Requirement\strut
\end{minipage} & \begin{minipage}[t]{0.47\columnwidth}\raggedright
\url{http://icao.int/iwxxm/1.1/req/xsd-aerodrome-surface-wind/valid}

The content model of this element shall have a value that matches the content model of iwxxm:AerodromeSurfaceWind.\strut
\end{minipage}\tabularnewline
\begin{minipage}[t]{0.47\columnwidth}\raggedright
Requirement\strut
\end{minipage} & \begin{minipage}[t]{0.47\columnwidth}\raggedright
\url{http://icao.int/iwxxm/1.1/req/xsd-aerodrome-surface-wind/mean-wind-speed}

The mean wind speed shall be stated using the XML element //iwxxm:AerodromeSurfaceWind/iwxxm:meanWindSpeed, with the unit of measure metres per second, knots or kilometres per hour. The unit of measure shall be indicated using the XML attribute //iwxxm:AerodromeSurfaceWind/iwxxm:meanWindSpeed/@uom with value ``m/s'' (metres per second), ``{[}kn\_i{]}'' (knots) or ``km/h'' (kilometres per hour).\strut
\end{minipage}\tabularnewline
\begin{minipage}[t]{0.47\columnwidth}\raggedright
Requirement\strut
\end{minipage} & \begin{minipage}[t]{0.47\columnwidth}\raggedright
\url{http://icao.int/iwxxm/1.1/req/xsd-aerodrome-surface-wind/variable-wind-direction}

If the wind direction is variable, then the XML attribute //iwxxm:AerodromeSurfaceWind/@variableDirection shall have the value ``true'' and XML element //iwxxm:AerodromeSurfaceWind/iwxxm:meanWindDirection shall be absent.\strut
\end{minipage}\tabularnewline
\begin{minipage}[t]{0.47\columnwidth}\raggedright
Requirement\strut
\end{minipage} & \begin{minipage}[t]{0.47\columnwidth}\raggedright
\url{http://icao.int/iwxxm/1.1/req/xsd-aerodrome-surface-wind/steady-wind-direction}

If the wind direction is not variable, then:

(i) The observed angle between true north and the mean direction from which the wind is blowing shall be expressed using XML element //iwxxm:AerodromeSurfaceWind/iwxxm:meanWindDirection, with the unit of measure indicated using the XML attribute //iwxxm:AerodromeSurfaceWind/iwxxm:meanWindDirection/@uom with value ``deg'';

(ii) The XML attribute //iwxxm:AerodromeSurfaceWind/@variableDirection shall be absent or have the value ``false''.\strut
\end{minipage}\tabularnewline
\begin{minipage}[t]{0.47\columnwidth}\raggedright
Requirement\strut
\end{minipage} & \begin{minipage}[t]{0.47\columnwidth}\raggedright
\url{http://icao.int/iwxxm/1.1/req/xsd-aerodrome-surface-wind/extreme-wind-direction}

If the extremes of wind direction variability are reported, then:

(i) The observed angle between true north and extreme clockwise direction from which the wind is blowing shall be expressed using XML element //iwxxm:AerodromeSurfaceWind/iwxxm:extremeClockWiseWindDirection;

(ii) The observed angle between true north and extreme counterclockwise direction from which the wind is blowing shall be expressed using XML element //iwxxm:AerodromeSurfaceWind/iwxxm:extremeCounterClockWiseWindDirection;

(iii) The unit of measure for each extreme wind direction shall be indicated using the XML attribute @uom with value ``deg''.\strut
\end{minipage}\tabularnewline
\begin{minipage}[t]{0.47\columnwidth}\raggedright
Requirement\strut
\end{minipage} & \begin{minipage}[t]{0.47\columnwidth}\raggedright
\url{http://icao.int/iwxxm/1.1/req/xsd-aerodrome-surface-wind/gust-speed}

If reported, the observed gust speed shall be stated using the XML element //iwxxm:AerodromeSurfaceWind/iwxxm:windGustSpeed and expressed in metres per second, knots or kilometres per hour.

The unit of measure shall be indicated using the XML attribute //iwxxm:AerodromeSurfaceWind/iwxxm:windGustSpeed/@uom with value ``m/s'' (metres per second), ``{[}kn\_i{]}'' (knots) or ``km/h'' (kilometres per hour).\strut
\end{minipage}\tabularnewline
\bottomrule
\end{longtable}

Notes:

1. The mean wind speed is the average wind speed observed over the previous 10~minutes.

2. The gust speed is the maximum wind speed observed over the previous 10~minutes.

3. Wind direction is reported as variable (VRB) if, during the 10-minute observation of mean wind speed, the variation of wind direction is (i) 180~degrees or more, or (ii) 60~degrees or more when the wind speed is less than 1.5~metres per second (3~knots).

4. Extreme directional variations of wind are reported if, during the 10-minute observation of mean wind speed, the variation of wind direction is 60 degrees or more and less than 180~degrees and the wind speed is 1.5~metres per second (3~knots) or more.

5. The absence of XML attribute //iwxxm:AerodromeSurfaceWind/@variableDirection implies a ``false'' value; for example, the wind direction is not variable.

6. Units of measurement are specified in accordance with 1.9 above.

7. The true north is the north point at which the meridian lines meet.

205-15-Ext.13 Requirements class: Meteorological aerodrome observation record

205-15-Ext.13.1 This requirements class is used to describe the aggregated set of meteorological conditions observed at an aerodrome.

205-15-Ext.13.2 XML elements describing the set of meteorological conditions observed at an aerodrome shall conform to all requirements specified in Table~205-15-Ext.13.

205-15-Ext.13.3 XML elements describing the set of meteorological conditions observed at an aerodrome shall conform to all requirements of all relevant dependencies specified in\\
Table~205-15-Ext.13.

Table~205-15-Ext.13. Requirements class xsd-meteorological-aerodrome-observation-record

\begin{longtable}[]{@{}ll@{}}
\toprule
Requirements class &\tabularnewline
\midrule
\endhead
\url{http://icao.int/iwxxm/1.1/req/xsd-meteorological-aerodrome-observation-record} &\tabularnewline
Target type & Data instance\tabularnewline
Name & Meteorological aerodrome observation record\tabularnewline
Dependency & \url{http://icao.int/iwxxm/1.1/req/xsd-aerodrome-runway-state}, 205-15-Ext.6\tabularnewline
Dependency & \url{http://icao.int/iwxxm/1.1/req/xsd-aerodrome-wind-shear}, 205-15-Ext.7\tabularnewline
Dependency & \url{http://icao.int/iwxxm/1.1/req/xsd-aerodrome-observed-clouds}, 205-15-Ext.8\tabularnewline
Dependency & \url{http://icao.int/iwxxm/1.1/req/xsd-aerodrome-runway-visual-range}, 205-15-Ext.9\tabularnewline
Dependency & \url{http://icao.int/iwxxm/1.1/req/xsd-aerodrome-sea-state}, 205-15-Ext.10\tabularnewline
Dependency & \url{http://icao.int/iwxxm/1.1/req/xsd-aerodrome-horizontal-visibility}, 205-15-Ext.11\tabularnewline
Dependency & \url{http://icao.int/iwxxm/1.1/req/xsd-aerodrome-surface-wind}, 205-15-Ext.12\tabularnewline
\begin{minipage}[t]{0.47\columnwidth}\raggedright
Requirement\strut
\end{minipage} & \begin{minipage}[t]{0.47\columnwidth}\raggedright
\url{http://icao.int/iwxxm/1.1/req/xsd-meteorological-aerodrome-observation-record/valid}

The content model of this element shall have a value that matches the content model of iwxxm:MeteorologicalAerodromeObservationRecord.\strut
\end{minipage}\tabularnewline
\begin{minipage}[t]{0.47\columnwidth}\raggedright
Requirement\strut
\end{minipage} & \begin{minipage}[t]{0.47\columnwidth}\raggedright
\url{http://icao.int/iwxxm/1.1/req/xsd-meteorological-aerodrome-observation-record/cavok}

If the conditions associated with CAVOK are observed, then:

(i) The XML attribute //iwxxm:MeteorologicalAerodromeObservationRecord/\\
@cloudAndVisibilityOK shall have the value ``true''; and

(ii) The following XML elements shall be absent:\\
//iwxxm:MeteorologicalAerodromeObservationRecord/iwxxm:visibility,\\
//iwxxm:MeteorologicalAerodromeObservationRecord/iwxxm:rvr,\\
//iwxxm:MeteorologicalAerodromeObservationRecord/iwxxm:presentWeather and //iwxxm:MeteorologicalAerodromeObservationRecord/iwxxm:cloud.\strut
\end{minipage}\tabularnewline
\begin{minipage}[t]{0.47\columnwidth}\raggedright
Requirement\strut
\end{minipage} & \begin{minipage}[t]{0.47\columnwidth}\raggedright
\url{http://icao.int/iwxxm/1.1/req/xsd-meteorological-aerodrome-observation-record/air-temperature}

The air temperature observed at the aerodrome shall be reported in Celsius (°C) using the XML element //iwxxm:MeteorologicalAerodromeObservationRecord/iwxxm:airTemperature. The value of the associated XML attribute @uom shall be ``Cel''.\strut
\end{minipage}\tabularnewline
\begin{minipage}[t]{0.47\columnwidth}\raggedright
Requirement\strut
\end{minipage} & \begin{minipage}[t]{0.47\columnwidth}\raggedright
\url{http://icao.int/iwxxm/1.1/req/xsd-meteorological-aerodrome-observation-record/dew-point-temperature}

The dewpoint temperature observed at the aerodrome shall be reported in Celsius (°C) using the XML element //iwxxm:MeteorologicalAerodromeObservationRecord/iwxxm:dewpointTemperature. The value of the associated XML attribute @uom shall be ``Cel''.\strut
\end{minipage}\tabularnewline
\begin{minipage}[t]{0.47\columnwidth}\raggedright
Requirement\strut
\end{minipage} & \begin{minipage}[t]{0.47\columnwidth}\raggedright
\url{http://icao.int/iwxxm/1.1/req/xsd-meteorological-aerodrome-observation-record/qnh}

The atmospheric pressure, known as QNH, observed at the aerodrome shall be reported in hectopascals (hPa) using the XML element //iwxxm:MeteorologicalAerodromeObservationRecord/iwxxm:qnh. The value of the associated XML attribute @uom shall be ``hPa''.\strut
\end{minipage}\tabularnewline
\begin{minipage}[t]{0.47\columnwidth}\raggedright
Requirement\strut
\end{minipage} & \begin{minipage}[t]{0.47\columnwidth}\raggedright
\url{http://icao.int/iwxxm/1.1/req/xsd-meteorological-aerodrome-observation-record/present-weather}

If present weather is reported, the value of XML attribute //iwxxm:MeteorologicalAerodromeObservationRecord/iwxxm:presentWeather/@xlink:href shall be the URI of a valid weather phenomenon code from Code table~D-7: Aerodrome present or forecast weather.\strut
\end{minipage}\tabularnewline
\begin{minipage}[t]{0.47\columnwidth}\raggedright
Requirement\strut
\end{minipage} & \begin{minipage}[t]{0.47\columnwidth}\raggedright
\url{http://icao.int/iwxxm/1.1/req/xsd-meteorological-aerodrome-observation-record/number-of-present-weather-codes}

No more than three present weather codes shall be reported.\strut
\end{minipage}\tabularnewline
\begin{minipage}[t]{0.47\columnwidth}\raggedright
Requirement\strut
\end{minipage} & \begin{minipage}[t]{0.47\columnwidth}\raggedright
\url{http://icao.int/iwxxm/1.1/req/xsd-meteorological-aerodrome-observation-record/recent-weather}

If recent weather is reported, the value of XML attribute //iwxxm:MeteorologicalAerodromeObservationRecord/iwxxm:recentWeather/@xlink:href shall be the URI of a valid weather phenomenon code from Code table~D-6: Aerodrome recent weather.\strut
\end{minipage}\tabularnewline
\begin{minipage}[t]{0.47\columnwidth}\raggedright
Requirement\strut
\end{minipage} & \begin{minipage}[t]{0.47\columnwidth}\raggedright
\url{http://icao.int/iwxxm/1.1/req/xsd-meteorological-aerodrome-observation-record/number-of-recent-weather-codes}

No more than three recent weather codes shall be reported.\strut
\end{minipage}\tabularnewline
\begin{minipage}[t]{0.47\columnwidth}\raggedright
Requirement\strut
\end{minipage} & \begin{minipage}[t]{0.47\columnwidth}\raggedright
\url{http://icao.int/iwxxm/1.1/req/xsd-meteorological-aerodrome-observation-record/surface-wind}

Surface wind conditions observed at the aerodrome shall be reported using the XML element //iwxxm:MeteorologicalAerodromeObservationRecord/iwxxm:surfaceWind containing a valid child element iwxxm:AerodromeSurfaceWind.\strut
\end{minipage}\tabularnewline
\begin{minipage}[t]{0.47\columnwidth}\raggedright
Requirement\strut
\end{minipage} & \begin{minipage}[t]{0.47\columnwidth}\raggedright
\url{http://icao.int/iwxxm/1.1/req/xsd-meteorological-aerodrome-observation-record/runway-state}

If reported, the surface conditions for a given runway direction shall be expressed using the XML element //iwxxm:MeteorologicalAerodromeObservationRecord/iwxxm:runwayState containing a valid child element iwxxm:AerodromeRunwayState.\strut
\end{minipage}\tabularnewline
\begin{minipage}[t]{0.47\columnwidth}\raggedright
Requirement\strut
\end{minipage} & \begin{minipage}[t]{0.47\columnwidth}\raggedright
\url{http://icao.int/iwxxm/1.1/req/xsd-meteorological-aerodrome-observation-record/wind-shear}

If reported, the wind shear conditions for the aerodrome shall be expressed using the XML element //iwxxm:MeteorologicalAerodromeObservationRecord/iwxxm:windShear containing a valid child element iwxxm:AerodromeWindShear.\strut
\end{minipage}\tabularnewline
\begin{minipage}[t]{0.47\columnwidth}\raggedright
Requirement\strut
\end{minipage} & \begin{minipage}[t]{0.47\columnwidth}\raggedright
\url{http://icao.int/iwxxm/1.1/req/xsd-meteorological-aerodrome-observation-record/cloud}

If reported, the cloud conditions observed at the aerodrome shall be expressed using the XML element //iwxxm:MeteorologicalAerodromeObservationRecord/iwxxm:cloud containing a valid child element iwxxm:AerodromeObservedClouds.\strut
\end{minipage}\tabularnewline
\begin{minipage}[t]{0.47\columnwidth}\raggedright
Requirement\strut
\end{minipage} & \begin{minipage}[t]{0.47\columnwidth}\raggedright
\url{http://icao.int/iwxxm/1.1/req/xsd-meteorological-aerodrome-observation-record/runway-visual-range}

If reported, the visual range conditions for a given runway direction shall be expressed using the XML element //iwxxm:MeteorologicalAerodromeObservationRecord/iwxxm:rvr containing a valid child element iwxxm:AerodromeRunwayVisualRange.\strut
\end{minipage}\tabularnewline
\begin{minipage}[t]{0.47\columnwidth}\raggedright
Requirement\strut
\end{minipage} & \begin{minipage}[t]{0.47\columnwidth}\raggedright
\url{http://icao.int/iwxxm/1.1/req/xsd-meteorological-aerodrome-observation-record/number-of-rvr-groups}

Visual range conditions shall be reported for no more than four runway directions.\strut
\end{minipage}\tabularnewline
\begin{minipage}[t]{0.47\columnwidth}\raggedright
Requirement\strut
\end{minipage} & \begin{minipage}[t]{0.47\columnwidth}\raggedright
\url{http://icao.int/iwxxm/1.1/req/xsd-meteorological-aerodrome-observation-record/sea-state}

If reported, the sea-state conditions observed at the aerodrome shall be expressed using the XML element //iwxxm:MeteorologicalAerodromeObservationRecord/iwxxm:seaState containing a valid child element iwxxm:AerodromeSeaState.\strut
\end{minipage}\tabularnewline
\begin{minipage}[t]{0.47\columnwidth}\raggedright
Requirement\strut
\end{minipage} & \begin{minipage}[t]{0.47\columnwidth}\raggedright
\url{http://icao.int/iwxxm/1.1/req/xsd-meteorological-aerodrome-observation-record/visibility}

If reported, the horizontal visibility conditions observed at the aerodrome shall be expressed using the XML element //iwxxm:MeteorologicalAerodromeObservationRecord/iwxxm:visibility containing a valid child element iwxxm:AerodromeHorizontalVisibility.\strut
\end{minipage}\tabularnewline
\begin{minipage}[t]{0.47\columnwidth}\raggedright
Recommendation\strut
\end{minipage} & \begin{minipage}[t]{0.47\columnwidth}\raggedright
\url{http://icao.int/iwxxm/1.1/req/xsd-meteorological-aerodrome-observation-record/present-weather-not-observable}

If present weather is not observable due to sensor failure or obstruction, the value of XML attribute //iwxxm:MeteorologicalAerodromeObservationRecord/iwxxm:presentWeather/@nilReason should indicate the URI ``\url{http://codes.wmo.int/common/nil/notObservable}''.\strut
\end{minipage}\tabularnewline
\bottomrule
\end{longtable}

Notes:

1. Units of measurement are specified in accordance with 1.9 above.

2. Cloud and visibility information is omitted when considered to be insignificant to aeronautical operations at an aerodrome. This occurs when: (i)~visibility exceeds 10~kilometres, (ii)~no cloud is present below 1~500~metres or the minimum sector altitude, whichever is greater, and there is no cumulonimbus at any height, and (iii) there is no weather of operational significance. These conditions are referred to as CAVOK. Use of CAVOK is specified in the \emph{Technical Regulations} (WMO-No.~49), Volume~II, Part~II, Appendix~3, 2.2.

3. The requirements for reporting the following are specified in the \emph{Technical Regulations} (WMO-No.~49), Volume~II, Part~II, Appendix~3:

(a) Air temperature and dewpoint temperature section~4.6

(b) Atmospheric pressure (QNH) section~4.7

(c) Present weather section~4.4

(d) Recent weather paragraph~4.8.1.1

(e) Surface wind conditions section~4.1

(f) Runway state paragraph~4.8.1.5

(g) Aerodrome wind shear paragraph~4.8.1.4

(h) Observed cloud conditions section~4.5

(i) Sea state paragraph~4.8.1.5

(j) Horizontal visibility section~4.2

4. Code table~D-7 is published online at \url{http://codes.wmo.int/49-2/AerodromePresentOrForecastWeather}.

5. Code table~D-6 is published online at \url{http://codes.wmo.int/49-2/AerodromeRecentWeather}.

6. Information on runway visual range shall be omitted if the prevailing visibility exceeds 1~500 metres. Details of the requirements for reporting runway visual range are specified in the \emph{Technical Regulations} (WMO-No.~49), Volume~II, Part~II, Appendix~3, 4.3.

205-15-Ext.14 Requirements class: Meteorological aerodrome observation

205-15-Ext.14.1 This requirements class restricts the content model for the XML element om:OM\_Observation such that the ``result'' of the observation describes the aggregated set of meteorological conditions observed at an aerodrome, the ``feature of interest'' is a representative point location within the aerodrome at which the meteorological conditions were observed and the ``procedure'' provides the set of information as specified by WMO.

Note: MeteorologicalAerodromeObservation is a subclass of ComplexSamplingMeasurement defined within METCE.

205-15-Ext.14.2 Instances of om:OM\_Observation with element om:type specifying \url{http://codes.wmo.int/49-2/observation-type/IWXXM/1.0/MeteorologicalAerodromeObservation} shall conform to all requirements in Table~205-15-Ext.14.

205-15-Ext.14.3 Instances of om:OM\_Observation with element om:type specifying \url{http://codes.wmo.int/49-2/observation-type/IWXXM/1.0/MeteorologicalAerodromeObservation} shall conform to all requirements of all relevant dependencies in Table~205-15-Ext.14 with the exception of those requirements listed as superseded in

.

205-15-Ext.14.4 The requirements and dependencies inherited from requirements class \url{http://def.wmo.int/metce/2013/req/xsd-complex-sampling-measurement} (as specified in 202-15-Ext.4) listed in Table~205-15-Ext.15 are superseded by requirements defined herein and shall no longer apply.

Note: XML implementation of iwxxm:MeteorologicalAerodromeObservation is dependent on:

-- OMXML {[}OGC/IS 10-025r1 Observations and Measurements 2.0 -- XML Implementation{]}.

Table~205-15-Ext.14. Requirements class xsd-meteorological-aerodrome-observation

\begin{longtable}[]{@{}ll@{}}
\toprule
Requirements class &\tabularnewline
\midrule
\endhead
\url{http://icao.int/iwxxm/1.1/req/xsd-meteorological-aerodrome-observation} &\tabularnewline
Target type & Data instance\tabularnewline
Name & Meteorological aerodrome observation\tabularnewline
Dependency & \url{http://www.opengis.net/spec/OMXML/2.0/req/observation}, OMXML clause~7.3\tabularnewline
Dependency & \url{http://www.opengis.net/spec/OMXML/2.0/req/sampling}, OMXML clause~7.14\tabularnewline
Dependency & \url{http://www.opengis.net/spec/OMXML/2.0/req/spatialSampling}, OMXML clause~7.15\tabularnewline
Dependency & \vtop{\hbox{\strut \url{http://def.wmo.int/metce/2013/req/xsd-complex-sampling-measurement},}\hbox{\strut 202-15-Ext.4}}\tabularnewline
Dependency & \url{http://icao.int/saf/1.1/req/xsd-aerodrome}, 204-15-Ext.4\tabularnewline
Dependency & \url{http://icao.int/iwxxm/1.1/req/xsd-meteorological-aerodrome-observation-record}, 205-15-Ext.13\tabularnewline
\begin{minipage}[t]{0.47\columnwidth}\raggedright
Requirement\strut
\end{minipage} & \begin{minipage}[t]{0.47\columnwidth}\raggedright
\url{http://icao.int/iwxxm/1.1/req/xsd-meteorological-aerodrome-observation/feature-of-interest}

The XML element //om:OM\_Observation/om:featureOfInterest shall contain a valid child element sams:SF\_SpatialSamplingFeature that describes the reference point to which the observed meteorological conditions apply.

The XML element //om:OM\_Observation/om:featureOfInterest/sams:SF\_SpatialSamplingFeature/sam:type shall have the value ``\url{http://www.opengis.net/def/samplingFeatureType/OGC-OM/2.0/SF_SamplingPoint}''.\strut
\end{minipage}\tabularnewline
\begin{minipage}[t]{0.47\columnwidth}\raggedright
Requirement\strut
\end{minipage} & \begin{minipage}[t]{0.47\columnwidth}\raggedright
\url{http://icao.int/iwxxm/1.1/req/xsd-meteorological-aerodrome-observation/sampled-feature}

The XML element //om:OM\_Observation/om:featureOfInterest/sams:SF\_ SpatialSamplingFeature/sam:sampledFeature shall contain a valid child element saf:Aerodrome that describes the aerodrome to which the observed meteorological conditions apply.\strut
\end{minipage}\tabularnewline
\begin{minipage}[t]{0.47\columnwidth}\raggedright
Requirement\strut
\end{minipage} & \begin{minipage}[t]{0.47\columnwidth}\raggedright
\url{http://icao.int/iwxxm/1.1/req/xsd-meteorological-aerodrome-observation/result}

If reported, the XML element //om:OM\_Observation/om:result shall contain a valid child element iwxxm:MeteorologicalAerodromeObservationRecord that describes the aggregated set of meteorological conditions observed at the target aerodrome.\strut
\end{minipage}\tabularnewline
\begin{minipage}[t]{0.47\columnwidth}\raggedright
Requirement\strut
\end{minipage} & \begin{minipage}[t]{0.47\columnwidth}\raggedright
\url{http://icao.int/iwxxm/1.1/req/xsd-meteorological-aerodrome-observation/phenomenon-time}

The XML element //om:OM\_Observation/om:phenomenonTime shall contain a valid child element gml:TimeInstant that describes the time at which the observation occurred.\strut
\end{minipage}\tabularnewline
\begin{minipage}[t]{0.47\columnwidth}\raggedright
Requirement\strut
\end{minipage} & \begin{minipage}[t]{0.47\columnwidth}\raggedright
\url{http://icao.int/iwxxm/1.1/req/xsd-meteorological-aerodrome-observation/result-time}

The XML element //om:OM\_Observation/om:resultTime shall contain a valid child element gml:TimeInstant that describes the time at which the observation was made available for dissemination.\strut
\end{minipage}\tabularnewline
\begin{minipage}[t]{0.47\columnwidth}\raggedright
Recommendation\strut
\end{minipage} & \begin{minipage}[t]{0.47\columnwidth}\raggedright
\url{http://icao.int/iwxxm/1.1/req/xsd-meteorological-aerodrome-observation/observed-property}

The XML attribute //om:OM\_Observation/om:observedProperty/@xlink:href should have the value ``\url{http://codes.wmo.int/49-2/observable-property/MeteorologicalAerodromeObservation}''.\strut
\end{minipage}\tabularnewline
\begin{minipage}[t]{0.47\columnwidth}\raggedright
Recommendation\strut
\end{minipage} & \begin{minipage}[t]{0.47\columnwidth}\raggedright
\url{http://icao.int/iwxxm/1.1/req/xsd-meteorological-aerodrome-observation/procedure}

The value of XML element //om:OM\_Observation/om:procedure/metce:Process/gml:description should be used to cite the Technical Regulations relating to meteorological aerodrome observations.\strut
\end{minipage}\tabularnewline
\bottomrule
\end{longtable}

Notes:

1. Dependency \url{http://www.opengis.net/spec/OMXML/2.0/req/observation} has associated conformance class\\
\url{http://www.opengis.net/spec/OMXML/2.0/conf/observation} (OMXML clause~A.1).

2. Dependency \url{http://www.opengis.net/spec/OMXML/2.0/req/sampling} has associated conformance class\\
\url{http://www.opengis.net/spec/OMXML/2.0/conf/sampling} (OMXML clause~A.12).

3. Dependency \url{http://www.opengis.net/spec/OMXML/2.0/req/spatialSampling} has associated conformance class \url{http://www.opengis.net/spec/OMXML/2.0/conf/spatialSampling} (OMXML clause~A.13).

4. URI \url{http://codes.wmo.int/49-2/observable-property/MeteorologicalAerodromeObservation} refers to an XML document that defines the aggregate set of observable properties relevant to a meteorological aerodrome observation.

5. The Technical Regulations relating to meteorological observations may be cited as: ``\emph{Technical Regulations} (WMO-No.~49), Volume~II, Part~II, Appendix~3 -- Technical specifications related to meteorological observations and reports''.

6. The time at which the observation is made available for dissemination may be a few minutes after the observation occurred.

7. In the case of NIL report (for example, to indicate that an anticipated meteorological aerodrome observation report is considered to be ``MISSING''), no meteorological conditions are provided. In these cases, the XML element //om:OM\_Observation/om:result has no child elements and the XML attribute //om:OM\_Observation/om:result/@nilReason is used to indicate why the ``result'' is absent.

Table~205-15-Ext.15. Superseded requirements and dependencies from xsd-complex-sampling-measurement

\begin{longtable}[]{@{}ll@{}}
\toprule
Superseded requirements and dependencies &\tabularnewline
\midrule
\endhead
Dependency & \url{http://www.opengis.net/spec/OMXML/2.0/req/complexObservation}, OMXML clause~7.10\tabularnewline
Dependency & \url{http://www.opengis.net/spec/SWE/2.0/req/xsd-simple-components}, SWE Common~2.0 clause~8.1\tabularnewline
Dependency & \url{http://www.opengis.net/spec/SWE/2.0/req/xsd-record-components}, SWE Common~2.0 clause~8.2\tabularnewline
Dependency & \url{http://www.opengis.net/spec/SWE/2.0/req/xsd-simple-encodings}, SWE Common 2.0 clause~8.5\tabularnewline
Dependency & \url{http://www.opengis.net/spec/SWE/2.0/req/general-encoding-rules}, SWE Common 2.0 clause~9.1\tabularnewline
Dependency & \url{http://www.opengis.net/spec/SWE/2.0/req/text-encoding-rules}, SWE Common~2.0 clause~9.2\tabularnewline
Dependency & \url{http://www.opengis.net/spec/SWE/2.0/req/xml-encoding-rules}, SWE Common~2.0 clause~9.3\tabularnewline
Requirement & \url{http://def.wmo.int/metce/2013/req/xsd-complex-sampling-measurement/xmlns-declaration-swe}, 202-15-Ext.4\tabularnewline
\bottomrule
\end{longtable}

205-15-Ext.15 Requirements class: Aerodrome surface wind trend forecast

205-15-Ext.15.1 This requirements class is used to describe the surface wind conditions forecast at an aerodrome as appropriate for inclusion in a trend forecast of a routine or special meteorological aerodrome report.

Note: The requirements for reporting the surface wind conditions within a trend forecast are specified in the \emph{Technical Regulations} (WMO-No.~49), Volume~II, Part~II, Appendix~5, 2.2.2.

205-15-Ext.15.2 XML elements describing surface wind conditions within a trend forecast shall conform to all requirements specified in Table~205-15-Ext.16.

205-15-Ext.15.3 XML elements describing surface wind conditions within a trend forecast shall conform to all requirements of all relevant dependencies specified in Table~205-15-Ext.16.

Table~205-15-Ext.16. Requirements class xsd-aerodrome-surface-wind-trend-forecast

\begin{longtable}[]{@{}ll@{}}
\toprule
Requirements class &\tabularnewline
\midrule
\endhead
\url{http://icao.int/iwxxm/1.1/req/xsd-aerodrome-surface-wind-trend-forecast} &\tabularnewline
Target type & Data instance\tabularnewline
Name & Aerodrome surface wind trend forecast\tabularnewline
\begin{minipage}[t]{0.47\columnwidth}\raggedright
Requirement\strut
\end{minipage} & \begin{minipage}[t]{0.47\columnwidth}\raggedright
\url{http://icao.int/iwxxm/1.1/req/xsd-aerodrome-surface-wind-trend-forecast/valid}

The content model of this element shall have a value that matches the content model of iwxxm:AerodromeSurfaceWindTrendForecast.\strut
\end{minipage}\tabularnewline
\begin{minipage}[t]{0.47\columnwidth}\raggedright
Requirement\strut
\end{minipage} & \begin{minipage}[t]{0.47\columnwidth}\raggedright
\url{http://icao.int/iwxxm/1.1/req/xsd-aerodrome-surface-wind-trend-forecast/mean-wind-speed}

The forecast mean wind speed shall be stated using the XML element //iwxxm:AerodromeSurfaceWindTrendForecast/iwxxm:meanWindSpeed, with the unit of measure metres per second, knots or kilometres per hour. The unit of measure shall be indicated using the XML attribute @uom with value ``m/s'' (metres per second), ``{[}kn\_i{]}'' (knots) or ``km/h'' (kilometres per hour).\strut
\end{minipage}\tabularnewline
\begin{minipage}[t]{0.47\columnwidth}\raggedright
Requirement\strut
\end{minipage} & \begin{minipage}[t]{0.47\columnwidth}\raggedright
\url{http://icao.int/iwxxm/1.1/req/xsd-aerodrome-surface-wind-trend-forecast/wind-direction}

If the forecast mean wind direction is reported, then the angle between true north and the mean direction from which the wind is forecast to be blowing shall be expressed using XML element //iwxxm:AerodromeSurfaceWindTrendForecast/iwxxm:meanWindDirection, with the unit of measure indicated using the XML attribute\\
@uom with value ``deg''.\strut
\end{minipage}\tabularnewline
\begin{minipage}[t]{0.47\columnwidth}\raggedright
Requirement\strut
\end{minipage} & \begin{minipage}[t]{0.47\columnwidth}\raggedright
\url{http://icao.int/iwxxm/1.1/req/xsd-aerodrome-surface-wind-trend-forecast/gust-speed}

If reported, the forecast gust speed shall be stated using the XML element //iwxxm:AerodromeSurfaceWindTrendForecast/iwxxm:windGustSpeed and expressed in metres per second, knots or kilometres per hour.

The unit of measure shall be indicated using the XML attribute //iwxxm:AerodromeSurfaceWind/iwxxm:windGustSpeed/@uom with value ``m/s'' (metres per second), ``{[}kn\_i{]}'' (knots) or ``km/h'' (kilometres per hour).\strut
\end{minipage}\tabularnewline
\bottomrule
\end{longtable}

Notes:

1. Units of measurement are specified in accordance with 1.9 above.

2. The true north is the north point at which the meridian lines meet.

205-15-Ext.16 Requirements class: Meteorological aerodrome trend forecast record

205-15-Ext.16.1 This requirements class is used to describe the aggregated set of meteorological conditions forecast at an aerodrome as appropriate for inclusion in a trend forecast of a routine or special meteorological aerodrome report.

205-15-Ext.16.2 XML elements describing the set of meteorological conditions for inclusion in a trend forecast shall conform to all requirements specified in Table~205-15-Ext.17.

205-15-Ext.16.3 XML elements describing the set of meteorological conditions for inclusion in a trend forecast shall conform to all requirements of all relevant dependencies specified in Table~205-15-Ext.17.

Table~205-15-Ext.17. Requirements class xsd-meteorological-aerodrome-trend-forecast-record

\begin{longtable}[]{@{}ll@{}}
\toprule
Requirements class &\tabularnewline
\midrule
\endhead
\url{http://icao.int/iwxxm/1.1/req/xsd-meteorological-aerodrome-trend-forecast-record} &\tabularnewline
Target type & Data instance\tabularnewline
Name & Meteorological aerodrome trend forecast record\tabularnewline
Dependency & \url{http://icao.int/iwxxm/1.1/req/xsd-aerodrome-cloud-forecast}, 205-15-Ext.5\tabularnewline
Dependency & \vtop{\hbox{\strut \url{http://icao.int/iwxxm/1.1/req/xsd-aerodrome-surface-wind-trend-forecast},}\hbox{\strut 205-15-Ext.15}}\tabularnewline
\begin{minipage}[t]{0.47\columnwidth}\raggedright
Requirement\strut
\end{minipage} & \begin{minipage}[t]{0.47\columnwidth}\raggedright
\url{http://icao.int/iwxxm/1.1/req/xsd-meteorological-aerodrome-trend-forecast-record/valid}

The content model of this element shall have a value that matches the content model of iwxxm:MeteorologicalAerodromeTrendForecastRecord.\strut
\end{minipage}\tabularnewline
\begin{minipage}[t]{0.47\columnwidth}\raggedright
Requirement\strut
\end{minipage} & \begin{minipage}[t]{0.47\columnwidth}\raggedright
\url{http://icao.int/iwxxm/1.1/req/xsd-meteorological-aerodrome-trend-forecast-record/change-indicator-nosig}

If no operationally significant changes to the meteorological conditions are forecast for the aerodrome, then the XML attribute //iwxxm:MeteorologicalAerodromeTrendForecastRecord/@changeIndicator shall have the value ``NO\_SIGNIFICANT\_CHANGES''.\strut
\end{minipage}\tabularnewline
\begin{minipage}[t]{0.47\columnwidth}\raggedright
Requirement\strut
\end{minipage} & \begin{minipage}[t]{0.47\columnwidth}\raggedright
\url{http://icao.int/iwxxm/1.1/req/xsd-meteorological-aerodrome-trend-forecast-record/change-indicator-becmg}

If the meteorological conditions forecast for the aerodrome are expected to reach or pass through specified values at a regular or irregular rate, then the XML attribute\\
//iwxxm:MeteorologicalAerodromeTrendForecastRecord/@changeIndicator shall have the value ``BECOMING''.\strut
\end{minipage}\tabularnewline
\begin{minipage}[t]{0.47\columnwidth}\raggedright
Requirement\strut
\end{minipage} & \begin{minipage}[t]{0.47\columnwidth}\raggedright
\url{http://icao.int/iwxxm/1.1/req/xsd-meteorological-aerodrome-trend-forecast-record/change-indicator-tempo}

If temporary fluctuations in the meteorological conditions forecast for the aerodrome are expected to occur, then the XML attribute //iwxxm:MeteorologicalAerodromeTrendForecastRecord/@changeIndicator shall have the value ``TEMPORARY\_FLUCTUATIONS''.\strut
\end{minipage}\tabularnewline
\begin{minipage}[t]{0.47\columnwidth}\raggedright
Requirement\strut
\end{minipage} & \begin{minipage}[t]{0.47\columnwidth}\raggedright
\url{http://icao.int/iwxxm/1.1/req/xsd-meteorological-aerodrome-trend-forecast-record/cavok}

If the conditions associated with CAVOK are forecast, then:

(i) The XML attribute //iwxxm:MeteorologicalAerodromeTrendForecastRecord/\\
@cloudAndVisibilityOK shall the have value ``true''; and

(ii) The following XML elements shall be absent: //iwxxm:MeteorologicalAerodromeTrendForecastRecord/iwxxm:prevailingVisibility, //iwxxm:MeteorologicalAerodromeTrendForecastRecord/iwxxm:prevailingVisibilityOperator, //iwxxm:MeteorologicalAerodromeTrendForecastRecord/iwxxm:forecastWeather and // iwxxm:MeteorologicalAerodromeTrendForecastRecord/iwxxm:cloud.\strut
\end{minipage}\tabularnewline
\begin{minipage}[t]{0.47\columnwidth}\raggedright
Requirement\strut
\end{minipage} & \begin{minipage}[t]{0.47\columnwidth}\raggedright
\url{http://icao.int/iwxxm/1.1/req/xsd-meteorological-aerodrome-trend-forecast-record/prevailing-visiblity}

If reported, the prevailing visibility shall be stated using the XML element\\
//iwxxm:MeteorologicalAerodromeTrendForecastRecord/iwxxm:prevailingVisibility with the unit of measure metres, indicated using the XML attribute //iwxxm:MeteorologicalAerodromeTrendForecastRecord/iwxxm:prevailingVisibility/@uom with value ``m''.\strut
\end{minipage}\tabularnewline
\begin{minipage}[t]{0.47\columnwidth}\raggedright
Requirement\strut
\end{minipage} & \begin{minipage}[t]{0.47\columnwidth}\raggedright
\url{http://icao.int/iwxxm/1.1/req/xsd-meteorological-aerodrome-trend-forecast-record/prevailing-visibility-exceeds-10000m}

If the prevailing visibility exceeds 10~000 metres, then the numeric value of XML element //iwxxm:MeteorologicalAerodromeTrendForecastRecord/iwxxm:prevailingVisibility shall be set to 10000 and the XML element //iwxxm:MeteorologicalAerodromeTrendForecastRecord/iwxxm:prevailingVisibilityOperator shall have the value ``ABOVE''.\strut
\end{minipage}\tabularnewline
\begin{minipage}[t]{0.47\columnwidth}\raggedright
Requirement\strut
\end{minipage} & \begin{minipage}[t]{0.47\columnwidth}\raggedright
\url{http://icao.int/iwxxm/1.1/req/xsd-meteorological-aerodrome-trend-forecast-record/prevailing-visibility-comparison-operator}

If present, the value of XML element //iwxxm:MeteorologicalAerodromeTrendForecastRecord/iwxxm:prevailingVisibilityOperator shall be one of the enumeration: ``ABOVE'' or ``BELOW''.\strut
\end{minipage}\tabularnewline
\begin{minipage}[t]{0.47\columnwidth}\raggedright
Requirement\strut
\end{minipage} & \begin{minipage}[t]{0.47\columnwidth}\raggedright
\url{http://icao.int/iwxxm/1.1/req/xsd-meteorological-aerodrome-trend-forecast-record/forecast-weather}

If forecast weather is reported, the value of XML attribute //iwxxm:MeteorologicalAerodromeTrendForecastRecord/iwxxm:forecastWeather/@xlink:href shall be the URI of a valid weather phenomenon code from Code table~D-7: Aerodrome present or forecast weather.\strut
\end{minipage}\tabularnewline
\begin{minipage}[t]{0.47\columnwidth}\raggedright
Requirement\strut
\end{minipage} & \begin{minipage}[t]{0.47\columnwidth}\raggedright
\url{http://icao.int/iwxxm/1.1/req/xsd-meteorological-aerodrome-trend-forecast-record/number-of-forecast-weather-codes}

No more than three forecast weather codes shall be reported.\strut
\end{minipage}\tabularnewline
\begin{minipage}[t]{0.47\columnwidth}\raggedright
Requirement\strut
\end{minipage} & \begin{minipage}[t]{0.47\columnwidth}\raggedright
\url{http://icao.int/iwxxm/1.1/req/xsd-meteorological-aerodrome-trend-forecast-record/surface-wind}

Surface wind conditions forecast for the aerodrome shall be reported using the XML element //iwxxm:MeteorologicalAerodromeTrendForecastRecord/iwxxm:surfaceWind containing a valid child element iwxxm:AerodromeSurfaceWindTrendForecast.\strut
\end{minipage}\tabularnewline
\begin{minipage}[t]{0.47\columnwidth}\raggedright
Requirement\strut
\end{minipage} & \begin{minipage}[t]{0.47\columnwidth}\raggedright
\url{http://icao.int/iwxxm/1.1/req/xsd-meteorological-aerodrome-trend-forecast-record/cloud}

If reported, the cloud conditions forecast for the aerodrome shall be expressed using the XML element //iwxxm:MeteorologicalAerodromeTrendForecastRecord/iwxxm:cloud containing a valid child element iwxxm:AerodromeCloudForecast.\strut
\end{minipage}\tabularnewline
\bottomrule
\end{longtable}

Notes:

1. Units of measurement are specified in accordance with 1.9 above.

2. Temporary fluctuations in the meteorological conditions occur when those conditions reach or pass specified values and last for a period of time less than one hour in each instance and, in the aggregate, cover less than one half the period during which the fluctuations are forecast to occur (\emph{Technical Regulations} (WMO-No.~49), Volume~II, Part~II, Appendix~5, 2.3.3).

3. The use of change groups is specified in the \emph{Technical Regulations} (WMO-No.~49), Volume~II, Part~II, Appendix~5, 2.3 and Appendix~3, Table~A3-3.

4. Cloud and visibility information is omitted when considered to be insignificant to aeronautical operations at an aerodrome. This occurs when: (i) visibility exceeds 10 kilometres, (ii) no cloud is present below 1~500 metres or the minimum sector altitude, whichever is greater, and there is no cumulonimbus at any height, and (iii) there is no weather of operational significance. These conditions are referred to as CAVOK. Use of CAVOK is specified in the \emph{Technical Regulations} (WMO-No.~49), Volume~II, Part~II, Appendix~3, 2.2.

5. Visibility for aeronautical purposes is defined as the greater of: (i) the greatest distance at which a black object of suitable dimensions, situated near the ground, can be seen and recognized when observed against a bright background; or (ii) the greatest distance at which lights in the vicinity of 1~000 candelas can be seen and identified against an unlit background.

6. Prevailing visibility is defined as the greatest visibility value observed which is reached within at least half the horizon circle or within at least half of the surface of the aerodrome. These areas could comprise contiguous or non-contiguous sectors.

7. The requirements for reporting the following within a trend forecast are specified in the \emph{Technical Regulations} (WMO-No.~49), Volume~II, Part~II, Appendix~5:

(a) Prevailing visibility conditions paragraph~2.2.3

(b) Forecast weather phenomena section~2.2.4

(c) Surface wind conditions paragraph~2.2.2

(d) Cloud conditions paragraph~2.2.5

8. The absence of XML element //iwxxm:MeteorologicalAerodromeTrendForecastRecord/iwxxm:prevailingVisibilityOperator indicates that the prevailing visibility has the numeric value reported.

9. Code table~D-7 is published online at \url{http://codes.wmo.int/49-2/AerodromePresentOrForecastWeather}.

205-15-Ext.17 Requirements class: Meteorological aerodrome trend forecast

205-15-Ext.17.1 This requirements class restricts the content model for the XML element om:OM\_Observation such that the ``result'' of the observation describes the aggregated set of meteorological conditions forecast at an aerodrome as appropriate for inclusion in a trend forecast, the ``feature of interest'' is a representative point location within the aerodrome for which the meteorological conditions were forecast and the ``procedure'' provides the set of information as specified by WMO.

Note: MeteorologicalAerodromeTrendForecast is a subclass of ComplexSamplingMeasurement defined within METCE.

205-15-Ext.17.2 Instances of om:OM\_Observation with element om:type specifying \url{http://codes.wmo.int/49-2/observation-type/IWXXM/1.0/MeteorologicalAerodromeTrendForecast} shall conform to all requirements in Table~205-15-Ext.18.

205-15-Ext.17.3 Instances of om:OM\_Observation with element om:type specifying \url{http://codes.wmo.int/49-2/observation-type/IWXXM/1.0/MeteorologicalAerodromeTrendForecast} shall conform to all requirements of all relevant dependencies in Table~205-15-Ext.18 with the exception of those requirements listed as superseded in 205-15-Ext.17.4.

205-15-Ext.17.4 The requirements and dependencies inherited from requirements class \url{http://def.wmo.int/metce/2013/req/xsd-complex-sampling-measurement} (as specified in 202-15-Ext.4) listed in Table~205-15-Ext.19 are superseded by requirements defined herein and shall no longer apply.

Note: XML implementation of iwxxm:MeteorologicalAerodromeTrendForecast is dependent on:

-- OMXML {[}OGC/IS 10-025r1 Observations and Measurements 2.0 -- XML Implementation{]}.

Table~205-15-Ext.18. Requirements class xsd-meteorological-aerodrome-trend-forecast

\begin{longtable}[]{@{}ll@{}}
\toprule
Requirements class &\tabularnewline
\midrule
\endhead
\url{http://icao.int/iwxxm/1.1/req/xsd-meteorological-aerodrome-trend-forecast} &\tabularnewline
Target type & Data instance\tabularnewline
Name & Meteorological aerodrome trend forecast\tabularnewline
Dependency & \url{http://www.opengis.net/spec/OMXML/2.0/req/observation}, OMXML clause~7.3\tabularnewline
Dependency & \url{http://www.opengis.net/spec/OMXML/2.0/req/sampling}, OMXML clause~7.14\tabularnewline
Dependency & \url{http://www.opengis.net/spec/OMXML/2.0/req/spatialSampling}, OMXML clause~7.15\tabularnewline
Dependency & \vtop{\hbox{\strut \url{http://def.wmo.int/metce/2013/req/xsd-complex-sampling-measurement},}\hbox{\strut 202-15-Ext.4}}\tabularnewline
Dependency & \url{http://icao.int/saf/1.1/req/xsd-aerodrome}, 204-15-Ext.4\tabularnewline
Dependency & \url{http://icao.int/iwxxm/1.1/req/xsd-meteorological-aerodrome-trend-forecast-record}, 205-15-Ext.16\tabularnewline
\begin{minipage}[t]{0.47\columnwidth}\raggedright
Requirement\strut
\end{minipage} & \begin{minipage}[t]{0.47\columnwidth}\raggedright
\href{http://icao.int/iwxxm/1.1/req/xsd-meteorological-trend-forecast/feature-of-interest}{http://icao.int/iwxxm/1.1/req/xsd-meteorological-aerodrome-trend-forecast/feature-of-interest}

The XML element //om:OM\_Observation/om:featureOfInterest shall contain a valid child element sams:SF\_SpatialSamplingFeature that describes the reference point to which the forecast meteorological conditions apply.

The XML element //om:OM\_Observation/om:featureOfInterest/sams:SF\_SpatialSamplingFeature/sam:type shall have the value ``\url{http://www.opengis.net/def/samplingFeatureType/OGC-OM/2.0/SF_SamplingPoint}''.\strut
\end{minipage}\tabularnewline
\begin{minipage}[t]{0.47\columnwidth}\raggedright
Requirement\strut
\end{minipage} & \begin{minipage}[t]{0.47\columnwidth}\raggedright
\url{http://icao.int/iwxxm/1.1/req/xsd-meteorological-aerodrome-trend-forecast/sampled-feature}

The XML element //om:OM\_Observation/om:featureOfInterest/sams:SF\_ SpatialSamplingFeature/sam:sampledFeature shall contain a valid child element saf:Aerodrome that describes the aerodrome to which the forecast meteorological conditions apply.\strut
\end{minipage}\tabularnewline
\begin{minipage}[t]{0.47\columnwidth}\raggedright
Requirement\strut
\end{minipage} & \begin{minipage}[t]{0.47\columnwidth}\raggedright
\url{http://icao.int/iwxxm/1.1/req/xsd-meteorological-aerodrome-trend-forecast/result}

The XML element //om:OM\_Observation/om:result shall contain a valid child element iwxxm:MeteorologicalAerodromeTrendForecastRecord that describes the aggregated set of meteorological conditions forecast for the target aerodrome.\strut
\end{minipage}\tabularnewline
\begin{minipage}[t]{0.47\columnwidth}\raggedright
Requirement\strut
\end{minipage} & \begin{minipage}[t]{0.47\columnwidth}\raggedright
\url{http://icao.int/iwxxm/1.1/req/xsd-meteorological-aerodrome-trend-forecast/phenomenon-time}

The XML element //om:OM\_Observation/om:phenomenonTime shall contain a valid child element gml:TimeInstant or gml:TimePeriod that describes the time for which the trend forecast is valid.\strut
\end{minipage}\tabularnewline
\begin{minipage}[t]{0.47\columnwidth}\raggedright
Requirement\strut
\end{minipage} & \begin{minipage}[t]{0.47\columnwidth}\raggedright
\url{http://icao.int/iwxxm/1.1/req/xsd-meteorological-aerodrome-trend-forecast/result-time}

The XML element //om:OM\_Observation/om:resultTime shall contain a valid child element gml:TimeInstant that describes the time at which the trend forecast was made available for dissemination.\strut
\end{minipage}\tabularnewline
\begin{minipage}[t]{0.47\columnwidth}\raggedright
Recommendation\strut
\end{minipage} & \begin{minipage}[t]{0.47\columnwidth}\raggedright
\url{http://icao.int/iwxxm/1.1/req/xsd-meteorological-aerodrome-trend-forecast/observed-property}

The XML attribute //om:OM\_Observation/om:observedProperty/@xlink:href should have the value ``\url{http://codes.wmo.int/49-2/observable-property/MeteorologicalAerodromeTrendForecast}''.\strut
\end{minipage}\tabularnewline
\begin{minipage}[t]{0.47\columnwidth}\raggedright
Recommendation\strut
\end{minipage} & \begin{minipage}[t]{0.47\columnwidth}\raggedright
\url{http://icao.int/iwxxm/1.1/req/xsd-meteorological-aerodrome-trend-forecast/procedure}

The value of XML element //om:OM\_Observation/om:procedure/metce:Process/gml:description should be used to cite the Technical Regulations relating to trend forecasts for aerodromes.\strut
\end{minipage}\tabularnewline
\bottomrule
\end{longtable}

Notes:

1. Dependency \url{http://www.opengis.net/spec/OMXML/2.0/req/observation} has associated conformance class\\
\url{http://www.opengis.net/spec/OMXML/2.0/conf/observation} (OMXML clause~A.1).

2. Dependency \url{http://www.opengis.net/spec/OMXML/2.0/req/sampling} has associated conformance class\\
\url{http://www.opengis.net/spec/OMXML/2.0/conf/sampling} (OMXML clause~A.12).

3. Dependency \url{http://www.opengis.net/spec/OMXML/2.0/req/spatialSampling} has associated conformance class \url{http://www.opengis.net/spec/OMXML/2.0/conf/spatialSampling} (OMXML clause~A.13).

4. URI \url{http://codes.wmo.int/49-2/observable-property/MeteorologicalAerodromeTrendForecast} refers to an XML document that defines the aggregate set of observable properties relevant to a trend forecast.

5. The Technical Regulations relating to trend forecasts may be cited as: ``\emph{Technical Regulations} (WMO-No.~49), Volume~II, Part~II, Appendix~5, section~2 -- Criteria related to trend forecasts''.

6. A forecast may be provided for a specific time instant or a time period. Requirements regarding the specification of time for which the forecast is valid are specified in the \emph{Technical Regulations} (WMO-No.~49), Volume~II, Part~II, Appendix~5, 2.3.

Table~205-15-Ext.19. Superseded requirements and dependencies from xsd-complex-sampling-measurement

\begin{longtable}[]{@{}ll@{}}
\toprule
Superseded requirements and dependencies &\tabularnewline
\midrule
\endhead
Dependency & \url{http://www.opengis.net/spec/OMXML/2.0/req/complexObservation}, OMXML clause~7.10\tabularnewline
Dependency & \url{http://www.opengis.net/spec/SWE/2.0/req/xsd-simple-components}, SWE Common~2.0 clause~8.1\tabularnewline
Dependency & \url{http://www.opengis.net/spec/SWE/2.0/req/xsd-record-components}, SWE Common~2.0 clause~8.2\tabularnewline
Dependency & \url{http://www.opengis.net/spec/SWE/2.0/req/xsd-simple-encodings}, SWE Common 2.0 clause~8.5\tabularnewline
Dependency & \url{http://www.opengis.net/spec/SWE/2.0/req/general-encoding-rules}, SWE Common~2.0 clause~9.1\tabularnewline
Dependency & \url{http://www.opengis.net/spec/SWE/2.0/req/text-encoding-rules}, SWE Common 2.0 clause~9.2\tabularnewline
Dependency & \url{http://www.opengis.net/spec/SWE/2.0/req/xml-encoding-rules}, SWE Common 2.0 clause~9.3\tabularnewline
Requirement & \url{http://def.wmo.int/metce/2013/req/xsd-complex-sampling-measurement/xmlns-declaration-swe}, 202-15-Ext.4\tabularnewline
\bottomrule
\end{longtable}

205-15-Ext.18 Requirements class: Meteorological aerodrome observation report

205-15-Ext.18.1 This requirements class is used to describe the report within which meteorological aerodrome observations, and optionally one or more trend forecasts, are provided.

Note: The reporting requirements for routine and special meteorological aerodrome reports are specified in the \emph{Technical Regulations} (WMO-No.~49), Volume~II, Part~II, Appendix~3 and Appendix~5, section~2.

205-15-Ext.18.2 XML elements describing routine or special meteorological aerodrome reports shall conform to all requirements specified in Table~205-15-Ext.20.

205-15-Ext.18.3 XML elements describing routine or special meteorological aerodrome reports shall conform to all requirements of all relevant dependencies specified in Table~205-15-Ext.20.

Table~205-15-Ext.20. Requirements class xsd-meteorological-aerodrome-observation-report

\begin{longtable}[]{@{}ll@{}}
\toprule
Requirements class &\tabularnewline
\midrule
\endhead
\url{http://icao.int/iwxxm/1.1/req/xsd-meteorological-aerodrome-observation-report} &\tabularnewline
Target type & Data instance\tabularnewline
Name & Meteorological aerodrome observation report\tabularnewline
Dependency & \vtop{\hbox{\strut \url{http://icao.int/iwxxm/1.1/req/xsd-meteorological-aerodrome-observation},}\hbox{\strut 205-15-Ext.14}}\tabularnewline
Dependency & \vtop{\hbox{\strut \url{http://icao.int/iwxxm/1.1/req/xsd-meteorological-aerodrome-trend-forecast},}\hbox{\strut 205-15-Ext.17}}\tabularnewline
\begin{minipage}[t]{0.47\columnwidth}\raggedright
Requirement\strut
\end{minipage} & \begin{minipage}[t]{0.47\columnwidth}\raggedright
\url{http://icao.int/iwxxm/1.1/req/xsd-meteorological-aerodrome-observation-report/valid}

The content model of this element shall have a value that matches the content model of iwxxm:MeteorologicalAerodromeObservationReport.\strut
\end{minipage}\tabularnewline
\begin{minipage}[t]{0.47\columnwidth}\raggedright
Requirement\strut
\end{minipage} & \begin{minipage}[t]{0.47\columnwidth}\raggedright
\url{http://icao.int/iwxxm/1.1/req/xsd-meteorological-aerodrome-observation-report/status}

The status of the report shall be indicated using the XML attribute @status with the value being one of the enumeration: ``NORMAL'', ``MISSING'' or ``CORRECTION''.\strut
\end{minipage}\tabularnewline
\begin{minipage}[t]{0.47\columnwidth}\raggedright
Requirement\strut
\end{minipage} & \begin{minipage}[t]{0.47\columnwidth}\raggedright
\url{http://icao.int/iwxxm/1.1/req/xsd-meteorological-aerodrome-observation-report/automated-station}

If the meteorological aerodrome observation included within the report has been generated by an automated system, the value of XML attribute @automatedStation shall be set to ``true''.\strut
\end{minipage}\tabularnewline
\begin{minipage}[t]{0.47\columnwidth}\raggedright
Requirement\strut
\end{minipage} & \begin{minipage}[t]{0.47\columnwidth}\raggedright
\url{http://icao.int/iwxxm/1.1/req/xsd-meteorological-aerodrome-observation-report/observation}

The XML element //iwxxm:observation shall contain a valid child element om:OM\_Observation of type MeteorologicalAerodromeObservation. The value of XML attribute //iwxxm:observation/om:OM\_Observation/om:type/@xlink:href shall be the URI ``\url{http://codes.wmo.int/49-2/observation-type/IWXXM/1.0/MeteorologicalAerodromeObservation}''.\strut
\end{minipage}\tabularnewline
\begin{minipage}[t]{0.47\columnwidth}\raggedright
Requirement\strut
\end{minipage} & \begin{minipage}[t]{0.47\columnwidth}\raggedright
\url{http://icao.int/iwxxm/1.1/req/xsd-meteorological-aerodrome-observation-report/trend-forecast}

If trend forecasts are reported, the value of XML element //iwxxm:trendForecast shall be a valid child element om:OM\_Observation of type MeteorologicalAerodromeTrendForecast.

For each trend forecast, the value of XML attribute //iwxxm:trendForecast/om:OM\_Observation/om:type/@xlink:href shall be the URI ``\url{http://codes.wmo.int/49-2/observation-type/IWXXM/1.0/MeteorologicalAerodromeTrendForecast}''.\strut
\end{minipage}\tabularnewline
\begin{minipage}[t]{0.47\columnwidth}\raggedright
Requirement\strut
\end{minipage} & \begin{minipage}[t]{0.47\columnwidth}\raggedright
\url{http://icao.int/iwxxm/1.1/req/xsd-meteorological-aerodrome-observation-report/number-of-trend-forecasts}

No more than three trend forecasts shall be reported.\strut
\end{minipage}\tabularnewline
\begin{minipage}[t]{0.47\columnwidth}\raggedright
Requirement\strut
\end{minipage} & \begin{minipage}[t]{0.47\columnwidth}\raggedright
\url{http://icao.int/iwxxm/1.1/req/xsd-meteorological-aerodrome-observation-report/unique-subject-aerodrome}

The observation and, if reported, trend forecasts shall refer to the same aerodrome. All values of XML element //om:OM\_Observation/om:featureOfInterest/sams:SF\_SpatialSamplingFeature/sam:sampledFeature/saf:Aerodrome/gml:identifier within the meteorological aerodrome observation report shall be identical.\strut
\end{minipage}\tabularnewline
\begin{minipage}[t]{0.47\columnwidth}\raggedright
Requirement\strut
\end{minipage} & \begin{minipage}[t]{0.47\columnwidth}\raggedright
\url{http://icao.int/iwxxm/1.1/req/xsd-meteorological-aerodrome-observation-report/nil-report}

If XML attribute @status has value ``MISSING'', then a NIL report shall be provided:

(i) XML element //iwxxm:observation/om:OM\_Observation/om:result shall have no child elements and XML attribute //iwxxm:observation/om:OM\_Observation/om:result/@nilReason shall provide an appropriate nil reason;

(ii) XML attribute @automatedStation shall be absent; and

(iii) XML element //iwxxm:trendForecast shall be absent.\strut
\end{minipage}\tabularnewline
\begin{minipage}[t]{0.47\columnwidth}\raggedright
Recommendation\strut
\end{minipage} & \begin{minipage}[t]{0.47\columnwidth}\raggedright
\url{http://icao.int/iwxxm/1.1/req/xsd-meteorological-aerodrome-observation-report/nosig}

If no change of operational significance is forecast, then a single XML element\\
//iwxxm:trendForecast should be included with no child elements therein and the value of XML attribute //iwxxm:trendForecast/@nilReason should indicate ``inapplicable''.\strut
\end{minipage}\tabularnewline
\bottomrule
\end{longtable}

Notes:

1. A report with status ``CORRECTED'' indicates that content has been amended to correct an error identified in an earlier report. The XML element //om:OM\_Observation/om:resultTime/gml:TimeInstant is used to reflect the dissemination time of the corrected report.

2. A report with status ``MISSING'' indicates that a routine report has not been provided on the anticipated timescales. Such a report does not contain details of any observed or forecast meteorological conditions and is often referred to as a ``NIL'' report.

3. The requirements for reporting the use of an automated system are specified in the \emph{Technical Regulations} (WMO-No.~49), Volume~II, Part~II, Appendix~3, 4.8.

4. If XML attribute @automatedStation is absent, then the value ``false'' is inferred; for example, the meteorological aerodrome observation included within the report has not been generated by an automated system.

5. Within an XML encoded meteorological aerodrome report, it is likely that only one instance of saf:Aerodrome will physically be present; subsequent assertions about the aerodrome may use xlinks to refer to the previously defined saf:Aerodrome element in order to keep the XML document size small. As such, validation of requirement \url{http://icao.int/iwxxm/1.1/req/xsd-meteorological-aerodrome-observation-report/unique-subject-aerodrome} is applied once any xlinks, if used, have been resolved.

6. Code table~D-1 provides a set of nil-reason codes and is published at \url{http://codes.wmo.int/common/nil}.

205-15-Ext.19 Requirements class: METAR

205-15-Ext.19.1 This requirements class is used to describe the routine meteorological aerodrome reports (METAR).

205-15-Ext.19.2 XML elements describing METARs shall conform to all requirements specified in Table~205-15-Ext.21.

205-15-Ext.19.3 XML elements describing METARs shall conform to all requirements of all relevant dependencies specified in Table~205-15-Ext.21.

Table~205-15-Ext.21. Requirements class xsd-metar

\begin{longtable}[]{@{}ll@{}}
\toprule
Requirements class &\tabularnewline
\midrule
\endhead
\url{http://icao.int/iwxxm/1.1/req/xsd-metar} &\tabularnewline
Target type & Data instance\tabularnewline
Name & METAR\tabularnewline
Dependency & \vtop{\hbox{\strut \url{http://icao.int/iwxxm/1.1/req/xsd-meteorological-aerodrome-observation-report},}\hbox{\strut 205-15-Ext.18}}\tabularnewline
\begin{minipage}[t]{0.47\columnwidth}\raggedright
Requirement\strut
\end{minipage} & \begin{minipage}[t]{0.47\columnwidth}\raggedright
\url{http://icao.int/iwxxm/1.1/req/xsd-metar/valid}

The content model of this element shall have a value that matches the content model of iwxxm:METAR.\strut
\end{minipage}\tabularnewline
\bottomrule
\end{longtable}

205-15-Ext.20 Requirements class: SPECI

205-15-Ext.20.1 This requirements class is used to describe the special meteorological aerodrome reports (SPECI).

205-15-Ext.20.2 XML elements describing SPECIs shall conform to all requirements specified in Table~205-15-Ext.22.

205-15-Ext.20.3 XML elements describing SPECIs shall conform to all requirements of all relevant dependencies specified in Table 205-15-Ext.22.

Table~205-15-Ext.22. Requirements class xsd-speci

\begin{longtable}[]{@{}ll@{}}
\toprule
Requirements class &\tabularnewline
\midrule
\endhead
\url{http://icao.int/iwxxm/1.1/req/xsd-speci} &\tabularnewline
Target type & Data instance\tabularnewline
Name & SPECI\tabularnewline
Dependency & \vtop{\hbox{\strut \url{http://icao.int/iwxxm/1.1/req/xsd-meteorological-aerodrome-observation-report},}\hbox{\strut 205-15-Ext.18}}\tabularnewline
\begin{minipage}[t]{0.47\columnwidth}\raggedright
Requirement\strut
\end{minipage} & \begin{minipage}[t]{0.47\columnwidth}\raggedright
\url{http://icao.int/iwxxm/1.1/req/xsd-speci/valid}

The content model of this element shall have a value that matches the content model of iwxxm:SPECI.\strut
\end{minipage}\tabularnewline
\bottomrule
\end{longtable}

205-15-Ext.21 Requirements class: Aerodrome surface wind forecast

205-15-Ext.21.1 This requirements class is used to describe the surface wind conditions forecast at an aerodrome as appropriate for inclusion in an aerodrome forecast (TAF) report.

Note: The requirements for reporting the surface wind conditions within a TAF are specified in the \emph{Technical Regulations} (WMO-No.~49), Volume~II, Part~II, Appendix~5, 1.2.1.

205-15-Ext.21.2 XML elements describing surface wind conditions forecast shall conform to all requirements specified in Table~205-15-Ext.23.

205-15-Ext.21.3 XML elements describing surface wind conditions forecast shall conform to all requirements of all relevant dependencies specified in Table~205-15-Ext.23.

Table~205-15-Ext.23. Requirements class xsd-aerodrome-surface-wind-forecast

\begin{longtable}[]{@{}ll@{}}
\toprule
Requirements class &\tabularnewline
\midrule
\endhead
\url{http://icao.int/iwxxm/1.1/req/xsd-aerodrome-surface-wind-forecast} &\tabularnewline
Target type & Data instance\tabularnewline
Name & Aerodrome surface wind forecast\tabularnewline
Dependency & \vtop{\hbox{\strut \url{http://icao.int/iwxxm/1.1/req/xsd-aerodrome-surface-wind-trend-forecast},}\hbox{\strut 205-15-Ext.15}}\tabularnewline
\begin{minipage}[t]{0.47\columnwidth}\raggedright
Requirement\strut
\end{minipage} & \begin{minipage}[t]{0.47\columnwidth}\raggedright
\url{http://icao.int/iwxxm/1.1/req/xsd-aerodrome-surface-wind-forecast/valid}

The content model of this element shall have a value that matches the content model of iwxxm:AerodromeSurfaceWindForecast.\strut
\end{minipage}\tabularnewline
\begin{minipage}[t]{0.47\columnwidth}\raggedright
Requirement\strut
\end{minipage} & \begin{minipage}[t]{0.47\columnwidth}\raggedright
\url{http://icao.int/iwxxm/1.1/req/xsd-aerodrome-surface-wind-forecast/variable-wind-direction}

If the wind direction is variable, then the XML attribute\\
//iwxxm:AerodromeSurfaceWindForecast/@variableWindDirection shall have the value ``true'' and XML element //iwxxm:AerodromeSurfaceWindForecast/iwxxm:meanWindDirection shall be absent.\strut
\end{minipage}\tabularnewline
\bottomrule
\end{longtable}

Note: Wind direction is reported as variable (VRB) if is not possible to forecast a prevailing surface wind direction due to expected variability, for example, during light wind conditions (less than 3 knots) or thunderstorms.

205-15-Ext.22 Requirements class: Aerodrome air temperature forecast

205-15-Ext.22.1 This requirements class is used to describe the temperature conditions forecast at an aerodrome as appropriate for inclusion in an aerodrome forecast (TAF) report, including the maximum and minimum temperature values and their time of occurrence.

Note: The requirements for reporting the temperature conditions at an aerodrome within a TAF are specified in the \emph{Technical Regulations} (WMO-No.~49), Volume~II, Part~II, Appendix~5, 1.2.5.

205-15-Ext.22.2 XML elements describing temperature conditions at an aerodrome shall conform to all requirements specified in Table~205-15-Ext.24.

205-15-Ext.22.3 XML elements describing temperature conditions at an aerodrome shall conform to all requirements of all relevant dependencies specified in Table~205-15-Ext.24.

Table~205-15-Ext.24. Requirements class xsd-aerodrome-air-temperature-forecast

\begin{longtable}[]{@{}ll@{}}
\toprule
Requirements class &\tabularnewline
\midrule
\endhead
\url{http://icao.int/iwxxm/1.1/req/xsd-aerodrome-air-temperature-forecast} &\tabularnewline
Target type & Data instance\tabularnewline
Name & Aerodrome air temperature forecast\tabularnewline
\begin{minipage}[t]{0.47\columnwidth}\raggedright
Requirement\strut
\end{minipage} & \begin{minipage}[t]{0.47\columnwidth}\raggedright
\url{http://icao.int/iwxxm/1.1/req/xsd-aerodrome-air-temperature-forecast/valid}

The content model of this element shall have a value that matches the content model of iwxxm:AerodromeAirTemperatureForecast.\strut
\end{minipage}\tabularnewline
\begin{minipage}[t]{0.47\columnwidth}\raggedright
Requirement\strut
\end{minipage} & \begin{minipage}[t]{0.47\columnwidth}\raggedright
\url{http://icao.int/iwxxm/1.1/req/xsd-aerodrome-air-temperature-forecast/maximum-temperature}

The maximum air temperature anticipated during the forecast period shall be reported in Celsius (°C) using the XML element //iwxxm:AerodromeAirTemperatureForecast/iwxxm:maximumAirTemperature. The value of the associated XML attribute @uom shall be ``Cel''.\strut
\end{minipage}\tabularnewline
\begin{minipage}[t]{0.47\columnwidth}\raggedright
Requirement\strut
\end{minipage} & \begin{minipage}[t]{0.47\columnwidth}\raggedright
\url{http://icao.int/iwxxm/1.1/req/xsd-aerodrome-air-temperature-forecast/maximum-temperature-time}

The XML element //iwxxm:AerodromeAirTemperatureForecast/iwxxm:maximumAirTemperatureTime shall contain a valid child element gml:TimeInstant that describes the time at which the maximum air temperature is anticipated to occur.\strut
\end{minipage}\tabularnewline
\begin{minipage}[t]{0.47\columnwidth}\raggedright
Requirement\strut
\end{minipage} & \begin{minipage}[t]{0.47\columnwidth}\raggedright
\url{http://icao.int/iwxxm/1.1/req/xsd-aerodrome-air-temperature-forecast/minimum-temperature}

The minimum air temperature anticipated during the forecast period shall be reported in Celsius (°C) using the XML element //iwxxm:AerodromeAirTemperatureForecast/iwxxm:minimumAirTemperature. The value of the associated XML attribute @uom shall be ``Cel''.\strut
\end{minipage}\tabularnewline
\begin{minipage}[t]{0.47\columnwidth}\raggedright
Requirement\strut
\end{minipage} & \begin{minipage}[t]{0.47\columnwidth}\raggedright
\url{http://icao.int/iwxxm/1.1/req/xsd-aerodrome-air-temperature-forecast/minimum-temperature-time}

The XML element //iwxxm:AerodromeAirTemperatureForecast/iwxxm:minimumAirTemperatureTime shall contain a valid child element gml:TimeInstant that describes the time at which the minimum air temperature is anticipated to occur.\strut
\end{minipage}\tabularnewline
\bottomrule
\end{longtable}

Note: Units of measurement are specified in accordance with 1.9 above.

205-15-Ext.23 Requirements class: Meteorological aerodrome forecast record

205-15-Ext.23.1 This requirements class is used to describe the aggregated set of meteorological conditions forecast at an aerodrome as appropriate for inclusion in a aerodrome forecast (TAF) report.

205-15-Ext.23.2 XML elements describing the set of meteorological conditions for inclusion in an aerodrome forecast shall conform to all requirements specified in Table~205-15-Ext.25.

205-15-Ext.23.3 XML elements describing the set of meteorological conditions for inclusion in an aerodrome forecast shall conform to all requirements of all relevant dependencies specified in Table~205-15-Ext.25.

Table~205-15-Ext.25. Requirements class xsd-meterological-aerodrome-forecast-record

\begin{longtable}[]{@{}ll@{}}
\toprule
Requirements class &\tabularnewline
\midrule
\endhead
\url{http://icao.int/iwxxm/1.1/req/xsd-meteorological-aerodrome-forecast-record} &\tabularnewline
Target type & Data instance\tabularnewline
Name & Meteorological aerodrome forecast record\tabularnewline
Dependency & \url{http://icao.int/iwxxm/1.1/req/xsd-aerodrome-cloud-forecast}, 205-15-Ext.5\tabularnewline
Dependency & \url{http://icao.int/iwxxm/1.1/req/xsd-aerodrome-surface-wind-forecast}, 205-15-Ext.21\tabularnewline
Dependency & \url{http://icao.int/iwxxm/1.1/req/xsd-aerodrome-air-temperature-forecast}, 205-15-Ext.22\tabularnewline
\begin{minipage}[t]{0.47\columnwidth}\raggedright
Requirement\strut
\end{minipage} & \begin{minipage}[t]{0.47\columnwidth}\raggedright
\url{http://icao.int/iwxxm/1.1/req/xsd-meteorological-aerodrome-forecast-record/valid}

The content model of this element shall have a value that matches the content model of iwxxm:MeteorologicalAerodromeForecastRecord.\strut
\end{minipage}\tabularnewline
\begin{minipage}[t]{0.47\columnwidth}\raggedright
Requirement\strut
\end{minipage} & \begin{minipage}[t]{0.47\columnwidth}\raggedright
\url{http://icao.int/iwxxm/1.1/req/xsd-meteorological-aerodrome-forecast-record/prevailing-forecast-conditions}

The XML attribute //iwxxm:MeteorologicalAerodromeForecastRecord/@changeIndicator shall be absent if the forecast describes the prevailing meteorological conditions.\strut
\end{minipage}\tabularnewline
\begin{minipage}[t]{0.47\columnwidth}\raggedright
Requirement\strut
\end{minipage} & \begin{minipage}[t]{0.47\columnwidth}\raggedright
\url{http://icao.int/iwxxm/1.1/req/xsd-meteorological-aerodrome-forecast-record/change-indicator-fm}

If the meteorological conditions forecast for the aerodrome are expected to change significantly and more or less completely to a different set of conditions, then the XML attribute //iwxxm:MeteorologicalAerodromeForecastRecord/@changeIndicator shall have the value ``FROM''.\strut
\end{minipage}\tabularnewline
\begin{minipage}[t]{0.47\columnwidth}\raggedright
Requirement\strut
\end{minipage} & \begin{minipage}[t]{0.47\columnwidth}\raggedright
\url{http://icao.int/iwxxm/1.1/req/xsd-meteorological-aerodrome-forecast-record/change-indicator-becmg}

If the meteorological conditions forecast for the aerodrome are expected to reach or pass through specified values at a regular or irregular rate, then the XML attribute\\
//iwxxm:MeteorologicalAerodromeForecastRecord/@changeIndicator shall have the value ``BECOMING''.\strut
\end{minipage}\tabularnewline
\begin{minipage}[t]{0.47\columnwidth}\raggedright
Requirement\strut
\end{minipage} & \begin{minipage}[t]{0.47\columnwidth}\raggedright
\url{http://icao.int/iwxxm/1.1/req/xsd-meteorological-aerodrome-forecast-record/change-indicator-tempo}

If temporary fluctuations in the meteorological conditions forecast for the aerodrome are expected to occur, then the XML attribute\\
//iwxxm:MeteorologicalAerodromeForecastRecord/@changeIndicator shall have the value ``TEMPORARY\_FLUCTUATIONS''.\strut
\end{minipage}\tabularnewline
\begin{minipage}[t]{0.47\columnwidth}\raggedright
Requirement\strut
\end{minipage} & \begin{minipage}[t]{0.47\columnwidth}\raggedright
\url{http://icao.int/iwxxm/1.1/req/xsd-meteorological-aerodrome-forecast-record/change-indicator-prob30}

If meteorological conditions forecast for the aerodrome have a 30\% probability of occurring, then the XML attribute //iwxxm:MeteorologicalAerodromeForecastRecord/@changeIndicator shall have the value ``PROBABILITY\_30''.\strut
\end{minipage}\tabularnewline
\begin{minipage}[t]{0.47\columnwidth}\raggedright
Requirement\strut
\end{minipage} & \begin{minipage}[t]{0.47\columnwidth}\raggedright
\url{http://icao.int/iwxxm/1.1/req/xsd-meteorological-aerodrome-forecast-record/change-indicator-prob30-tempo}

If the temporary fluctuations in meteorological conditions forecast have a 30\% probability of occurring, then the XML attribute\\
//iwxxm:MeteorologicalAerodromeForecastRecord/@changeIndicator shall have the value ``PROBABILITY\_30\_TEMPORARY\_FLUCTUATIONS''.\strut
\end{minipage}\tabularnewline
\begin{minipage}[t]{0.47\columnwidth}\raggedright
Requirement\strut
\end{minipage} & \begin{minipage}[t]{0.47\columnwidth}\raggedright
\url{http://icao.int/iwxxm/1.1/req/xsd-meteorological-aerodrome-forecast-record/change-indicator-prob40}

If meteorological conditions forecast for the aerodrome have a 40\% probability of occurring, then the XML attribute //iwxxm:MeteorologicalAerodromeForecastRecord/@changeIndicator shall have the value ``PROBABILITY\_40''.\strut
\end{minipage}\tabularnewline
\begin{minipage}[t]{0.47\columnwidth}\raggedright
Requirement\strut
\end{minipage} & \begin{minipage}[t]{0.47\columnwidth}\raggedright
\url{http://icao.int/iwxxm/1.1/req/xsd-meteorological-aerodrome-forecast-record/change-indicator-prob40-tempo}

If the temporary fluctuations in meteorological conditions forecast have a 40\% probability of occurring, then the XML attribute\\
//iwxxm:MeteorologicalAerodromeForecastRecord/@changeIndicator shall have the value ``PROBABILITY\_40\_TEMPORARY\_FLUCTUATIONS''.\strut
\end{minipage}\tabularnewline
\begin{minipage}[t]{0.47\columnwidth}\raggedright
Requirement\strut
\end{minipage} & \begin{minipage}[t]{0.47\columnwidth}\raggedright
\url{http://icao.int/iwxxm/1.1/req/xsd-meteorological-aerodrome-forecast-record/cavok}

If the conditions associated with CAVOK are forecast, then:

(i) The XML attribute //iwxxm:MeteorologicalAerodromeForecastRecord/\\
@cloudAndVisibilityOK shall have the value ``true''; and

(ii) The following XML elements shall be absent:\\
//iwxxm:MeteorologicalAerodromeForecastRecord/iwxxm:prevailingVisibility,\\
//iwxxm:MeteorologicalAerodromeForecastRecord/iwxxm:prevailingVisibilityOperator,\\
//iwxxm:MeteorologicalAerodromeForecastRecord/iwxxm:weather and\\
//iwxxm:MeteorologicalAerodromeForecastRecord/iwxxm:cloud.\strut
\end{minipage}\tabularnewline
\begin{minipage}[t]{0.47\columnwidth}\raggedright
Requirement\strut
\end{minipage} & \begin{minipage}[t]{0.47\columnwidth}\raggedright
\url{http://icao.int/iwxxm/1.1/req/xsd-meteorological-aerodrome-forecast-record/prevailing-visiblity}

If reported, the prevailing visibility shall be stated using the XML element\\
//iwxxm:MeteorologicalAerodromeForecastRecord/iwxxm:prevailingVisibility with the unit of measure metres, indicated using the XML attribute\\
//iwxxm:MeteorologicalAerodromeForecastRecord/iwxxm:prevailingVisibility/@uom with value ``m''.\strut
\end{minipage}\tabularnewline
\begin{minipage}[t]{0.47\columnwidth}\raggedright
Requirement\strut
\end{minipage} & \begin{minipage}[t]{0.47\columnwidth}\raggedright
\url{http://icao.int/iwxxm/1.1/req/xsd-meteorological-aerodrome-forecast-record/prevailing-visibility-exceeds-10000m}

If the prevailing visibility exceeds 10~000 metres, then the numeric value of XML element //iwxxm:MeteorologicalAerodromeForecastRecord/iwxxm:prevailingVisibility shall be set to 10000 and the XML element\\
//iwxxm:MeteorologicalAerodromeForecastRecord/iwxxm:prevailingVisibilityOperator shall have the value ``ABOVE''.\strut
\end{minipage}\tabularnewline
\begin{minipage}[t]{0.47\columnwidth}\raggedright
Requirement\strut
\end{minipage} & \begin{minipage}[t]{0.47\columnwidth}\raggedright
\url{http://icao.int/iwxxm/1.1/req/xsd-meteorological-aerodrome-forecast-record/prevailing-visibility-comparison-operator}

If present, the value of XML element //iwxxm:MeteorologicalAerodromeForecastRecord/iwxxm:prevailingVisibilityOperator shall be one of the enumeration: ``ABOVE'' or ``BELOW''.\strut
\end{minipage}\tabularnewline
\begin{minipage}[t]{0.47\columnwidth}\raggedright
Requirement\strut
\end{minipage} & \begin{minipage}[t]{0.47\columnwidth}\raggedright
\url{http://icao.int/iwxxm/1.1/req/xsd-meteorological-aerodrome-forecast-record/temperature}

If reported, the temperature conditions forecast for the aerodrome shall be expressed using the XML element //iwxxm:MeteorologicalAerodromeForecastRecord/iwxxm:temperature containing a valid child element iwxxm:AerodromeAirTemperatureForecast.\strut
\end{minipage}\tabularnewline
\begin{minipage}[t]{0.47\columnwidth}\raggedright
Requirement\strut
\end{minipage} & \begin{minipage}[t]{0.47\columnwidth}\raggedright
\url{http://icao.int/iwxxm/1.1/req/xsd-meteorological-aerodrome-forecast-record/number-of-temperature-groups}

No more than two sets of temperature conditions shall be reported.\strut
\end{minipage}\tabularnewline
\begin{minipage}[t]{0.47\columnwidth}\raggedright
Requirement\strut
\end{minipage} & \begin{minipage}[t]{0.47\columnwidth}\raggedright
\url{http://icao.int/iwxxm/1.1/req/xsd-meteorological-aerodrome-forecast-record/cloud}

If reported, the cloud conditions forecast for the aerodrome shall be expressed using the XML element //iwxxm:MeteorologicalAerodromeForecastRecord/iwxxm:cloud containing a valid child element iwxxm:AerodromeCloudForecast.\strut
\end{minipage}\tabularnewline
\begin{minipage}[t]{0.47\columnwidth}\raggedright
Requirement\strut
\end{minipage} & \begin{minipage}[t]{0.47\columnwidth}\raggedright
\url{http://icao.int/iwxxm/1.1/req/xsd-meteorological-aerodrome-forecast-record/forecast-weather}

If forecast weather is reported, the value of XML attribute\\
//iwxxm:MeteorologicalAerodromeForecastRecord/iwxxm:forecastWeather/@xlink:href shall be the URI of a valid weather phenomenon code from Code table~D-7: Aerodrome present or forecast weather.\strut
\end{minipage}\tabularnewline
\begin{minipage}[t]{0.47\columnwidth}\raggedright
Requirement\strut
\end{minipage} & \begin{minipage}[t]{0.47\columnwidth}\raggedright
\url{http://icao.int/iwxxm/1.1/req/xsd-meteorological-aerodrome-forecast-record/number-of-forecast-weather-codes}

No more than three forecast weather codes shall be reported.\strut
\end{minipage}\tabularnewline
\begin{minipage}[t]{0.47\columnwidth}\raggedright
Requirement\strut
\end{minipage} & \begin{minipage}[t]{0.47\columnwidth}\raggedright
\url{http://icao.int/iwxxm/1.1/req/xsd-meteorological-aerodrome-forecast-record/surface-wind}

Surface wind conditions forecast for the aerodrome shall be reported using the XML element //iwxxm:MeteorologicalAerodromeForecastRecord/iwxxm:surfaceWind containing a valid child element iwxxm:AerodromeSurfaceWindForecast.\strut
\end{minipage}\tabularnewline
\bottomrule
\end{longtable}

Notes:

1. Units of measurement are specified in accordance with 1.9 above.

2. Temporary fluctuations in the meteorological conditions occur when those conditions reach or pass specified values and last for a period of time less than one hour in each instance and, in the aggregate, cover less than one half the period during which the fluctuations are forecast to occur (\emph{Technical Regulations} (WMO-No.~49), Volume~II, Part~II, Appendix~5, 2.3.3).

3. The use of change groups and time indicators within a TAF is specified in the \emph{Technical Regulations} (WMO-No.~49), Volume~II, Part~II, Appendix~5, 1.3 and Table~A5-2.

4. The use of probability groups and time indicators within a TAF is specified in the \emph{Technical Regulations} (WMO-No.~49), Volume~II, Part~II, Appendix~5, 1.4 and Table~A5-2.

5. Cloud and visibility information is omitted when considered to be insignificant to aeronautical operations at an aerodrome. This occurs when: (i) visibility exceeds 10 kilometres, (ii) no cloud is present below 1~500 metres or the minimum sector altitude, whichever is greater, and there is no cumulonimbus at any height, and (iii) there is no weather of operational significance. These conditions are referred to as CAVOK. Use of CAVOK is specified in the \emph{Technical Regulations} (WMO-No.~49), Volume~II, Part~II, Appendix~3, 2.2.

6. Visibility for aeronautical purposes is defined as the greater of: (i) the greatest distance at which a black object of suitable dimensions, situated near the ground, can be seen and recognized when observed against a bright background; or (ii) the greatest distance at which lights in the vicinity of 1~000 candelas can be seen and identified against an unlit background.

7. Prevailing visibility is defined as the greatest visibility value observed which is reached within at least half the horizon circle or within at least half of the surface of the aerodrome. These areas could comprise contiguous or non-contiguous sectors.

8. The requirements for reporting the following within an aerodrome forecast are specified in the \emph{Technical Regulations} (WMO-No.~49), Volume~II, Part~II, Appendix~5:

(a) Prevailing visibility conditions paragraph~1.2.2

(b) Temperature conditions paragraph~1.2.5

(c) Cloud conditions paragraph~1.2.4

(d) Forecast weather phenomena paragraph~1.2.3

(e) Surface wind conditions paragraph~1.2.1

9. The absence of XML element //iwxxm:MeteorologicalAerodromeForecastRecord/iwxxm:prevailingVisibilityOperator indicates that the prevailing visibility has the numeric value reported.

10. Code table~D-7 is published online at \url{http://codes.wmo.int/49-2/AerodromePresentOrForecastWeather}.

205-15-Ext.24 Requirements class: Meteorological aerodrome forecast

205-15-Ext.24.1 This requirements class restricts the content model for the XML element om:OM\_Observation such that the ``result'' of the observation describes the aggregated set of meteorological conditions forecast at an aerodrome as appropriate for inclusion in an aerodrome forecast (TAF) report, the ``feature of interest'' is a representative point location within the aerodrome for which the meteorological conditions were forecast and the ``procedure'' provides the set of information as specified by WMO.

Note: MeteorologicalAerodromeForecast is a subclass of ComplexSamplingMeasurement defined within METCE.

205-15-Ext.24.2 Instances of om:OM\_Observation with element om:type specifying \url{http://codes.wmo.int/49-2/observation-type/IWXXM/1.0/MeteorologicalAerodromeForecast} shall conform to all requirements in Table~205-15-Ext.26.

205-15-Ext.24.3 Instances of om:OM\_Observation with element om:type specifying \url{http://codes.wmo.int/49-2/observation-type/IWXXM/1.0/MeteorologicalAerodromeForecast} shall conform to all requirements of all relevant dependencies in Table~205-15-Ext.26 with the exception of those requirements listed as superseded in 205-15-Ext.24.4.

205-15-Ext.24.4 The requirements and dependencies inherited from requirements class \url{http://def.wmo.int/metce/2013/req/xsd-complex-sampling-measurement} (as specified in 202-15-Ext.4) listed in Table~205-15-Ext.27 are superseded by requirements defined herein and shall no longer apply.

Note: XML implementation of iwxxm:MeteorologicalAerodromeForecast is dependent on:

-- OMXML {[}OGC/IS 10-025r1 Observations and Measurements 2.0 -- XML Implementation{]}.

Table~205-15-Ext.26. Requirements class xsd-meteorological-aerodrome-forecast

\begin{longtable}[]{@{}ll@{}}
\toprule
Requirements class &\tabularnewline
\midrule
\endhead
\url{http://icao.int/iwxxm/1.1/req/xsd-meteorological-aerodrome-forecast} &\tabularnewline
Target type & Data instance\tabularnewline
Name & Meteorological aerodrome forecast\tabularnewline
Dependency & \url{http://www.opengis.net/spec/OMXML/2.0/req/observation}, OMXML clause~7.3\tabularnewline
Dependency & \url{http://www.opengis.net/spec/OMXML/2.0/req/sampling}, OMXML clause~7.14\tabularnewline
Dependency & \url{http://www.opengis.net/spec/OMXML/2.0/req/spatialSampling}, OMXML clause~7.15\tabularnewline
Dependency & \vtop{\hbox{\strut \url{http://def.wmo.int/metce/2013/req/xsd-complex-sampling-measurement},}\hbox{\strut 202-15-Ext.4}}\tabularnewline
Dependency & \url{http://icao.int/saf/1.1/req/xsd-aerodrome}, 204-15-Ext.4\tabularnewline
Dependency & \vtop{\hbox{\strut \url{http://icao.int/iwxxm/1.1/req/xsd-meteorological-aerodrome-forecast-record},}\hbox{\strut 205-15-Ext.23}}\tabularnewline
\begin{minipage}[t]{0.47\columnwidth}\raggedright
Requirement\strut
\end{minipage} & \begin{minipage}[t]{0.47\columnwidth}\raggedright
\href{http://icao.int/iwxxm/1.1/req/xsd-meteorological-forecast/feature-of-interest}{http://icao.int/iwxxm/1.1/req/xsd-meteorological-aerodrome-forecast/feature-of-interest}

The XML element //om:OM\_Observation/om:featureOfInterest shall contain a valid child element sams:SF\_SpatialSamplingFeature that describes the reference point to which the forecast meteorological conditions apply.

The XML element //om:OM\_Observation/om:featureOfInterest/sams:SF\_SpatialSamplingFeature/sam:type shall have the value ``\url{http://www.opengis.net/def/samplingFeatureType/OGC-OM/2.0/SF_SamplingPoint}''.\strut
\end{minipage}\tabularnewline
\begin{minipage}[t]{0.47\columnwidth}\raggedright
Requirement\strut
\end{minipage} & \begin{minipage}[t]{0.47\columnwidth}\raggedright
\url{http://icao.int/iwxxm/1.1/req/xsd-meteorological-aerodrome-forecast/sampled-feature}

The XML element //om:OM\_Observation/om:featureOfInterest/sams:SF\_ SpatialSamplingFeature/sam:sampledFeature shall contain a valid child element saf:Aerodrome that describes the aerodrome to which the forecast meteorological conditions apply.\strut
\end{minipage}\tabularnewline
\begin{minipage}[t]{0.47\columnwidth}\raggedright
Requirement\strut
\end{minipage} & \begin{minipage}[t]{0.47\columnwidth}\raggedright
\url{http://icao.int/iwxxm/1.1/req/xsd-meteorological-aerodrome-forecast/result}

If reported, the XML element //om:OM\_Observation/om:result shall contain a valid child element iwxxm:MeteorologicalAerodromeForecastRecord that describes the aggregated set of meteorological conditions forecast for the target aerodrome.\strut
\end{minipage}\tabularnewline
\begin{minipage}[t]{0.47\columnwidth}\raggedright
Requirement\strut
\end{minipage} & \begin{minipage}[t]{0.47\columnwidth}\raggedright
\url{http://icao.int/iwxxm/1.1/req/xsd-meteorological-aerodrome-forecast/phenomenon-time}

The XML element //om:OM\_Observation/om:phenomenonTime shall contain a valid child element gml:TimeInstant or gml:TimePeriod that describes the time for which the forecast is valid.\strut
\end{minipage}\tabularnewline
\begin{minipage}[t]{0.47\columnwidth}\raggedright
Requirement\strut
\end{minipage} & \begin{minipage}[t]{0.47\columnwidth}\raggedright
\url{http://icao.int/iwxxm/1.1/req/xsd-meteorological-aerodrome-forecast/result-time}

The XML element //om:OM\_Observation/om:resultTime shall contain a valid child element gml:TimeInstant that describes the time at which the forecast was made available for dissemination.\strut
\end{minipage}\tabularnewline
\begin{minipage}[t]{0.47\columnwidth}\raggedright
Recommendation\strut
\end{minipage} & \begin{minipage}[t]{0.47\columnwidth}\raggedright
\url{http://icao.int/iwxxm/1.1/req/xsd-meteorological-aerodrome-forecast/observed-property}

The XML attribute //om:OM\_Observation/om:observedProperty/@xlink:href should have the value ``\url{http://codes.wmo.int/49-2/observable-property/MeteorologicalAerodromeForecast}''.\strut
\end{minipage}\tabularnewline
\begin{minipage}[t]{0.47\columnwidth}\raggedright
Recommendation\strut
\end{minipage} & \begin{minipage}[t]{0.47\columnwidth}\raggedright
\url{http://icao.int/iwxxm/1.1/req/xsd-meteorological-aerodrome-forecast/procedure}

The value of XML element //om:OM\_Observation/om:procedure/metce:Process/gml:description should be used to cite the Technical Regulations relating to meteorological aerodrome forecasts.\strut
\end{minipage}\tabularnewline
\bottomrule
\end{longtable}

Notes:

1. Dependency \url{http://www.opengis.net/spec/OMXML/2.0/req/observation} has associated conformance class\\
\url{http://www.opengis.net/spec/OMXML/2.0/conf/observation} (OMXML clause~A.1).

2. Dependency \url{http://www.opengis.net/spec/OMXML/2.0/req/sampling} has associated conformance class\\
\url{http://www.opengis.net/spec/OMXML/2.0/conf/sampling} (OMXML clause~A.12).

3. Dependency \url{http://www.opengis.net/spec/OMXML/2.0/req/spatialSampling} has associated conformance class \url{http://www.opengis.net/spec/OMXML/2.0/conf/spatialSampling} (OMXML clause~A.13).

4. URI \url{http://codes.wmo.int/49-2/observable-property/MeteorologicalAerodromeForecast} refers to an XML document that defines the aggregate set of observable properties relevant to an aerodrome forecast.

5. The Technical Regulations relating to forecasts may be cited as: ``\emph{Technical Regulations} (WMO-No.~49), Volume~II, Part~II, Appendix~5, section~1 -- Criteria related to TAF''.

6. A forecast may be provided for a specific time instant or a time period. Requirements regarding the specification of time for which the forecast is valid are specified at the \emph{Technical Regulation}s (WMO-No.~49), Volume~II, Part~II, Appendix~5, 1.3.

7. In the case of NIL report (for example, to indicate that an anticipated TAF is considered to be ``MISSING''), no meteorological conditions are provided. In these cases, the XML element //om:OM\_Observation/om:result has no child elements and the XML attribute //om:OM\_Observation/om:result/@nilReason is used to indicate why the ``result'' is absent.

Table~205-15-Ext.27. Superseded requirements and dependencies from xsd-complex-sampling-measurement

\begin{longtable}[]{@{}ll@{}}
\toprule
Superseded requirements and dependencies &\tabularnewline
\midrule
\endhead
Dependency & \url{http://www.opengis.net/spec/OMXML/2.0/req/complexObservation}, OMXML clause~7.10\tabularnewline
Dependency & \vtop{\hbox{\strut \url{http://www.opengis.net/spec/SWE/2.0/req/xsd-simple-components}, SWE}\hbox{\strut Common 2.0 clause~8.1}}\tabularnewline
Dependency & \vtop{\hbox{\strut \url{http://www.opengis.net/spec/SWE/2.0/req/xsd-record-components}, SWE}\hbox{\strut Common 2.0 clause~8.2}}\tabularnewline
Dependency & \url{http://www.opengis.net/spec/SWE/2.0/req/xsd-simple-encodings}, SWE Common 2.0 clause~8.5\tabularnewline
Dependency & \vtop{\hbox{\strut \url{http://www.opengis.net/spec/SWE/2.0/req/general-encoding-rules}, SWE}\hbox{\strut Common 2.0 clause~9.1}}\tabularnewline
Dependency & \url{http://www.opengis.net/spec/SWE/2.0/req/text-encoding-rules}, SWE Common 2.0 clause~9.2\tabularnewline
Dependency & \url{http://www.opengis.net/spec/SWE/2.0/req/xml-encoding-rules}, SWE Common 2.0 clause~9.3\tabularnewline
Requirement & \url{http://def.wmo.int/metce/2013/req/xsd-complex-sampling-measurement/xmlns-declaration-swe}, 202-15-Ext.4\tabularnewline
\bottomrule
\end{longtable}

205-15-Ext.25 Requirements class: TAF

205-15-Ext.25.1 This requirements class is used to describe the aerodrome forecast (TAF) report within which a base forecast, and optionally one or more change forecasts, is provided.

Note: The reporting requirements for aerodrome forecasts are specified in the \emph{Technical Regulations} (WMO-No.~49), Volume~II, Part~II, Appendix~3 and Appendix~5, section~1.

205-15-Ext.25.2 XML elements describing TAFs shall conform to all requirements specified in Table~205-15-Ext.28.

205-15-Ext.25.3 XML elements describing TAFs shall conform to all requirements of all relevant dependencies specified in Table~205-15-Ext.28.

Table~205-15-Ext.28. Requirements class xsd-taf

\begin{longtable}[]{@{}ll@{}}
\toprule
Requirements class &\tabularnewline
\midrule
\endhead
\url{http://icao.int/iwxxm/1.1/req/xsd-taf} &\tabularnewline
Target type & Data instance\tabularnewline
Name & TAF\tabularnewline
Dependency & \vtop{\hbox{\strut \url{http://icao.int/iwxxm/1.1/req/xsd-meteorological-aerodrome-forecast},}\hbox{\strut 205-15-Ext.24}}\tabularnewline
\begin{minipage}[t]{0.47\columnwidth}\raggedright
Requirement\strut
\end{minipage} & \begin{minipage}[t]{0.47\columnwidth}\raggedright
\url{http://icao.int/iwxxm/1.1/req/xsd-taf/valid}

The content model of this element shall have a value that matches the content model of iwxxm:TAF.\strut
\end{minipage}\tabularnewline
\begin{minipage}[t]{0.47\columnwidth}\raggedright
Requirement\strut
\end{minipage} & \begin{minipage}[t]{0.47\columnwidth}\raggedright
\url{http://icao.int/iwxxm/1.1/req/xsd-taf/status}

The status of the TAF shall be indicated using the XML attribute //iwxxm:TAF/@status with the value being one of the enumeration: ``NORMAL'', ``AMENDMENT'', ``CANCELLATION'', ``CORRECTION'' or ``MISSING''.\strut
\end{minipage}\tabularnewline
\begin{minipage}[t]{0.47\columnwidth}\raggedright
Requirement\strut
\end{minipage} & \begin{minipage}[t]{0.47\columnwidth}\raggedright
\url{http://icao.int/iwxxm/1.1/req/xsd-taf/issue-time}

The XML element //iwxxm:TAF/iwxxm:issueTime shall contain a valid child element gml:TimeInstant that describes the time at which the TAF was issued.\strut
\end{minipage}\tabularnewline
\begin{minipage}[t]{0.47\columnwidth}\raggedright
Requirement\strut
\end{minipage} & \begin{minipage}[t]{0.47\columnwidth}\raggedright
\url{http://icao.int/iwxxm/1.1/req/xsd-taf/base-forecast}

If the prevailing forecast conditions for the valid period of the TAF are reported, then:

(i) The XML element //iwxxm:TAF/iwxxm:baseForecast shall contain a valid child element om:OM\_Observation of type MeteorologicalAerodromeForecast;

(ii) The value of XML attribute //iwxxm:TAF/iwxxm:baseForecast/om:OM\_Observation/om:type/@xlink:href shall be the URI ``\url{http://codes.wmo.int/49-2/observation-type/IWXXM/1.0/MeteorologicalAerodromeForecast}''; and

(iii) The XML attribute //iwxxm:TAF/iwxxm:baseForecast/om:OM\_Observation/om:result/iwxxm:MeteorologicalAerodromeForecastRecord/@changeIndicator shall be absent.\strut
\end{minipage}\tabularnewline
\begin{minipage}[t]{0.47\columnwidth}\raggedright
Requirement\strut
\end{minipage} & \begin{minipage}[t]{0.47\columnwidth}\raggedright
\url{http://icao.int/iwxxm/1.1/req/xsd-taf/change-forecast}

If change forecasts or forecasts with probability of occurrence are reported, then:

(i) The XML element //iwxxm:TAF/iwxxm:changeForecast shall contain a valid child element om:OM\_Observation of type MeteorologicalAerodromeForecast;

(ii) The value of XML attribute //iwxxm:TAF/iwxxm:changeForecast/om:OM\_Observation/om:type/@xlink:href shall be the URI ``\url{http://codes.wmo.int/49-2/observation-type/IWXXM/1.0/MeteorologicalAerodromeForecast}'';

(iii) The XML element //iwxxm:TAF/iwxxm:changeForecast/om:OM\_Observation/om:result/iwxxm:MeteorologicalAerodromeForecastRecord/iwxxm:temperature shall be absent; and

(iv) The XML attribute //iwxxm:TAF/iwxxm:baseForecast/om:OM\_Observation/om:result/iwxxm:MeteorologicalAerodromeForecastRecord/@changeIndicator shall be one of the enumeration: ``BECOMING'', ``TEMPORARY\_FLUCTUATIONS'', ``FROM'', ``PROBABILITY\_30'', ``PROBABILITY\_30\_TEMPORARY\_FLUCTUATIONS'', ``PROBABILITY\_40'' or ``PROBABILITY\_40\_TEMPORARY\_FLUCTUATIONS''.\strut
\end{minipage}\tabularnewline
\begin{minipage}[t]{0.47\columnwidth}\raggedright
Requirement\strut
\end{minipage} & \begin{minipage}[t]{0.47\columnwidth}\raggedright
\url{http://icao.int/iwxxm/1.1/req/xsd-taf/unique-subject-aerodrome}

The base forecast and, if reported, change forecasts shall refer to the same aerodrome. All values of XML element //iwxxm:TAF/*/om:OM\_Observation/om:featureOfInterest/sams:SF\_SpatialSamplingFeature/sam:sampledFeature/saf:Aerodrome/gml:identifier within the TAF shall be identical.\strut
\end{minipage}\tabularnewline
\begin{minipage}[t]{0.47\columnwidth}\raggedright
Requirement\strut
\end{minipage} & \begin{minipage}[t]{0.47\columnwidth}\raggedright
\url{http://icao.int/iwxxm/1.1/req/xsd-taf/status-normal}

If the status of the TAF is ``NORMAL'' (as specified by XML attribute //iwxxm:TAF/@status), then:

(i) The prevailing meteorological conditions anticipated during the valid period of the TAF shall be reported using the XML element //iwxxm:TAF/iwxxm:baseForecast;

(ii) The valid time period of the TAF shall be given using the XML element //iwxxm:TAF/iwxxm:validTime/gml:TimePeriod;

(iii) The XML element //iwxxm:TAF/iwxxm:previousReportAerodrome shall be absent; and

(iv) The XML element //iwxxm:TAF/iwxxm:previousReportValidPeriod shall be absent.\strut
\end{minipage}\tabularnewline
\begin{minipage}[t]{0.47\columnwidth}\raggedright
Requirement\strut
\end{minipage} & \begin{minipage}[t]{0.47\columnwidth}\raggedright
\url{http://icao.int/iwxxm/1.1/req/xsd-taf/status-amendment-or-correction}

If the status of the TAF is ``AMENDMENT'' or ``CORRECTION'' (as specified by XML attribute //iwxxm:TAF/@status), then:

(i) The prevailing meteorological conditions anticipated during the valid period of the TAF shall be reported using the XML element //iwxxm:TAF/iwxxm:baseForecast;

(ii) The valid time period of the TAF shall be given using the XML element //iwxxm:TAF/iwxxm:validTime/gml:TimePeriod; and

(iii) The valid time period for the TAF that has been amended or corrected shall be reported using the XML element //iwxxm:TAF/iwxxm:previousReportValidPeriod/gml:TimePeriod.\strut
\end{minipage}\tabularnewline
\begin{minipage}[t]{0.47\columnwidth}\raggedright
Requirement\strut
\end{minipage} & \begin{minipage}[t]{0.47\columnwidth}\raggedright
\url{http://icao.int/iwxxm/1.1/req/xsd-taf/status-cancellation}

If the status of the TAF is ``CANCELLATION'' (as specified by XML attribute //iwxxm:TAF/@status), then:

(i) The XML element //iwxxm:TAF/iwxxm:baseForecast shall be absent;

(ii) The XML element //iwxxm:TAF/iwxxm:changeForecast shall be absent;

(iii) The time period for which TAF reports at the subject aerodrome are cancelled shall be given using the XML element //iwxxm:TAF/iwxxm:validTime/gml:TimePeriod;

(iv) The aerodrome for which TAF reports are cancelled shall be reported using the XML element //iwxxm:TAF/iwxxm:previousReportAerodrome/saf:Aerodrome; and

(v) The valid time period for the TAF that has been cancelled shall be reported using the XML element //iwxxm:TAF/iwxxm:previousReportValidPeriod/gml:TimePeriod.\strut
\end{minipage}\tabularnewline
\begin{minipage}[t]{0.47\columnwidth}\raggedright
Requirement\strut
\end{minipage} & \begin{minipage}[t]{0.47\columnwidth}\raggedright
\url{http://icao.int/iwxxm/1.1/req/xsd-taf/nil-report-status-missing}

If the status of the TAF is ``MISSING'' (as specified by XML attribute //iwxxm:TAF/@status), then:

(i) The XML element //iwxxm:TAF/iwxxm:baseForecast shall contain valid child element om:OM\_Observation of type MeteorologicalAerodromeForecast;

(ii) The value of XML element //iwxxm:TAF/iwxxm:baseForecast/om:OM\_Observation/om:featureOfInterest/sams:SF\_SpatialSamplingFeature/sam:sampledFeature/saf:Aerodrome shall indicate the aerodrome for which the TAF is missing;

(iii) The XML element //iwxxm:TAF/iwxxm:baseForecast/om:OM\_Observation/om:result shall have no child elements and XML attribute //iwxxm:TAF/iwxxm:baseForecast/om:OM\_Observation/om:result/@nilReason shall provide an appropriate nil reason;

(iv) The XML element //iwxxm:TAF/iwxxm:changeForecast shall be absent;

(v) The XML element //iwxxm:TAF/iwxxm:validTime shall be absent;

(vi) The XML element //iwxxm:TAF/iwxxm:previousReportAerodrome shall be absent; and

(vii) The XML element //iwxxm:TAF/iwxxm:previousReportValidPeriod shall be absent.\strut
\end{minipage}\tabularnewline
\begin{minipage}[t]{0.47\columnwidth}\raggedright
Recommendation\strut
\end{minipage} & \begin{minipage}[t]{0.47\columnwidth}\raggedright
\url{http://icao.int/iwxxm/1.1/req/xsd-taf/number-of-change-forecasts}

The number of change forecasts should be kept to a minimum, and no more than five change forecasts should be reported in normal circumstances.\strut
\end{minipage}\tabularnewline
\begin{minipage}[t]{0.47\columnwidth}\raggedright
Recommendation\strut
\end{minipage} & \begin{minipage}[t]{0.47\columnwidth}\raggedright
\url{http://icao.int/iwxxm/1.1/req/xsd-taf/issue-time-matches-result-time}

The TAF issue time (specified by XML element //iwxxm:TAF/iwxxm:issueTime/gml:TimeInstant) should match the result time for each of the forecasts provided within the TAF (specified by XML element //iwxxm:TAF/*/om:OM\_Observation/om:resultTime/gml:TimeInstant).\strut
\end{minipage}\tabularnewline
\begin{minipage}[t]{0.47\columnwidth}\raggedright
Recommendation\strut
\end{minipage} & \begin{minipage}[t]{0.47\columnwidth}\raggedright
\url{http://icao.int/iwxxm/1.1/req/xsd-taf/valid-time-includes-all-phenomenon-times}

The valid times of all forecasts included in the TAF (specified by XML element\\
//iwxxm:TAF/*/om:OM\_Observation/om:phenomenonTime/*) should occur within the valid time period of the TAF (specified by XML element //iwxxm:TAF/iwxxm:validTime/gml:TimePeriod).\strut
\end{minipage}\tabularnewline
\begin{minipage}[t]{0.47\columnwidth}\raggedright
Recommendation\strut
\end{minipage} & \begin{minipage}[t]{0.47\columnwidth}\raggedright
\url{http://icao.int/iwxxm/1.1/req/xsd-taf/status-amendment-or-correction-previous-aerodrome}

If the status of the TAF is ``AMENDMENT'' or ``CORRECTION'' (as specified by XML attribute //iwxxm:TAF/@status), then the aerodrome that was the subject of the TAF that has been amended or corrected should be reported using the XML element\\
//iwxxm:TAF/iwxxm:previousReportAerodrome/saf:Aerodrome.\strut
\end{minipage}\tabularnewline
\bottomrule
\end{longtable}

Notes:

1. The requirements for reporting the following are specified in the \emph{Technical Regulations} (WMO-No.~49), Volume~II, Part~II, Appendix~5:

(a) Change forecasts section~1.3

(b) Probability forecasts with probability of occurrence paragraph~1.4

2. Guidance regarding the number of change forecasts or forecasts with probability of occurrence is given in the \emph{Technical Regulations} (WMO-No.~49), Volume~II, Part~II, Appendix~5, 1.5.

3. A report with status ``MISSING'' indicates that a routine report has not been provided on the anticipated timescales. Such a report does not contain details of any forecast meteorological conditions and is often referred to as a ``NIL'' report

4. Within an XML encoded TAF, it is likely that only one instance of saf:Aerodrome will physically be present; subsequent assertions about the aerodrome may use xlinks to refer to the previously defined saf:Aerodrome element in order to keep the XML document size small. As such, validation of requirement \url{http://icao.int/iwxxm/1.1/req/xsd-taf/unique-subject-aerodrome} is applied once any xlinks, if used, have been resolved.

5. Code table~D-1 provides a set of nil-reason codes and is published at \url{http://codes.wmo.int/common/nil}.

205-15-Ext.26 Requirements class: Evolving meteorological condition

205-15-Ext.26.1 This requirements class is used to describe the presence of a specific SIGMET phenomenon such as volcanic ash or thunderstorm, along with expected changes to the intensity of the phenomenon, its speed and direction of motion. The geometric extent of the SIGMET phenomenon is specified as a two-dimensional horizontal region with bounded vertical extent.

205-15-Ext.26.2 XML elements describing the characteristics of a SIGMET phenomenon shall conform to all requirements specified in Table~205-15-Ext.29.

205-15-Ext.26.3 XML elements describing the characteristics of a SIGMET phenomenon shall conform to all requirements of all relevant dependencies specified in Table~205-15-Ext.29.

Table~205-15-Ext.29. Requirements class xsd-evolving-meteorological-condition

\begin{longtable}[]{@{}ll@{}}
\toprule
Requirements class &\tabularnewline
\midrule
\endhead
\url{http://icao.int/iwxxm/1.1/req/xsd-evolving-meteorological-condition} &\tabularnewline
Target type & Data instance\tabularnewline
Name & Evolving meteorological condition\tabularnewline
Dependency & \url{http://icao.int/saf/1.1/req/xsd-airspace-volume}, 204-15-Ext.8\tabularnewline
\begin{minipage}[t]{0.47\columnwidth}\raggedright
Requirement\strut
\end{minipage} & \begin{minipage}[t]{0.47\columnwidth}\raggedright
\url{http://icao.int/iwxxm/1.1/req/xsd-evolving-meteorological-condition/valid}

The content model of this element shall have a value that matches the content model of iwxxm:EvolvingMeteorologicalCondition.\strut
\end{minipage}\tabularnewline
\begin{minipage}[t]{0.47\columnwidth}\raggedright
Requirement\strut
\end{minipage} & \begin{minipage}[t]{0.47\columnwidth}\raggedright
\url{http://icao.int/iwxxm/1.1/req/xsd-evolving-meteorological-condition/intensity-change}

The anticipated change in intensity of the SIGMET phenomenon shall be indicated using the XML attribute //iwxxm:EvolvingMeteorologicalCondition/@intensityChange with the value being one of the enumeration: ``NO\_CHANGE'', ``WEAKEN'' or ``INTENSIFY''.\strut
\end{minipage}\tabularnewline
\begin{minipage}[t]{0.47\columnwidth}\raggedright
Requirement\strut
\end{minipage} & \begin{minipage}[t]{0.47\columnwidth}\raggedright
\url{http://icao.int/iwxxm/1.1/req/xsd-evolving-meteorological-condition/geometry}

The geometric extent of the SIGMET phenomenon shall be reported using the XML element //iwxxm:EvolvingMeteorologicalCondition/iwxxm:geometry with valid child element saf:AirspaceVolume.\strut
\end{minipage}\tabularnewline
\begin{minipage}[t]{0.47\columnwidth}\raggedright
Requirement\strut
\end{minipage} & \begin{minipage}[t]{0.47\columnwidth}\raggedright
\url{http://icao.int/iwxxm/1.1/req/xsd-evolving-meteorological-condition/speed-of-motion}

The speed of motion of the SIGMET phenomenon shall be reported using the XML element //iwxxm:EvolvingMeteorologicalCondition/iwxxm:speedOfMotion, with the unit of measure metres per second, knots or kilometres per hour.

The unit of measure shall be indicated using the XML attribute\\
//iwxxm:EvolvingMeteorologicalCondition/iwxxm:speedOfMotion/@uom with value ``m/s'' (metres per second), ``{[}kn\_i{]}'' (knots) or ``km/h'' (kilometres per hour).\strut
\end{minipage}\tabularnewline
\begin{minipage}[t]{0.47\columnwidth}\raggedright
Requirement\strut
\end{minipage} & \begin{minipage}[t]{0.47\columnwidth}\raggedright
\url{http://icao.int/iwxxm/1.1/req/xsd-evolving-meteorological-condition/direction-of-motion}

If reported, the angle between true north and the direction of motion of the SIGMET phenomenon shall be given in degrees using the XML element\\
//iwxxm:EvolvingMeteorologicalCondition/iwxxm:directionOfMotion.

The unit of measure shall be indicated using the XML attribute\\
//iwxxm:EvolvingMeteorologicalCondition/iwxxm:directionOfMotion/@uom with value ``deg''.\strut
\end{minipage}\tabularnewline
\begin{minipage}[t]{0.47\columnwidth}\raggedright
Recommendation\strut
\end{minipage} & \begin{minipage}[t]{0.47\columnwidth}\raggedright
\url{http://icao.int/iwxxm/1.1/req/xsd-evolving-meteorological-condition/stationary-phenomenon}

If the SIGMET phenomenon is not moving (indicated by the XML element\\
//iwxxm:EvolvingMeteorologicalCondition/iwxxm:speedOfMotion having numeric value zero), XML element //iwxxm:EvolvingMeteorologicalCondition/iwxxm:directionOfMotion should be absent.\strut
\end{minipage}\tabularnewline
\bottomrule
\end{longtable}

Notes:

1. Units of measurement are specified in accordance with 1.9 above.

2. The true north is the north point at which the meridian lines meet.

205-15-Ext.27 Requirements class: SIGMET evolving condition analysis

205-15-Ext.27.1 This requirements class is used to describe the details of how the characteristics of a SIGMET phenomenon were evaluated and is based on the observation pattern from ISO~19156:2011, Geographic information -- Observations and measurements. This requirements class is applicable to both the observation and forecasting of SIGMET phenomenon characteristics.

205-15-Ext.27.2 This requirements class restricts the content model of om:OM\_Observation such that the ``result'' of the observation describes the characteristics of a SIGMET phenomenon (including geometric extent, expected intensity change, speed and direction of motion), the ``feature of interest'' is the bounded extent of the airspace for which the SIGMET report is issued and the ``procedure'' provides the set of information as specified by WMO.

Note: SIGMETEvolvingConditionAnalysis is a subclass of SamplingObservation defined within METCE.

205-15-Ext.27.3 Instances of om:OM\_Observation with element om:type specifying \url{http://codes.wmo.int/49-2/observation-type/IWXXM/1.0/SIGMETEvolvingConditionAnalysis} shall conform to all requirements in Table~205-15-Ext.30.

205-15-Ext.27.4 Instances of om:OM\_Observation with element om:type specifying \url{http://codes.wmo.int/49-2/observation-type/IWXXM/1.0/SIGMETEvolvingConditionAnalysis} shall conform to all requirements of all relevant dependencies in Table~205-15-Ext.30.

Note: XML implementation of iwxxm:SIGMETEvolvingConditionAnalysis is dependent on:

-- OMXML {[}OGC/IS 10-025r1 Observations and Measurements 2.0 -- XML Implementation{]}.

Table~205-15-Ext.30. Requirements class xsd-sigmet-evolving-condition-analysis

\begin{longtable}[]{@{}ll@{}}
\toprule
Requirements class &\tabularnewline
\midrule
\endhead
\url{http://icao.int/iwxxm/1.1/req/xsd-sigmet-evolving-condition-analysis} &\tabularnewline
Target type & Data instance\tabularnewline
Name & SIGMET evolving condition analysis\tabularnewline
Dependency & \url{http://www.opengis.net/spec/OMXML/2.0/req/observation}, OMXML clause~7.3\tabularnewline
Dependency & \url{http://www.opengis.net/spec/OMXML/2.0/req/sampling}, OMXML clause~7.14\tabularnewline
Dependency & \url{http://www.opengis.net/spec/OMXML/2.0/req/spatialSampling}, OMXML clause~7.15\tabularnewline
Dependency & \url{http://def.wmo.int/metce/2013/req/xsd-sampling-observation}, 202-15-Ext.6\tabularnewline
Dependency & \url{http://icao.int/saf/1.1/req/xsd-airspace}, 204-15-Ext.9\tabularnewline
Dependency & \url{http://icao.int/iwxxm/1.1/req/xsd-evolving-meteorological-condition}, 205-15-Ext.26\tabularnewline
\begin{minipage}[t]{0.47\columnwidth}\raggedright
Requirement\strut
\end{minipage} & \begin{minipage}[t]{0.47\columnwidth}\raggedright
\url{http://icao.int/iwxxm/1.1/req/xsd-sigmet-evolving-condition-analysis/feature-of-interest}

The XML element //om:OM\_Observation/om:featureOfInterest shall contain a valid child element sams:SF\_SpatialSamplingFeature that describes the horizontal extent of the airspace for which the SIGMET report is issued -- a sampling surface.

The XML element //om:OM\_Observation/om:featureOfInterest/sams:SF\_SpatialSamplingFeature/sam:type shall have the value ``\url{http://www.opengis.net/def/samplingFeatureType/OGC-OM/2.0/SF_SamplingSurface}''.\strut
\end{minipage}\tabularnewline
\begin{minipage}[t]{0.47\columnwidth}\raggedright
Requirement\strut
\end{minipage} & \begin{minipage}[t]{0.47\columnwidth}\raggedright
\url{http://icao.int/iwxxm/1.1/req/xsd-sigmet-evolving-condition-analysis/sampled-feature}

The XML element //om:OM\_Observation/om:featureOfInterest/sams:SF\_ SpatialSamplingFeature/sam:sampledFeature shall contain a valid child element saf:Airspace that describes the airspace for which the SIGMET report is issued.\strut
\end{minipage}\tabularnewline
\begin{minipage}[t]{0.47\columnwidth}\raggedright
Requirement\strut
\end{minipage} & \begin{minipage}[t]{0.47\columnwidth}\raggedright
\url{http://icao.int/iwxxm/1.1/req/xsd-sigmet-evolving-condition-analysis/result}

If reported, the XML element //om:OM\_Observation/om:result shall contain a valid child element iwxxm:EvolvingMeteorologicalCondition that describes the characteristics of the SIGMET phenomenon (geometric extent, expected intensity change, speed and direction of motion).\strut
\end{minipage}\tabularnewline
\begin{minipage}[t]{0.47\columnwidth}\raggedright
Requirement\strut
\end{minipage} & \begin{minipage}[t]{0.47\columnwidth}\raggedright
\url{http://icao.int/iwxxm/1.1/req/xsd-sigmet-evolving-condition-analysis/phenomenon-time}

The XML element //om:OM\_Observation/om:phenomenonTime shall contain a valid child element gml:TimeInstant that describes the time at which the SIGMET phenomenon was observed or the time for which the characteristics of the SIGMET phenomenon have been forecast.\strut
\end{minipage}\tabularnewline
\begin{minipage}[t]{0.47\columnwidth}\raggedright
Requirement\strut
\end{minipage} & \begin{minipage}[t]{0.47\columnwidth}\raggedright
\url{http://icao.int/iwxxm/1.1/req/xsd-sigmet-evolving-condition-analysis/result-time}

The XML element //om:OM\_Observation/om:resultTime shall contain a valid child element gml:TimeInstant that describes the time at which the details of the SIGMET phenomenon were made available for dissemination.\strut
\end{minipage}\tabularnewline
\begin{minipage}[t]{0.47\columnwidth}\raggedright
Recommendation\strut
\end{minipage} & \begin{minipage}[t]{0.47\columnwidth}\raggedright
\url{http://icao.int/iwxxm/1.1/req/xsd-sigmet-evolving-condition-analysis/observed-property}

The XML attribute //om:OM\_Observation/om:observedProperty/@xlink:href should have a value that is the URI of a valid term from Code table~D-10: Significant weather phenomena.\strut
\end{minipage}\tabularnewline
\begin{minipage}[t]{0.47\columnwidth}\raggedright
Recommendation\strut
\end{minipage} & \begin{minipage}[t]{0.47\columnwidth}\raggedright
\url{http://icao.int/iwxxm/1.1/req/xsd-sigmet-evolving-condition-analysis/procedure}

The value of XML element //om:OM\_Observation/om:procedure/metce:Process/gml:description should be used to cite the Technical Regulations relating to the provision of SIGMET reports.\strut
\end{minipage}\tabularnewline
\bottomrule
\end{longtable}

Notes:

1. Dependency \url{http://www.opengis.net/spec/OMXML/2.0/req/observation} has associated conformance class\\
\url{http://www.opengis.net/spec/OMXML/2.0/conf/observation} (OMXML clause~A.1).

2. Dependency \url{http://www.opengis.net/spec/OMXML/2.0/req/sampling} has associated conformance class\\
\url{http://www.opengis.net/spec/OMXML/2.0/conf/sampling} (OMXML clause~A.12).

3. Dependency \url{http://www.opengis.net/spec/OMXML/2.0/req/spatialSampling} has associated conformance class \url{http://www.opengis.net/spec/OMXML/2.0/conf/spatialSampling} (OMXML clause~A.13).

4. Code table~D-10 is published online at \url{http://codes.wmo.int/49-2/SigWxPhenomena}.

5. The Technical Regulations relating to the provision of SIGMET reports may be cited as: ``\emph{Technical Regulations} (WMO-No.~49), Volume~II, Part~II, Appendix~6, section~1 -- Specifications related to SIGMET information''.

6. In the case of SIGMET cancellation, no characteristics of a SIGMET phenomenon are provided. In these cases, the XML element //om:OM\_Observation/om:result has no child elements and the XML attribute //om:OM\_Observation/om:result/@nilReason is used to indicate why the ``result'' is absent.

205-15-Ext.28 Requirements class: Meteorological position

205-15-Ext.28.1 This requirements class is used to describe the forecast position and extent of a specific SIGMET phenomenon, such as volcanic ash or thunderstorm, at the end of the valid period of the SIGMET report. The geometric extent of the SIGMET phenomenon is specified as a two-dimensional horizontal region with bounded vertical extent.

205-15-Ext.28.2 XML elements describing only the geometry of a SIGMET phenomenon shall conform to all requirements specified in Table~205-15-Ext.31.

205-15-Ext.28.3 XML elements describing only the geometry of a SIGMET phenomenon shall conform to all requirements of all relevant dependencies specified in Table~205-15-Ext.31.

Table~205-15-Ext.31. Requirements class xsd-meteorological-position

\begin{longtable}[]{@{}ll@{}}
\toprule
Requirements class &\tabularnewline
\midrule
\endhead
\url{http://icao.int/iwxxm/1.1/req/xsd-meteorological-position} &\tabularnewline
Target type & Data instance\tabularnewline
Name & Meteorological position\tabularnewline
Dependency & \url{http://icao.int/saf/1.1/req/xsd-airspace-volume}, 204-15-Ext.8\tabularnewline
\begin{minipage}[t]{0.47\columnwidth}\raggedright
Requirement\strut
\end{minipage} & \begin{minipage}[t]{0.47\columnwidth}\raggedright
\url{http://icao.int/iwxxm/1.1/req/xsd-meteorological-position/valid}

The content model of this element shall have a value that matches the content model of iwxxm:MeteorologicalPosition.\strut
\end{minipage}\tabularnewline
\begin{minipage}[t]{0.47\columnwidth}\raggedright
Requirement\strut
\end{minipage} & \begin{minipage}[t]{0.47\columnwidth}\raggedright
\url{http://icao.int/iwxxm/1.1/req/xsd-meteorological-position/geometry}

The geometric extent of the SIGMET phenomenon shall be reported using the XML element //iwxxm:MeteorologicalPosition/iwxxm:geometry with valid child element saf:AirspaceVolume.\strut
\end{minipage}\tabularnewline
\bottomrule
\end{longtable}

205-15-Ext.29 Requirements class: Meteorological position collection

205-15-Ext.29.1 This requirements class is used to describe a collection of geometries for a specific SIGMET phenomenon, such as volcanic ash or thunderstorm, at the end of the valid period of the SIGMET report.

205-15-Ext.29.2 XML elements describing a collection of geometries for a SIGMET phenomenon shall conform to all requirements specified in Table~205-15-Ext.32.

205-15-Ext.29.3 XML elements describing a collection of geometries for a SIGMET phenomenon shall conform to all requirements of all relevant dependencies specified in Table~205-15-Ext.32.

Table~205-15-Ext.32. Requirements class xsd-meteorological-position-collection

\begin{longtable}[]{@{}ll@{}}
\toprule
Requirements class &\tabularnewline
\midrule
\endhead
\url{http://icao.int/iwxxm/1.1/req/xsd-meteorological-position-collection} &\tabularnewline
Target type & Data instance\tabularnewline
Name & Meteorological position collection\tabularnewline
Dependency & \url{http://icao.int/iwxxm/1.1/req/xsd-meteorological-position}, 205-15-Ext.28\tabularnewline
\begin{minipage}[t]{0.47\columnwidth}\raggedright
Requirement\strut
\end{minipage} & \begin{minipage}[t]{0.47\columnwidth}\raggedright
\url{http://icao.int/iwxxm/1.1/req/xsd-meteorological-position-collection/valid}

The content model of this element shall have a value that matches the content model of iwxxm:MeteorologicalPositionCollection.\strut
\end{minipage}\tabularnewline
\begin{minipage}[t]{0.47\columnwidth}\raggedright
Requirement\strut
\end{minipage} & \begin{minipage}[t]{0.47\columnwidth}\raggedright
\url{http://icao.int/iwxxm/1.1/req/xsd-meteorological-position-collection/members}

If reported, the geometries for a specific SIGMET phenomenon shall be provided using the XML element //iwxxm:MeteorologicalPositionCollection/iwxxm:member with valid child element iwxxm:MeteorologicalPosition.\strut
\end{minipage}\tabularnewline
\bottomrule
\end{longtable}

205-15-Ext.30 Requirements class: SIGMET position analysis

205-15-Ext.30.1 This requirements class is used to describe the details of how the forecast position of a SIGMET phenomenon at the end of the valid period of a SIGMET report was evaluated.

205-15-Ext.30.2 This requirements class restricts the content model of om:OM\_Observation such that the ``result'' of the observation describes the collection of forecast positions of a specific SIGMET phenomenon, the ``feature of interest'' is the bounded extent of the airspace for which the SIGMET report is issued and the ``procedure'' provides the set of information as specified by WMO.

Note. SIGMETPositionAnalysis is a subclass of SamplingObservation defined within METCE.

205-15-Ext.30.3 Instances of om:OM\_Observation with element om:type specifying \url{http://codes.wmo.int/49-2/observation-type/IWXXM/1.0/SIGMETPositionAnalysis} shall conform to all requirements in Table~205-15-Ext.33.

205-15-Ext.30.4 Instances of om:OM\_Observation with element om:type specifying \url{http://codes.wmo.int/49-2/observation-type/IWXXM/1.0/SIGMETPositionAnalysis} shall conform to all requirements of all relevant dependencies in Table~205-15-Ext.33.

Note: XML implementation of iwxxm:SIGMETPositionAnalysis is dependent on:

-- OMXML {[}OGC/IS 10-025r1 Observations and Measurements 2.0 -- XML Implementation{]}.

Table~205-15-Ext.33. Requirements class xsd-sigmet-position-analysis

\begin{longtable}[]{@{}ll@{}}
\toprule
Requirements class &\tabularnewline
\midrule
\endhead
\url{http://icao.int/iwxxm/1.1/req/xsd-sigmet-position-analysis} &\tabularnewline
Target type & Data instance\tabularnewline
Name & SIGMET position analysis\tabularnewline
Dependency & \url{http://www.opengis.net/spec/OMXML/2.0/req/observation}, OMXML clause~7.3\tabularnewline
Dependency & \url{http://www.opengis.net/spec/OMXML/2.0/req/sampling}, OMXML clause~7.14\tabularnewline
Dependency & \url{http://www.opengis.net/spec/OMXML/2.0/req/spatialSampling}, OMXML clause~7.15\tabularnewline
Dependency & \url{http://def.wmo.int/metce/2013/req/xsd-sampling-observation}, 202-15-Ext.6\tabularnewline
Dependency & \url{http://icao.int/saf/1.1/req/xsd-airspace}, 204-15-Ext.9\tabularnewline
Dependency & \url{http://icao.int/iwxxm/1.1/req/xsd-meteorological-position-collection}, 205-15-Ext.29\tabularnewline
\begin{minipage}[t]{0.47\columnwidth}\raggedright
Requirement\strut
\end{minipage} & \begin{minipage}[t]{0.47\columnwidth}\raggedright
\url{http://icao.int/iwxxm/1.1/req/xsd-sigmet-position-analysis/feature-of-interest}

The XML element //om:OM\_Observation/om:featureOfInterest shall contain a valid child element sams:SF\_SpatialSamplingFeature that describes the horizontal extent of the airspace for which the SIGMET report is issued -- a sampling surface.

The XML element //om:OM\_Observation/om:featureOfInterest/sams:SF\_SpatialSamplingFeature/sam:type shall have the value ``\url{http://www.opengis.net/def/samplingFeatureType/OGC-OM/2.0/SF_SamplingSurface}''.\strut
\end{minipage}\tabularnewline
\begin{minipage}[t]{0.47\columnwidth}\raggedright
Requirement\strut
\end{minipage} & \begin{minipage}[t]{0.47\columnwidth}\raggedright
\url{http://icao.int/iwxxm/1.1/req/xsd-sigmet-position-analysis/sampled-feature}

The XML element //om:OM\_Observation/om:featureOfInterest/sams:SF\_ SpatialSamplingFeature/sam:sampledFeature shall contain a valid child element saf:Airspace that describes the airspace for which the SIGMET report is issued.\strut
\end{minipage}\tabularnewline
\begin{minipage}[t]{0.47\columnwidth}\raggedright
Requirement\strut
\end{minipage} & \begin{minipage}[t]{0.47\columnwidth}\raggedright
\url{http://icao.int/iwxxm/1.1/req/xsd-sigmet-position-analysis/result}

The XML element //om:OM\_Observation/om:result shall contain a valid child element iwxxm:MeteorologicalPositionCollection that describes the collection of forecast positions of a specific SIGMET phenomenon.\strut
\end{minipage}\tabularnewline
\begin{minipage}[t]{0.47\columnwidth}\raggedright
Requirement\strut
\end{minipage} & \begin{minipage}[t]{0.47\columnwidth}\raggedright
\url{http://icao.int/iwxxm/1.1/req/xsd-sigmet-position-analysis/phenomenon-time}

The XML element //om:OM\_Observation/om:phenomenonTime shall contain a valid child element gml:TimeInstant that describes the time for which the collection of positions of the SIGMET phenomenon have been forecast.\strut
\end{minipage}\tabularnewline
\begin{minipage}[t]{0.47\columnwidth}\raggedright
Requirement\strut
\end{minipage} & \begin{minipage}[t]{0.47\columnwidth}\raggedright
\url{http://icao.int/iwxxm/1.1/req/xsd-sigmet-position-analysis/result-time}

The XML element //om:OM\_Observation/om:resultTime shall contain a valid child element gml:TimeInstant that describes the time at which the details of the SIGMET phenomenon positions were made available for dissemination.\strut
\end{minipage}\tabularnewline
\begin{minipage}[t]{0.47\columnwidth}\raggedright
Recommendation\strut
\end{minipage} & \begin{minipage}[t]{0.47\columnwidth}\raggedright
\url{http://icao.int/iwxxm/1.1/req/xsd-sigmet-position-analysis/observed-property}

The XML attribute //om:OM\_Observation/om:observedProperty/@xlink:href should have a value that is the URI of a valid term from Code table~D-10: Significant weather phenomena.\strut
\end{minipage}\tabularnewline
\begin{minipage}[t]{0.47\columnwidth}\raggedright
Recommendation\strut
\end{minipage} & \begin{minipage}[t]{0.47\columnwidth}\raggedright
\url{http://icao.int/iwxxm/1.1/req/xsd-sigmet-position-analysis/procedure}

The value of XML element //om:OM\_Observation/om:procedure/metce:Process/gml:description should be used to cite the Technical Regulations relating to the provision of SIGMET reports.\strut
\end{minipage}\tabularnewline
\bottomrule
\end{longtable}

Notes:

1. Dependency \url{http://www.opengis.net/spec/OMXML/2.0/req/observation} has associated conformance class\\
\url{http://www.opengis.net/spec/OMXML/2.0/conf/observation} (OMXML clause~A.1).

2. Dependency \url{http://www.opengis.net/spec/OMXML/2.0/req/sampling} has associated conformance class\\
\url{http://www.opengis.net/spec/OMXML/2.0/conf/sampling} (OMXML clause~A.12).

3. Dependency \url{http://www.opengis.net/spec/OMXML/2.0/req/spatialSampling} has associated conformance class \url{http://www.opengis.net/spec/OMXML/2.0/conf/spatialSampling} (OMXML clause~A.13).

4. Code table~D-10 is published online at \url{http://codes.wmo.int/49-2/SigWxPhenomena}.

5. The Technical Regulations relating to the provision of SIGMET reports may be cited as: ``\emph{Technical Regulations} (WMO-No.~49), Volume~II, Part~II, Appendix~6, section~1 -- Specifications related to SIGMET information''.

205-15-Ext.31 Requirements class: SIGMET

205-15-Ext.31.1 This requirements class is used to describe the SIGMET report within which the characteristics of a specific SIGMET phenomenon are described.

Note: The reporting requirements for SIGMETs are specified in the \emph{Technical Regulations} (WMO-No.~49), Volume~II, Part~II, Appendix~6, section~1.

205-15-Ext.31.2 XML elements describing SIGMET reports shall conform to all requirements specified in Table~205-15-Ext.34.

205-15-Ext.31.3 XML elements describing SIGMET reports shall conform to all requirements of all relevant dependencies specified in Table~205-15-Ext.34.

Table~205-15-Ext.34. Requirements class xsd-sigmet

\begin{longtable}[]{@{}ll@{}}
\toprule
Requirements class &\tabularnewline
\midrule
\endhead
\url{http://icao.int/iwxxm/1.1/req/xsd-sigmet} &\tabularnewline
Target type & Data instance\tabularnewline
Name & SIGMET\tabularnewline
Dependency & \url{http://icao.int/saf/1.1/req/xsd-aeronautical-service-provision-units}, 204-15-Ext.7\tabularnewline
Dependency & \url{http://icao.int/iwxxm/1.1/req/xsd-sigmet-evolving-condition-analysis}, 205-15-Ext.27\tabularnewline
Dependency & \url{http://icao.int/iwxxm/1.1/req/xsd-sigmet-position-analysis}, 205-15-Ext.30\tabularnewline
\begin{minipage}[t]{0.47\columnwidth}\raggedright
Requirement\strut
\end{minipage} & \begin{minipage}[t]{0.47\columnwidth}\raggedright
\url{http://icao.int/iwxxm/1.1/req/xsd-sigmet/valid}

The content model of this element shall have a value that matches the content model of iwxxm:SIGMET.\strut
\end{minipage}\tabularnewline
\begin{minipage}[t]{0.47\columnwidth}\raggedright
Requirement\strut
\end{minipage} & \begin{minipage}[t]{0.47\columnwidth}\raggedright
\url{http://icao.int/iwxxm/1.1/req/xsd-sigmet/status}

The status of the SIGMET shall be indicated using the XML attribute @status with the value being one of the enumeration: ``NORMAL'' or ``CANCELLATION''.\strut
\end{minipage}\tabularnewline
\begin{minipage}[t]{0.47\columnwidth}\raggedright
Requirement\strut
\end{minipage} & \begin{minipage}[t]{0.47\columnwidth}\raggedright
\url{http://icao.int/iwxxm/1.1/req/xsd-sigmet/issuing-air-traffic-services-unit}

The air traffic services unit responsible for the subject airspace shall be indicated using the XML element //iwxxm:issuingAirTrafficServicesUnit with a valid child element saf:Unit.\strut
\end{minipage}\tabularnewline
\begin{minipage}[t]{0.47\columnwidth}\raggedright
Requirement\strut
\end{minipage} & \begin{minipage}[t]{0.47\columnwidth}\raggedright
\url{http://icao.int/iwxxm/1.1/req/xsd-sigmet/originating-meteorological-watch-office}

The meteorological watch office that originated the SIGMET report shall be indicated using the XML element //iwxxm:originatingMeteorologicalWatchOffice with a valid child element saf:Unit.

The value of XML element //iwxxm:issuingAirTrafficServicesUnit/saf:Unit/saf:type shall be ``MWO'' (Meteorological Watch Office).\strut
\end{minipage}\tabularnewline
\begin{minipage}[t]{0.47\columnwidth}\raggedright
Requirement\strut
\end{minipage} & \begin{minipage}[t]{0.47\columnwidth}\raggedright
\url{http://icao.int/iwxxm/1.1/req/xsd-sigmet/sequence-number}

The sequence number of this SIGMET report shall be indicated using XML element\\
//iwxxm:sequenceNumber.\strut
\end{minipage}\tabularnewline
\begin{minipage}[t]{0.47\columnwidth}\raggedright
Requirement\strut
\end{minipage} & \begin{minipage}[t]{0.47\columnwidth}\raggedright
\url{http://icao.int/iwxxm/1.1/req/xsd-sigmet/valid-period}

The period of validity of this SIGMET report shall be indicated using XML element\\
//iwxxm:validPeriod with valid child element gml:TimePeriod.\strut
\end{minipage}\tabularnewline
\begin{minipage}[t]{0.47\columnwidth}\raggedright
Requirement\strut
\end{minipage} & \begin{minipage}[t]{0.47\columnwidth}\raggedright
\url{http://icao.int/iwxxm/1.1/req/xsd-sigmet/phenomenon}

The XML attribute //iwxxm:phenomenon/@xlink:href shall have a value that is the URI of a valid term from Code table~D-10: Significant weather phenomena.\strut
\end{minipage}\tabularnewline
\begin{minipage}[t]{0.47\columnwidth}\raggedright
Requirement\strut
\end{minipage} & \begin{minipage}[t]{0.47\columnwidth}\raggedright
\url{http://icao.int/iwxxm/1.1/req/xsd-sigmet/unique-subject-airspace}

All SIGMET analyses included in the report shall refer to the same airspace. All values of XML element //om:OM\_Observation/om:featureOfInterest/sams:SF\_SpatialSamplingFeature/sam:sampledFeature/saf:Airspace/gml:identifier within the SIGMET shall be identical.\strut
\end{minipage}\tabularnewline
\begin{minipage}[t]{0.47\columnwidth}\raggedright
Requirement\strut
\end{minipage} & \begin{minipage}[t]{0.47\columnwidth}\raggedright
\url{http://icao.int/iwxxm/1.1/req/xsd-sigmet/analysis}

If reported, XML element //iwxxm:analysis shall contain a valid child element\\
//om:OM\_Observation of type SIGMETEvolvingConditionAnalysis. The value of XML attribute //iwxxm:analysis/om:OM\_Observation/om:type/@xlink:href shall be the URI ``\url{http://codes.wmo.int/49-2/observation-type/IWXXM/1.0/SIGMETEvolvingConditionAnalysis}''.\strut
\end{minipage}\tabularnewline
\begin{minipage}[t]{0.47\columnwidth}\raggedright
Requirement\strut
\end{minipage} & \begin{minipage}[t]{0.47\columnwidth}\raggedright
\url{http://icao.int/iwxxm/1.1/req/xsd-sigmet/forecast-position-analysis}

If reported, the forecast position of the phenomenon shall be reported using the XML element //iwxxm:forecastPositionAnalysis with valid child element\\
//om:OM\_Observation of type SIGMETPositionAnalysis.

The value of XML attribute //iwxxm:forecastPositionAnalysis/om:OM\_Observation/om:type/@xlink:href shall be the URI ``\url{http://codes.wmo.int/49-2/observation-type/IWXXM/1.0/SIGMETPositionAnalysis}''.\strut
\end{minipage}\tabularnewline
\begin{minipage}[t]{0.47\columnwidth}\raggedright
Requirement\strut
\end{minipage} & \begin{minipage}[t]{0.47\columnwidth}\raggedright
\url{http://icao.int/iwxxm/1.1/req/xsd-sigmet/status-normal}

If the status of the SIGMET is ``NORMAL'' (as specified by XML attribute @status), then:

(i) The characteristics of the SIGMET phenomenon shall be reported using one or more of the XML element //iwxxm:analysis;

(ii) Each XML element //iwxxm:analysis shall contain a valid element //iwxxm:analysis/om:OM\_Observation/om:result/iwxxm:EvolvingMeteorologicalCondition within which the characteristics of the SIGMET phenomenon are described;

(iii) The XML element //iwxxm:cancelledSequenceNumber shall be absent; and

(iv) The XML element //iwxxm:cancelledValidPeriod shall be absent.\strut
\end{minipage}\tabularnewline
\begin{minipage}[t]{0.47\columnwidth}\raggedright
Requirement\strut
\end{minipage} & \begin{minipage}[t]{0.47\columnwidth}\raggedright
\url{http://icao.int/iwxxm/1.1/req/xsd-sigmet/status-cancellation}

If the status of the SIGMET is ``CANCELLATION'' (as specified by XML attribute\\
@status), then:

(i) The details of the airspace for which the SIGMET has been cancelled shall be provided by a single instance of XML element //iwxxm:analysis;

(ii) The XML element //iwxxm:analysis/om:OM\_Observation/om:result shall have no child elements and XML attribute //iwxxm:analysis/om:OM\_Observation/om:result/@nilReason shall provide an appropriate nil reason;

(iii) The value of XML element //iwxxm:cancelledSequenceNumber shall indicate the sequence number of the SIGMET that has been cancelled; and

(iv) The XML element //iwxxm:cancelledValidPeriod shall contain a valid child element gml:TimePeriod that indicates the validity period of the SIGMET that has been cancelled.\strut
\end{minipage}\tabularnewline
\begin{minipage}[t]{0.47\columnwidth}\raggedright
Recommendation\strut
\end{minipage} & \begin{minipage}[t]{0.47\columnwidth}\raggedright
\url{http://icao.int/iwxxm/1.1/req/xsd-sigmet/issuing-air-traffic-services-unit-type}

The value of XML element //iwxxm:SIGMET/iwxxm:issuingAirTrafficServicesUnit/saf:Unit/saf:type should be one of the enumeration: ``ATSU'' (Air Traffic Services Unit) or ``FIC'' (Flight Information Centre).\strut
\end{minipage}\tabularnewline
\begin{minipage}[t]{0.47\columnwidth}\raggedright
Recommendation\strut
\end{minipage} & \begin{minipage}[t]{0.47\columnwidth}\raggedright
\url{http://icao.int/iwxxm/1.1/req/xsd-sigmet/valid-period-start-matches-result-time}

The start time of the validity period of the SIGMET report (expressed using XML element //iwxxm:validPeriod/gml:TimePeriod/gml:beginPosition) should match the result time of each SIGMET analysis included within the report (expressed using XML element //om:OM\_Observation/om:resultTime/gml:TimeInstant/gml:timePosition).\strut
\end{minipage}\tabularnewline
\begin{minipage}[t]{0.47\columnwidth}\raggedright
Recommendation\strut
\end{minipage} & \begin{minipage}[t]{0.47\columnwidth}\raggedright
\url{http://icao.int/iwxxm/1.1/req/xsd-sigmet/valid-time-includes-all-phenomenon-times}

The observation and/or forecast times of all SIGMET analyses and, if reported, forecast position analyses included in the report (specified by XML element\\
//om:OM\_Observation/om:phenomenonTime/*) should occur within the valid time period of the SIGMET (specified by XML element //iwxxm:validPeriod/gml:TimePeriod).\strut
\end{minipage}\tabularnewline
\begin{minipage}[t]{0.47\columnwidth}\raggedright
Recommendation\strut
\end{minipage} & \begin{minipage}[t]{0.47\columnwidth}\raggedright
\url{http://icao.int/iwxxm/1.1/req/xsd-sigmet/7-point-definition-of-airspace-volume}

The horizontal extent of any airspace volumes enclosing a SIGMET phenomenon (reported using XML element //om:OM\_Observation/om:result/*/iwxxm:geometry/saf:AirspaceVolume/saf:horizontalProjection) should use no more than seven points to define the bounding polygon.\strut
\end{minipage}\tabularnewline
\bottomrule
\end{longtable}

Notes:

1. Requirements relating to sequence numbers within SIGMET reports are specified in the \emph{Technical Regulations} (WMO-No.~49), Volume~II, Part~II, Appendix~6, 1.1.3.

2. Requirements for reporting the SIGMET phenomenon are specified in the \emph{Technical Regulations} (WMO-No.~49), Volume~II, Part~II, Appendix~6, 1.1.4.

3. A forecast position may be provided for a volcanic ash cloud, the centre of a tropical cyclone or other hazardous phenomena at the end of the validity period of the SIGMET message.

4. Within an XML encoded SIGMET, it is likely that only one instance of saf:Airspace will physically be present; subsequent assertions about the airspace may use xlinks to refer to the previously defined saf:Airspace element in order to keep the XML document size small. As such, validation of requirement \url{http://icao.int/iwxxm/1.1/req/xsd-sigmet/unique-subject-airspace} is applied once any xlinks, if used, have been resolved.

5. Code table~D-1 provides a set of nil-reason codes and is published at \url{http://codes.wmo.int/common/nil}.

6. Code table~D-10 is published online at \url{http://codes.wmo.int/49-2/SigWxPhenomena}.

205-15-Ext.32 Requirements class: Volcanic ash SIGMET

205-15-Ext.32.1 This requirements class is used to describe the volcanic ash (VA) SIGMET report, which includes additional information about the source volcano and the forecast position of the volcanic ash at the end of the validity period of the SIGMET.

205-15-Ext.32.2 XML elements describing VA SIGMET reports shall conform to all requirements specified in Table~205-15-Ext.35.

205-15-Ext.32.3 XML elements describing VA SIGMET reports shall conform to all requirements of all relevant dependencies specified in Table~205-15-Ext.35.

Table~205-15-Ext.35. Requirements class xsd-volcanic-ash-sigmet

\begin{longtable}[]{@{}ll@{}}
\toprule
Requirements class &\tabularnewline
\midrule
\endhead
\url{http://icao.int/iwxxm/1.1/req/xsd-volcanic-ash-sigmet} &\tabularnewline
Target type & Data instance\tabularnewline
Name & Volcanic ash SIGMET\tabularnewline
Dependency & \url{http://def.wmo.int/metce/2013/req/xsd-erupting-volcano}, 202-15-Ext.8\tabularnewline
Dependency & \url{http://icao.int/iwxxm/1.1/req/xsd-sigmet}, 205-15-Ext.31\tabularnewline
\begin{minipage}[t]{0.47\columnwidth}\raggedright
Requirement\strut
\end{minipage} & \begin{minipage}[t]{0.47\columnwidth}\raggedright
\url{http://icao.int/iwxxm/1.1/req/xsd-volcanic-ash-sigmet/valid}

The content model of this element shall have a value that matches the content model of iwxxm:VolcanicAshSIGMET.\strut
\end{minipage}\tabularnewline
\begin{minipage}[t]{0.47\columnwidth}\raggedright
Requirement\strut
\end{minipage} & \begin{minipage}[t]{0.47\columnwidth}\raggedright
\url{http://icao.int/iwxxm/1.1/req/xsd-volcanic-ash-sigmet/source-volcano}

Details of the volcano that is the source of the volcanic ash shall be reported using the XML element //iwxxm:eruptingvolcano with valid child element metce:Volcano (or element in the substitution group of metce:Volcano).\strut
\end{minipage}\tabularnewline
\begin{minipage}[t]{0.47\columnwidth}\raggedright
Requirement\strut
\end{minipage} & \begin{minipage}[t]{0.47\columnwidth}\raggedright
\url{http://icao.int/iwxxm/1.1/req/xsd-volcanic-ash-sigmet/phenomenon}

The XML attribute //iwxxm:phenomenon/@xlink:href shall have a value that is the URI ``\url{http://codes.wmo.int/49-2/SigWxPhenomena/VA}''.\strut
\end{minipage}\tabularnewline
\bottomrule
\end{longtable}

205-15-Ext.33 Requirements class: Tropical cyclone SIGMET

205-15-Ext.33.1 This requirements class is used to describe the tropical cyclone (TC) SIGMET report, which includes additional information about the tropical cyclone itself and the forecast position of the tropical cyclone at the end of the validity period of the SIGMET.

205-15-Ext.33.2 XML elements describing TC SIGMET reports shall conform to all requirements specified in Table~205-15-Ext.36.

205-15-Ext.33.3 XML elements describing TC SIGMET reports shall conform to all requirements of all relevant dependencies specified in Table~205-15-Ext.36.

Table~205-15-Ext.36. Requirements class xsd-tropical-cyclone-sigmet

\begin{longtable}[]{@{}ll@{}}
\toprule
Requirements class &\tabularnewline
\midrule
\endhead
\url{http://icao.int/iwxxm/1.1/req/xsd-tropical-cyclone-sigmet} &\tabularnewline
Target type & Data instance\tabularnewline
Name & Tropical cyclone SIGMET\tabularnewline
Dependency & \url{http://def.wmo.int/metce/2013/req/xsd-tropical-cyclone}, 202-15-Ext.9\tabularnewline
Dependency & \url{http://icao.int/iwxxm/1.1/req/xsd-sigmet}, 205-15-Ext.31\tabularnewline
\begin{minipage}[t]{0.47\columnwidth}\raggedright
Requirement\strut
\end{minipage} & \begin{minipage}[t]{0.47\columnwidth}\raggedright
\url{http://icao.int/iwxxm/1.1/req/xsd-tropical-cyclone-sigmet/valid}

The content model of this element shall have a value that matches the content model of iwxxm:TropicalCycloneSIGMET.\strut
\end{minipage}\tabularnewline
\begin{minipage}[t]{0.47\columnwidth}\raggedright
Requirement\strut
\end{minipage} & \begin{minipage}[t]{0.47\columnwidth}\raggedright
\url{http://icao.int/iwxxm/1.1/req/xsd-tropical-cyclone-sigmet/cyclone}

Details of the tropical cyclone shall be reported using the XML element\\
//iwxxm:tropicalCyclone with valid child element metce:TropicalCyclone.\strut
\end{minipage}\tabularnewline
\begin{minipage}[t]{0.47\columnwidth}\raggedright
Requirement\strut
\end{minipage} & \begin{minipage}[t]{0.47\columnwidth}\raggedright
\url{http://icao.int/iwxxm/1.1/req/xsd-tropical-cyclone-sigmet/phenomenon}

The XML attribute //iwxxm:phenomenon/@xlink:href shall have a value that is the URI ``\url{http://codes.wmo.int/49-2/SigWxPhenomena/TC}''.\strut
\end{minipage}\tabularnewline
\bottomrule
\end{longtable}

FM~205-16 IWXXM-XML ICAO Meteorological Information Exchange Model (IWXXM 2.1)

205-16.1 Scope

205-16.1.1 IWXXM-XML shall be used to represent observations and forecasts, and reports thereof, for international civil aviation, as specified by the \emph{Technical Regulations} (WMO-No.~49), Volume~II -- Meteorological Service for International Air Navigation.

205-16.1.2 IWXXM-XML includes provision for aerodrome routine meteorological reports (METAR), aerodrome special meteorological reports (SPECI), aerodrome forecast (TAF) reports, SIGMET information, AIRMET information, Tropical Cyclone Advisory and Volcanic Ash Advisory.

Notes:

1. SIGMET information is information issued by a meteorological watch office concerning the occurrence or expected occurrence of specified en-route weather phenomena that may affect the safety of aircraft operations.

2. AIRMET Information is information issued by a meteorological watch office concerning the occurrence or expected occurrence of specified en-route weather phenomena that may affect the safety of low-level aircraft operations and which was not already included in the forecast issued for low-level flights in the flight information region concerned or sub-area thereof.

3. Tropical Cyclone Advisory Information is information issued by a Tropical Cyclone Advisory Centre (TCAC) regarding the position, forecast direction and speed of movement, central pressure and maximum surface wind of tropical cyclones.

4. Volcanic Ash Advisory Information is information issued by a Volcanic Ash Advisory Centre (VAAC) regarding the lateral and vertical extent and forecast movement of volcanic ash in the atmosphere following volcanic eruptions.

205-16.1.3 The requirements classes defined in IWXXM-XML are listed in Table~205-16.1.

Table~205-16.1. Requirements classes defined in IWXXM-XML

\begin{longtable}[]{@{}ll@{}}
\toprule
Requirements classes &\tabularnewline
\midrule
\endhead
Requirements class & \url{http://icao.int/iwxxm/2.1/req/xsd-meteorological-aerodrome-observation-report}, 205-16.4\tabularnewline
Requirements class & \url{http://icao.int/iwxxm/2.1/req/xsd-speci}, 205-16.5\tabularnewline
Requirements class & \url{http://icao.int/iwxxm/2.1/req/xsd-metar}, 205-16.6\tabularnewline
Requirements class & \url{http://icao.int/iwxxm/2.1/req/xsd-meteorological-aerodrome-trend-forecast-record}, 205-16.7\tabularnewline
Requirements class & \url{http://icao.int/iwxxm/2.1/req/xsd-meteorological-aerodrome-observation-record}, 205-16.8\tabularnewline
Requirements class & \url{http://icao.int/iwxxm/2.1/req/xsd-aerodrome-runway-state}, 205-16.9\tabularnewline
Requirements class & \url{http://icao.int/iwxxm/2.1/req/xsd-aerodrome-runway-visual-range}, 205-16.10\tabularnewline
Requirements class & \url{http://icao.int/iwxxm/2.1/req/xsd-aerodrome-sea-state}, 205-16.11\tabularnewline
Requirements class & \url{http://icao.int/iwxxm/2.1/req/xsd-aerodrome-wind-shear}, 205-16.12\tabularnewline
Requirements class & \url{http://icao.int/iwxxm/2.1/req/xsd-aerodrome-observed-clouds}, 205-16.13\tabularnewline
Requirements class & \url{http://icao.int/iwxxm/2.1/req/xsd-aerodrome-surface-wind}, 205-16.14\tabularnewline
Requirements class & \url{http://icao.int/iwxxm/2.1/req/xsd-aerodrome-horizontal-visibility}, 205-16.15\tabularnewline
Requirements class & \url{http://icao.int/iwxxm/2.1/req/xsd-taf}, 205-16.16\tabularnewline
Requirements class & \url{http://icao.int/iwxxm/2.1/req/xsd-meteorological-aerodrome-forecast-record}, 205-16.17\tabularnewline
Requirements class & \vtop{\hbox{\strut \url{http://icao.int/iwxxm/2.1/req/xsd-aerodrome-air-temperature-forecast},}\hbox{\strut 205-16.18}}\tabularnewline
Requirements class & \url{http://icao.int/iwxxm/2.1/req/xsd-sigmet-position}, 205-16.19\tabularnewline
Requirements class & \url{http://icao.int/iwxxm/2.1/req/xsd-sigmet-position-collection}, 205-16.20\tabularnewline
Requirements class & \url{http://icao.int/iwxxm/2.1/req/xsd-sigmet}, 205-16.21\tabularnewline
Requirements class & \url{http://icao.int/iwxxm/2.1/req/xsd-sigmet-evolving-condition-collection}, 205‑16.22\tabularnewline
Requirements class & \url{http://icao.int/iwxxm/2.1/req/xsd-sigmet-evolving-condition}, 205-16.23\tabularnewline
Requirements class & \url{http://icao.int/iwxxm/2.1/req/xsd-tropical-cyclone-sigmet}, 205-16.24\tabularnewline
Requirements class & \href{http://icao.int/iwxxm/1.1/req/xsd-volcanic-ash-sigmet}{http://icao.int/iwxxm/2.1/req/xsd-volcanic-ash-sigmet}, 205-16.25\tabularnewline
Requirements class & \url{http://icao.int/iwxxm/2.1/req/xsd-airmet}, 205-16.26\tabularnewline
Requirements class & \url{http://icao.int/iwxxm/2.1/req/xsd-airmet-evolving-condition-collection}, 205‑16.27\tabularnewline
Requirements class & \url{http://icao.int/iwxxm/2.1/req/xsd-airmet-evolving-condition}, 205-16.28\tabularnewline
Requirements class & \url{http://icao.int/iwxxm/2.1/req/xsd-tropical-cyclone-advisory}, 205-16.29\tabularnewline
Requirements class & \url{http://icao.int/iwxxm/2.1/req/xsd-tropical-cyclone-observed-conditions}, 205‑16.30\tabularnewline
Requirements class & \url{http://icao.int/iwxxm/2.1/req/xsd-tropical-cyclone-forecast-conditions}, 205‑16.31\tabularnewline
Requirements class & \url{http://icao.int/iwxxm/2.1/req/xsd-volcanic-ash-advisory}, 205-16.32\tabularnewline
Requirements class & \url{http://icao.int/iwxxm/2.1/req/xsd-volcanic-ash-conditions}, 205-16.33\tabularnewline
Requirements class & \url{http://icao.int/iwxxm/2.1/req/xsd-volcanic-ash-cloud}, 205-16.34\tabularnewline
Requirements class & \url{http://icao.int/iwxxm/2.1/req/xsd-report}, 205-16.35\tabularnewline
Requirements class & \url{http://icao.int/iwxxm/2.1/req/xsd-aerodrome-cloud-forecast}, 205-16.36\tabularnewline
Requirements class & \url{http://icao.int/iwxxm/2.1/req/xsd-aerodrome-surface-wind-forecast}, 205-16.37\tabularnewline
Requirements class & \url{http://icao.int/iwxxm/2.1/req/xsd-aerodrome-surface-wind-trend-forecast}, 205-16.38\tabularnewline
Requirements class & \url{http://icao.int/iwxxm/2.1/req/xsd-cloud-layer}, 205-16.39\tabularnewline
Requirements class & \url{http://icao.int/iwxxm/2.1/req/xsd-angle-with-nil-reason}, 205-16.40\tabularnewline
Requirements class & \url{http://icao.int/iwxxm/2.1/req/xsd-distance-with-nil-reason}, 205-16.41\tabularnewline
Requirements class & \url{http://icao.int/iwxxm/2.1/req/xsd-length-with-nil-reason}, 205-16.42\tabularnewline
\bottomrule
\end{longtable}

205-16.2 XML schema for IWXXM-XML

Representations of information in IWXXM-XML shall declare the XML namespaces listed in Table~205-16.2 and Table~205-16.3.

Notes:

1. Additional namespace declarations may be required depending on the XML elements used within IWXXM‑XML.

2. The XML schema is packaged in five XML schema documents (XSD) describing one XML namespace: \url{http://icao.int/iwxxm/2.1}.

3. Schematron schemas providing additional constraints are provided as an external file to the XSD defining IWXXM-XML. The canonical location of this file is \url{http://schemas.wmo.int/iwxxm/2.1/rule/iwxxm.sch}.

Table~205-16.2. XML namespaces defined for IWXXM-XML

\begin{longtable}[]{@{}lll@{}}
\toprule
XML namespace & Default namespace prefix & Canonical location of all-components schema document\tabularnewline
\midrule
\endhead
\url{http://icao.int/iwxxm/2.1} & iwxxm & \url{http://schemas.wmo.int/iwxxm/2.1/iwxxm.xsd}\tabularnewline
\bottomrule
\end{longtable}

Table~205-16.3. External XML namespaces used in IWXXM-XML

\begin{longtable}[]{@{}llll@{}}
\toprule
Standard & XML namespace & Default namespace prefix & Canonical location of all-components schema document\tabularnewline
\midrule
\endhead
XML schema & \url{http://www.w3.org/2001/XMLSchema} & xs &\tabularnewline
Schematron & \url{http://purl.oclc.org/dsdl/schematron} & sch &\tabularnewline
XSLT v2 & \url{http://www.w3.org/1999/XSL/Transform} & xsl &\tabularnewline
XML Linking Language & \url{http://www.w3.org/1999/xlink} & xlink & \url{http://www.w3.org/1999/xlink.xsd}\tabularnewline
ISO 19136:2007 GML & \url{http://www.opengis.net/gml/3.2} & gml & \url{http://schemas.opengis.net/gml/3.2.1/gml.xsd}\tabularnewline
ISO/TS 19139:2007 metadata XML implementation & \url{http://www.isotc211.org/2005/gmd} & gmd & \url{http://standards.iso.org/ittf/PubliclyAvailableStandards/ISO_19139_Schemas/gmd/gmd.xsd}\tabularnewline
OGC OMXML & \url{http://www.opengis.net/om/2.0} & om & \url{http://schemas.opengis.net/om/2.0/observation.xsd}\tabularnewline
OGC OMXML & \url{http://www.opengis.net/sampling/2.0} & sam & \url{http://schemas.opengis.net/sampling/2.0/samplingFeature.xsd}\tabularnewline
OGC OMXML & \url{http://www.opengis.net/samplingSpatial/2.0} & sams & \url{http://schemas.opengis.net/samplingSpatial/2.0/spatialSamplingFeature.xsd}\tabularnewline
\vtop{\hbox{\strut FM 202-16}\hbox{\strut METCE-XML}} & \url{http://def.wmo.int/metce/2013} & metce & \url{http://schemas.wmo.int/metce/1.2/metce.xsd}\tabularnewline
FM 203-15 Ext. OPM-XML & \url{http://def.wmo.int/opm/2013} & opm & \url{http://schemas.wmo.int/opm/1.1/opm.xsd}\tabularnewline
AIXM 5.1.1 & \url{http://www.aixm.aero/schema/5.1.1/} & aixm & \url{http://www.aixm.aero/schema/5.1.1/AIXM_Features.xsd}\tabularnewline
\bottomrule
\end{longtable}

205-16.3 Virtual typing

In accordance with OMXML (clause~7.2), the specialization of OM\_Observation is provided through Schematron restriction. The om:type element shall be used to specify the type of OM\_Observation that is being encoded using the URI for the corresponding observation type listed in Code table~D-4.

Notes:

1. Code table~D-4 is described in Appendix~A.

2. Code table~D-4 is published online at \url{http://codes.wmo.int/49-2/observation-type/IWXXM/2.1}.

3. The URI for each observation type is composed by appending the \emph{notation} to the \emph{code-space}. As an example, the URI of MeteorologicalAerodromeForecast is \url{http://codes.wmo.int/49-2/observation-type/IWXXM/2.1/MeteorologicalAerodromeForecast}.

4. Each URI will resolve to provide further information about the associated observation type.

205-16.4 Requirements class: Meteorological aerodrome observation report

205-16.4.1 This requirements class is used to describe the report within which meteorological aerodrome observations, and optionally one or more trend forecasts, are provided.

Note: The reporting requirements for routine and special meteorological aerodrome reports are specified in the \emph{Technical Regulations} (WMO-No.~49), Volume~II, Part~II, Appendix~3 and Appendix~5, section~2.

205-16.4.2 XML elements describing routine or special meteorological aerodrome reports shall conform to all requirements specified in Table~205-16.4.

205-16.4.3 XML elements describing routine or special meteorological aerodrome reports shall conform to all requirements of all relevant dependencies specified in Table~205-16.4.

Table~205-16.4. Requirements class xsd-meteorological-aerodrome-observation-report

\begin{longtable}[]{@{}ll@{}}
\toprule
Requirements class &\tabularnewline
\midrule
\endhead
\url{http://icao.int/iwxxm/2.1/req/xsd-meteorological-aerodrome-observation-report} &\tabularnewline
Target type & Data instance\tabularnewline
Name & Meteorological aerodrome observation report\tabularnewline
\begin{minipage}[t]{0.47\columnwidth}\raggedright
Requirement\strut
\end{minipage} & \begin{minipage}[t]{0.47\columnwidth}\raggedright
\url{http://icao.int/iwxxm/2.1/req/xsd-meteorological-aerodrome-observation-report/valid}

The content model of this element shall have a value that matches the content model of iwxxm:MeteorologicalAerodromeObservationReport.\strut
\end{minipage}\tabularnewline
\begin{minipage}[t]{0.47\columnwidth}\raggedright
Requirement\strut
\end{minipage} & \begin{minipage}[t]{0.47\columnwidth}\raggedright
\url{http://icao.int/iwxxm/2.1/req/xsd-meteorological-aerodrome-observation-report/status}

The status of the report shall be indicated using the XML attribute @status with the value being one of the enumeration: ``NORMAL'', ``MISSING'' or ``CORRECTION''.\strut
\end{minipage}\tabularnewline
\begin{minipage}[t]{0.47\columnwidth}\raggedright
Requirement\strut
\end{minipage} & \begin{minipage}[t]{0.47\columnwidth}\raggedright
\url{http://icao.int/iwxxm/2.1/req/xsd-meteorological-aerodrome-observation-report/automated-station}

If the meteorological aerodrome observation included within the report has been generated by an automated system, the value of XML attribute @automatedStation shall be set to ``true''.\strut
\end{minipage}\tabularnewline
\begin{minipage}[t]{0.47\columnwidth}\raggedright
Requirement\strut
\end{minipage} & \begin{minipage}[t]{0.47\columnwidth}\raggedright
\url{http://icao.int/iwxxm/2.1/req/xsd-meteorological-aerodrome-observation-report/observation}

The XML element //iwxxm:observation shall contain a valid child element om:OM\_Observation of type MeteorologicalAerodromeObservation. The value of XML attribute //iwxxm:observation/om:OM\_Observation/om:type/@xlink:href shall be the URI ``\url{http://codes.wmo.int/49-2/observation-type/IWXXM/2.1/MeteorologicalAerodromeObservation}''.\strut
\end{minipage}\tabularnewline
\begin{minipage}[t]{0.47\columnwidth}\raggedright
Requirement\strut
\end{minipage} & \begin{minipage}[t]{0.47\columnwidth}\raggedright
\url{http://icao.int/iwxxm/2.1/req/xsd-meteorological-aerodrome-observation-report/trend-forecast}

If trend forecasts are reported, the value of XML element //iwxxm:trendForecast shall be a valid child element om:OM\_Observation of type MeteorologicalAerodromeTrendForecast.

For each trend forecast, the value of XML attribute //iwxxm:trendForecast/om:OM\_Observation/om:type/@xlink:href shall be the URI ``\url{http://codes.wmo.int/49-2/observation-type/IWXXM/2.1/MeteorologicalAerodromeTrendForecast}''.\strut
\end{minipage}\tabularnewline
\begin{minipage}[t]{0.47\columnwidth}\raggedright
Requirement\strut
\end{minipage} & \begin{minipage}[t]{0.47\columnwidth}\raggedright
\url{http://icao.int/iwxxm/2.1/req/xsd-meteorological-aerodrome-observation-report/number-of-trend-forecasts}

No more than three trend forecasts shall be reported.\strut
\end{minipage}\tabularnewline
\begin{minipage}[t]{0.47\columnwidth}\raggedright
Requirement\strut
\end{minipage} & \begin{minipage}[t]{0.47\columnwidth}\raggedright
\url{http://icao.int/iwxxm/2.1/req/xsd-meteorological-aerodrome-observation-report/unique-subject-aerodrome}

The observation and, if reported, trend forecasts shall refer to the same aerodrome. All values of XML element //om:OM\_Observation/om:featureOfInterest/sams:SF\_SpatialSamplingFeature/sam:sampledFeature/aixm:AirportHeliport/gml:identifier within the meteorological aerodrome observation report shall be identical.\strut
\end{minipage}\tabularnewline
\begin{minipage}[t]{0.47\columnwidth}\raggedright
Requirement\strut
\end{minipage} & \begin{minipage}[t]{0.47\columnwidth}\raggedright
\url{http://icao.int/iwxxm/2.1/req/xsd-meteorological-aerodrome-observation-report/nil-report}

If XML attribute @status has value ``MISSING'', then a NIL report shall be provided:

(i) XML element //iwxxm:observation/om:OM\_Observation/om:result shall have no child elements and XML attribute //iwxxm:observation/om:OM\_Observation/om:result/@nilReason shall provide an appropriate nil reason;

(ii) XML attribute @automatedStation shall be absent; and

(iii) XML element //iwxxm:trendForecast shall be absent.\strut
\end{minipage}\tabularnewline
\begin{minipage}[t]{0.47\columnwidth}\raggedright
Recommendation\strut
\end{minipage} & \begin{minipage}[t]{0.47\columnwidth}\raggedright
\url{http://icao.int/iwxxm/2.1/req/xsd-meteorological-aerodrome-observation-report/nosig}

If no change of operational significance is forecast, then a single XML element\\
//iwxxm:trendForecast should be included with no child elements therein and the value of XML attribute //iwxxm:trendForecast/@nilReason should indicate ``inapplicable''.\strut
\end{minipage}\tabularnewline
\bottomrule
\end{longtable}

Notes:

1. A report with status ``CORRECTED'' indicates that content has been amended to correct an error identified in an earlier report. The XML element //om:OM\_Observation/om:resultTime/gml:TimeInstant is used to reflect the dissemination time of the corrected report.

2. A report with status ``MISSING'' indicates that a routine report has not been provided on the anticipated timescales. Such a report does not contain details of any observed or forecast meteorological conditions and is often referred to as a ``NIL'' report.

3. The requirements for reporting the use of an automated system are specified in the \emph{Technical Regulations} (WMO-No.~49), Volume~II, Part~II, Appendix~3, 4.8.

4. If XML attribute @automatedStation is absent, then the value ``false'' is inferred; for example, the meteorological aerodrome observation included within the report has not been generated by an automated system.

5. Within an XML encoded meteorological aerodrome report, it is likely that only one instance of aixm:AirportHeliport will physically be present; subsequent assertions about the aerodrome may use xlinks to refer to the previously defined aixm:AirportHeliport element in order to keep the XML document size small. As such, validation of requirement \url{http://icao.int/iwxxm/2.1/req/xsd-meteorological-aerodrome-observation-report/unique-subject-aerodrome} is applied once any xlinks, if used, have been resolved.

6. Code table~D-1 provides a set of nil-reason codes and is published at \url{http://codes.wmo.int/common/nil}.

205-16.5 Requirements class: SPECI

205-16.5.1 This requirements class is used to describe the special meteorological aerodrome reports (SPECI).

205-16.5.2 XML elements describing SPECIs shall conform to all requirements specified in Table~205-16.5.

205-16.5.3 XML elements describing SPECIs shall conform to all requirements of all relevant dependencies specified in Table~205-16.5.

Table~205-16.5. Requirements class xsd-speci

\begin{longtable}[]{@{}ll@{}}
\toprule
Requirements class &\tabularnewline
\midrule
\endhead
\url{http://icao.int/iwxxm/2.1/req/xsd-speci} &\tabularnewline
Target type & Data instance\tabularnewline
Name & SPECI\tabularnewline
Dependency & \url{http://icao.int/iwxxm/2.1/req/xsd-meteorological-aerodrome-observation-report}, 205-16.4\tabularnewline
\begin{minipage}[t]{0.47\columnwidth}\raggedright
Requirement\strut
\end{minipage} & \begin{minipage}[t]{0.47\columnwidth}\raggedright
\url{http://icao.int/iwxxm/2.1/req/xsd-speci/valid}

The content model of this element shall have a value that matches the content model of iwxxm:SPECI.\strut
\end{minipage}\tabularnewline
\bottomrule
\end{longtable}

205-16.6 Requirements class: METAR

205-16.6.1 This requirements class is used to describe the routine meteorological aerodrome reports (METAR).

205-16.6.2 XML elements describing METARs shall conform to all requirements specified in Table~205-16.6.

205-16.6.3 XML elements describing METARs shall conform to all requirements of all relevant dependencies specified in Table~205-16.6.

Table~205-16.6. Requirements class xsd-metar

\begin{longtable}[]{@{}ll@{}}
\toprule
Requirements class &\tabularnewline
\midrule
\endhead
\url{http://icao.int/iwxxm/2.1/req/xsd-metar} &\tabularnewline
Target type & Data instance\tabularnewline
Name & METAR\tabularnewline
Dependency & \url{http://icao.int/iwxxm/2.1/req/xsd-meteorological-aerodrome-observation-report}, 205-16.4\tabularnewline
\begin{minipage}[t]{0.47\columnwidth}\raggedright
Requirement\strut
\end{minipage} & \begin{minipage}[t]{0.47\columnwidth}\raggedright
\url{http://icao.int/iwxxm/2.1/req/xsd-metar/valid}

The content model of this element shall have a value that matches the content model of iwxxm:METAR.\strut
\end{minipage}\tabularnewline
\bottomrule
\end{longtable}

205-16.7 Requirements class: Meteorological aerodrome trend forecast record

205-16.7.1 This requirements class is used to describe the aggregated set of meteorological conditions forecast at an aerodrome as appropriate for inclusion in a trend forecast of a routine or special meteorological aerodrome report.

205-16.7.2 XML elements describing the set of meteorological conditions for inclusion in a trend forecast shall conform to all requirements specified in Table~205-16.7.

205-16.7.3 XML elements describing the set of meteorological conditions for inclusion in a trend forecast shall conform to all requirements of all relevant dependencies specified in Table~205‑16.7.

Table~205-16.7. Requirements class xsd-meteorological-aerodrome-trend-forecast-record

\begin{longtable}[]{@{}ll@{}}
\toprule
Requirements class &\tabularnewline
\midrule
\endhead
\url{http://icao.int/iwxxm/2.1/req/xsd-meteorological-aerodrome-trend-forecast-record} &\tabularnewline
Target type & Data instance\tabularnewline
Name & Meteorological aerodrome trend forecast record\tabularnewline
Dependency & \url{http://icao.int/iwxxm/2.1/req/xsd-aerodrome-cloud-forecast}, 205-16.36\tabularnewline
Dependency & \url{http://icao.int/iwxxm/2.1/req/xsd-aerodrome-surface-wind-trend-forecast}, 205-16.38\tabularnewline
\begin{minipage}[t]{0.47\columnwidth}\raggedright
Requirement\strut
\end{minipage} & \begin{minipage}[t]{0.47\columnwidth}\raggedright
\url{http://icao.int/iwxxm/2.1/req/xsd-meteorological-aerodrome-trend-forecast-record/valid}

The content model of this element shall have a value that matches the content model of iwxxm:MeteorologicalAerodromeTrendForecastRecord.\strut
\end{minipage}\tabularnewline
\begin{minipage}[t]{0.47\columnwidth}\raggedright
Requirement\strut
\end{minipage} & \begin{minipage}[t]{0.47\columnwidth}\raggedright
\url{http://icao.int/iwxxm/2.1/req/xsd-meteorological-aerodrome-trend-forecast-record/change-indicator-nosig}

If no operationally significant changes to the meteorological conditions are forecast for the aerodrome, then the XML attribute //iwxxm:MeteorologicalAerodromeTrendForecastRecord/@changeIndicator shall have the value ``NO\_SIGNIFICANT\_CHANGES''.\strut
\end{minipage}\tabularnewline
\begin{minipage}[t]{0.47\columnwidth}\raggedright
Requirement\strut
\end{minipage} & \begin{minipage}[t]{0.47\columnwidth}\raggedright
\url{http://icao.int/iwxxm/2.1/req/xsd-meteorological-aerodrome-trend-forecast-record/change-indicator-becmg}

If the meteorological conditions forecast for the aerodrome are expected to reach or pass through specified values at a regular or irregular rate, then the XML attribute\\
//iwxxm:MeteorologicalAerodromeTrendForecastRecord/@changeIndicator shall have the value ``BECOMING''.\strut
\end{minipage}\tabularnewline
\begin{minipage}[t]{0.47\columnwidth}\raggedright
Requirement\strut
\end{minipage} & \begin{minipage}[t]{0.47\columnwidth}\raggedright
\url{http://icao.int/iwxxm/2.1/req/xsd-meteorological-aerodrome-trend-forecast-record/change-indicator-tempo}

If temporary fluctuations in the meteorological conditions forecast for the aerodrome are expected to occur, then the XML attribute //iwxxm:MeteorologicalAerodromeTrendForecastRecord/\\
@changeIndicator shall have the value ``TEMPORARY\_FLUCTUATIONS''.\strut
\end{minipage}\tabularnewline
\begin{minipage}[t]{0.47\columnwidth}\raggedright
Requirement\strut
\end{minipage} & \begin{minipage}[t]{0.47\columnwidth}\raggedright
\url{http://icao.int/iwxxm/2.1/req/xsd-meteorological-aerodrome-trend-forecast-record/cavok}

If the conditions associated with CAVOK are forecast, then:

(i) The XML attribute //iwxxm:MeteorologicalAerodromeTrendForecastRecord/@cloudAndVisibilityOK shall the have value ``true''; and

(ii) The following XML elements shall be absent: //iwxxm:MeteorologicalAerodromeTrendForecastRecord/iwxxm:prevailingVisibility, //iwxxm:MeteorologicalAerodromeTrendForecastRecord/iwxxm:prevailingVisibilityOperator, //iwxxm:MeteorologicalAerodromeTrendForecastRecord/iwxxm:forecastWeather and\\
// iwxxm:MeteorologicalAerodromeTrendForecastRecord/iwxxm:cloud.\strut
\end{minipage}\tabularnewline
\begin{minipage}[t]{0.47\columnwidth}\raggedright
Requirement\strut
\end{minipage} & \begin{minipage}[t]{0.47\columnwidth}\raggedright
\url{http://icao.int/iwxxm/2.1/req/xsd-meteorological-aerodrome-trend-forecast-record/prevailing-visiblity}

If reported, the prevailing visibility shall be stated using the XML element\\
//iwxxm:MeteorologicalAerodromeTrendForecastRecord/iwxxm:prevailingVisibility with the unit of measure metres, indicated using the XML attribute //iwxxm:MeteorologicalAerodromeTrendForecastRecord/iwxxm:prevailingVisibility/@uom with value ``m''.\strut
\end{minipage}\tabularnewline
\begin{minipage}[t]{0.47\columnwidth}\raggedright
Requirement\strut
\end{minipage} & \begin{minipage}[t]{0.47\columnwidth}\raggedright
\url{http://icao.int/iwxxm/2.1/req/xsd-meteorological-aerodrome-trend-forecast-record/prevailing-visibility-exceeds-10000m}

If the prevailing visibility exceeds 10~000 metres, then the numeric value of XML element //iwxxm:MeteorologicalAerodromeTrendForecastRecord/iwxxm:prevailingVisibility shall be set to 10000 and the XML element //iwxxm:MeteorologicalAerodromeTrendForecastRecord/iwxxm:prevailingVisibilityOperator shall have the value ``ABOVE''.\strut
\end{minipage}\tabularnewline
\begin{minipage}[t]{0.47\columnwidth}\raggedright
Requirement\strut
\end{minipage} & \begin{minipage}[t]{0.47\columnwidth}\raggedright
\url{http://icao.int/iwxxm/2.1/req/xsd-meteorological-aerodrome-trend-forecast-record/prevailing-visibility-comparison-operator}

If present, the value of XML element //iwxxm:MeteorologicalAerodromeTrendForecastRecord/iwxxm:prevailingVisibilityOperator shall be one of the enumeration: ``ABOVE'' or ``BELOW''.\strut
\end{minipage}\tabularnewline
\begin{minipage}[t]{0.47\columnwidth}\raggedright
Requirement\strut
\end{minipage} & \begin{minipage}[t]{0.47\columnwidth}\raggedright
\url{http://icao.int/iwxxm/2.1/req/xsd-meteorological-aerodrome-trend-forecast-record/forecast-weather}

If forecast weather is reported, the value of XML attribute //iwxxm:MeteorologicalAerodromeTrendForecastRecord/iwxxm:forecastWeather/@xlink:href shall be the URI of a valid weather phenomenon code from Code table~D-7: Aerodrome present or forecast weather.\strut
\end{minipage}\tabularnewline
\begin{minipage}[t]{0.47\columnwidth}\raggedright
Requirement\strut
\end{minipage} & \begin{minipage}[t]{0.47\columnwidth}\raggedright
\url{http://icao.int/iwxxm/2.1/req/xsd-meteorological-aerodrome-trend-forecast-record/number-of-forecast-weather-codes}

No more than three forecast weather codes shall be reported.\strut
\end{minipage}\tabularnewline
\begin{minipage}[t]{0.47\columnwidth}\raggedright
Requirement\strut
\end{minipage} & \begin{minipage}[t]{0.47\columnwidth}\raggedright
\url{http://icao.int/iwxxm/2.1/req/xsd-meteorological-aerodrome-trend-forecast-record/surface-wind}

Surface wind conditions forecast for the aerodrome shall be reported using the XML element //iwxxm:MeteorologicalAerodromeTrendForecastRecord/iwxxm:surfaceWind containing a valid child element iwxxm:AerodromeSurfaceWindTrendForecast.\strut
\end{minipage}\tabularnewline
\begin{minipage}[t]{0.47\columnwidth}\raggedright
Requirement\strut
\end{minipage} & \begin{minipage}[t]{0.47\columnwidth}\raggedright
\url{http://icao.int/iwxxm/2.1/req/xsd-meteorological-aerodrome-trend-forecast-record/cloud}

If reported, the cloud conditions forecast for the aerodrome shall be expressed using the XML element //iwxxm:MeteorologicalAerodromeTrendForecastRecord/iwxxm:cloud containing a valid child element iwxxm:AerodromeCloudForecast.\strut
\end{minipage}\tabularnewline
\bottomrule
\end{longtable}

Notes:

1. Units of measurement are specified in accordance with 1.9 above.

2. Temporary fluctuations in the meteorological conditions occur when those conditions reach or pass specified values and last for a period of time less than one hour in each instance and, in the aggregate, cover less than one half the period during which the fluctuations are forecast to occur (\emph{Technical Regulations} (WMO-No.~49), Volume~II, Part~II, Appendix~5, 2.3.3).

3. The use of change groups is specified in the \emph{Technical Regulations} (WMO-No.~49), Volume~II, Part~II, Appendix~5, 2.3 and Appendix~3, Table~A3-3.

4. Cloud and visibility information is omitted when considered to be insignificant to aeronautical operations at an aerodrome. This occurs when: (i) visibility exceeds 10 kilometres, (ii) no cloud is present below 1~500 metres or the minimum sector altitude, whichever is greater, and there is no cumulonimbus at any height, and (iii) there is no weather of operational significance. These conditions are referred to as CAVOK. Use of CAVOK is specified in the \emph{Technical Regulations} (WMO-No.~49), Volume~II, Part~II, Appendix~3, 2.2.

5. Visibility for aeronautical purposes is defined as the greater of: (i) the greatest distance at which a black object of suitable dimensions, situated near the ground, can be seen and recognized when observed against a bright background; or (ii) the greatest distance at which lights in the vicinity of 1~000 candelas can be seen and identified against an unlit background.

6. Prevailing visibility is defined as the greatest visibility value observed which is reached within at least half the horizon circle or within at least half of the surface of the aerodrome. These areas could comprise contiguous or non-contiguous sectors.

7. The requirements for reporting the following within a trend forecast are specified in the \emph{Technical Regulations} (WMO-No.~49), Volume~II, Part~II, Appendix~5:

(a) Prevailing visibility conditions paragraph~2.2.3

(b) Forecast weather phenomena section~2.2.4

(c) Surface wind conditions paragraph~2.2.2

(d) Cloud conditions paragraph~2.2.5

8. The absence of XML element //iwxxm:MeteorologicalAerodromeTrendForecastRecord/iwxxm:prevailingVisibilityOperator indicates that the prevailing visibility has the numeric value reported.

9. Code table~D-7 is published online at \url{http://codes.wmo.int/49-2/AerodromePresentOrForecastWeather}.

205-16.8 Requirements class: Meteorological aerodrome observation record

205-16.8.1 This requirements class is used to describe the aggregated set of meteorological conditions observed at an aerodrome.

205-16.8.2 XML elements describing the set of meteorological conditions observed at an aerodrome shall conform to all requirements specified in Table~205-16.8.

205-16.8.3 XML elements describing the set of meteorological conditions observed at an aerodrome shall conform to all requirements of all relevant dependencies specified in Table~205‑16.8.

Table~205-16.8. Requirements class xsd-meteorological-aerodrome-observation-record

\begin{longtable}[]{@{}ll@{}}
\toprule
Requirements class &\tabularnewline
\midrule
\endhead
\url{http://icao.int/iwxxm/2.1/req/xsd-meteorological-aerodrome-observation-record} &\tabularnewline
Target type & Data instance\tabularnewline
Name & Meteorological aerodrome observation record\tabularnewline
Dependency & \url{http://icao.int/iwxxm/2.1/req/xsd-aerodrome-runway-state}, 205-16.9\tabularnewline
Dependency & \url{http://icao.int/iwxxm/2.1/req/xsd-aerodrome-wind-shear}, 205-16.12\tabularnewline
Dependency & \url{http://icao.int/iwxxm/2.1/req/xsd-aerodrome-observed-clouds}, 205-16.13\tabularnewline
Dependency & \url{http://icao.int/iwxxm/2.1/req/xsd-aerodrome-runway-visual-range}, 205-16.10\tabularnewline
Dependency & \url{http://icao.int/iwxxm/2.1/req/xsd-aerodrome-sea-state}, 205-16.11\tabularnewline
Dependency & \url{http://icao.int/iwxxm/2.1/req/xsd-aerodrome-horizontal-visibility}, 205-16.15\tabularnewline
Dependency & \url{http://icao.int/iwxxm/2.1/req/xsd-aerodrome-surface-wind}, 205-16.14\tabularnewline
\begin{minipage}[t]{0.47\columnwidth}\raggedright
Requirement\strut
\end{minipage} & \begin{minipage}[t]{0.47\columnwidth}\raggedright
\url{http://icao.int/iwxxm/2.1/req/xsd-meteorological-aerodrome-observation-record/valid}

The content model of this element shall have a value that matches the content model of iwxxm:MeteorologicalAerodromeObservationRecord.\strut
\end{minipage}\tabularnewline
\begin{minipage}[t]{0.47\columnwidth}\raggedright
Requirement\strut
\end{minipage} & \begin{minipage}[t]{0.47\columnwidth}\raggedright
\url{http://icao.int/iwxxm/2.1/req/xsd-meteorological-aerodrome-observation-record/cavok}

If the conditions associated with CAVOK are observed, then:

(i) The XML attribute //iwxxm:MeteorologicalAerodromeObservationRecord/\\
@cloudAndVisibilityOK shall have the value ``true''; and

(ii) The following XML elements shall be absent:\\
//iwxxm:MeteorologicalAerodromeObservationRecord/iwxxm:visibility,\\
//iwxxm:MeteorologicalAerodromeObservationRecord/iwxxm:rvr,\\
//iwxxm:MeteorologicalAerodromeObservationRecord/iwxxm:presentWeather\\
and //iwxxm:MeteorologicalAerodromeObservationRecord/iwxxm:cloud.\strut
\end{minipage}\tabularnewline
\begin{minipage}[t]{0.47\columnwidth}\raggedright
Requirement\strut
\end{minipage} & \begin{minipage}[t]{0.47\columnwidth}\raggedright
\url{http://icao.int/iwxxm/2.1/req/xsd-meteorological-aerodrome-observation-record/air-temperature}

The air temperature observed at the aerodrome shall be reported in Celsius (°C) using the XML element //iwxxm:MeteorologicalAerodromeObservationRecord/iwxxm:airTemperature. The value of the associated XML attribute @uom shall be ``Cel''.\strut
\end{minipage}\tabularnewline
\begin{minipage}[t]{0.47\columnwidth}\raggedright
Requirement\strut
\end{minipage} & \begin{minipage}[t]{0.47\columnwidth}\raggedright
\url{http://icao.int/iwxxm/2.1/req/xsd-meteorological-aerodrome-observation-record/dew-point-temperature}

The dewpoint temperature observed at the aerodrome shall be reported in Celsius (°C) using the XML element //iwxxm:MeteorologicalAerodromeObservationRecord/iwxxm:dewpointTemperature. The value of the associated XML attribute @uom shall be ``Cel''.\strut
\end{minipage}\tabularnewline
\begin{minipage}[t]{0.47\columnwidth}\raggedright
Requirement\strut
\end{minipage} & \begin{minipage}[t]{0.47\columnwidth}\raggedright
\url{http://icao.int/iwxxm/2.1/req/xsd-meteorological-aerodrome-observation-record/qnh}

The atmospheric pressure, known as QNH, observed at the aerodrome shall be reported in hectopascals (hPa) using the XML element //iwxxm:MeteorologicalAerodromeObservationRecord/iwxxm:qnh. The value of the associated XML attribute @uom shall be ``hPa''.\strut
\end{minipage}\tabularnewline
\begin{minipage}[t]{0.47\columnwidth}\raggedright
Requirement\strut
\end{minipage} & \begin{minipage}[t]{0.47\columnwidth}\raggedright
\url{http://icao.int/iwxxm/2.1/req/xsd-meteorological-aerodrome-observation-record/present-weather}

If present weather is reported, the value of XML attribute //iwxxm:MeteorologicalAerodromeObservationRecord/iwxxm:presentWeather/@xlink:href shall be the URI of a valid weather phenomenon code from Code table~D-7: Aerodrome present or forecast weather.\strut
\end{minipage}\tabularnewline
\begin{minipage}[t]{0.47\columnwidth}\raggedright
Requirement\strut
\end{minipage} & \begin{minipage}[t]{0.47\columnwidth}\raggedright
\url{http://icao.int/iwxxm/2.1/req/xsd-meteorological-aerodrome-observation-record/number-of-present-weather-codes}

No more than three present weather codes shall be reported.\strut
\end{minipage}\tabularnewline
\begin{minipage}[t]{0.47\columnwidth}\raggedright
Requirement\strut
\end{minipage} & \begin{minipage}[t]{0.47\columnwidth}\raggedright
\url{http://icao.int/iwxxm/2.1/req/xsd-meteorological-aerodrome-observation-record/recent-weather}

If recent weather is reported, the value of XML attribute //iwxxm:MeteorologicalAerodromeObservationRecord/iwxxm:recentWeather/@xlink:href shall be the URI of a valid weather phenomenon code from Code table~D-6: Aerodrome recent weather.\strut
\end{minipage}\tabularnewline
\begin{minipage}[t]{0.47\columnwidth}\raggedright
Requirement\strut
\end{minipage} & \begin{minipage}[t]{0.47\columnwidth}\raggedright
\url{http://icao.int/iwxxm/2.1/req/xsd-meteorological-aerodrome-observation-record/number-of-recent-weather-codes}

No more than three recent weather codes shall be reported.\strut
\end{minipage}\tabularnewline
\begin{minipage}[t]{0.47\columnwidth}\raggedright
Requirement\strut
\end{minipage} & \begin{minipage}[t]{0.47\columnwidth}\raggedright
\url{http://icao.int/iwxxm/2.1/req/xsd-meteorological-aerodrome-observation-record/surface-wind}

Surface wind conditions observed at the aerodrome shall be reported using the XML element //iwxxm:MeteorologicalAerodromeObservationRecord/iwxxm:surfaceWind containing a valid child element iwxxm:AerodromeSurfaceWind.\strut
\end{minipage}\tabularnewline
\begin{minipage}[t]{0.47\columnwidth}\raggedright
Requirement\strut
\end{minipage} & \begin{minipage}[t]{0.47\columnwidth}\raggedright
\url{http://icao.int/iwxxm/2.1/req/xsd-meteorological-aerodrome-observation-record/runway-state}

If reported, the surface conditions for a given runway direction shall be expressed using the XML element //iwxxm:MeteorologicalAerodromeObservationRecord/iwxxm:runwayState containing a valid child element iwxxm:AerodromeRunwayState.\strut
\end{minipage}\tabularnewline
\begin{minipage}[t]{0.47\columnwidth}\raggedright
Requirement\strut
\end{minipage} & \begin{minipage}[t]{0.47\columnwidth}\raggedright
\url{http://icao.int/iwxxm/2.1/req/xsd-meteorological-aerodrome-observation-record/wind-shear}

If reported, the wind shear conditions for the aerodrome shall be expressed using the XML element //iwxxm:MeteorologicalAerodromeObservationRecord/iwxxm:windShear containing a valid child element iwxxm:AerodromeWindShear.\strut
\end{minipage}\tabularnewline
\begin{minipage}[t]{0.47\columnwidth}\raggedright
Requirement\strut
\end{minipage} & \begin{minipage}[t]{0.47\columnwidth}\raggedright
\url{http://icao.int/iwxxm/2.1/req/xsd-meteorological-aerodrome-observation-record/cloud}

If reported, the cloud conditions observed at the aerodrome shall be expressed using the XML element //iwxxm:MeteorologicalAerodromeObservationRecord/iwxxm:cloud containing a valid child element iwxxm:AerodromeObservedClouds.\strut
\end{minipage}\tabularnewline
\begin{minipage}[t]{0.47\columnwidth}\raggedright
Requirement\strut
\end{minipage} & \begin{minipage}[t]{0.47\columnwidth}\raggedright
\url{http://icao.int/iwxxm/2.1/req/xsd-meteorological-aerodrome-observation-record/runway-visual-range}

If reported, the visual range conditions for a given runway direction shall be expressed using the XML element //iwxxm:MeteorologicalAerodromeObservationRecord/iwxxm:rvr containing a valid child element iwxxm:AerodromeRunwayVisualRange.\strut
\end{minipage}\tabularnewline
\begin{minipage}[t]{0.47\columnwidth}\raggedright
Requirement\strut
\end{minipage} & \begin{minipage}[t]{0.47\columnwidth}\raggedright
\url{http://icao.int/iwxxm/2.1/req/xsd-meteorological-aerodrome-observation-record/number-of-rvr-groups}

Visual range conditions shall be reported for no more than four runway directions.\strut
\end{minipage}\tabularnewline
\begin{minipage}[t]{0.47\columnwidth}\raggedright
Requirement\strut
\end{minipage} & \begin{minipage}[t]{0.47\columnwidth}\raggedright
\url{http://icao.int/iwxxm/2.1/req/xsd-meteorological-aerodrome-observation-record/sea-state}

If reported, the sea-state conditions observed at the aerodrome shall be expressed using the XML element //iwxxm:MeteorologicalAerodromeObservationRecord/iwxxm:seaState containing a valid child element iwxxm:AerodromeSeaState.\strut
\end{minipage}\tabularnewline
\begin{minipage}[t]{0.47\columnwidth}\raggedright
Requirement\strut
\end{minipage} & \begin{minipage}[t]{0.47\columnwidth}\raggedright
\url{http://icao.int/iwxxm/2.1/req/xsd-meteorological-aerodrome-observation-record/visibility}

If reported, the horizontal visibility conditions observed at the aerodrome shall be expressed using the XML element //iwxxm:MeteorologicalAerodromeObservationRecord/iwxxm:visibility containing a valid child element iwxxm:AerodromeHorizontalVisibility.\strut
\end{minipage}\tabularnewline
\begin{minipage}[t]{0.47\columnwidth}\raggedright
Recommendation\strut
\end{minipage} & \begin{minipage}[t]{0.47\columnwidth}\raggedright
\url{http://icao.int/iwxxm/2.1/req/xsd-meteorological-aerodrome-observation-record/present-weather-not-observable}

If present weather is not observable due to sensor failure or obstruction, the value of XML attribute //iwxxm:MeteorologicalAerodromeObservationRecord/iwxxm:presentWeather/@nilReason should indicate the URI ``\url{http://codes.wmo.int/common/nil/notObservable}''.\strut
\end{minipage}\tabularnewline
\bottomrule
\end{longtable}

Notes:

1. Units of measurement are specified in accordance with 1.9 above.

2. Cloud and visibility information is omitted when considered to be insignificant to aeronautical operations at an aerodrome. This occurs when: (i)~visibility exceeds 10~kilometres, (ii)~no cloud is present below 1~500~metres or the minimum sector altitude, whichever is greater, and there is no cumulonimbus at any height, and (iii) there is no weather of operational significance. These conditions are referred to as CAVOK. Use of CAVOK is specified in the \emph{Technical Regulations} (WMO-No.~49), Volume~II, Part~II, Appendix~3, 2.2.

3. The requirements for reporting the following are specified in the \emph{Technical Regulations} (WMO-No.~49), Volume~II, Part~II, Appendix~3:

(a) Air temperature and dewpoint temperature section~4.6

(b) Atmospheric pressure (QNH) section~4.7

(c) Present weather section~4.4

(d) Recent weather paragraph~4.8.1.1

(e) Surface wind conditions section~4.1

(f) Runway state paragraph~4.8.1.5

(g) Aerodrome wind shear paragraph~4.8.1.4

(h) Observed cloud conditions section~4.5

(i) Sea state paragraph~4.8.1.5

(j) Horizontal visibility section~4.2

4. Code table~D-7 is published online at \url{http://codes.wmo.int/49-2/AerodromePresentOrForecastWeather}.

5. Code table~D-6 is published online at \url{http://codes.wmo.int/49-2/AerodromeRecentWeather}.

6. Information on runway visual range shall be omitted if the prevailing visibility exceeds 1~500~metres. Details of the requirements for reporting runway visual range are specified in the \emph{Technical Regulations} (WMO-No.~49), Volume~II, Part~II, Appendix~3, 4.3.

205-16.9 Requirements class: Aerodrome runway state

205-16.9.1 This requirements class is used to describe the observed runway state.

Notes:

1. Representations providing more detailed information may be used if required.

2. The requirements for reporting runway state are specified in the \emph{Technical Regulations} (WMO-No.~49), Volume~II, Part~II, Appendix~3, 4.8.1.5.

205-16.9.2 XML elements describing observed runway state shall conform to all requirements specified in Table~205-16.9.

205-16.9.3 XML elements describing observed runway state shall conform to all requirements of all relevant dependencies specified in Table~205-16.9.

Table~205-16.9. Requirements class xsd-aerodrome-runway-state

\begin{longtable}[]{@{}ll@{}}
\toprule
Requirements class &\tabularnewline
\midrule
\endhead
\url{http://icao.int/iwxxm/2.1/req/xsd-aerodrome-runway-state} &\tabularnewline
Target type & Data instance\tabularnewline
Name & Aerodrome runway state\tabularnewline
\begin{minipage}[t]{0.47\columnwidth}\raggedright
Requirement\strut
\end{minipage} & \begin{minipage}[t]{0.47\columnwidth}\raggedright
\url{http://icao.int/iwxxm/2.1/req/xsd-aerodrome-runway-state/valid}

The content model of this element shall have a value that matches the content model of iwxxm:AerodromeRunwayState.\strut
\end{minipage}\tabularnewline
\begin{minipage}[t]{0.47\columnwidth}\raggedright
Requirement\strut
\end{minipage} & \begin{minipage}[t]{0.47\columnwidth}\raggedright
\url{http://icao.int/iwxxm/2.1/req/xsd-aerodrome-runway-state/applicable-runway}

If XML attribute //iwxxm:AerodromeRunwayState/@allRunways is absent or has value ``false'', then XML element //iwxxm:AerodromeRunwayState/iwxxm:runway, with valid child element //iwxxm:AerodromeRunwayState/iwxxm:runway/aixm:RunwayDirection, shall be used to indicate the runway direction to which these conditions apply.\strut
\end{minipage}\tabularnewline
\begin{minipage}[t]{0.47\columnwidth}\raggedright
Requirement\strut
\end{minipage} & \begin{minipage}[t]{0.47\columnwidth}\raggedright
\url{http://icao.int/iwxxm/2.1/req/xsd-aerodrome-runway-state/all-runways}

If XML attribute //iwxxm:AerodromeRunwayState/@allRunways has value ``true'', then XML element //iwxxm:AerodromeRunwayState/iwxxm:runway shall be absent.\strut
\end{minipage}\tabularnewline
\begin{minipage}[t]{0.47\columnwidth}\raggedright
Requirement\strut
\end{minipage} & \begin{minipage}[t]{0.47\columnwidth}\raggedright
\url{http://icao.int/iwxxm/2.1/req/xsd-aerodrome-runway-state/snow-closure}

If the aerodrome is closed due to an extreme deposit of snow, XML attribute\\
//iwxxm:AerodromeRunwayState/@snowClosure shall have the value ``true''.\strut
\end{minipage}\tabularnewline
\begin{minipage}[t]{0.47\columnwidth}\raggedright
Requirement\strut
\end{minipage} & \begin{minipage}[t]{0.47\columnwidth}\raggedright
\url{http://icao.int/iwxxm/2.1/req/xsd-aerodrome-runway-state/cleared}

If the runway has been cleared of meteorological deposits, then XML attribute //iwxxm:AerodromeRunwayState/@cleared shall have the value ``true'' and XML elements //iwxxm:AerodromeRunwayState/iwxxm:depositType, //iwxxm:AerodromeRunwayState/iwxxm:contamination, //iwxxm:AerodromeRunwayState/iwxxm:depthOfDeposit and //iwxxm:AerodromeRunwayState/iwxxm:estimatedSurfaceFriction shall be absent.\strut
\end{minipage}\tabularnewline
\begin{minipage}[t]{0.47\columnwidth}\raggedright
Requirement\strut
\end{minipage} & \begin{minipage}[t]{0.47\columnwidth}\raggedright
\url{http://icao.int/iwxxm/2.1/req/xsd-aerodrome-runway-state/surface-friction-estimate}

If reported, the estimated surface friction shall be stated using the XML element\\
//iwxxm:AerodromeRunwayState/iwxxm:estimatedSurfaceFriction and shall have numeric value greater than 0.0 and less than or equal to 0.9.\strut
\end{minipage}\tabularnewline
\begin{minipage}[t]{0.47\columnwidth}\raggedright
Requirement\strut
\end{minipage} & \begin{minipage}[t]{0.47\columnwidth}\raggedright
\url{http://icao.int/iwxxm/2.1/req/xsd-aerodrome-runway-state/surface-friction-estimate-unit-of-measure}

If reported, the estimated surface friction shall be expressed as a unitless ratio with the value of XML attribute //iwxxm:AerodromeRunwayState/iwxxm:estimatedSurfaceFriction/@uom specified as ``\url{http://www.opengis.net/def/uom/OGC/1.0/unity}''.\strut
\end{minipage}\tabularnewline
\begin{minipage}[t]{0.47\columnwidth}\raggedright
Requirement\strut
\end{minipage} & \begin{minipage}[t]{0.47\columnwidth}\raggedright
\url{http://icao.int/iwxxm/2.1/req/xsd-aerodrome-runway-state/unreliable-surface-friction-estimate}

If the surface friction estimate for the runway is considered to be unreliable, then XML attribute //iwxxm:AerodromeRunwayState/\\
@estimatedSurfaceFrictionUnreliable shall have the value ``true''.\strut
\end{minipage}\tabularnewline
\begin{minipage}[t]{0.47\columnwidth}\raggedright
Requirement\strut
\end{minipage} & \begin{minipage}[t]{0.47\columnwidth}\raggedright
\url{http://icao.int/iwxxm/2.1/req/xsd-aerodrome-runway-state/unreliable-surface-friction-estimate-true}

If XML attribute //iwxxm:AerodromeRunwayState/\\
@estimatedSurfaceFrictionUnreliable has value ``true'', then XML element //iwxxm:AerodromeRunwayState/iwxxm:estimatedSurfaceFriction shall be absent.\strut
\end{minipage}\tabularnewline
\begin{minipage}[t]{0.47\columnwidth}\raggedright
Requirement\strut
\end{minipage} & \begin{minipage}[t]{0.47\columnwidth}\raggedright
\url{http://icao.int/iwxxm/2.1/req/xsd-aerodrome-runway-state/deposit-type-code}

If deposit type is reported, then the value of XML attribute //iwxxm:AerodromeRunwayState/iwxxm:depositType/@xlink:href shall be the URI of the valid term from Volume~I.2, FM 94 BUFR, Code table 0 20 086: Runway deposits.\strut
\end{minipage}\tabularnewline
\begin{minipage}[t]{0.47\columnwidth}\raggedright
Requirement\strut
\end{minipage} & \begin{minipage}[t]{0.47\columnwidth}\raggedright
\url{http://icao.int/iwxxm/2.1/req/xsd-aerodrome-runway-state/contamination-code}

If runway contamination is reported, then the value of XML attribute //iwxxm:AerodromeRunwayState/iwxxm:contamination/@xlink:href shall be the URI of the valid term from Volume~I.2, FM 94 BUFR, Code table 0 20 087: Runway contamination.\strut
\end{minipage}\tabularnewline
\begin{minipage}[t]{0.47\columnwidth}\raggedright
Recommendation\strut
\end{minipage} & \begin{minipage}[t]{0.47\columnwidth}\raggedright
\url{http://icao.int/iwxxm/2.1/req/xsd-aerodrome-runway-state/snow-closure-affects-all-runways}

If XML attribute //iwxxm:AerodromeRunwayState/@snowClosure has value ``true'', then XML //iwxxm:AerodromeRunwayState/@allRunways should also have value ``true''; snow closure affects all runways at an aerodrome.\strut
\end{minipage}\tabularnewline
\begin{minipage}[t]{0.47\columnwidth}\raggedright
Recommendation\strut
\end{minipage} & \begin{minipage}[t]{0.47\columnwidth}\raggedright
\url{http://icao.int/iwxxm/2.1/req/xsd-aerodrome-runway-state/deposit-depth-unit-of-measure}

If reported, the depth of deposit should be expressed in millimetres, with the value of XML attribute //iwxxm:AerodromeRunwayState/iwxxm:depthOfDeposit/@uom specified as ``mm''.\strut
\end{minipage}\tabularnewline
\bottomrule
\end{longtable}

Notes:

1. For convenience, FM~94 BUFR, Code table 0 20 086 from Volume~I.2, is published online at \url{http://codes.wmo.int/bufr4/codeflag/0-20-086}.

2. Runway contamination is expressed as a percentage of the total runway area that is contaminated according to a predefined set of categories: less than 10\%, between 11\% and 25\%, between 25\% and 50\% and more than 50\%. These categories are listed in Volume~I.2, FM~94 BUFR, Code table~0 20 087: Runway contamination. For convenience, this code table is published online at \url{http://codes.wmo.int/bufr4/codeflag/0-20-087}.

3. Units of measurement are specified in accordance with 1.9 above.

205-16.10 Requirements class: Aerodrome runway visual range

205-16.10.1 This requirements class is used to describe runway visual range for a specific runway direction at an aerodrome.

Note: The requirements for reporting runway visual range are specified in the \emph{Technical Regulations} (WMO-No.~49), Volume~II, Part~II, Appendix~3, 4.3.

205-16.10.2 XML elements describing runway visual range shall conform to all requirements specified in Table~205-16.10.

205-16.10.3 XML elements describing runway visual range shall conform to all requirements of all relevant dependencies specified in Table~205-16.10.

Table~205-16.10. Requirements class xsd-aerodrome-runway-visual-range

\begin{longtable}[]{@{}ll@{}}
\toprule
Requirements class &\tabularnewline
\midrule
\endhead
\url{http://icao.int/iwxxm/2.1/req/xsd-aerodrome-runway-visual-range} &\tabularnewline
Target type & Data instance\tabularnewline
Name & Aerodrome runway visual range\tabularnewline
\begin{minipage}[t]{0.47\columnwidth}\raggedright
Requirement\strut
\end{minipage} & \begin{minipage}[t]{0.47\columnwidth}\raggedright
\url{http://icao.int/iwxxm/2.1/req/xsd-aerodrome-runway-visual-range/valid}

The content model of this element shall have a value that matches the content model of iwxxm:RunwayVisualRange.\strut
\end{minipage}\tabularnewline
\begin{minipage}[t]{0.47\columnwidth}\raggedright
Requirement\strut
\end{minipage} & \begin{minipage}[t]{0.47\columnwidth}\raggedright
\url{http://icao.int/iwxxm/2.1/req/xsd-aerodrome-runway-visual-range/applicable-runway}

The XML element //iwxxm:AerodromeRunwayVisualRange/iwxxm:runway, with valid child element //iwxxm:AerodromeRunwayState/iwxxm:runway/aixm:RunwayDirection, shall be used to indicate the runway direction to which these visual range conditions apply.\strut
\end{minipage}\tabularnewline
\begin{minipage}[t]{0.47\columnwidth}\raggedright
Requirement\strut
\end{minipage} & \begin{minipage}[t]{0.47\columnwidth}\raggedright
\url{http://icao.int/iwxxm/2.1/req/xsd-aerodrome-runway-visual-range/mean-rvr}

The XML element //iwxxm:AerodromeRunwayVisualRange/iwxxm:meanRVR shall be used to express the 10-minute average for observed runway visual range or, if a marked discontinuity in visual range occurs during the 10-minute period, the average runway visual range following that marked discontinuity.\strut
\end{minipage}\tabularnewline
\begin{minipage}[t]{0.47\columnwidth}\raggedright
Requirement\strut
\end{minipage} & \begin{minipage}[t]{0.47\columnwidth}\raggedright
\url{http://icao.int/iwxxm/2.1/req/xsd-aerodrome-runway-visual-range/mean-rvr-unit-of-measure}

The mean runway visual range shall be reported in metres. The unit of measure shall be indicated using the XML attribute //iwxxm:AerodromeRunwayVisualRange/iwxxm:meanRVR/@uom with value ``m''.\strut
\end{minipage}\tabularnewline
\begin{minipage}[t]{0.47\columnwidth}\raggedright
Requirement\strut
\end{minipage} & \begin{minipage}[t]{0.47\columnwidth}\raggedright
\url{http://icao.int/iwxxm/2.1/req/xsd-aerodrome-runway-visual-range/mean-rvr-exceeds-2000m}

If the mean runway visual range exceeds 2~000 metres, then the numeric value of XML element //iwxxm:AerodromeRunwayVisualRange/iwxxm:meanRVR shall be set to 2000 and the XML element //iwxxm:AerodromeRunwayVisualRange/iwxxm:meanRVROperator shall have the value ``ABOVE''.\strut
\end{minipage}\tabularnewline
\begin{minipage}[t]{0.47\columnwidth}\raggedright
Requirement\strut
\end{minipage} & \begin{minipage}[t]{0.47\columnwidth}\raggedright
\url{http://icao.int/iwxxm/2.1/req/xsd-aerodrome-runway-visual-range/mean-rvr-comparison-operator}

If present, the value of XML element //iwxxm:AerodromeRunwayVisualRange/iwxxm:meanRVROperator shall be one of the enumeration: ``ABOVE'' or ``BELOW''.\strut
\end{minipage}\tabularnewline
\begin{minipage}[t]{0.47\columnwidth}\raggedright
Requirement\strut
\end{minipage} & \begin{minipage}[t]{0.47\columnwidth}\raggedright
\url{http://icao.int/iwxxm/2.1/req/xsd-aerodrome-runway-visual-range/upward-or-downward-visual-range-tendency}

If the runway visual range values observed in the 10-minute period have shown a distinct tendency, such that the mean during the first 5~minutes varies by 100~metres or more when compared with the second 5~minutes, this shall be indicated using the XML element //iwxxm:AerodromeRunwayVisualRange/iwxxm:pastTendency with value ``UPWARD'' (visual range is increasing) or ``DOWNWARD'' (visual range is decreasing) as appropriate.\strut
\end{minipage}\tabularnewline
\begin{minipage}[t]{0.47\columnwidth}\raggedright
Recommendation\strut
\end{minipage} & \begin{minipage}[t]{0.47\columnwidth}\raggedright
\url{http://icao.int/iwxxm/2.1/req/xsd-aerodrome-runway-visual-range/no-change-in-visual-range-tendency}

If the runway visual range values observed in the 10-minute period have not shown a distinct tendency, this should be indicated using the XML element //iwxxm:AerodromeRunwayVisualRange/iwxxm:pastTendency with value ``NO\_CHANGE''.\strut
\end{minipage}\tabularnewline
\bottomrule
\end{longtable}

Notes:

1. Units of measurement are specified in accordance with 1.9 above.

2. The absence of XML element //iwxxm:AerodromeRunwayVisualRange/iwxxm:meanRVROperator indicates that the mean runway visual range has the numeric value reported.

3. The absence of XML element //iwxxm:AerodromeRunwayVisualRange/iwxxm:pastTendency indicates that no distinct tendency in visual range has been observed.

205-16.11 Requirements class: Aerodrome sea state

205-16.11.1 This requirements class is used to describe an aggregated set of sea-state conditions reported at an aerodrome.

Note: The requirements for reporting sea state are specified in the \emph{Technical Regulations} (WMO-No.~49), Volume~II, Part~II, Appendix~3, 4.8.1.5.

205-16.11.2 XML elements describing sea state shall conform to all requirements specified in Table~205-16.11.

205-16.11.3 XML elements describing sea state shall conform to all requirements of all relevant dependencies specified in Table~205-16.11.

Table~205-16.11. Requirements class xsd-aerodrome-sea-state

\begin{longtable}[]{@{}ll@{}}
\toprule
Requirements class &\tabularnewline
\midrule
\endhead
\url{http://icao.int/iwxxm/2.1/req/xsd-aerodrome-sea-state} &\tabularnewline
Target type & Data instance\tabularnewline
Name & Aerodrome sea state\tabularnewline
\begin{minipage}[t]{0.47\columnwidth}\raggedright
Requirement\strut
\end{minipage} & \begin{minipage}[t]{0.47\columnwidth}\raggedright
\url{http://icao.int/iwxxm/2.1/req/xsd-aerodrome-sea-state/valid}

The content model of this element shall have a value that matches the content model of iwxxm:AerodromeSeaState.\strut
\end{minipage}\tabularnewline
\begin{minipage}[t]{0.47\columnwidth}\raggedright
Requirement\strut
\end{minipage} & \begin{minipage}[t]{0.47\columnwidth}\raggedright
\url{http://icao.int/iwxxm/2.1/req/xsd-aerodrome-sea-state/sea-surface-temperature}

The sea-surface temperature shall be reported in Celsius (°C) using the XML element //iwxxm:AerdromeSeaState/iwxxm:seaSurfaceTemperature. The value of the associated XML attribute //iwxxm:AerdromeSeaState/iwxxm:seaSurfaceTemperature/@uom shall be ``Cel''.\strut
\end{minipage}\tabularnewline
\begin{minipage}[t]{0.47\columnwidth}\raggedright
Requirement\strut
\end{minipage} & \begin{minipage}[t]{0.47\columnwidth}\raggedright
\url{http://icao.int/iwxxm/2.1/req/xsd-aerodrome-sea-state/either-significant-wave-height-or-sea-state}

When significant wave height is reported, sea state shall not be reported.

When sea state is reported, significant wave height shall not be reported.\strut
\end{minipage}\tabularnewline
\begin{minipage}[t]{0.47\columnwidth}\raggedright
Requirement\strut
\end{minipage} & \begin{minipage}[t]{0.47\columnwidth}\raggedright
\url{http://icao.int/iwxxm/2.1/req/xsd-aerodrome-sea-state/significant-wave-height}

If reported, the observed significant wave height shall be expressed using the XML element //iwxxm:AerodromeSeaState/iwxxm:significantWaveHeight.\strut
\end{minipage}\tabularnewline
\begin{minipage}[t]{0.47\columnwidth}\raggedright
Requirement\strut
\end{minipage} & \begin{minipage}[t]{0.47\columnwidth}\raggedright
\url{http://icao.int/iwxxm/2.1/req/xsd-aerodrome-sea-state/sea-state-code}

If sea state is reported, then the value of XML attribute //iwxxm:AerodromeSeaState/iwxxm:seaState/@xlink:href shall be the URI of the valid term from Volume~I.2, FM 94 BUFR, Code table 0 22 061: State of the sea.\strut
\end{minipage}\tabularnewline
\begin{minipage}[t]{0.47\columnwidth}\raggedright
Recommendation\strut
\end{minipage} & \begin{minipage}[t]{0.47\columnwidth}\raggedright
\url{http://icao.int/iwxxm/2.1/req/xsd-aerodrome-sea-state/significant-wave-height-unit-of-measure}

The significant wave height should be reported in metres. The unit of measure should be indicated using the XML attribute //iwxxm:AerodromeSeaState/iwxxm:significantWaveHeight/@uom with value~``m''.\strut
\end{minipage}\tabularnewline
\bottomrule
\end{longtable}

Notes:

1. Units of measurement are specified in accordance with 1.9 above.

2. The term sea-surface temperature is generally meant to be representative of the upper few metres of the ocean as opposed to the skin temperature.

3. For convenience, FM~94 BUFR, Code table~0~22~061 from Volume~I.2 is published online at \url{http://codes.wmo.int/bufr4/codeflag/0-22-061}.

205-16.12 Requirements class: Aerodrome wind shear

205-16.12.1 This requirements class is used to describe the aerodrome wind shear. The class is targeted at providing a basic description of the wind shear as required for civil aviation purposes -- currently limited to indicating whether a wind shear threshold has been exceeded.

Notes:

1. The information on wind shear includes, but is not necessarily limited to, wind shear of a non-transitory nature such as might be associated with low-level temperature inversions or local topography.

2. Representations providing more detailed information may be used if required.

3. The requirements for reporting aerodrome wind shear are specified in the \emph{Technical Regulations} (WMO-No.~49), Volume~II, Part~II, Appendix~3, 4.8.1.4.

205-16.12.2 XML elements describing aerodrome wind shear shall conform to all requirements specified in Table~205-16.12.

205-16.12.3 XML elements describing aerodrome wind shear shall conform to all requirements of all relevant dependencies specified in Table~205-16.12.

Table~205-16.12. Requirements class xsd-aerodrome-wind-shear

\begin{longtable}[]{@{}ll@{}}
\toprule
Requirements class &\tabularnewline
\midrule
\endhead
\url{http://icao.int/iwxxm/2.1/req/xsd-aerodrome-wind-shear} &\tabularnewline
Target type & Data instance\tabularnewline
Name & Aerodrome wind shear\tabularnewline
\begin{minipage}[t]{0.47\columnwidth}\raggedright
Requirement\strut
\end{minipage} & \begin{minipage}[t]{0.47\columnwidth}\raggedright
\url{http://icao.int/iwxxm/2.1/req/xsd-aerodrome-wind-shear/valid}

The content model of this element shall have a value that matches the content model of iwxxm:AerodromeWindShear.\strut
\end{minipage}\tabularnewline
\begin{minipage}[t]{0.47\columnwidth}\raggedright
Requirement\strut
\end{minipage} & \begin{minipage}[t]{0.47\columnwidth}\raggedright
\url{http://icao.int/iwxxm/2.1/req/xsd-aerodrome-wind-shear/applicable-runways}

If XML attribute //iwxxm:AerodromeWindShear/@allRunways is absent or has value ``false'', then one or more XML elements //iwxxm:AerodromeWindShear/iwxxm:runway, each with valid child element //iwxxm:AerodromeWindShear/iwxxm:runway/aixm:RunwayDirection, shall be used to indicate the set of runway directions to which wind shear conditions apply.\strut
\end{minipage}\tabularnewline
\begin{minipage}[t]{0.47\columnwidth}\raggedright
Requirement\strut
\end{minipage} & \begin{minipage}[t]{0.47\columnwidth}\raggedright
\url{http://icao.int/iwxxm/2.1/req/xsd-aerodrome-wind-shear/all-runways}

If XML attribute //iwxxm:AerodromeWindShear/@allRunways has value ``true'', then XML element //iwxxm:AerodromeWindShear/iwxxm:runway shall be absent.\strut
\end{minipage}\tabularnewline
\bottomrule
\end{longtable}

205-16.13 Requirements class: Aerodrome observed clouds

205-16.13.1 This requirements class is used to describe observed cloud conditions at an aerodrome. The class is targeted at providing a basic description of the observed cloud conditions as required for civil aviation purposes.

Notes:

1. Representations providing more detailed information may be used if required.

2. The requirements for reporting observed cloud conditions are specified in the \emph{Technical Regulations} (WMO-No.~49), Volume~II, Part~II, Appendix~3, 4.5.

205-16.13.2 XML elements describing observed cloud conditions shall conform to all requirements specified in Table~205-16.13.

205-16.13.3 XML elements describing observed cloud conditions shall conform to all requirements of all relevant dependencies specified in Table~205-16.13.

Table~205-16.13. Requirements class xsd-aerodrome-observed-clouds

\begin{longtable}[]{@{}ll@{}}
\toprule
Requirements class &\tabularnewline
\midrule
\endhead
\url{http://icao.int/iwxxm/2.1/req/xsd-aerodrome-observed-clouds} &\tabularnewline
Target type & Data instance\tabularnewline
Name & Aerodrome observed clouds\tabularnewline
Dependency & \url{http://icao.int/iwxxm/2.1/req/xsd-cloud-layer}, 205-16.39\tabularnewline
\begin{minipage}[t]{0.47\columnwidth}\raggedright
Requirement\strut
\end{minipage} & \begin{minipage}[t]{0.47\columnwidth}\raggedright
\url{http://icao.int/iwxxm/2.1/req/xsd-aerodrome-observed-clouds/valid}

The content model of this element shall have a value that matches the content model of iwxxm:AerodromeObservedClouds.\strut
\end{minipage}\tabularnewline
\begin{minipage}[t]{0.47\columnwidth}\raggedright
Requirement\strut
\end{minipage} & \begin{minipage}[t]{0.47\columnwidth}\raggedright
\url{http://icao.int/iwxxm/2.1/req/xsd-aerodrome-observed-clouds/amount-and-height-not-detectable-by-auto-system}

When an automatic observing system observes cumulonimbus clouds or towering cumulus clouds but the amount and height cannot be observed, the XML attribute\\
//iwxxm:AerodromeObservedClouds/@amountAndHeightUnobservableByAutoSystem shall have the value set to ``true'' and XML element //iwxxm:AerodromeObservedClouds/iwxxm:layer shall be absent.\strut
\end{minipage}\tabularnewline
\begin{minipage}[t]{0.47\columnwidth}\raggedright
Requirement\strut
\end{minipage} & \begin{minipage}[t]{0.47\columnwidth}\raggedright
\url{http://icao.int/iwxxm/2.1/req/xsd-aerodrome-observed-clouds/either-vertical-visibility-or-cloud-layers}

When vertical visibility is reported, cloud layers shall not be reported.

When cloud layers are reported, vertical visibility shall not be reported.\strut
\end{minipage}\tabularnewline
\begin{minipage}[t]{0.47\columnwidth}\raggedright
Requirement\strut
\end{minipage} & \begin{minipage}[t]{0.47\columnwidth}\raggedright
\url{http://icao.int/iwxxm/2.1/req/xsd-aerodrome-observed-clouds/vertical-visibility}

When cloud of operational significance is observed but the amount and height cannot be observed, then the XML element //iwxxm:AerodromeObservedClouds/iwxxm:verticalVisibility shall be used to report the vertical visibility.\strut
\end{minipage}\tabularnewline
\begin{minipage}[t]{0.47\columnwidth}\raggedright
Requirement\strut
\end{minipage} & \begin{minipage}[t]{0.47\columnwidth}\raggedright
\url{http://icao.int/iwxxm/2.1/req/xsd-aerodrome-observed-clouds/vertical-visibility-unit-of-measure}

If the vertical visibility is reported then the vertical distance shall be expressed in metres or feet. The unit of measure shall be indicated using the XML attribute\\
//iwxxm:AerodromeObservedClouds/iwxxm:verticalVisibility/@uom with value ``m'' (metres) or ``{[}ft\_i{]}'' (feet).\strut
\end{minipage}\tabularnewline
\begin{minipage}[t]{0.47\columnwidth}\raggedright
Requirement\strut
\end{minipage} & \begin{minipage}[t]{0.47\columnwidth}\raggedright
\url{http://icao.int/iwxxm/2.1/req/xsd-aerodrome-observed-clouds/cloud-layers}

When the amount and height of cloud of operational significance are observed, then the XML element //iwxxm:AerodromeObservedClouds/iwxxm:layer, containing a valid child element //iwxxm:AerodromeObservedClouds/iwxxm:layer/iwxxm:CloudLayer, shall be used to describe each cloud layer.\strut
\end{minipage}\tabularnewline
\begin{minipage}[t]{0.47\columnwidth}\raggedright
Requirement\strut
\end{minipage} & \begin{minipage}[t]{0.47\columnwidth}\raggedright
\href{http://icao.int/iwxxm/2.1/req/xsd-aerodrome-cloud-forecast/number-of-cloud-layers}{http://icao.int/iwxxm/2.1/req/xsd-aerodrome-observed-clouds/number-of-cloud-layers}

No more than four cloud layers shall be reported. If more than four significant cloud layers are observed, then the four most significant cloud layers with respect to aviation operations shall be prioritized.\strut
\end{minipage}\tabularnewline
\bottomrule
\end{longtable}

Notes:

1. Cloud of operational significance includes cloud below 1~500~metres or the highest minimum sector altitude, whichever is greater, and cumulonimbus whenever present.

2. Vertical visibility is defined as the vertical visual range into an obscuring medium.

3. Units of measurement are specified in accordance with 1.9 above.

205-16.14 Requirements class: Aerodrome surface wind

205-16.14.1 This requirements class is used to describe the surface wind conditions observed at an aerodrome.

Note: The requirements for reporting surface wind conditions are specified in the \emph{Technical Regulations} (WMO-No.~49), Volume~II, Part~II, Appendix~3, 4.1.

205-16.14.2 XML elements describing surface wind conditions shall conform to all requirements specified in Table~205-16.14.

205-16.14.3 XML elements describing surface wind conditions shall conform to all requirements of all relevant dependencies specified in Table~205-16.14.

Table~205-16.14. Requirements class xsd-aerodrome-surface-wind

\begin{longtable}[]{@{}ll@{}}
\toprule
Requirements class &\tabularnewline
\midrule
\endhead
\url{http://icao.int/iwxxm/2.1/req/xsd-aerodrome-surface-wind} &\tabularnewline
Target type & Data instance\tabularnewline
Name & Aerodrome surface wind\tabularnewline
\begin{minipage}[t]{0.47\columnwidth}\raggedright
Requirement\strut
\end{minipage} & \begin{minipage}[t]{0.47\columnwidth}\raggedright
\url{http://icao.int/iwxxm/2.1/req/xsd-aerodrome-surface-wind/valid}

The content model of this element shall have a value that matches the content model of iwxxm:AerodromeSurfaceWind.\strut
\end{minipage}\tabularnewline
\begin{minipage}[t]{0.47\columnwidth}\raggedright
Requirement\strut
\end{minipage} & \begin{minipage}[t]{0.47\columnwidth}\raggedright
\url{http://icao.int/iwxxm/2.1/req/xsd-aerodrome-surface-wind/mean-wind-speed}

The mean wind speed shall be stated using the XML element //iwxxm:AerodromeSurfaceWind/iwxxm:meanWindSpeed, with the unit of measure metres per second, knots or kilometres per hour. The unit of measure shall be indicated using the XML attribute //iwxxm:AerodromeSurfaceWind/iwxxm:meanWindSpeed/@uom with value ``m/s'' (metres per second), ``{[}kn\_i{]}'' (knots) or ``km/h'' (kilometres per hour).\strut
\end{minipage}\tabularnewline
\begin{minipage}[t]{0.47\columnwidth}\raggedright
Requirement\strut
\end{minipage} & \begin{minipage}[t]{0.47\columnwidth}\raggedright
\url{http://icao.int/iwxxm/2.1/req/xsd-aerodrome-surface-wind/variable-wind-direction}

If the wind direction is variable, then the XML attribute //iwxxm:AerodromeSurfaceWind/@variableDirection shall have the value ``true'' and XML element //iwxxm:AerodromeSurfaceWind/iwxxm:meanWindDirection shall be absent.\strut
\end{minipage}\tabularnewline
\begin{minipage}[t]{0.47\columnwidth}\raggedright
Requirement\strut
\end{minipage} & \begin{minipage}[t]{0.47\columnwidth}\raggedright
\url{http://icao.int/iwxxm/2.1/req/xsd-aerodrome-surface-wind/steady-wind-direction}

If the wind direction is not variable, then:

(i) The observed angle between true north and the mean direction from which the wind is blowing shall be expressed using XML element //iwxxm:AerodromeSurfaceWind/iwxxm:meanWindDirection, with the unit of measure indicated using the XML attribute //iwxxm:AerodromeSurfaceWind/iwxxm:meanWindDirection/@uom with value ``deg'';

(ii) The XML attribute //iwxxm:AerodromeSurfaceWind/@variableDirection shall be absent or have the value ``false''.\strut
\end{minipage}\tabularnewline
\begin{minipage}[t]{0.47\columnwidth}\raggedright
Requirement\strut
\end{minipage} & \begin{minipage}[t]{0.47\columnwidth}\raggedright
\url{http://icao.int/iwxxm/2.1/req/xsd-aerodrome-surface-wind/extreme-wind-direction}

If the extremes of wind direction variability are reported, then:

(i) The observed angle between true north and extreme clockwise direction from which the wind is blowing shall be expressed using XML element //iwxxm:AerodromeSurfaceWind/iwxxm:extremeClockWiseWindDirection;

(ii) The observed angle between true north and extreme counterclockwise direction from which the wind is blowing shall be expressed using XML element //iwxxm:AerodromeSurfaceWind/iwxxm:extremeCounterClockWiseWindDirection;

(iii) The unit of measure for each extreme wind direction shall be indicated using the XML attribute @uom with value ``deg''.\strut
\end{minipage}\tabularnewline
\begin{minipage}[t]{0.47\columnwidth}\raggedright
Requirement\strut
\end{minipage} & \begin{minipage}[t]{0.47\columnwidth}\raggedright
\url{http://icao.int/iwxxm/2.1/req/xsd-aerodrome-surface-wind/gust-speed}

If reported, the observed gust speed shall be stated using the XML element //iwxxm:AerodromeSurfaceWind/iwxxm:windGustSpeed and expressed in metres per second, knots or kilometres per hour.

The unit of measure shall be indicated using the XML attribute //iwxxm:AerodromeSurfaceWind/iwxxm:windGustSpeed/@uom with value ``m/s'' (metres per second), ``{[}kn\_i{]}'' (knots) or ``km/h'' (kilometres per hour).\strut
\end{minipage}\tabularnewline
\bottomrule
\end{longtable}

Notes:

1. The mean wind speed is the average wind speed observed over the previous 10~minutes.

2. The gust speed is the maximum wind speed observed over the previous 10~minutes.

3. Wind direction is reported as variable (VRB) if, during the 10-minute observation of mean wind speed, the variation of wind direction is (i) 180~degrees or more, or (ii) 60~degrees or more when the wind speed is less than 1.5~metres per second (3~knots).

4. Extreme directional variations of wind are reported if, during the 10-minute observation of mean wind speed, the variation of wind direction is 60 degrees or more and less than 180~degrees and the wind speed is 1.5~metres per second (3~knots) or more.

5. The absence of XML attribute //iwxxm:AerodromeSurfaceWind/@variableDirection implies a ``false'' value; for example, the wind direction is not variable.

6. Units of measurement are specified in accordance with 1.9 above.

7. The true north is the north point at which the meridian lines meet.

205-16.15 Requirements class: Aerodrome horizontal visibility

205-16.15.1 This requirements class is used to describe the horizontal visibility conditions observed at an aerodrome.

Note: The requirements for reporting horizontal visibility are specified in the \emph{Technical Regulations} (WMO-No.~49), Volume~II, Part~II, Appendix~3, 4.2.

205-16.15.2 XML elements describing horizontal visibility shall conform to all requirements specified in Table~205-16.15.

205-16.15.3 XML elements describing horizontal visibility shall conform to all requirements of all relevant dependencies specified in Table~205-16.15.

Table~205-16.15. Requirements class xsd-aerodrome-horizontal-visibility

\begin{longtable}[]{@{}ll@{}}
\toprule
Requirements class &\tabularnewline
\midrule
\endhead
\url{http://icao.int/iwxxm/2.1/req/xsd-aerodrome-horizontal-visibility} &\tabularnewline
Target type & Data instance\tabularnewline
Name & Aerodrome horizontal visibility\tabularnewline
\begin{minipage}[t]{0.47\columnwidth}\raggedright
Requirement\strut
\end{minipage} & \begin{minipage}[t]{0.47\columnwidth}\raggedright
\url{http://icao.int/iwxxm/2.1/req/xsd-aerodrome-horizontal-visibility/valid}

The content model of this element shall have a value that matches the content model of iwxxm:AerodromeHorizontalVisibility.\strut
\end{minipage}\tabularnewline
\begin{minipage}[t]{0.47\columnwidth}\raggedright
Requirement\strut
\end{minipage} & \begin{minipage}[t]{0.47\columnwidth}\raggedright
\href{http://icao.int/iwxxm/1.1/req/xsd-aerodrome-horizontal-visibility/prevailing-visibility}{http://icao.int/iwxxm/2.1/req/xsd-aerodrome-horizontal-visibility/prevailing-visibility}

The prevailing visibility shall be stated using the XML element //iwxxm:AerodromeHorizontalVisibility/iwxxm:prevailingVisibility with the unit of measure metres, indicated using the XML attribute //iwxxm:AerodromeHorizontalVisibility/iwxxm:prevailingVisibility/@uom with value ``m''.\strut
\end{minipage}\tabularnewline
\begin{minipage}[t]{0.47\columnwidth}\raggedright
Requirement\strut
\end{minipage} & \begin{minipage}[t]{0.47\columnwidth}\raggedright
\href{http://icao.int/iwxxm/1.1/req/xsd-aerodrome-horizontal-visibility/prevailing-visibility-exceeds-10000m}{http://icao.int/iwxxm/2.1/req/xsd-aerodrome-horizontal-visibility/prevailing-visibility-exceeds-10000m}

If the prevailing visibility exceeds 10~000 metres, then the numeric value of XML element //iwxxm:AerodromeHorizontalVisibility/iwxxm:prevailingVisibility shall be set to 10000 and the XML element //iwxxm:AerodromeHorizontalVisibility/iwxxm:prevailingVisibilityOperator shall have the value ``ABOVE''.\strut
\end{minipage}\tabularnewline
\begin{minipage}[t]{0.47\columnwidth}\raggedright
Requirement\strut
\end{minipage} & \begin{minipage}[t]{0.47\columnwidth}\raggedright
\href{http://icao.int/iwxxm/1.1/req/xsd-aerodrome-horizontal-visibility/prevailing-visibility-comparison-operator}{http://icao.int/iwxxm/2.1/req/xsd-aerodrome-horizontal-visibility/prevailing-visibility-comparison-operator}

If present, the value of XML element //iwxxm:AerodromeHorizontalVisibility/iwxxm:prevailingVisibilityOperator shall be one of the enumeration: ``ABOVE'' or ``BELOW''.\strut
\end{minipage}\tabularnewline
\begin{minipage}[t]{0.47\columnwidth}\raggedright
Requirement\strut
\end{minipage} & \begin{minipage}[t]{0.47\columnwidth}\raggedright
\href{http://icao.int/iwxxm/1.1/req/xsd-aerodrome-horizontal-visibility/minimum-visibility}{http://icao.int/iwxxm/2.1/req/xsd-aerodrome-horizontal-visibility/minimum-visibility}

If reported, the minimum visibility shall be expressed using XML element //iwxxm:AerodromeHorizontalVisibility/iwxxm:minimumVisibility with the unit of measure metres, indicated using the XML attribute //iwxxm:AerodromeHorizontalVisibility/iwxxm:minimumVisibility/@uom with value ``m''.\strut
\end{minipage}\tabularnewline
\begin{minipage}[t]{0.47\columnwidth}\raggedright
Requirement\strut
\end{minipage} & \begin{minipage}[t]{0.47\columnwidth}\raggedright
\href{http://icao.int/iwxxm/1.1/req/xsd-aerodrome-horizontal-visibility/minimum-visibility-direction}{http://icao.int/iwxxm/2.1/req/xsd-aerodrome-horizontal-visibility/minimum-visibility-direction}

If reported, the observed angle between true north and the direction of minimum visibility shall be expressed in degrees using XML element //iwxxm:AerodromeHorizontalVisibility/iwxxm:mininumVisibilityDirection, with the unit of measure indicated using the XML attribute //iwxxm:AerodromeHorizontalVisibility/iwxxm:mininumVisibilityDirection/@uom with value ``deg''.\strut
\end{minipage}\tabularnewline
\bottomrule
\end{longtable}

Notes:

1. Units of measurement are specified in accordance with 1.9 above.

2. Visibility for aeronautical purposes is defined as the greater of: (i) the greatest distance at which a black object of suitable dimensions, situated near the ground, can be seen and recognized when observed against a bright background; or (ii) the greatest distance at which lights in the vicinity of 1~000~candelas can be seen and identified against an unlit background.

3. Prevailing visibility is defined as the greatest visibility value observed which is reached within at least half the horizon circle or within at least half of the surface of the aerodrome. These areas could comprise contiguous or non-contiguous sectors.

4. The absence of XML element //iwxxm:AerodromeHorizontalVisibility/iwxxm:prevailingVisibilityOperator indicates that the prevailing visibility has the numeric value reported.

5. The conditions for reporting minimum visibility are that the visibility is not the same in different directions and (i)~when the lowest visibility is different from the prevailing visibility and less than 1~500~metres or less than 50\% of the prevailing visibility and less than 5~000~metres, or (ii) when the visibility is fluctuating rapidly and the prevailing visibility cannot be determined.

6. When reporting minimum visibility, the general direction of the minimum visibility in relation to the aerodrome should be reported unless the visibility is fluctuating rapidly.

7. The true north is the north point at which the meridian lines meet.

205-16.16 Requirements class: TAF

205-16.16.1 This requirements class is used to describe the aerodrome forecast (TAF) report within which a base forecast, and optionally one or more change forecasts, is provided.

Note: The reporting requirements for aerodrome forecasts are specified in the \emph{Technical Regulations} (WMO-No.~49), Volume~II, Part~II, Appendix~3 and Appendix~5, section~1.

205-16.16.2 XML elements describing TAFs shall conform to all requirements specified in Table~205-16.16.

205-16.16.3 XML elements describing TAFs shall conform to all requirements of all relevant dependencies specified in Table~205-16.16.

Table~205-16.16. Requirements class xsd-taf

\begin{longtable}[]{@{}ll@{}}
\toprule
Requirements class &\tabularnewline
\midrule
\endhead
\href{http://icao.int/iwxxm/1.1/req/xsd-taf}{http://icao.int/iwxxm/2.1/req/xsd-taf} &\tabularnewline
Target type & Data instance\tabularnewline
Name & TAF\tabularnewline
\begin{minipage}[t]{0.47\columnwidth}\raggedright
Requirement\strut
\end{minipage} & \begin{minipage}[t]{0.47\columnwidth}\raggedright
\href{http://icao.int/iwxxm/1.1/req/xsd-taf/valid}{http://icao.int/iwxxm/2.1/req/xsd-taf/valid}

The content model of this element shall have a value that matches the content model of iwxxm:TAF.\strut
\end{minipage}\tabularnewline
\begin{minipage}[t]{0.47\columnwidth}\raggedright
Requirement\strut
\end{minipage} & \begin{minipage}[t]{0.47\columnwidth}\raggedright
\href{http://icao.int/iwxxm/1.1/req/xsd-taf/status}{http://icao.int/iwxxm/2.1/req/xsd-taf/status}

The status of the TAF shall be indicated using the XML attribute //iwxxm:TAF/@status with the value being one of the enumeration: ``NORMAL'', ``AMENDMENT'', ``CANCELLATION'', ``CORRECTION'' or ``MISSING''.\strut
\end{minipage}\tabularnewline
\begin{minipage}[t]{0.47\columnwidth}\raggedright
Requirement\strut
\end{minipage} & \begin{minipage}[t]{0.47\columnwidth}\raggedright
\href{http://icao.int/iwxxm/1.1/req/xsd-taf/issue-time}{http://icao.int/iwxxm/2.1/req/xsd-taf/issue-time}

The XML element //iwxxm:TAF/iwxxm:issueTime shall contain a valid child element gml:TimeInstant that describes the time at which the TAF was issued.\strut
\end{minipage}\tabularnewline
\begin{minipage}[t]{0.47\columnwidth}\raggedright
Requirement\strut
\end{minipage} & \begin{minipage}[t]{0.47\columnwidth}\raggedright
\href{http://icao.int/iwxxm/1.1/req/xsd-taf/base-forecast}{http://icao.int/iwxxm/2.1/req/xsd-taf/base-forecast}

If the prevailing forecast conditions for the valid period of the TAF are reported, then:

(i) The XML element //iwxxm:TAF/iwxxm:baseForecast shall contain a valid child element om:OM\_Observation of type MeteorologicalAerodromeForecast;

(ii) The value of XML attribute //iwxxm:TAF/iwxxm:baseForecast/om:OM\_Observation/om:type/@xlink:href shall be the URI ``\url{http://codes.wmo.int/49-2/observation-type/IWXXM/1.0/MeteorologicalAerodromeForecast}''; and

(iii) The XML attribute //iwxxm:TAF/iwxxm:baseForecast/om:OM\_Observation/om:result/iwxxm:MeteorologicalAerodromeForecastRecord/@changeIndicator shall be absent.\strut
\end{minipage}\tabularnewline
\begin{minipage}[t]{0.47\columnwidth}\raggedright
Requirement\strut
\end{minipage} & \begin{minipage}[t]{0.47\columnwidth}\raggedright
\href{http://icao.int/iwxxm/1.1/req/xsd-taf/change-forecast}{http://icao.int/iwxxm/2.1/req/xsd-taf/change-forecast}

If change forecasts or forecasts with probability of occurrence are reported, then:

(i) The XML element //iwxxm:TAF/iwxxm:changeForecast shall contain a valid child element om:OM\_Observation of type MeteorologicalAerodromeForecast;

(ii) The value of XML attribute //iwxxm:TAF/iwxxm:changeForecast/om:OM\_Observation/om:type/@xlink:href shall be the URI ``\url{http://codes.wmo.int/49-2/observation-type/IWXXM/1.0/MeteorologicalAerodromeForecast}'';

(iii) The XML element //iwxxm:TAF/iwxxm:changeForecast/om:OM\_Observation/om:result/iwxxm:MeteorologicalAerodromeForecastRecord/iwxxm:temperature shall be absent; and

(iv) The XML attribute //iwxxm:TAF/iwxxm:baseForecast/om:OM\_Observation/om:result/iwxxm:MeteorologicalAerodromeForecastRecord/@changeIndicator shall be one of the enumeration: ``BECOMING'', ``TEMPORARY\_FLUCTUATIONS'', ``FROM'', ``PROBABILITY\_30'', ``PROBABILITY\_30\_TEMPORARY\_FLUCTUATIONS'', ``PROBABILITY\_40'' or ``PROBABILITY\_40\_TEMPORARY\_FLUCTUATIONS''.\strut
\end{minipage}\tabularnewline
\begin{minipage}[t]{0.47\columnwidth}\raggedright
Requirement\strut
\end{minipage} & \begin{minipage}[t]{0.47\columnwidth}\raggedright
\href{http://icao.int/iwxxm/1.1/req/xsd-taf/unique-subject-aerodrome}{http://icao.int/iwxxm/2.1/req/xsd-taf/unique-subject-aerodrome}

The base forecast and, if reported, change forecasts shall refer to the same aerodrome. All values of XML element //iwxxm:TAF/*/om:OM\_Observation/om:featureOfInterest/sams:SF\_SpatialSamplingFeature/sam:sampledFeature/aixm:AirportHeliport/gml:identifier within the TAF shall be identical.\strut
\end{minipage}\tabularnewline
\begin{minipage}[t]{0.47\columnwidth}\raggedright
Requirement\strut
\end{minipage} & \begin{minipage}[t]{0.47\columnwidth}\raggedright
\href{http://icao.int/iwxxm/1.1/req/xsd-taf/status-normal}{http://icao.int/iwxxm/2.1/req/xsd-taf/status-normal}

If the status of the TAF is ``NORMAL'' (as specified by XML attribute //iwxxm:TAF/@status), then:

(i) The prevailing meteorological conditions anticipated during the valid period of the TAF shall be reported using the XML element //iwxxm:TAF/iwxxm:baseForecast;

(ii) The valid time period of the TAF shall be given using the XML element //iwxxm:TAF/iwxxm:validTime/gml:TimePeriod;

(iii) The XML element //iwxxm:TAF/iwxxm:previousReportAerodrome shall be absent; and

(iv) The XML element //iwxxm:TAF/iwxxm:previousReportValidPeriod shall be absent.\strut
\end{minipage}\tabularnewline
\begin{minipage}[t]{0.47\columnwidth}\raggedright
Requirement\strut
\end{minipage} & \begin{minipage}[t]{0.47\columnwidth}\raggedright
\href{http://icao.int/iwxxm/1.1/req/xsd-taf/status-amendment-or-correction}{http://icao.int/iwxxm/2.1/req/xsd-taf/status-amendment-or-correction}

If the status of the TAF is ``AMENDMENT'' or ``CORRECTION'' (as specified by XML attribute //iwxxm:TAF/@status), then:

(i) The prevailing meteorological conditions anticipated during the valid period of the TAF shall be reported using the XML element //iwxxm:TAF/iwxxm:baseForecast;

(ii) The valid time period of the TAF shall be given using the XML element //iwxxm:TAF/iwxxm:validTime/gml:TimePeriod; and

(iii) The valid time period for the TAF that has been amended or corrected shall be reported using the XML element //iwxxm:TAF/iwxxm:previousReportValidPeriod/gml:TimePeriod.\strut
\end{minipage}\tabularnewline
\begin{minipage}[t]{0.47\columnwidth}\raggedright
Requirement\strut
\end{minipage} & \begin{minipage}[t]{0.47\columnwidth}\raggedright
http://icao.int/iwxxm/2.1/req/xsd-taf/status-cancellation

If the status of the TAF is ``CANCELLATION'' (as specified by XML attribute //iwxxm:TAF/@status), then:

(i) The XML element //iwxxm:TAF/iwxxm:baseForecast shall be absent;

(ii) The XML element //iwxxm:TAF/iwxxm:changeForecast shall be absent;

(iii) The time period for which TAF reports at the subject aerodrome are cancelled shall be given using the XML element //iwxxm:TAF/iwxxm:validTime/gml:TimePeriod;

(iv) The aerodrome for which TAF reports are cancelled shall be reported using the XML element //iwxxm:TAF/iwxxm:previousReportAerodrome/aixm:AirportHeliport; and

(v) The valid time period for the TAF that has been cancelled shall be reported using the XML element //iwxxm:TAF/iwxxm:previousReportValidPeriod/gml:TimePeriod.\strut
\end{minipage}\tabularnewline
\begin{minipage}[t]{0.47\columnwidth}\raggedright
Requirement\strut
\end{minipage} & \begin{minipage}[t]{0.47\columnwidth}\raggedright
http://icao.int/iwxxm/2.1/req/xsd-taf/nil-report-status-missing

If the status of the TAF is ``MISSING'' (as specified by XML attribute //iwxxm:TAF/@status), then:

(i) The XML element //iwxxm:TAF/iwxxm:baseForecast shall contain valid child element om:OM\_Observation of type MeteorologicalAerodromeForecast;

(ii) The value of XML element //iwxxm:TAF/iwxxm:baseForecast/om:OM\_Observation/om:featureOfInterest/sams:SF\_SpatialSamplingFeature/sam:sampledFeature/aixm:AirportHeliport shall indicate the aerodrome for which the TAF is missing;

(iii) The XML element //iwxxm:TAF/iwxxm:baseForecast/om:OM\_Observation/om:result shall have no child elements and XML attribute //iwxxm:TAF/iwxxm:baseForecast/om:OM\_Observation/om:result/@nilReason shall provide an appropriate nil reason;

(iv) The XML element //iwxxm:TAF/iwxxm:changeForecast shall be absent;

(v) The XML element //iwxxm:TAF/iwxxm:validTime shall be absent;

(vi) The XML element //iwxxm:TAF/iwxxm:previousReportAerodrome shall be absent; and

(vii) The XML element //iwxxm:TAF/iwxxm:previousReportValidPeriod shall be absent.\strut
\end{minipage}\tabularnewline
\begin{minipage}[t]{0.47\columnwidth}\raggedright
Recommendation\strut
\end{minipage} & \begin{minipage}[t]{0.47\columnwidth}\raggedright
\href{http://icao.int/iwxxm/1.1/req/xsd-taf/number-of-change-forecasts}{http://icao.int/iwxxm/2.1/req/xsd-taf/number-of-change-forecasts}

The number of change forecasts should be kept to a minimum, and no more than five change forecasts should be reported in normal circumstances.\strut
\end{minipage}\tabularnewline
\begin{minipage}[t]{0.47\columnwidth}\raggedright
Recommendation\strut
\end{minipage} & \begin{minipage}[t]{0.47\columnwidth}\raggedright
\href{http://icao.int/iwxxm/1.1/req/xsd-taf/issue-time-matches-result-time}{http://icao.int/iwxxm/2.1/req/xsd-taf/issue-time-matches-result-time}

The TAF issue time (specified by XML element //iwxxm:TAF/iwxxm:issueTime/gml:TimeInstant) should match the result time for each of the forecasts provided within the TAF (specified by XML element //iwxxm:TAF/*/om:OM\_Observation/om:resultTime/gml:TimeInstant).\strut
\end{minipage}\tabularnewline
\begin{minipage}[t]{0.47\columnwidth}\raggedright
Recommendation\strut
\end{minipage} & \begin{minipage}[t]{0.47\columnwidth}\raggedright
http://icao.int/iwxxm/2.1/req/xsd-taf/valid-time-includes-all-phenomenon-times

The valid times of all forecasts included in the TAF (specified by XML element\\
//iwxxm:TAF/*/om:OM\_Observation/om:phenomenonTime/*) should occur within the valid time period of the TAF (specified by XML element //iwxxm:TAF/iwxxm:validTime/gml:TimePeriod).\strut
\end{minipage}\tabularnewline
\begin{minipage}[t]{0.47\columnwidth}\raggedright
Recommendation\strut
\end{minipage} & \begin{minipage}[t]{0.47\columnwidth}\raggedright
\href{http://icao.int/iwxxm/1.1/req/xsd-taf/status-amendment-or-correction-previous-aerodrome}{http://icao.int/iwxxm/2.1/req/xsd-taf/status-amendment-or-correction-previous-aerodrome}

If the status of the TAF is ``AMENDMENT'' or ``CORRECTION'' (as specified by XML attribute //iwxxm:TAF/@status), then the aerodrome that was the subject of the TAF that has been amended or corrected should be reported using the XML element\\
//iwxxm:TAF/iwxxm:previousReportAerodrome/aixm:AirportHeliport\strut
\end{minipage}\tabularnewline
\bottomrule
\end{longtable}

Notes:

1. The requirements for reporting the following are specified in the \emph{Technical Regulations} (WMO-No.~49), Volume~II, Part~II, Appendix~5:

(a) Change forecasts section~1.3

(b) Probability forecasts with probability of occurrence paragraph~1.4

2. Guidance regarding the number of change forecasts or forecasts with probability of occurrence is given in the \emph{Technical Regulations} (WMO-No.~49), Volume~II, Part~II, Appendix~5, 1.5.

3. A report with status ``MISSING'' indicates that a routine report has not been provided on the anticipated timescales. Such a report does not contain details of any forecast meteorological conditions and is often referred to as a ``NIL'' report

4. Within an XML encoded TAF, it is likely that only one instance of aixm:AirportHeliport will physically be present; subsequent assertions about the aerodrome may use xlinks to refer to the previously defined aixm:AirportHeliport element in order to keep the XML document size small. As such, validation of requirement \url{http://icao.int/iwxxm/2.1/req/xsd-taf/unique-subject-aerodrome} is applied once any xlinks, if used, have been resolved.

5. Code table~D-1 provides a set of nil-reason codes and is published at \url{http://codes.wmo.int/common/nil}.

205-16.17 Requirements class: Meteorological aerodrome forecast record

205-16.17.1 This requirements class is used to describe the aggregated set of meteorological conditions forecast at an aerodrome as appropriate for inclusion in a aerodrome forecast (TAF) report.

205-16.17.2 XML elements describing the set of meteorological conditions for inclusion in an aerodrome forecast shall conform to all requirements specified in Table~205-16.17.

205-16.17.3 XML elements describing the set of meteorological conditions for inclusion in an aerodrome forecast shall conform to all requirements of all relevant dependencies specified in Table~205-16.17.

Table~205-16.17. Requirements class xsd-meterological-aerodrome-forecast-record

\begin{longtable}[]{@{}ll@{}}
\toprule
Requirements class &\tabularnewline
\midrule
\endhead
\href{http://icao.int/iwxxm/1.1/req/xsd-meteorological-aerodrome-forecast-record}{http://icao.int/iwxxm/2.1/req/xsd-meteorological-aerodrome-forecast-record} &\tabularnewline
Target type & Data instance\tabularnewline
Name & Meteorological aerodrome forecast record\tabularnewline
Dependency & \href{http://icao.int/iwxxm/1.1/req/xsd-aerodrome-cloud-forecast}{http://icao.int/iwxxm/2.1/req/xsd-aerodrome-cloud-forecast}, 205-16.36\tabularnewline
Dependency & \href{http://icao.int/iwxxm/1.1/req/xsd-aerodrome-surface-wind-forecast}{http://icao.int/iwxxm/2.1/req/xsd-aerodrome-surface-wind-forecast}, 205-16.37\tabularnewline
Dependency & \href{http://icao.int/iwxxm/1.1/req/xsd-aerodrome-air-temperature-forecast}{http://icao.int/iwxxm/2.1/req/xsd-aerodrome-air-temperature-forecast}, 205-16.18\tabularnewline
\begin{minipage}[t]{0.47\columnwidth}\raggedright
Requirement\strut
\end{minipage} & \begin{minipage}[t]{0.47\columnwidth}\raggedright
\href{http://icao.int/iwxxm/1.1/req/xsd-meteorological-aerodrome-forecast-record/valid}{http://icao.int/iwxxm/2.1/req/xsd-meteorological-aerodrome-forecast-record/valid}

The content model of this element shall have a value that matches the content model of iwxxm:MeteorologicalAerodromeForecastRecord.\strut
\end{minipage}\tabularnewline
\begin{minipage}[t]{0.47\columnwidth}\raggedright
Requirement\strut
\end{minipage} & \begin{minipage}[t]{0.47\columnwidth}\raggedright
\href{http://icao.int/iwxxm/1.1/req/xsd-meteorological-aerodrome-forecast-record/prevailing-forecast-conditions}{http://icao.int/iwxxm/2.1/req/xsd-meteorological-aerodrome-forecast-record/prevailing-forecast-conditions}

The XML attribute //iwxxm:MeteorologicalAerodromeForecastRecord/@changeIndicator shall be absent if the forecast describes the prevailing meteorological conditions.\strut
\end{minipage}\tabularnewline
\begin{minipage}[t]{0.47\columnwidth}\raggedright
Requirement\strut
\end{minipage} & \begin{minipage}[t]{0.47\columnwidth}\raggedright
\href{http://icao.int/iwxxm/1.1/req/xsd-meteorological-aerodrome-forecast-record/change-indicator-fm}{http://icao.int/iwxxm/2.1/req/xsd-meteorological-aerodrome-forecast-record/change-indicator-fm}

If the meteorological conditions forecast for the aerodrome are expected to change significantly and more or less completely to a different set of conditions, then the XML attribute //iwxxm:MeteorologicalAerodromeForecastRecord/@changeIndicator shall have the value ``FROM''.\strut
\end{minipage}\tabularnewline
\begin{minipage}[t]{0.47\columnwidth}\raggedright
Requirement\strut
\end{minipage} & \begin{minipage}[t]{0.47\columnwidth}\raggedright
\href{http://icao.int/iwxxm/1.1/req/xsd-meteorological-aerodrome-forecast-record/change-indicator-becmg}{http://icao.int/iwxxm/2.1/req/xsd-meteorological-aerodrome-forecast-record/change-indicator-becmg}

If the meteorological conditions forecast for the aerodrome are expected to reach or pass through specified values at a regular or irregular rate, then the XML attribute\\
//iwxxm:MeteorologicalAerodromeForecastRecord/@changeIndicator shall have the value ``BECOMING''.\strut
\end{minipage}\tabularnewline
\begin{minipage}[t]{0.47\columnwidth}\raggedright
Requirement\strut
\end{minipage} & \begin{minipage}[t]{0.47\columnwidth}\raggedright
\href{http://icao.int/iwxxm/1.1/req/xsd-meteorological-aerodrome-forecast-record/change-indicator-tempo}{http://icao.int/iwxxm/2.1/req/xsd-meteorological-aerodrome-forecast-record/change-indicator-tempo}

If temporary fluctuations in the meteorological conditions forecast for the aerodrome are expected to occur, then the XML attribute\\
//iwxxm:MeteorologicalAerodromeForecastRecord/@changeIndicator shall have the value ``TEMPORARY\_FLUCTUATIONS''.\strut
\end{minipage}\tabularnewline
\begin{minipage}[t]{0.47\columnwidth}\raggedright
Requirement\strut
\end{minipage} & \begin{minipage}[t]{0.47\columnwidth}\raggedright
\href{http://icao.int/iwxxm/1.1/req/xsd-meteorological-aerodrome-forecast-record/change-indicator-prob30}{http://icao.int/iwxxm/2.1/req/xsd-meteorological-aerodrome-forecast-record/change-indicator-prob30}

If meteorological conditions forecast for the aerodrome have a 30\% probability of occurring, then the XML attribute //iwxxm:MeteorologicalAerodromeForecastRecord/@changeIndicator shall have the value ``PROBABILITY\_30''.\strut
\end{minipage}\tabularnewline
\begin{minipage}[t]{0.47\columnwidth}\raggedright
Requirement\strut
\end{minipage} & \begin{minipage}[t]{0.47\columnwidth}\raggedright
\href{http://icao.int/iwxxm/1.1/req/xsd-meteorological-aerodrome-forecast-record/change-indicator-prob30-tempo}{http://icao.int/iwxxm/2.1/req/xsd-meteorological-aerodrome-forecast-record/change-indicator-prob30-tempo}

If the temporary fluctuations in meteorological conditions forecast have a 30\% probability of occurring, then the XML attribute\\
//iwxxm:MeteorologicalAerodromeForecastRecord/@changeIndicator shall have the value ``PROBABILITY\_30\_TEMPORARY\_FLUCTUATIONS''.\strut
\end{minipage}\tabularnewline
\begin{minipage}[t]{0.47\columnwidth}\raggedright
Requirement\strut
\end{minipage} & \begin{minipage}[t]{0.47\columnwidth}\raggedright
\href{http://icao.int/iwxxm/1.1/req/xsd-meteorological-aerodrome-forecast-record/change-indicator-prob40}{http://icao.int/iwxxm/2.1/req/xsd-meteorological-aerodrome-forecast-record/change-indicator-prob40}

If meteorological conditions forecast for the aerodrome have a 40\% probability of occurring, then the XML attribute //iwxxm:MeteorologicalAerodromeForecastRecord/@changeIndicator shall have the value ``PROBABILITY\_40''.\strut
\end{minipage}\tabularnewline
\begin{minipage}[t]{0.47\columnwidth}\raggedright
Requirement\strut
\end{minipage} & \begin{minipage}[t]{0.47\columnwidth}\raggedright
\href{http://icao.int/iwxxm/1.1/req/xsd-meteorological-aerodrome-forecast-record/change-indicator-prob40-tempo}{http://icao.int/iwxxm/2.1/req/xsd-meteorological-aerodrome-forecast-record/change-indicator-prob40-tempo}

If the temporary fluctuations in meteorological conditions forecast have a 40\% probability of occurring, then the XML attribute\\
//iwxxm:MeteorologicalAerodromeForecastRecord/@changeIndicator shall have the value ``PROBABILITY\_40\_TEMPORARY\_FLUCTUATIONS''.\strut
\end{minipage}\tabularnewline
\begin{minipage}[t]{0.47\columnwidth}\raggedright
Requirement\strut
\end{minipage} & \begin{minipage}[t]{0.47\columnwidth}\raggedright
\href{http://icao.int/iwxxm/1.1/req/xsd-meteorological-aerodrome-forecast-record/cavok}{http://icao.int/iwxxm/2.1/req/xsd-meteorological-aerodrome-forecast-record/cavok}

If the conditions associated with CAVOK are forecast, then:

(i) The XML attribute //iwxxm:MeteorologicalAerodromeForecastRecord/\\
@cloudAndVisibilityOK shall have the value ``true''; and

(ii) The following XML elements shall be absent: //iwxxm:MeteorologicalAerodromeForecastRecord/iwxxm:prevailingVisibility,\\
//iwxxm:MeteorologicalAerodromeForecastRecord/iwxxm:prevailingVisibilityOperator,\\
//iwxxm:MeteorologicalAerodromeForecastRecord/iwxxm:weather and\\
//iwxxm:MeteorologicalAerodromeForecastRecord/iwxxm:cloud.\strut
\end{minipage}\tabularnewline
\begin{minipage}[t]{0.47\columnwidth}\raggedright
Requirement\strut
\end{minipage} & \begin{minipage}[t]{0.47\columnwidth}\raggedright
\href{http://icao.int/iwxxm/1.1/req/xsd-meteorological-aerodrome-forecast-record/prevailing-visiblity}{http://icao.int/iwxxm/2.1/req/xsd-meteorological-aerodrome-forecast-record/prevailing-visiblity}

If reported, the prevailing visibility shall be stated using the XML element\\
//iwxxm:MeteorologicalAerodromeForecastRecord/iwxxm:prevailingVisibility with the unit of measure metres, indicated using the XML attribute\\
//iwxxm:MeteorologicalAerodromeForecastRecord/iwxxm:prevailingVisibility/@uom with value ``m''.\strut
\end{minipage}\tabularnewline
\begin{minipage}[t]{0.47\columnwidth}\raggedright
Requirement\strut
\end{minipage} & \begin{minipage}[t]{0.47\columnwidth}\raggedright
\href{http://icao.int/iwxxm/1.1/req/xsd-meteorological-aerodrome-forecast-record/prevailing-visibility-exceeds-10000m}{http://icao.int/iwxxm/2.1/req/xsd-meteorological-aerodrome-forecast-record/prevailing-visibility-exceeds-10000m}

If the prevailing visibility exceeds 10~000 metres, then the numeric value of XML element //iwxxm:MeteorologicalAerodromeForecastRecord/iwxxm:prevailingVisibility shall be set to 10000 and the XML element\\
//iwxxm:MeteorologicalAerodromeForecastRecord/iwxxm:prevailingVisibilityOperator shall have the value ``ABOVE''.\strut
\end{minipage}\tabularnewline
\begin{minipage}[t]{0.47\columnwidth}\raggedright
Requirement\strut
\end{minipage} & \begin{minipage}[t]{0.47\columnwidth}\raggedright
\href{http://icao.int/iwxxm/1.1/req/xsd-meteorological-aerodrome-forecast-record/prevailing-visibility-comparison-operator}{http://icao.int/iwxxm/2.1/req/xsd-meteorological-aerodrome-forecast-record/prevailing-visibility-comparison-operator}

If present, the value of XML element //iwxxm:MeteorologicalAerodromeForecastRecord/iwxxm:prevailingVisibilityOperator shall be one of the enumeration: ``ABOVE'' or ``BELOW''.\strut
\end{minipage}\tabularnewline
\begin{minipage}[t]{0.47\columnwidth}\raggedright
Requirement\strut
\end{minipage} & \begin{minipage}[t]{0.47\columnwidth}\raggedright
\href{http://icao.int/iwxxm/1.1/req/xsd-meteorological-aerodrome-forecast-record/temperature}{http://icao.int/iwxxm/2.1/req/xsd-meteorological-aerodrome-forecast-record/temperature}

If reported, the temperature conditions forecast for the aerodrome shall be expressed using the XML element //iwxxm:MeteorologicalAerodromeForecastRecord/iwxxm:temperature containing a valid child element iwxxm:AerodromeAirTemperatureForecast.\strut
\end{minipage}\tabularnewline
\begin{minipage}[t]{0.47\columnwidth}\raggedright
Requirement\strut
\end{minipage} & \begin{minipage}[t]{0.47\columnwidth}\raggedright
\href{http://icao.int/iwxxm/1.1/req/xsd-meteorological-aerodrome-forecast-record/number-of-temperature-groups}{http://icao.int/iwxxm/2.1/req/xsd-meteorological-aerodrome-forecast-record/number-of-temperature-groups}

No more than two sets of temperature conditions shall be reported.\strut
\end{minipage}\tabularnewline
\begin{minipage}[t]{0.47\columnwidth}\raggedright
Requirement\strut
\end{minipage} & \begin{minipage}[t]{0.47\columnwidth}\raggedright
\href{http://icao.int/iwxxm/1.1/req/xsd-meteorological-aerodrome-forecast-record/cloud}{http://icao.int/iwxxm/2.1/req/xsd-meteorological-aerodrome-forecast-record/cloud}

If reported, the cloud conditions forecast for the aerodrome shall be expressed using the XML element //iwxxm:MeteorologicalAerodromeForecastRecord/iwxxm:cloud containing a valid child element iwxxm:AerodromeCloudForecast.\strut
\end{minipage}\tabularnewline
\begin{minipage}[t]{0.47\columnwidth}\raggedright
Requirement\strut
\end{minipage} & \begin{minipage}[t]{0.47\columnwidth}\raggedright
\href{http://icao.int/iwxxm/1.1/req/xsd-meteorological-aerodrome-forecast-record/forecast-weather}{http://icao.int/iwxxm/2.1/req/xsd-meteorological-aerodrome-forecast-record/forecast-weather}

If forecast weather is reported, the value of XML attribute\\
//iwxxm:MeteorologicalAerodromeForecastRecord/iwxxm:forecastWeather/@xlink:href shall be the URI of a valid weather phenomenon code from Code table~D-7: Aerodrome present or forecast weather.\strut
\end{minipage}\tabularnewline
\begin{minipage}[t]{0.47\columnwidth}\raggedright
Requirement\strut
\end{minipage} & \begin{minipage}[t]{0.47\columnwidth}\raggedright
\href{http://icao.int/iwxxm/1.1/req/xsd-meteorological-aerodrome-forecast-record/number-of-forecast-weather-codes}{http://icao.int/iwxxm/2.1/req/xsd-meteorological-aerodrome-forecast-record/number-of-forecast-weather-codes}

No more than three forecast weather codes shall be reported.\strut
\end{minipage}\tabularnewline
\begin{minipage}[t]{0.47\columnwidth}\raggedright
Requirement\strut
\end{minipage} & \begin{minipage}[t]{0.47\columnwidth}\raggedright
\href{http://icao.int/iwxxm/1.1/req/xsd-meteorological-aerodrome-forecast-record/surface-wind}{http://icao.int/iwxxm/2.1/req/xsd-meteorological-aerodrome-forecast-record/surface-wind}

Surface wind conditions forecast for the aerodrome shall be reported using the XML element //iwxxm:MeteorologicalAerodromeForecastRecord/iwxxm:surfaceWind containing a valid child element iwxxm:AerodromeSurfaceWindForecast.\strut
\end{minipage}\tabularnewline
\bottomrule
\end{longtable}

Notes:

1. Units of measurement are specified in accordance with 1.9 above.

2. Temporary fluctuations in the meteorological conditions occur when those conditions reach or pass specified values and last for a period of time less than one hour in each instance and, in the aggregate, cover less than one half the period during which the fluctuations are forecast to occur (\emph{Technical Regulations} (WMO-No.~49), Volume~II, Part~II, Appendix~5, 2.3.3).

3. The use of change groups and time indicators within a TAF is specified in the \emph{Technical Regulations} (WMO-No.~49), Volume~II, Part~II, Appendix~5, 1.3 and Table~A5-2.

4. The use of probability groups and time indicators within a TAF is specified in the \emph{Technical Regulations} (WMO-No.~49), Volume~II, Part~II, Appendix~5, 1.4 and Table~A5-2.

5. Cloud and visibility information is omitted when considered to be insignificant to aeronautical operations at an aerodrome. This occurs when: (i) visibility exceeds 10 kilometres, (ii) no cloud is present below 1~500 metres or the minimum sector altitude, whichever is greater, and there is no cumulonimbus at any height, and (iii) there is no weather of operational significance. These conditions are referred to as CAVOK. Use of CAVOK is specified in the \emph{Technical Regulations} (WMO-No.~49), Volume~II, Part~II, Appendix~3, 2.2.

6. Visibility for aeronautical purposes is defined as the greater of: (i) the greatest distance at which a black object of suitable dimensions, situated near the ground, can be seen and recognized when observed against a bright background; or (ii) the greatest distance at which lights in the vicinity of 1~000 candelas can be seen and identified against an unlit background.

7. Prevailing visibility is defined as the greatest visibility value observed which is reached within at least half the horizon circle or within at least half of the surface of the aerodrome. These areas could comprise contiguous or non-contiguous sectors.

8. The requirements for reporting the following within an aerodrome forecast are specified in the \emph{Technical Regulations} (WMO-No.~49), Volume~II, Part~II, Appendix~5:

(a) Prevailing visibility conditions paragraph~1.2.2

(b) Temperature conditions paragraph~1.2.5

(c) Cloud conditions paragraph~1.2.4

(d) Forecast weather phenomena paragraph~1.2.3

(e) Surface wind conditions paragraph~1.2.1

9. The absence of XML element //iwxxm:MeteorologicalAerodromeForecastRecord/iwxxm:prevailingVisibilityOperator indicates that the prevailing visibility has the numeric value reported.

10. Code table~D-7 is published online at \url{http://codes.wmo.int/49-2/AerodromePresentOrForecastWeather}.

205-16.18 Requirements class: Aerodrome air temperature forecast

205-16.18.1 This requirements class is used to describe the temperature conditions forecast at an aerodrome as appropriate for inclusion in an aerodrome forecast (TAF) report, including the maximum and minimum temperature values and their time of occurrence.

Note: The requirements for reporting the temperature conditions at an aerodrome within a TAF are specified in the \emph{Technical Regulations} (WMO-No.~49), Volume~II, Part~II, Appendix~5, 1.2.5.

205-16.18.2 XML elements describing temperature conditions at an aerodrome shall conform to all requirements specified in Table~205-16.18.

205-16.18.3 XML elements describing temperature conditions at an aerodrome shall conform to all requirements of all relevant dependencies specified in Table~205-16.18.

Table~205-16.18. Requirements class xsd-aerodrome-air-temperature-forecast

\begin{longtable}[]{@{}ll@{}}
\toprule
Requirements class &\tabularnewline
\midrule
\endhead
\href{http://icao.int/iwxxm/1.1/req/xsd-aerodrome-air-temperature-forecast}{http://icao.int/iwxxm/2.1/req/xsd-aerodrome-air-temperature-forecast} &\tabularnewline
Target type & Data instance\tabularnewline
Name & Aerodrome air temperature forecast\tabularnewline
\begin{minipage}[t]{0.47\columnwidth}\raggedright
Requirement\strut
\end{minipage} & \begin{minipage}[t]{0.47\columnwidth}\raggedright
\href{http://icao.int/iwxxm/1.1/req/xsd-aerodrome-air-temperature-forecast/valid}{http://icao.int/iwxxm/2.1/req/xsd-aerodrome-air-temperature-forecast/valid}

The content model of this element shall have a value that matches the content model of iwxxm:AerodromeAirTemperatureForecast.\strut
\end{minipage}\tabularnewline
\begin{minipage}[t]{0.47\columnwidth}\raggedright
Requirement\strut
\end{minipage} & \begin{minipage}[t]{0.47\columnwidth}\raggedright
http://icao.int/iwxxm/2.1/req/xsd-aerodrome-air-temperature-forecast/maximum-temperature

The maximum air temperature anticipated during the forecast period shall be reported in Celsius (°C) using the XML element //iwxxm:AerodromeAirTemperatureForecast/iwxxm:maximumAirTemperature. The value of the associated XML attribute @uom shall be ``Cel''.\strut
\end{minipage}\tabularnewline
\begin{minipage}[t]{0.47\columnwidth}\raggedright
Requirement\strut
\end{minipage} & \begin{minipage}[t]{0.47\columnwidth}\raggedright
\href{http://icao.int/iwxxm/1.1/req/xsd-aerodrome-air-temperature-forecast/maximum-temperature-time}{http://icao.int/iwxxm/2.1/req/xsd-aerodrome-air-temperature-forecast/maximum-temperature-time}

The XML element //iwxxm:AerodromeAirTemperatureForecast/iwxxm:maximumAirTemperatureTime shall contain a valid child element gml:TimeInstant that describes the time at which the maximum air temperature is anticipated to occur.\strut
\end{minipage}\tabularnewline
\begin{minipage}[t]{0.47\columnwidth}\raggedright
Requirement\strut
\end{minipage} & \begin{minipage}[t]{0.47\columnwidth}\raggedright
\href{http://icao.int/iwxxm/1.1/req/xsd-aerodrome-air-temperature-forecast/minimum-temperature}{http://icao.int/iwxxm/2.1/req/xsd-aerodrome-air-temperature-forecast/minimum-temperature}

The minimum air temperature anticipated during the forecast period shall be reported in Celsius (°C) using the XML element //iwxxm:AerodromeAirTemperatureForecast/iwxxm:minimumAirTemperature. The value of the associated XML attribute @uom shall be ``Cel''.\strut
\end{minipage}\tabularnewline
\begin{minipage}[t]{0.47\columnwidth}\raggedright
Requirement\strut
\end{minipage} & \begin{minipage}[t]{0.47\columnwidth}\raggedright
\href{http://icao.int/iwxxm/1.1/req/xsd-aerodrome-air-temperature-forecast/minimum-temperature-time}{http://icao.int/iwxxm/2.1/req/xsd-aerodrome-air-temperature-forecast/minimum-temperature-time}

The XML element //iwxxm:AerodromeAirTemperatureForecast/iwxxm:minimumAirTemperatureTime shall contain a valid child element gml:TimeInstant that describes the time at which the minimum air temperature is anticipated to occur.\strut
\end{minipage}\tabularnewline
\bottomrule
\end{longtable}

Note: Units of measurement are specified in accordance with 1.9 above.

205-16.19 Requirements class: SIGMET position

205-16.19.1 This requirements class is used to describe the forecast position and extent of a specific SIGMET phenomenon, such as volcanic ash or thunderstorm, at the end of the valid period of the SIGMET report. The geometric extent of the SIGMET phenomenon is specified as a two-dimensional horizontal region with bounded vertical extent.

205-16.19.2 XML elements describing only the geometry of a SIGMET phenomenon shall conform to all requirements specified in Table~205-16.19.

205-16.19.3 XML elements describing only the geometry of a SIGMET phenomenon shall conform to all requirements of all relevant dependencies specified in Table~205-16.19.

Table~205-16.19. Requirements class xsd-sigmet-position

\begin{longtable}[]{@{}ll@{}}
\toprule
Requirements class &\tabularnewline
\midrule
\endhead
\href{http://icao.int/iwxxm/1.1/req/xsd-sigmet-position}{http://icao.int/iwxxm/2.1/req/xsd-sigmet-position} &\tabularnewline
Target type & Data instance\tabularnewline
Name & SIGMET position\tabularnewline
\begin{minipage}[t]{0.47\columnwidth}\raggedright
Requirement\strut
\end{minipage} & \begin{minipage}[t]{0.47\columnwidth}\raggedright
\href{http://icao.int/iwxxm/1.1/req/xsd-meteorological-position/valid}{http://icao.int/iwxxm/2.1/req/xsd-sigmet-position/valid}

The content model of this element shall have a value that matches the content model of iwxxm:SIGMETPosition.\strut
\end{minipage}\tabularnewline
\begin{minipage}[t]{0.47\columnwidth}\raggedright
Requirement\strut
\end{minipage} & \begin{minipage}[t]{0.47\columnwidth}\raggedright
\href{http://icao.int/iwxxm/1.1/req/xsd-meteorological-position/geometry}{http://icao.int/iwxxm/2.1/req/xsd-sigmet-position/geometry}

The geometric extent of the SIGMET phenomenon shall be reported using the XML element //iwxxm:SIGMETPosition/iwxxm:geometry with valid child element aixm:AirspaceVolume.\strut
\end{minipage}\tabularnewline
\bottomrule
\end{longtable}

205-16.20 Requirements class: SIGMET position collection

205-16.20.1 This requirements class is used to describe a collection of geometries for a specific SIGMET phenomenon, such as volcanic ash or thunderstorm, at the end of the valid period of the SIGMET report.

205-16.20.2 XML elements describing a collection of geometries for a SIGMET phenomenon shall conform to all requirements specified in Table~205-16.20.

205-16.20.3 XML elements describing a collection of geometries for a SIGMET phenomenon shall conform to all requirements of all relevant dependencies specified in Table~205-16.20.

Table~205-16.20. Requirements class xsd-sigmet-position-collection

\begin{longtable}[]{@{}ll@{}}
\toprule
Requirements class &\tabularnewline
\midrule
\endhead
\href{http://icao.int/iwxxm/1.1/req/xsd-meteorological-position-collection}{http://icao.int/iwxxm/2.1/req/xsd-sigmet-position-collection} &\tabularnewline
Target type & Data instance\tabularnewline
Name & SIGMET position collection\tabularnewline
Dependency & \href{http://icao.int/iwxxm/1.1/req/xsd-meteorological-position}{http://icao.int/iwxxm/2.1/req/xsd-sigmet-position}, 205-16.19\tabularnewline
\begin{minipage}[t]{0.47\columnwidth}\raggedright
Requirement\strut
\end{minipage} & \begin{minipage}[t]{0.47\columnwidth}\raggedright
\href{http://icao.int/iwxxm/1.1/req/xsd-meteorological-position-collection/valid}{http://icao.int/iwxxm/2.1/req/xsd-sigmet-position-collection/valid}

The content model of this element shall have a value that matches the content model of iwxxm: SIGMETPositionCollection.\strut
\end{minipage}\tabularnewline
\begin{minipage}[t]{0.47\columnwidth}\raggedright
Requirement\strut
\end{minipage} & \begin{minipage}[t]{0.47\columnwidth}\raggedright
\href{http://icao.int/iwxxm/1.1/req/xsd-meteorological-position-collection/members}{http://icao.int/iwxxm/2.1/req/xsd-sigmet-position-collection/members}

If reported, the geometries for a specific SIGMET phenomenon shall be provided using the XML element //iwxxm:SIGMETPositionCollection/iwxxm:member with valid child element iwxxm:SIGMETPosition.\strut
\end{minipage}\tabularnewline
\bottomrule
\end{longtable}

205-16.21 Requirements class: SIGMET

205-16.21.1 This requirements class is used to describe the SIGMET report within which the characteristics of a specific SIGMET phenomenon are described.

Note: The reporting requirements for SIGMETs are specified in the \emph{Technical Regulations} (WMO-No.~49), Volume~II, Part~II, Appendix~6, section~1.

205-16.21.2 XML elements describing SIGMET reports shall conform to all requirements specified in Table~205-16.21.

205-16.21.3 XML elements describing SIGMET reports shall conform to all requirements of all relevant dependencies specified in Table~205-16.21.

Table~205-16.21. Requirements class xsd-sigmet

\begin{longtable}[]{@{}ll@{}}
\toprule
Requirements class &\tabularnewline
\midrule
\endhead
\href{http://icao.int/iwxxm/1.1/req/xsd-sigmet}{http://icao.int/iwxxm/2.1/req/xsd-sigmet} &\tabularnewline
Target type & Data instance\tabularnewline
Name & SIGMET\tabularnewline
\begin{minipage}[t]{0.47\columnwidth}\raggedright
Requirement\strut
\end{minipage} & \begin{minipage}[t]{0.47\columnwidth}\raggedright
\href{http://icao.int/iwxxm/1.1/req/xsd-sigmet/valid}{http://icao.int/iwxxm/2.1/req/xsd-sigmet/valid}

The content model of this element shall have a value that matches the content model of iwxxm:SIGMET.\strut
\end{minipage}\tabularnewline
\begin{minipage}[t]{0.47\columnwidth}\raggedright
Requirement\strut
\end{minipage} & \begin{minipage}[t]{0.47\columnwidth}\raggedright
\href{http://icao.int/iwxxm/1.1/req/xsd-sigmet/status}{http://icao.int/iwxxm/2.1/req/xsd-sigmet/status}

The status of the SIGMET shall be indicated using the XML attribute @status with the value being one of the enumeration: ``NORMAL'' or ``CANCELLATION''.\strut
\end{minipage}\tabularnewline
\begin{minipage}[t]{0.47\columnwidth}\raggedright
Requirement\strut
\end{minipage} & \begin{minipage}[t]{0.47\columnwidth}\raggedright
\href{http://icao.int/iwxxm/1.1/req/xsd-sigmet/issuing-air-traffic-services-unit}{http://icao.int/iwxxm/2.1/req/xsd-sigmet/issuing-air-traffic-services-unit}

The air traffic services unit responsible for the subject airspace shall be indicated using the XML element //iwxxm:issuingAirTrafficServicesUnit with a valid child element aixm:Unit.\strut
\end{minipage}\tabularnewline
\begin{minipage}[t]{0.47\columnwidth}\raggedright
Requirement\strut
\end{minipage} & \begin{minipage}[t]{0.47\columnwidth}\raggedright
\href{http://icao.int/iwxxm/1.1/req/xsd-sigmet/originating-meteorological-watch-office}{http://icao.int/iwxxm/2.1/req/xsd-sigmet/originating-meteorological-watch-office}

The meteorological watch office that originated the SIGMET report shall be indicated using the XML element //iwxxm:originatingMeteorologicalWatchOffice with a valid child element aixm:Unit.

The value of XML element //iwxxm:issuingAirTrafficServicesUnit/saf:Unit/saf:type shall be ``MWO'' (Meteorological Watch Office).\strut
\end{minipage}\tabularnewline
\begin{minipage}[t]{0.47\columnwidth}\raggedright
Requirement\strut
\end{minipage} & \begin{minipage}[t]{0.47\columnwidth}\raggedright
\href{http://icao.int/iwxxm/1.1/req/xsd-sigmet/sequence-number}{http://icao.int/iwxxm/2.1/req/xsd-sigmet/sequence-number}

The sequence number of this SIGMET report shall be indicated using XML element\\
//iwxxm:sequenceNumber.\strut
\end{minipage}\tabularnewline
\begin{minipage}[t]{0.47\columnwidth}\raggedright
Requirement\strut
\end{minipage} & \begin{minipage}[t]{0.47\columnwidth}\raggedright
\href{http://icao.int/iwxxm/1.1/req/xsd-sigmet/valid-period}{http://icao.int/iwxxm/2.1/req/xsd-sigmet/valid-period}

The period of validity of this SIGMET report shall be indicated using XML element\\
//iwxxm:validPeriod with valid child element gml:TimePeriod.\strut
\end{minipage}\tabularnewline
\begin{minipage}[t]{0.47\columnwidth}\raggedright
Requirement\strut
\end{minipage} & \begin{minipage}[t]{0.47\columnwidth}\raggedright
\href{http://icao.int/iwxxm/1.1/req/xsd-sigmet/phenomenon}{http://icao.int/iwxxm/2.1/req/xsd-sigmet/phenomenon}

The XML attribute //iwxxm:phenomenon/@xlink:href shall have a value that is the URI of a valid term from Code table~D-10: Significant weather phenomena.\strut
\end{minipage}\tabularnewline
\begin{minipage}[t]{0.47\columnwidth}\raggedright
Requirement\strut
\end{minipage} & \begin{minipage}[t]{0.47\columnwidth}\raggedright
\href{http://icao.int/iwxxm/1.1/req/xsd-sigmet/unique-subject-airspace}{http://icao.int/iwxxm/2.1/req/xsd-sigmet/unique-subject-airspace}

All SIGMET analyses included in the report shall refer to the same airspace. All values of XML element //om:OM\_Observation/om:featureOfInterest/sams:SF\_SpatialSamplingFeature/sam:sampledFeature/aixm:Airspace/gml:identifier within the SIGMET shall be identical.\strut
\end{minipage}\tabularnewline
\begin{minipage}[t]{0.47\columnwidth}\raggedright
Requirement\strut
\end{minipage} & \begin{minipage}[t]{0.47\columnwidth}\raggedright
\href{http://icao.int/iwxxm/1.1/req/xsd-sigmet/analysis}{http://icao.int/iwxxm/2.1/req/xsd-sigmet/analysis}

If reported, XML element //iwxxm:analysis shall contain a valid child element\\
//om:OM\_Observation of type SIGMETEvolvingConditionAnalysis. The value of XML attribute //iwxxm:analysis/om:OM\_Observation/om:type/@xlink:href shall be the URI ``\url{http://codes.wmo.int/49-2/observation-type/IWXXM/2.1/SIGMETEvolvingConditionAnalysis}''.\strut
\end{minipage}\tabularnewline
\begin{minipage}[t]{0.47\columnwidth}\raggedright
Requirement\strut
\end{minipage} & \begin{minipage}[t]{0.47\columnwidth}\raggedright
\href{http://icao.int/iwxxm/1.1/req/xsd-sigmet/forecast-position-analysis}{http://icao.int/iwxxm/2.1/req/xsd-sigmet/forecast-position-analysis}

If reported, the forecast position of the phenomenon shall be reported using the XML element //iwxxm:forecastPositionAnalysis with valid child element\\
//om:OM\_Observation of type SIGMETPositionAnalysis.

The value of XML attribute //iwxxm:forecastPositionAnalysis/om:OM\_Observation/om:type/@xlink:href shall be the URI ``\url{http://codes.wmo.int/49-2/observation-type/IWXXM/2.1/SIGMETPositionAnalysis}''.\strut
\end{minipage}\tabularnewline
\begin{minipage}[t]{0.47\columnwidth}\raggedright
Requirement\strut
\end{minipage} & \begin{minipage}[t]{0.47\columnwidth}\raggedright
\href{http://icao.int/iwxxm/1.1/req/xsd-sigmet/status-normal}{http://icao.int/iwxxm/2.1/req/xsd-sigmet/status-normal}

If the status of the SIGMET is ``NORMAL'' (as specified by XML attribute @status), then:

(i) The characteristics of the SIGMET phenomenon shall be reported using one or more of the XML element //iwxxm:analysis;

(ii) Each XML element //iwxxm:analysis shall contain a valid element //iwxxm:analysis/om:OM\_Observation/om:result/iwxxm:EvolvingMeteorologicalCondition within which the characteristics of the SIGMET phenomenon are described;

(iii) The XML element //iwxxm:cancelledSequenceNumber shall be absent; and

(iv) The XML element //iwxxm:cancelledValidPeriod shall be absent.\strut
\end{minipage}\tabularnewline
\begin{minipage}[t]{0.47\columnwidth}\raggedright
Requirement\strut
\end{minipage} & \begin{minipage}[t]{0.47\columnwidth}\raggedright
\href{http://icao.int/iwxxm/1.1/req/xsd-sigmet/status-cancellation}{http://icao.int/iwxxm/2.1/req/xsd-sigmet/status-cancellation}

If the status of the SIGMET is ``CANCELLATION'' (as specified by XML attribute\\
@status), then:

(i) The details of the airspace for which the SIGMET has been cancelled shall be provided by a single instance of XML element //iwxxm:analysis;

(ii) The XML element //iwxxm:analysis/om:OM\_Observation/om:result shall have no child elements and XML attribute //iwxxm:analysis/om:OM\_Observation/om:result/@nilReason shall provide an appropriate nil reason;

(iii) The value of XML element //iwxxm:cancelledSequenceNumber shall indicate the sequence number of the SIGMET that has been cancelled; and

(iv) The XML element //iwxxm:cancelledValidPeriod shall contain a valid child element gml:TimePeriod that indicates the validity period of the SIGMET that has been cancelled.\strut
\end{minipage}\tabularnewline
\begin{minipage}[t]{0.47\columnwidth}\raggedright
Recommendation\strut
\end{minipage} & \begin{minipage}[t]{0.47\columnwidth}\raggedright
\href{http://icao.int/iwxxm/1.1/req/xsd-sigmet/issuing-air-traffic-services-unit-type}{http://icao.int/iwxxm/2.1/req/xsd-sigmet/issuing-air-traffic-services-unit-type}

The value of XML element //iwxxm:SIGMET/iwxxm:issuingAirTrafficServicesUnit/aixm:Unit/aixm:type should be one of the enumeration: ``ATSU'' (Air Traffic Services Unit) or ``FIC'' (Flight Information Centre).\strut
\end{minipage}\tabularnewline
\begin{minipage}[t]{0.47\columnwidth}\raggedright
Recommendation\strut
\end{minipage} & \begin{minipage}[t]{0.47\columnwidth}\raggedright
\href{http://icao.int/iwxxm/1.1/req/xsd-sigmet/valid-period-start-matches-result-time}{http://icao.int/iwxxm/2.1/req/xsd-sigmet/valid-period-start-matches-result-time}

The start time of the validity period of the SIGMET report (expressed using XML element //iwxxm:validPeriod/gml:TimePeriod/gml:beginPosition) should match the result time of each SIGMET analysis included within the report (expressed using XML element //om:OM\_Observation/om:resultTime/gml:TimeInstant/gml:timePosition).\strut
\end{minipage}\tabularnewline
\begin{minipage}[t]{0.47\columnwidth}\raggedright
Recommendation\strut
\end{minipage} & \begin{minipage}[t]{0.47\columnwidth}\raggedright
\href{http://icao.int/iwxxm/1.1/req/xsd-sigmet/valid-time-includes-all-phenomenon-times}{http://icao.int/iwxxm/2.1/req/xsd-sigmet/valid-time-includes-all-phenomenon-times}

The observation and/or forecast times of all SIGMET analyses and, if reported, forecast position analyses included in the report (specified by XML element\\
//om:OM\_Observation/om:phenomenonTime/*) should occur within the valid time period of the SIGMET (specified by XML element //iwxxm:validPeriod/gml:TimePeriod).\strut
\end{minipage}\tabularnewline
\begin{minipage}[t]{0.47\columnwidth}\raggedright
Recommendation\strut
\end{minipage} & \begin{minipage}[t]{0.47\columnwidth}\raggedright
\href{http://icao.int/iwxxm/1.1/req/xsd-sigmet/7-point-definition-of-airspace-volume}{http://icao.int/iwxxm/2.1/req/xsd-sigmet/7-point-definition-of-airspace-volume}

The horizontal extent of any airspace volumes enclosing a SIGMET phenomenon (reported using XML element //om:OM\_Observation/om:result/*/iwxxm:geometry/aixm:AirspaceVolume/aixm:horizontalProjection) should use no more than seven points to define the bounding polygon.\strut
\end{minipage}\tabularnewline
\bottomrule
\end{longtable}

Notes:

1. Requirements relating to sequence numbers within SIGMET reports are specified in the \emph{Technical Regulations} (WMO-No.~49), Volume~II, Part~II, Appendix~6, 1.1.3.

2. Requirements for reporting the SIGMET phenomenon are specified in the \emph{Technical Regulations} (WMO-No.~49), Volume~II, Part~II, Appendix~6, 1.1.4.

3. A forecast position may be provided for a volcanic ash cloud, the centre of a tropical cyclone or other hazardous phenomena at the end of the validity period of the SIGMET message.

4. Within an XML encoded SIGMET, it is likely that only one instance of aixm:Airspace will physically be present; subsequent assertions about the airspace may use xlinks to refer to the previously defined aixm:Airspace element in order to keep the XML document size small. As such, validation of requirement \url{http://icao.int/iwxxm/2.1/req/xsd-sigmet/unique-subject-airspace} is applied once any xlinks, if used, have been resolved.

5. Code table~D-1 provides a set of nil-reason codes and is published at \url{http://codes.wmo.int/common/nil}.

6. Code table~D-10 is published online at \url{http://codes.wmo.int/49-2/SigWxPhenomena}.

205-16.22 Requirements class: SIGMET evolving condition collection

205-16.22.1 This requirements class is used to define a collection of SIGMET phenomena described by the requirements class SIGMET evolving condition, each representing a location where SIGMET observed or forecast conditions exist.

205-16.22.2 XML elements describing the characteristics of a SIGMET phenomenon shall conform to all requirements specified in Table~205-16.22.

205-16.22.3 XML elements describing the characteristics of a SIGMET phenomenon shall conform to all requirements of all relevant dependencies specified in Table~205-16.22.

Table~205-16.22. Requirements class xsd-sigmet-evolving-condition-collection

\begin{longtable}[]{@{}ll@{}}
\toprule
Requirements class &\tabularnewline
\midrule
\endhead
\href{http://icao.int/iwxxm/2.0/req/xsd-evolving-meteorological-condition}{http://icao.int/iwxxm/2.1/req/xsd-sigmet-evolving-condition}-collection &\tabularnewline
Target type & Data instance\tabularnewline
Name & SIGMET evolving condition collection\tabularnewline
\begin{minipage}[t]{0.47\columnwidth}\raggedright
Requirement\strut
\end{minipage} & \begin{minipage}[t]{0.47\columnwidth}\raggedright
\href{http://icao.int/iwxxm/2.0/req/xsd-evolving-meteorological-condition/valid}{http://icao.int/iwxxm/2.1/req/xsd-sigmet-evolving-condition-collection/valid}

The content model of this element shall have a value that matches the content model of iwxxm:SIGMETEvolvingConditionCollection.\strut
\end{minipage}\tabularnewline
\begin{minipage}[t]{0.47\columnwidth}\raggedright
Requirement\strut
\end{minipage} & \begin{minipage}[t]{0.47\columnwidth}\raggedright
\href{http://icao.int/iwxxm/2.0/req/xsd-evolving-meteorological-condition/valid}{http://icao.int/iwxxm/2.1/req/xsd-sigmet-evolving-condition-collection/}time-indicator

The content model of this element shall have a value that matches the content model of iwxxm:SIGMETEvolvingConditionCollection/@TimeIndicator.\strut
\end{minipage}\tabularnewline
\bottomrule
\end{longtable}

205-16.23 Requirements class: SIGMET evolving condition

205-16.23.1 This requirements class is used to describe the presence of a specific SIGMET phenomenon such as volcanic ash or thunderstorm, along with expected changes to the intensity of the phenomenon, its speed and direction of motion. The geometric extent of the SIGMET phenomenon is specified as a two-dimensional horizontal region with bounded vertical extent.

205-16.23.2 XML elements describing the characteristics of a SIGMET phenomenon shall conform to all requirements specified in Table~205-16.23.

205-16.23.3 XML elements describing the characteristics of a SIGMET phenomenon shall conform to all requirements of all relevant dependencies specified in Table~205-16.23.

Table~205-16.23. Requirements class xsd-sigmet-evolving-condition

\begin{longtable}[]{@{}ll@{}}
\toprule
Requirements class &\tabularnewline
\midrule
\endhead
\href{http://icao.int/iwxxm/1.1/req/xsd-evolving-meteorological-condition}{http://icao.int/iwxxm/2.1/req/xsd-sigmet-evolving-condition} &\tabularnewline
Target type & Data instance\tabularnewline
Name & SIGMET evolving condition\tabularnewline
\begin{minipage}[t]{0.47\columnwidth}\raggedright
Requirement\strut
\end{minipage} & \begin{minipage}[t]{0.47\columnwidth}\raggedright
\url{http://icao.int/iwxxm/2.1/req/xsd-sigmet-evolving-condition/valid}

The content model of this element shall have a value that matches the content model of iwxxm:SIGMETEvolvingCondition.\strut
\end{minipage}\tabularnewline
\begin{minipage}[t]{0.47\columnwidth}\raggedright
Requirement\strut
\end{minipage} & \begin{minipage}[t]{0.47\columnwidth}\raggedright
\url{http://icao.int/iwxxm/2.1/req/xsd-sigmet-evolving-condition/intensity-change}

The anticipated change in intensity of the SIGMET phenomenon shall be indicated using the XML attribute //iwxxm:SIGMETEvolvingCondition/@intensityChange with the value being one of the enumeration: ``NO\_CHANGE'', ``WEAKEN'' or ``INTENSIFY''.\strut
\end{minipage}\tabularnewline
\begin{minipage}[t]{0.47\columnwidth}\raggedright
Requirement\strut
\end{minipage} & \begin{minipage}[t]{0.47\columnwidth}\raggedright
\href{http://icao.int/iwxxm/1.1/req/xsd-sigmet-evolving-condition/geometry}{http://icao.int/iwxxm/2.1/req/xsd-sigmet-evolving-condition/geometry}

The geometric extent of the SIGMET phenomenon shall be reported using the XML element //iwxxm:SIGMETEvolvingCondition/iwxxm:geometry with valid child element aixm:AirspaceVolume.\strut
\end{minipage}\tabularnewline
\begin{minipage}[t]{0.47\columnwidth}\raggedright
Requirement\strut
\end{minipage} & \begin{minipage}[t]{0.47\columnwidth}\raggedright
\href{http://icao.int/iwxxm/1.1/req/xsd-sigmet-evolving-condition/speed-of-motion}{http://icao.int/iwxxm/2.1/req/xsd-sigmet-evolving-condition/speed-of-motion}

The speed of motion of the SIGMET phenomenon shall be reported using the XML element //iwxxm:SIGMETEvolvingCondition/iwxxm:speedOfMotion, with the unit of measure metres per second, knots or kilometres per hour.

The unit of measure shall be indicated using the XML attribute\\
//iwxxm:SIGMETEvolvingCondition/iwxxm:speedOfMotion/@uom with value ``m/s'' (metres per second), ``{[}kn\_i{]}'' (knots) or ``km/h'' (kilometres per hour).\strut
\end{minipage}\tabularnewline
\begin{minipage}[t]{0.47\columnwidth}\raggedright
Requirement\strut
\end{minipage} & \begin{minipage}[t]{0.47\columnwidth}\raggedright
\url{http://icao.int/iwxxm/2.1/req/xsd-sigmet-evolving-condition/direction-of-motion}

If reported, the angle between true north and the direction of motion of the SIGMET phenomenon shall be given in degrees using the XML element\\
//iwxxm:SIGMETEvolvingCondition/iwxxm:directionOfMotion.

The unit of measure shall be indicated using the XML attribute\\
//iwxxm:SIGMETEvolvingCondition/iwxxm:directionOfMotion/@uom with value ``deg''.\strut
\end{minipage}\tabularnewline
Recommendation & \vtop{\hbox{\strut \href{http://icao.int/iwxxm/1.1/req/xsd-igmet-evolving-condition/stationary-phenomenon}{http://icao.int/iwxxm/2.1/req/xsd-sigmet-evolving-condition/stationary-phenomenon}If the SIGMET phenomenon is not moving (indicated by the XML element}\hbox{\strut //iwxxm:SIGMETEvolvingCondition/iwxxm:speedOfMotion having numeric value zero), XML element //iwxxm:SIGMETEvolvingCondition/iwxxm:directionOfMotion should be absent.}}\tabularnewline
\bottomrule
\end{longtable}

Notes:

1. Units of measurement are specified in accordance with 1.9 above.

2. The true north is the north point at which the meridian lines meet.

205-16.24 Requirements class: Tropical cyclone SIGMET

205-16.24.1 This requirements class is used to describe the tropical cyclone (TC) SIGMET report, which includes additional information about the tropical cyclone itself and the forecast position of the tropical cyclone at the end of the validity period of the SIGMET.

205-16.24.2 XML elements describing TC SIGMET reports shall conform to all requirements specified in Table~205-16.24.

205-16.24.3 XML elements describing TC SIGMET reports shall conform to all requirements of all relevant dependencies specified in Table~205-16.24.

Table~205-16.24. Requirements class xsd-tropical-cyclone-sigmet

\begin{longtable}[]{@{}ll@{}}
\toprule
Requirements class &\tabularnewline
\midrule
\endhead
\href{http://icao.int/iwxxm/1.1/req/xsd-tropical-cyclone-sigmet}{http://icao.int/iwxxm/2.1/req/xsd-tropical-cyclone-sigmet} &\tabularnewline
Target type & Data instance\tabularnewline
Name & Tropical cyclone SIGMET\tabularnewline
Dependency & \url{http://def.wmo.int/metce/2013/req/xsd-tropical-cyclone}, 202-16.24\tabularnewline
Dependency & \href{http://icao.int/iwxxm/1.1/req/xsd-sigmet}{http://icao.int/iwxxm/2.1/req/xsd-sigmet}, 205-16.21\tabularnewline
\begin{minipage}[t]{0.47\columnwidth}\raggedright
Requirement\strut
\end{minipage} & \begin{minipage}[t]{0.47\columnwidth}\raggedright
\href{http://icao.int/iwxxm/1.1/req/xsd-tropical-cyclone-sigmet/valid}{http://icao.int/iwxxm/2.1/req/xsd-tropical-cyclone-sigmet/valid}

The content model of this element shall have a value that matches the content model of iwxxm:TropicalCycloneSIGMET.\strut
\end{minipage}\tabularnewline
\begin{minipage}[t]{0.47\columnwidth}\raggedright
Requirement\strut
\end{minipage} & \begin{minipage}[t]{0.47\columnwidth}\raggedright
\href{http://icao.int/iwxxm/1.1/req/xsd-tropical-cyclone-sigmet/cyclone}{http://icao.int/iwxxm/2.1/req/xsd-tropical-cyclone-sigmet/cyclone}

Details of the tropical cyclone shall be reported using the XML element\\
//iwxxm:tropicalCyclone with valid child element metce:TropicalCyclone.\strut
\end{minipage}\tabularnewline
\begin{minipage}[t]{0.47\columnwidth}\raggedright
Requirement\strut
\end{minipage} & \begin{minipage}[t]{0.47\columnwidth}\raggedright
http://icao.int/iwxxm/2.1/req/xsd-tropical-cyclone-sigmet/phenomenon

The XML attribute //iwxxm:phenomenon/@xlink:href shall have a value that is the URI ``\url{http://codes.wmo.int/49-2/SigWxPhenomena/TC}''.\strut
\end{minipage}\tabularnewline
\bottomrule
\end{longtable}

205-16.25 Requirements class: Volcanic ash SIGMET

205-16.25.1 This requirements class is used to describe the volcanic ash (VA) SIGMET report, which includes additional information about the source volcano and the forecast position of the volcanic ash at the end of the validity period of the SIGMET.

205-16.25.2 XML elements describing VA SIGMET reports shall conform to all requirements specified in Table~205-16.25.

205-16.25.3 XML elements describing VA SIGMET reports shall conform to all requirements of all relevant dependencies specified in Table~205-16.25.

Table~205-16.25. Requirements class xsd-volcanic-ash-sigmet

\begin{longtable}[]{@{}ll@{}}
\toprule
Requirements class &\tabularnewline
\midrule
\endhead
\href{http://icao.int/iwxxm/1.1/req/xsd-volcanic-ash-sigmet}{http://icao.int/iwxxm/2.1/req/xsd-volcanic-ash-sigmet} &\tabularnewline
Target type & Data instance\tabularnewline
Name & Volcanic ash SIGMET\tabularnewline
Dependency & \url{http://def.wmo.int/metce/2013/req/xsd-erupting-volcano}, 202-16.8\tabularnewline
Dependency & \href{http://icao.int/iwxxm/1.1/req/xsd-sigmet}{http://icao.int/iwxxm/2.1/req/xsd-sigmet}, 205-16.21\tabularnewline
\begin{minipage}[t]{0.47\columnwidth}\raggedright
Requirement\strut
\end{minipage} & \begin{minipage}[t]{0.47\columnwidth}\raggedright
\href{http://icao.int/iwxxm/1.1/req/xsd-volcanic-ash-sigmet/valid}{http://icao.int/iwxxm/2.1/req/xsd-volcanic-ash-sigmet/valid}

The content model of this element shall have a value that matches the content model of iwxxm:VolcanicAshSIGMET.\strut
\end{minipage}\tabularnewline
\begin{minipage}[t]{0.47\columnwidth}\raggedright
Requirement\strut
\end{minipage} & \begin{minipage}[t]{0.47\columnwidth}\raggedright
\href{http://icao.int/iwxxm/1.1/req/xsd-volcanic-ash-sigmet/source-volcano}{http://icao.int/iwxxm/2.1/req/xsd-volcanic-ash-sigmet/source-volcano}

Details of the volcano that is the source of the volcanic ash shall be reported using the XML element //iwxxm:eruptingvolcano with valid child element metce:Volcano (or element in the substitution group of metce:Volcano).\strut
\end{minipage}\tabularnewline
\begin{minipage}[t]{0.47\columnwidth}\raggedright
Requirement\strut
\end{minipage} & \begin{minipage}[t]{0.47\columnwidth}\raggedright
\href{http://icao.int/iwxxm/1.1/req/xsd-volcanic-ash-sigmet/phenomenon}{http://icao.int/iwxxm/2.1/req/xsd-volcanic-ash-sigmet/phenomenon}

The XML attribute //iwxxm:phenomenon/@xlink:href shall have a value that is the URI ``\url{http://codes.wmo.int/49-2/SigWxPhenomena/VA}''.\strut
\end{minipage}\tabularnewline
\bottomrule
\end{longtable}

205-16.26 Requirements class: AIRMET

205-16.26.1 This requirements class is used to describe the AIRMET report within which the characteristics of a specific AIRMET phenomenon are described.

Note: The reporting requirements for AIRMETs are specified in the \emph{Technical Regulations} (WMO-No.~49), Volume~II, Part~II, Appendix~6, section~2.

205-16.26.2 XML elements describing AIRMET reports shall conform to all requirements specified in Table~205-16.26.

205-16.26.3 XML elements describing AIRMET reports shall conform to all requirements of all relevant dependencies specified in Table~205-16.26.

Table 205-16.26. Requirements class xsd-airmet

\begin{longtable}[]{@{}ll@{}}
\toprule
Requirements class &\tabularnewline
\midrule
\endhead
\url{http://icao.int/iwxxm/2.1/req/xsd-airmet} &\tabularnewline
Target type & Data instance\tabularnewline
Name & AIRMET\tabularnewline
\begin{minipage}[t]{0.47\columnwidth}\raggedright
Requirement\strut
\end{minipage} & \begin{minipage}[t]{0.47\columnwidth}\raggedright
\url{http://icao.int/iwxxm/2.1/req/xsd-airmet/valid}

The content model of this element shall have a value that matches the content model of iwxxm:AIRMET.\strut
\end{minipage}\tabularnewline
\begin{minipage}[t]{0.47\columnwidth}\raggedright
Requirement\strut
\end{minipage} & \begin{minipage}[t]{0.47\columnwidth}\raggedright
\href{http://icao.int/iwxxm/2.0/req/xsd-airmet/status}{http://icao.int/iwxxm/2.1/req/xsd-airmet/status}

The status of the AIRMET shall be indicated using the XML attribute @status with the value being one of the enumeration: ``NORMAL'' or ``CANCELLATION''.\strut
\end{minipage}\tabularnewline
\begin{minipage}[t]{0.47\columnwidth}\raggedright
Requirement\strut
\end{minipage} & \begin{minipage}[t]{0.47\columnwidth}\raggedright
\href{http://icao.int/iwxxm/2.0/req/xsd-airmet/issuing-air-traffic-services-unit}{http://icao.int/iwxxm/2.1/req/xsd-airmet/issuing-air-traffic-services-unit}

The air traffic services unit responsible for the subject airspace shall be indicated using the XML element //iwxxm:issuingAirTrafficServicesUnit with a valid child element aixm:Unit.\strut
\end{minipage}\tabularnewline
\begin{minipage}[t]{0.47\columnwidth}\raggedright
Requirement\strut
\end{minipage} & \begin{minipage}[t]{0.47\columnwidth}\raggedright
\href{http://icao.int/iwxxm/2.0/req/xsd-airmet/originating-meteorological-watch-office}{http://icao.int/iwxxm/2.1/req/xsd-airmet/originating-meteorological-watch-office}

The meteorological watch office that originated the AIRMET report shall be indicated using the XML element //iwxxm:originatingMeteorologicalWatchOffice with a valid child element aixm:Unit.

The value of XML element //iwxxm:issuingAirTrafficServicesUnit/aixm:Unit/aixm:type shall be ``MWO'' (Meteorological Watch Office).\strut
\end{minipage}\tabularnewline
\begin{minipage}[t]{0.47\columnwidth}\raggedright
Requirement\strut
\end{minipage} & \begin{minipage}[t]{0.47\columnwidth}\raggedright
\href{http://icao.int/iwxxm/2.0/req/xsd-airmet/sequence-number}{http://icao.int/iwxxm/2.1/req/xsd-airmet/sequence-number}

The sequence number of this AIRMET report shall be indicated using XML element //iwxxm:sequenceNumber.\strut
\end{minipage}\tabularnewline
\begin{minipage}[t]{0.47\columnwidth}\raggedright
Requirement\strut
\end{minipage} & \begin{minipage}[t]{0.47\columnwidth}\raggedright
\href{http://icao.int/iwxxm/2.0/req/xsd-airmet/valid-period}{http://icao.int/iwxxm/2.1/req/xsd-airmet/valid-period}

The period of validity of this AIRMET report shall be indicated using XML element\\
//iwxxm:validPeriod with valid child element gml:TimePeriod.\strut
\end{minipage}\tabularnewline
\begin{minipage}[t]{0.47\columnwidth}\raggedright
Requirement\strut
\end{minipage} & \begin{minipage}[t]{0.47\columnwidth}\raggedright
\href{http://icao.int/iwxxm/2.0/req/xsd-airmet/phenomenon}{http://icao.int/iwxxm/2.1/req/xsd-airmet/phenomenon}

The XML attribute //iwxxm:phenomenon/@xlink:href shall have a value that is the URI of a valid term from Code table D-10: Significant weather phenomena.\strut
\end{minipage}\tabularnewline
\begin{minipage}[t]{0.47\columnwidth}\raggedright
Requirement\strut
\end{minipage} & \begin{minipage}[t]{0.47\columnwidth}\raggedright
\href{http://icao.int/iwxxm/2.0/req/xsd-airmet/unique-subject-airspace}{http://icao.int/iwxxm/2.1/req/xsd-airmet/unique-subject-airspace}

All AIRMET analyses included in the report shall refer to the same airspace. All values of XML element //om:OM\_Observation/om:featureOfInterest/sams:SF\_SpatialSamplingFeature/sam:sampledFeature/aixm:Airspace/gml:identifier within the AIRMET shall be identical.\strut
\end{minipage}\tabularnewline
\begin{minipage}[t]{0.47\columnwidth}\raggedright
Requirement\strut
\end{minipage} & \begin{minipage}[t]{0.47\columnwidth}\raggedright
\href{http://icao.int/iwxxm/2.0/req/xsd-airmet/analysis}{http://icao.int/iwxxm/2.1/req/xsd-airmet/analysis}

If reported, XML element //iwxxm:analysis shall contain a valid child element\\
//om:OM\_Observation of type AIRMETEvolvingConditionAnalysis. The value of XML attribute //iwxxm:analysis/om:OM\_Observation/om:type/@xlink:href shall be the URI ``\url{http://codes.wmo.int/49-2/observation-type/IWXXM/2.1/AIRMETEvolvingConditionAnalysis}''.\strut
\end{minipage}\tabularnewline
\begin{minipage}[t]{0.47\columnwidth}\raggedright
Requirement\strut
\end{minipage} & \begin{minipage}[t]{0.47\columnwidth}\raggedright
\href{http://icao.int/iwxxm/2.0/req/xsd-airmet/status-normal}{http://icao.int/iwxxm/2.1/req/xsd-airmet/status-normal}

If the status of the AIRMET is ``NORMAL'' (as specified by XML attribute @status), then:

(i) The characteristics of the AIRMET phenomenon shall be reported using one or more of the XML element //iwxxm:analysis;

(ii) Each XML element //iwxxm:analysis shall contain a valid element //iwxxm:analysis/om:OM\_Observation/om:result/iwxxm:EvolvingMeteorologicalCondition within which the characteristics of the AIRMET phenomenon are described;

(iii) The XML element //iwxxm:cancelledSequenceNumber shall be absent; and

(iv) The XML element //iwxxm:cancelledValidPeriod shall be absent.\strut
\end{minipage}\tabularnewline
\begin{minipage}[t]{0.47\columnwidth}\raggedright
Requirement\strut
\end{minipage} & \begin{minipage}[t]{0.47\columnwidth}\raggedright
\href{http://icao.int/iwxxm/2.0/req/xsd-airmet/status-cancellation}{http://icao.int/iwxxm/2.1/req/xsd-airmet/status-cancellation}

If the status of the AIRMET is ``CANCELLATION'' (as specified by XML attribute @ status), then:

(i) The details of the airspace for which the AIRMET has been cancelled shall be provided by a single instance of XML element //iwxxm:analysis;

(ii) The XML element //iwxxm:analysis/om:OM\_Observation/om:result shall have no child elements and XML attribute //iwxxm:analysis/om:OM\_Observation/om:result/@nilReason shall provide an appropriate nil reason;

(iii) The value of XML element //iwxxm:cancelledSequenceNumber shall indicate the sequence number of the AIRMET that has been cancelled; and

(iv) The XML element //iwxxm:cancelledValidPeriod shall contain a valid child element gml:TimePeriod that indicates the validity period of the AIRMET that has been cancelled.\strut
\end{minipage}\tabularnewline
\begin{minipage}[t]{0.47\columnwidth}\raggedright
Recommendation\strut
\end{minipage} & \begin{minipage}[t]{0.47\columnwidth}\raggedright
\href{http://icao.int/iwxxm/2.0/req/xsd-airmet/issuing-air-traffic-services-unit-type}{http://icao.int/iwxxm/2.1/req/xsd-airmet/issuing-air-traffic-services-unit-type}

The value of XML element //iwxxm:AIRMET/iwxxm:issuingAirTrafficServicesUnit/aixm:Unit/aixm:type should be one of the enumeration: ``ATSU'' (Air Traffic Services Unit) or ``FIC'' (Flight Information Centre).\strut
\end{minipage}\tabularnewline
\begin{minipage}[t]{0.47\columnwidth}\raggedright
Recommendation\strut
\end{minipage} & \begin{minipage}[t]{0.47\columnwidth}\raggedright
\href{http://icao.int/iwxxm/2.0/req/xsd-airmet/valid-period-start-matches-result-time}{http://icao.int/iwxxm/2.1/req/xsd-airmet/valid-period-start-matches-result-time}

The start time of the validity period of the AIRMET report (expressed using XML element //iwxxm:validPeriod/gml:TimePeriod/gml:beginPosition) should match the result time of each AIRMET analysis included within the report (expressed using XML element //om:OM\_Observation/om:resultTime/gml:TimeInstant/gml:timePosition).\strut
\end{minipage}\tabularnewline
\begin{minipage}[t]{0.47\columnwidth}\raggedright
Recommendation\strut
\end{minipage} & \begin{minipage}[t]{0.47\columnwidth}\raggedright
\href{http://icao.int/iwxxm/2.0/req/xsd-airmet/valid-time-includes-all-phenomenon-times}{http://icao.int/iwxxm/2.1/req/xsd-airmet/valid-time-includes-all-phenomenon-times}

The observation and/or forecast times of all AIRMET analyses and, if reported, forecast position analyses included in the report (specified by XML element // om:OM\_Observation/om:phenomenonTime/*) should occur within the valid time period of the AIRMET (specified by XML element //iwxxm:validPeriod/gml:TimePeriod).\strut
\end{minipage}\tabularnewline
\begin{minipage}[t]{0.47\columnwidth}\raggedright
Recommendation\strut
\end{minipage} & \begin{minipage}[t]{0.47\columnwidth}\raggedright
\href{http://icao.int/iwxxm/2.0/req/xsd-airmet/7-point-definition-of-airspace-volume}{http://icao.int/iwxxm/2.1/req/xsd-airmet/7-point-definition-of-airspace-volume}

The horizontal extent of any airspace volumes enclosing an AIRMET phenomenon (reported using XML element //om:OM\_Observation/om:result/*/iwxxm:geometry/aixm:AirspaceVolume/aixm:horizontalProjection) should use no more than seven points to define the bounding polygon.\strut
\end{minipage}\tabularnewline
\bottomrule
\end{longtable}

Notes:

1. Requirements relating to sequence numbers within AIRMET reports are specified in the \emph{Technical Regulations} (WMO-No.~49), Volume~II, Part~II, Appendix~6, 2.1.2.

2. Requirements for reporting the AIRMET phenomenon are specified in the \emph{Technical Regulations} (WMO-No.~49), Volume~II, Part~II, Appendix~6, 2.1.4.

3. Within an XML encoded AIRMET, it is likely that only one instance of aixm:Airspace will physically be present; subsequent assertions about the airspace may use xlinks to refer to the previously defined aixm:Airspace element in order to keep the XML document size small. As such, validation of requirement \url{http://icao.int/iwxxm/2.1/req/xsd-airmet/unique-subject-airspace} is applied once any xlinks, if used, have been resolved.

4. Code table D-1 provides a set of nil-reason codes and is published at \href{http://codes.wmo.int/common/nil}{http://codes.wmo.int/common/nil.}

5. Code table D-10 is published online at \href{http://codes.wmo.int/49-2/AirWxPhenomena}{http://codes.wmo.int/49-2/AirWxPhenomena.}

205-16.27 Requirements class: AIRMET evolving condition collection

205-16.27.1 This requirements class is used to define a collection of AIRMET phenomena described by the requirements class AIRMET evolving condition, each representing a location where AIRMET observed or forecast conditions exist.

205-16.27.2 XML elements describing the characteristics of an AIRMET phenomenon shall conform to all requirements specified in Table~205-16.27.

205-16.27.3 XML elements describing the characteristics of an AIRMET phenomenon shall conform to all requirements of all relevant dependencies specified in Table~205-16.27.

Table~205-16.27. Requirements class xsd-airmet-evolving-condition-collection

\begin{longtable}[]{@{}ll@{}}
\toprule
Requirements class &\tabularnewline
\midrule
\endhead
\url{http://icao.int/iwxxm/2.1/req/xsd-airmet-evolving-condition-collection} &\tabularnewline
Target type & Data instance\tabularnewline
Name & AIRMET evolving condition collection\tabularnewline
\begin{minipage}[t]{0.47\columnwidth}\raggedright
Requirement\strut
\end{minipage} & \begin{minipage}[t]{0.47\columnwidth}\raggedright
\url{http://icao.int/iwxxm/2.1/req/xsd-airmet-evolving-condition-collection/valid}

The content model of this element shall have a value that matches the content model of iwxxm:AIRMETEvolvingConditionCollection.\strut
\end{minipage}\tabularnewline
\begin{minipage}[t]{0.47\columnwidth}\raggedright
Requirement\strut
\end{minipage} & \begin{minipage}[t]{0.47\columnwidth}\raggedright
\url{http://icao.int/iwxxm/2.1/req/xsd-airmet-evolving-condition-collection/time-indicator}

The content model of this element shall have a value that matches the content model of iwxxm:AIRMETEvolvingConditionCollection/@TimeIndicator.\strut
\end{minipage}\tabularnewline
\bottomrule
\end{longtable}

205-16.28 Requirements class: AIRMET evolving condition

205-16.28.1 This requirements class is used to describe the presence of a specific AIRMET phenomenon such as thunderstorms or mountain obscuration, along with expected changes to the intensity of the phenomenon, its speed and direction of motion. The geometric extent of the AIRMET phenomenon is specified as a two-dimensional horizontal region with bounded vertical extent.

205-16.28.2 XML elements describing the characteristics of an AIRMET phenomenon shall conform to all requirements specified in Table 205-16.28.

205-16.28.3 XML elements describing the characteristics of an AIRMET phenomenon shall conform to all requirements of all relevant dependencies specified in Table 205-16.28.

Table 205-16. 28. Requirements class xsd-airmet-evolving-condition

\begin{longtable}[]{@{}ll@{}}
\toprule
Requirements class &\tabularnewline
\midrule
\endhead
\url{http://icao.int/iwxxm/2.1/req/xsd-airmet-evolving-condition} &\tabularnewline
Target type & Data instance\tabularnewline
Name & AIRMET evolving condition\tabularnewline
\begin{minipage}[t]{0.47\columnwidth}\raggedright
Requirement\strut
\end{minipage} & \begin{minipage}[t]{0.47\columnwidth}\raggedright
\url{http://icao.int/iwxxm/2.1/req/xsd-airmet-evolving-condition/valid}

The content model of this element shall have a value that matches the content model of iwxxm:AIRMETEvolvingCondition.\strut
\end{minipage}\tabularnewline
\begin{minipage}[t]{0.47\columnwidth}\raggedright
Requirement\strut
\end{minipage} & \begin{minipage}[t]{0.47\columnwidth}\raggedright
\url{http://icao.int/iwxxm/2.1/req/xsd-airmet-evolving-condition/intensity-change}

The anticipated change in intensity of the AIRMET observed phenomenon shall be indicated using the XML attribute //iwxxm:AIRMETEvolvingCondition/@intensityChange with the value being one of the enumeration: ``NO\_CHANGE'', ``WEAKEN'' or ``INTENSIFY''.\strut
\end{minipage}\tabularnewline
\begin{minipage}[t]{0.47\columnwidth}\raggedright
Requirement\strut
\end{minipage} & \begin{minipage}[t]{0.47\columnwidth}\raggedright
\url{http://icao.int/iwxxm/2.1/req/xsd-airmet-evolving-condition/geometry}

The geometric extent of the AIRMET phenomenon shall be reported using the XML element //iwxxm:AIRMETEvolvingCondition/iwxxm:geometry with valid child element aixm:AirspaceVolume.\strut
\end{minipage}\tabularnewline
\begin{minipage}[t]{0.47\columnwidth}\raggedright
Requirement\strut
\end{minipage} & \begin{minipage}[t]{0.47\columnwidth}\raggedright
\url{http://icao.int/iwxxm/2.1/req/xsd-airmet-evolving-condition/speed-of-motion}

The speed of motion of the AIRMET phenomenon shall be reported using the XML element //iwxxm:AIRMETEvolvingCondition/iwxxm:speedOfMotion, with the unit of measure metres per second, knots or kilometres per hour.

The unit of measure shall be indicated using the XML attribute

//iwxxm:AIRMETEvolvingCondition/iwxxm:speedOfMotion/@uom with value ``m/s'' (metres per second), ``{[}kn\_i{]}'' (knots) or ``km/h'' (kilometres per hour).\strut
\end{minipage}\tabularnewline
\begin{minipage}[t]{0.47\columnwidth}\raggedright
Requirement\strut
\end{minipage} & \begin{minipage}[t]{0.47\columnwidth}\raggedright
\url{http://icao.int/iwxxm/2.1/req/xsd-airmet-evolving-condition/direction-of-motion}

If reported, the angle between true north and the direction of motion of the AIRMET phenomenon shall be given in degrees using the XML element //iwxxm:AIRMETEvolvingCondition/iwxxm:directionOfMotion.

The unit of measure shall be indicated using the XML attribute //iwxxm:AIRMETEvolvingCondition/iwxxm:directionOfMotion/@uom with value ``deg''.\strut
\end{minipage}\tabularnewline
\begin{minipage}[t]{0.47\columnwidth}\raggedright
Recommendation\strut
\end{minipage} & \begin{minipage}[t]{0.47\columnwidth}\raggedright
\url{http://icao.int/iwxxm/2.1/req/xsd-airmet-evolving-condition/stationary-phenomenon}

If the AIRMET phenomenon is not moving (indicated by the XML element //iwxxm:AIRMETEvolvingCondition/iwxxm:speedOfMotion having numeric value zero), XML element //iwxxm:AIRMET EvolvingCondition/iwxxm:directionOfMotion should be absent.\strut
\end{minipage}\tabularnewline
\bottomrule
\end{longtable}

Notes:

1. Units of measurement are specified in accordance with 1.9 above.

2. The true north is the north point at which the meridian lines meet.

205-16.29 Requirements class: Tropical Cyclone Advisory

205-16.29.1 This requirements class is used to describe the Tropical Cyclone Advisory report within which the characteristics of a specific Tropical Cyclone Advisory phenomenon are described.

Note: The reporting requirements for Tropical Cyclone Advisories are specified in the \emph{Technical Regulations} (WMO-No.~49), Volume~II, Part~II, Appendix~2, section~5.

205-16.29.2 XML elements describing Tropical Cyclone Advisory reports shall conform to all requirements specified in Table 205-16.29.

205-16.29.3 XML elements describing Tropical Cyclone Advisory reports shall conform to all requirements of all relevant dependencies specified in Table~205-16.29.

Table 205-16.29. Requirements class xsd-tropical-cyclone-advisory

\begin{longtable}[]{@{}ll@{}}
\toprule
Requirements class &\tabularnewline
\midrule
\endhead
\href{http://icao.int/iwxxm/2.0/req/xsd-tropical-cyclone-advisory}{http://icao.int/iwxxm/2.1/req/xsd-tropical-cyclone-advisory} &\tabularnewline
Target type & Data instance\tabularnewline
Name & Tropical Cyclone Advisory\tabularnewline
\begin{minipage}[t]{0.47\columnwidth}\raggedright
Requirement\strut
\end{minipage} & \begin{minipage}[t]{0.47\columnwidth}\raggedright
\href{http://icao.int/iwxxm/2.0/req/xsd-tropical-cyclone-advisory/valid}{http://icao.int/iwxxm/2.1/req/xsd-tropical-cyclone-advisory/valid}

The content model of this element shall have a value that matches the content model of iwxxm:TropicalCycloneAdvisory.\strut
\end{minipage}\tabularnewline
\begin{minipage}[t]{0.47\columnwidth}\raggedright
Requirement\strut
\end{minipage} & \begin{minipage}[t]{0.47\columnwidth}\raggedright
\href{http://icao.int/iwxxm/2.0/req/xsd-tropical-cyclone-advisory/issuing-tropical-cyclone-advisory-centre}{http://icao.int/iwxxm/2.1/req/xsd-tropical-cyclone-advisory/issuing-tropical-cyclone-advisory-centre}

The air traffic services unit responsible for the subject airspace shall be indicated using the XML element //iwxxm:issuingTropicalCycloneAdvisoryCentre with a valid child element aixm:Unit.\strut
\end{minipage}\tabularnewline
\begin{minipage}[t]{0.47\columnwidth}\raggedright
Requirement\strut
\end{minipage} & \begin{minipage}[t]{0.47\columnwidth}\raggedright
\href{http://icao.int/iwxxm/2.0/req/xsd-tropical-cyclone-advisory/advisory-number}{http://icao.int/iwxxm/2.1/req/xsd-tropical-cyclone-advisory/advisory-number}

The advisory number of this Tropical Cyclone Advisory report shall be indicated using XML element //iwxxm:advisoryNumber.\strut
\end{minipage}\tabularnewline
\begin{minipage}[t]{0.47\columnwidth}\raggedright
Requirement\strut
\end{minipage} & \begin{minipage}[t]{0.47\columnwidth}\raggedright
\href{http://icao.int/iwxxm/2.0/req/xsd-tropical-cyclone-advisory/issue-time}{http://icao.int/iwxxm/2.1/req/xsd-tropical-cyclone-advisory/issue-time}

The issuance time of this Tropical Cyclone Advisory report shall be indicated using XML element //iwxxm:issueTime with valid child element gml:TimeInstant.\strut
\end{minipage}\tabularnewline
\begin{minipage}[t]{0.47\columnwidth}\raggedright
Requirement\strut
\end{minipage} & \begin{minipage}[t]{0.47\columnwidth}\raggedright
\href{http://icao.int/iwxxm/2.0/req/xsd-tropical-cyclone-advisory/observation}{http://icao.int/iwxxm/2.1/req/xsd-tropical-cyclone-advisory/observation}

If reported, XML element //iwxxm:observation shall contain a valid child element //om:OM\_Observation of type TropicalCycloneObservedConditions. The value of XML attribute //iwxxm:observation/om:OM\_Observation/om:type/@xlink:href shall be the URI ``\url{http://codes.wmo.int/49-2/observation-type/IWXXM/2.1/TropicalCycloneObservedConditions}''.\strut
\end{minipage}\tabularnewline
\begin{minipage}[t]{0.47\columnwidth}\raggedright
Requirement\strut
\end{minipage} & \begin{minipage}[t]{0.47\columnwidth}\raggedright
\href{http://icao.int/iwxxm/2.0/req/xsd-tropical-cyclone-advisory/forecast}{http://icao.int/iwxxm/2.1/req/xsd-tropical-cyclone-advisory/forecast}

If reported, XML element //iwxxm:forecast shall contain a valid child element\\
//om:OM\_Observation of type TropicalCycloneForecastConditions. The value of XML attribute //iwxxm:forecast/om:OM\_Observation/om:type/@xlink:href shall be the URI ``\url{http://codes.wmo.int/49-2/observation-type/IWXXM/2.1/TropicalCycloneForecastConditions}''.\strut
\end{minipage}\tabularnewline
\bottomrule
\end{longtable}

Notes:

1. Requirements for reporting the Tropical Cyclone Advisory phenomenon are specified in the \emph{Technical Regulations} (WMO-No.~49), Volume~II, Part~II, Appendix~2, 5.1.1.

2. Code table D-1 provides a set of nil-reason codes and is published at \href{http://codes.wmo.int/common/nil}{http://codes.wmo.int/common/nil.}

205-16.30 Requirements class: Tropical cyclone observed conditions

205-16.30.1 This requirements class is used to describe the presence of a specific condition observed phenomenon, along with expected changes. The geometric extent of the phenomenon is specified as a two-dimensional horizontal region with bounded vertical extent.

205-16.30.2 XML elements describing the characteristics of a tropical cyclone observed phenomenon shall conform to all requirements specified in Table 205-16.30.

205-16.30.3 XML elements describing the characteristics of a tropical cyclone observed phenomenon shall conform to all requirements of all relevant dependencies specified in Table~205-16.30.

Table 205-16.30. Requirements class xsd-tropical-cyclone-observed-conditions

\begin{longtable}[]{@{}ll@{}}
\toprule
Requirements class &\tabularnewline
\midrule
\endhead
\url{http://icao.int/iwxxm/2.1/req/xsd-tropical-cyclone-observed-conditions} &\tabularnewline
Target type & Data instance\tabularnewline
Name & Tropical cyclone observed conditions\tabularnewline
\begin{minipage}[t]{0.47\columnwidth}\raggedright
Requirement\strut
\end{minipage} & \begin{minipage}[t]{0.47\columnwidth}\raggedright
\href{http://icao.int/iwxxm/2.0/req/xsd-tropical-cyclone-observed-conditions/valid}{http://icao.int/iwxxm/2.1/req/xsd-tropical-cyclone-observed-conditions/valid}

The content model of this element shall have a value that matches the content model of iwxxm:TropicalCycloneObservedConditions.\strut
\end{minipage}\tabularnewline
\begin{minipage}[t]{0.47\columnwidth}\raggedright
Requirement\strut
\end{minipage} & \begin{minipage}[t]{0.47\columnwidth}\raggedright
\url{http://icao.int/iwxxm/2.1/req/xsd-tropical-cyclone-observed-conditions/geometry}

The geometric extent of the tropical cyclone observed phenomenon shall be reported using the XML element //iwxxm:forecast/om:OM\_Observation/om:featureOfInterest/sams:SF\_SpatialSamplingFeature/sams:shape with valid child element gml:Point.\strut
\end{minipage}\tabularnewline
\begin{minipage}[t]{0.47\columnwidth}\raggedright
Requirement\strut
\end{minipage} & \begin{minipage}[t]{0.47\columnwidth}\raggedright
\href{http://icao.int/iwxxm/2.0/req/xsd-tropical-cyclone-observed-conditions/speed-of-motion}{http://icao.int/iwxxm/2.1/req/xsd-tropical-cyclone-observed-conditions/speed-of-motion}

The speed of motion of the tropical cyclone observed phenomenon shall be reported using the XML element //iwxxm:TropicalCycloneObservedConditions/iwxxm:movementSpeed, with the unit of measure metres per second, knots or kilometres per hour.

The unit of measure shall be indicated using the XML attribute

//iwxxm:TropicalCycloneObservedConditions/iwxxm:movementSpeed/@uom with value ``m/s'' (metres per second), ``{[}kn\_i{]}'' (knots) or ``km/h'' (kilometres per hour).\strut
\end{minipage}\tabularnewline
\begin{minipage}[t]{0.47\columnwidth}\raggedright
Requirement\strut
\end{minipage} & \begin{minipage}[t]{0.47\columnwidth}\raggedright
\href{http://icao.int/iwxxm/2.0/req/xsd-tropical-cyclone-observed-conditions/direction-of-motion}{http://icao.int/iwxxm/2.1/req/xsd-tropical-cyclone-observed-conditions/direction-of-motion}

If reported, the angle between true north and the direction of motion of the tropical cyclone observed phenomenon shall be given in degrees using the XML element\\
//iwxxm:TropicalCycloneObservedConditions/iwxxm:movementDirection.

The unit of measure shall be indicated using the XML attribute //iwxxm:TropicalCycloneObservedConditions/iwxxm:directionOfMotion/@uom with value ``deg''.\strut
\end{minipage}\tabularnewline
\bottomrule
\end{longtable}

Note: Units of measurement are specified in accordance with 1.9 above.

205-16.31 Requirements class: Tropical cyclone forecast conditions

205-16.31.1 This requirements class is used to describe the presence of a specific tropical cyclone forecast phenomenon, along with expected changes. The geometric extent of the phenomenon is specified as a two-dimensional horizontal region with bounded vertical extent.

205-16.31.2 XML elements describing the characteristics of a tropical cyclone forecast phenomenon shall conform to all requirements specified in Table 205-16.31.

205-16.31.3 XML elements describing the characteristics of a tropical cyclone forecast phenomenon shall conform to all requirements of all relevant dependencies specified in Table~205-16.31.

Table 205-16.31. Requirements class xsd-tropical-cyclone-forecast-conditions

\begin{longtable}[]{@{}ll@{}}
\toprule
Requirements class &\tabularnewline
\midrule
\endhead
\href{http://icao.int/iwxxm/2.0/req/xsd-tropical-cyclone-forecast-conditions}{http://icao.int/iwxxm/2.1/req/xsd-tropical-cyclone-forecast-conditions} &\tabularnewline
Target type & Data instance\tabularnewline
Name & Tropical cyclone forecast conditions\tabularnewline
\begin{minipage}[t]{0.47\columnwidth}\raggedright
Requirement\strut
\end{minipage} & \begin{minipage}[t]{0.47\columnwidth}\raggedright
\url{http://icao.int/iwxxm/2.1/req/xsd-tropical-cyclone-forecast-conditions/valid}

The content model of this element shall have a value that matches the content model of iwxxm:TropicalCycloneForecastConditions.\strut
\end{minipage}\tabularnewline
\begin{minipage}[t]{0.47\columnwidth}\raggedright
Requirement\strut
\end{minipage} & \begin{minipage}[t]{0.47\columnwidth}\raggedright
\url{http://icao.int/iwxxm/2.1/req/xsd-tropical-cyclone-forecast-conditions/geometry}

The geometric extent of the tropical cyclone forecast phenomenon shall be reported using the XML element //iwxxm:observation/om:OM\_Observation/om:featureOfInterest/sams:SF\_SpatialSamplingFeature/sams:shape with valid child element gml:Point.\strut
\end{minipage}\tabularnewline
\bottomrule
\end{longtable}

Note: Units of measurement are specified in accordance with 1.9 above.

205-16.32 Requirements class: Volcanic Ash Advisory

205-16.32.1 This requirements class is used to describe the volcanic ash advisory report within which the characteristics of a specific Volcanic Ash Advisory phenomenon are described.

Note: The reporting requirements for Volcanic Ash Advisories are specified in the \emph{Technical Regulations} (WMO-No.~49), Volume~II, Part~II, Appendix~2, section~3.

205-16.32.2 XML elements describing Volcanic Ash Advisory reports shall conform to all requirements specified in Table 205-16.32.

205-16.32.3 XML elements describing Volcanic Ash Advisory reports shall conform to all requirements of all relevant dependencies specified in Table~205-16.32.

Table 205-16.32. Requirements class xsd-volcanic-ash-advisory

\begin{longtable}[]{@{}ll@{}}
\toprule
Requirements class &\tabularnewline
\midrule
\endhead
\href{http://icao.int/iwxxm/2.0/req/xsd-volcanic-ash-advisory}{http://icao.int/iwxxm/2.1/req/xsd-volcanic-ash-advisory} &\tabularnewline
Target type & Data instance\tabularnewline
Name & Volcanic Ash Advisory\tabularnewline
\begin{minipage}[t]{0.47\columnwidth}\raggedright
Requirement\strut
\end{minipage} & \begin{minipage}[t]{0.47\columnwidth}\raggedright
\href{http://icao.int/iwxxm/2.0/req/xsd-volcanic-ash-advisory/valid}{http://icao.int/iwxxm/2.1/req/xsd-volcanic-ash-advisory/valid}

The content model of this element shall have a value that matches the content model of iwxxm:VolcanicAshAdvisory.\strut
\end{minipage}\tabularnewline
\begin{minipage}[t]{0.47\columnwidth}\raggedright
Requirement\strut
\end{minipage} & \begin{minipage}[t]{0.47\columnwidth}\raggedright
\href{http://icao.int/iwxxm/2.0/req/xsd-volcanic-ash-advisory/issuing-volcanic-ash-advisory-centre}{http://icao.int/iwxxm/2.1/req/xsd-volcanic-ash-advisory/issuing-volcanic-ash-advisory-centre}

The air traffic services unit responsible for the subject airspace shall be indicated using the XML element //iwxxm:issuingVolcanicAshAdvisoryCentre with a valid child element aixm:Unit.\strut
\end{minipage}\tabularnewline
\begin{minipage}[t]{0.47\columnwidth}\raggedright
Requirement\strut
\end{minipage} & \begin{minipage}[t]{0.47\columnwidth}\raggedright
\href{http://icao.int/iwxxm/2.0/req/xsd-volcanic-ash-advisory/advisory-number}{http://icao.int/iwxxm/2.1/req/xsd-volcanic-ash-advisory/advisory-number}

The advisory number of this Volcanic Ash Advisory report shall be indicated using XML element //iwxxm:advisoryNumber.\strut
\end{minipage}\tabularnewline
\begin{minipage}[t]{0.47\columnwidth}\raggedright
Requirement\strut
\end{minipage} & \begin{minipage}[t]{0.47\columnwidth}\raggedright
\href{http://icao.int/iwxxm/2.0/req/xsd-volcanic-ash-advisory/issue-time}{http://icao.int/iwxxm/2.1/req/xsd-volcanic-ash-advisory/issue-time}

The issuance time of this Volcanic Ash Advisory report shall be indicated using XML element //iwxxm:issueTime with valid child element gml:TimeInstant.\strut
\end{minipage}\tabularnewline
\begin{minipage}[t]{0.47\columnwidth}\raggedright
Requirement\strut
\end{minipage} & \begin{minipage}[t]{0.47\columnwidth}\raggedright
\href{http://icao.int/iwxxm/2.0/req/xsd-volcanic-ash-advisory/analysis}{http://icao.int/iwxxm/2.1/req/xsd-volcanic-ash-advisory/analysis}

If reported, XML element //iwxxm:analysis shall contain a valid child element

//om:OM\_Observation of type VolcanicAshConditions. The value of XML attribute\\
//iwxxm:analysis/om:OM\_Observation/om:type/@xlink:href shall be the URI ``\url{http://codes.wmo.int/49-2/observation-type/IWXXM/2.1/VolcanicAshConditions}''.\strut
\end{minipage}\tabularnewline
\bottomrule
\end{longtable}

Notes:

1. Requirements for reporting the Volcanic Ash Advisory phenomenon are specified in the \emph{Technical Regulations} (WMO-No.~49), Volume~II, Part~II, Appendix~2, 3.1.1.

2. Code table D-1 provides a set of nil-reason codes and is published at \href{http://codes.wmo.int/common/nil}{http://codes.wmo.int/common/nil.}

205-16.33 Requirements class: Volcanic ash conditions

205-16.33.1 This requirements class is used to describe the presence of a specific volcanic ash phenomenon.

205-16.33.2 XML elements describing the characteristics of a volcanic ash phenomenon shall conform to all requirements specified in Table 205-16.33.

205-16.33.3 XML elements describing the characteristics of a volcanic ash phenomenon shall conform to all requirements of all relevant dependencies specified in Table 205-16.33.

Table 205-16.33. Requirements class xsd-volcanic-ash-conditions

\begin{longtable}[]{@{}ll@{}}
\toprule
Requirements class &\tabularnewline
\midrule
\endhead
\url{http://icao.int/iwxxm/2.1/req/xsd-volcanic-ash-conditions} &\tabularnewline
Target type & Data instance\tabularnewline
Name & Volcanic ash conditions\tabularnewline
\begin{minipage}[t]{0.47\columnwidth}\raggedright
Requirement\strut
\end{minipage} & \begin{minipage}[t]{0.47\columnwidth}\raggedright
\url{http://icao.int/iwxxm/2.1/req/xsd-volcanic-ash-conditions/valid}

The content model of this element shall have a value that matches the content model of iwxxm:VolcanicAshConditions.\strut
\end{minipage}\tabularnewline
\bottomrule
\end{longtable}

205-16.34 Requirements class: Volcanic ash cloud

205-16.34.1 This requirements class is used to describe the presence of specific volcanic ash clouds.

205-16.34.2 XML elements describing the characteristics of a volcanic ash cloud shall conform to all requirements specified in Table 205-16.34.

205-16.34.3 XML elements describing the characteristics of a volcanic ash cloud shall conform to all requirements of all relevant dependencies specified in Table 205-16.34.

Table 205-16.34. Requirements class xsd-volcanic-ash-cloud

\begin{longtable}[]{@{}ll@{}}
\toprule
Requirements class &\tabularnewline
\midrule
\endhead
\href{http://icao.int/iwxxm/2.0/req/xsd-volcanic-ash-cloud}{http://icao.int/iwxxm/2.1/req/xsd-volcanic-ash-cloud} &\tabularnewline
Target type & Data instance\tabularnewline
Name & Volcanic ash cloud\tabularnewline
\begin{minipage}[t]{0.47\columnwidth}\raggedright
Requirement\strut
\end{minipage} & \begin{minipage}[t]{0.47\columnwidth}\raggedright
\href{http://icao.int/iwxxm/2.0/req/xsd-volcanic-ash-cloud/valid}{http://icao.int/iwxxm/2.1/req/xsd-volcanic-ash-cloud/valid}

The content model of this element shall have a value that matches the content model of iwxxm:VolcanicAshCloud.\strut
\end{minipage}\tabularnewline
\begin{minipage}[t]{0.47\columnwidth}\raggedright
Requirement\strut
\end{minipage} & \begin{minipage}[t]{0.47\columnwidth}\raggedright
\href{http://icao.int/iwxxm/2.0/req/xsd-volcanic-ash-cloud/geometry}{http://icao.int/iwxxm/2.1/req/xsd-volcanic-ash-cloud/geometry}

The geometric extent of the volcanic ash cloud shall be reported using the XML element //iwxxm:VolcanicAshCloud/iwxxm:ashCloudExtent with valid child element aixm:AirspaceVolume.\strut
\end{minipage}\tabularnewline
\begin{minipage}[t]{0.47\columnwidth}\raggedright
Requirement\strut
\end{minipage} & \begin{minipage}[t]{0.47\columnwidth}\raggedright
\href{http://icao.int/iwxxm/2.0/req/xsd-}{http://icao.int/iwxxm/2.1/req/xsd-}volcanic-ash-cloud/speed-of-motion

The speed of motion of the volcanic ash cloud observed phenomenon shall be reported using the XML element //iwxxm:VolcanicAshCloud/iwxxm:speedOfMotion, with the unit of measure metres per second, knots or kilometres per hour.

The unit of measure shall be indicated using the XML attribute //iwxxm:VolcanicAshCloud/iwxxm:movementSpeed/@uom with value ``m/s'' (metres per second), ``{[}kn\_i{]}'' (knots) or ``km/h'' (kilometres per hour).\strut
\end{minipage}\tabularnewline
\begin{minipage}[t]{0.47\columnwidth}\raggedright
Requirement\strut
\end{minipage} & \begin{minipage}[t]{0.47\columnwidth}\raggedright
\href{http://icao.int/iwxxm/2.0/req/xsd-volcanic-ash-cloud/direction-of-motion}{http://icao.int/iwxxm/2.1/req/xsd-volcanic-ash-cloud/direction-of-motion}

If reported, the angle between true north and the direction of motion of the volcanic ash cloud shall be given in degrees using the XML element //iwxxm:VolcanicAshCloud/iwxxm:directionOfMotion.

The unit of measure shall be indicated using the XML attribute //iwxxm: VolcanicAshCloud/iwxxm:directionOfMotion/@uom with value ``deg''.\strut
\end{minipage}\tabularnewline
\bottomrule
\end{longtable}

Note: Units of measurement are specified in accordance with 1.9 above.

205-16.35 Requirements class: Report

205-16.35.1 This requirements class is used to describe the report within which the characteristics common to all IWXXM reports (such as METAR and SIGMET) are described. The report type cannot be instantiated directly, but report members are included on all sub-types.

205-16.35.2 XML elements describing report shall conform to all requirements specified in Table 205-16.35.

205-16.35.3 XML elements describing report shall conform to all requirements of all relevant dependencies specified in Table 205-16.35.

Table 205-16.35. Requirements class xsd-report

\begin{longtable}[]{@{}ll@{}}
\toprule
Requirements class &\tabularnewline
\midrule
\endhead
\href{http://icao.int/iwxxm/2.0/req/xsd-report}{http://icao.int/iwxxm/2.1/req/xsd-report} &\tabularnewline
Target type & Abstract data instance\tabularnewline
Name & Report\tabularnewline
\begin{minipage}[t]{0.47\columnwidth}\raggedright
Requirement\strut
\end{minipage} & \begin{minipage}[t]{0.47\columnwidth}\raggedright
\href{http://icao.int/iwxxm/2.0/req/xsd-report/permissible-usage}{http://icao.int/iwxxm/2.1/req/xsd-report/permissible-usage}

The permissible usage (operational or non-operational) for the report shall be indicated using the XML attribute //iwxxm:Report/@permissibleUsage.\strut
\end{minipage}\tabularnewline
\begin{minipage}[t]{0.47\columnwidth}\raggedright
Requirement\strut
\end{minipage} & \begin{minipage}[t]{0.47\columnwidth}\raggedright
\href{http://icao.int/iwxxm/2.0/req/xsd-report/permissible-usage-reason}{http://icao.int/iwxxm/2.1/req/xsd-report/permissible-usage-reason}

When the permissible usage is non-operational, the reason for this usage (test or exercise) shall be indicated using XML attribute

//iwxxm:Report/@permissibleUsageReason.\strut
\end{minipage}\tabularnewline
\begin{minipage}[t]{0.47\columnwidth}\raggedright
Requirement\strut
\end{minipage} & \begin{minipage}[t]{0.47\columnwidth}\raggedright
\href{http://icao.int/iwxxm/2.0/req/xsd-report/permissible-usage-supplementary}{http://icao.int/iwxxm/2.1/req/xsd-report/permissible-usage-supplementary}

When the permissible usage is non-operational, the human-readable descriptive supplementary information shall be indicated using XML attribute

//iwxxm:Report/@permissibleUsageSupplementary.\strut
\end{minipage}\tabularnewline
\begin{minipage}[t]{0.47\columnwidth}\raggedright
Requirement\strut
\end{minipage} & \begin{minipage}[t]{0.47\columnwidth}\raggedright
\href{http://icao.int/iwxxm/2.0/req/xsd-report/translatedBulletinId}{http://icao.int/iwxxm/2.1/req/xsd-report/translatedBulletinId}

If reported on a report that was translated from the Traditional Alphanumeric Code (TAC) report, XML attribute //iwxxm:Report/@translatedBulletinId shall contain the bulletin ID of the form `TTAAiiCCCYYGGgg'.\strut
\end{minipage}\tabularnewline
\begin{minipage}[t]{0.47\columnwidth}\raggedright
Requirement\strut
\end{minipage} & \begin{minipage}[t]{0.47\columnwidth}\raggedright
\href{http://icao.int/iwxxm/2.0/req/xsd-report/translatedBulletinReceptionTime}{http://icao.int/iwxxm/2.1/req/xsd-report/translatedBulletinReceptionTime}

If reported on a report that was translated from TAC, XML attribute //iwxxm:Report/@translatedBulletinReceptionTime shall contain the time at which the bulletin was received by the translation centre.\strut
\end{minipage}\tabularnewline
\begin{minipage}[t]{0.47\columnwidth}\raggedright
Requirement\strut
\end{minipage} & \begin{minipage}[t]{0.47\columnwidth}\raggedright
\href{http://icao.int/iwxxm/2.0/req/xsd-report/translatedCentreDesignator}{http://icao.int/iwxxm/2.1/req/xsd-report/translatedCentreDesignator}

If reported on a report that was translated from TAC, XML attribute //iwxxm:Report/@translatedCentreDesignator shall contain the ICAO designator of the translation centre.\strut
\end{minipage}\tabularnewline
\begin{minipage}[t]{0.47\columnwidth}\raggedright
Requirement\strut
\end{minipage} & \begin{minipage}[t]{0.47\columnwidth}\raggedright
\href{http://icao.int/iwxxm/2.0/req/xsd-report/translatedCentreName}{http://icao.int/iwxxm/2.1/req/xsd-report/translatedCentreName}

If reported on a report that was translated from TAC, XML attribute //iwxxm:Report/@translatedCentreName shall contain the name of the translation centre.\strut
\end{minipage}\tabularnewline
\begin{minipage}[t]{0.47\columnwidth}\raggedright
Requirement\strut
\end{minipage} & \begin{minipage}[t]{0.47\columnwidth}\raggedright
\href{http://icao.int/iwxxm/2.0/req/xsd-report/translationTime}{http://icao.int/iwxxm/2.1/req/xsd-report/translationTime}

If reported on a report that was translated from TAC, XML attribute //iwxxm:Report/@translationTime shall contain the time at which the bulletin was translated.\strut
\end{minipage}\tabularnewline
\begin{minipage}[t]{0.47\columnwidth}\raggedright
Requirement\strut
\end{minipage} & \begin{minipage}[t]{0.47\columnwidth}\raggedright
\url{http://icao.int/iwxxm/2.1/req/xsd-report/translatedFailedTAC}

If reported on a report that was translated from TAC which could not be completely translated, XML attribute //iwxxm:Report/@translationFailedTAC shall contain the original TAC report that was not translated. When translation fails, only the report type (i.e. SIGMET or METAR), translation information and other basic report metadata should be provided. In this case no translated content will be included other than the original TAC. Permissible usage may be set as normal and TAC that failed translation may still be used for operational purposes, but under no circumstances should partially translated content be distributed or marked as operational.\strut
\end{minipage}\tabularnewline
\bottomrule
\end{longtable}

205-16.36 Requirements class: Aerodrome cloud forecast

205-16.36.1 This requirements class is used to describe forecast cloud conditions at an aerodrome. The class is targeted at providing a basic description of the forecast cloud conditions as required for civil aviation purposes.

Notes:

1. Representations providing more detailed information may be used if required.

2. The requirements for reporting forecast cloud conditions are specified in the \emph{Technical Regulations} (WMO-No.~49), Volume~II, Part~II, Appendix~5, 2.2.5 and 1.2.4.

205-16.36.2 XML elements describing forecast cloud conditions shall conform to all requirements specified in Table~205-16.36.

205-16.36.3 XML elements describing forecast cloud conditions shall conform to all requirements of all relevant dependencies specified in Table~205-16.36.

Table~205-16.36. Requirements class xsd-aerodrome-cloud-forecast

\begin{longtable}[]{@{}ll@{}}
\toprule
Requirements class &\tabularnewline
\midrule
\endhead
\href{http://icao.int/iwxxm/1.1/req/xsd-aerodrome-cloud-forecast}{http://icao.int/iwxxm/2.1/req/xsd-aerodrome-cloud-forecast} &\tabularnewline
Target type & Data instance\tabularnewline
Name & Aerodrome cloud forecast\tabularnewline
Dependency & \href{http://icao.int/iwxxm/1.1/req/xsd-cloud-layer}{http://icao.int/iwxxm/2.1/req/xsd-cloud-layer}, 205-16.39\tabularnewline
\begin{minipage}[t]{0.47\columnwidth}\raggedright
Requirement\strut
\end{minipage} & \begin{minipage}[t]{0.47\columnwidth}\raggedright
\href{http://icao.int/iwxxm/1.1/req/xsd-aerodrome-cloud-forecast/valid}{http://icao.int/iwxxm/2.1/req/xsd-aerodrome-cloud-forecast/valid}

The content model of this element shall have a value that matches the content model of iwxxm:AerodromeCloudForecast.\strut
\end{minipage}\tabularnewline
\begin{minipage}[t]{0.47\columnwidth}\raggedright
Requirement\strut
\end{minipage} & \begin{minipage}[t]{0.47\columnwidth}\raggedright
\href{http://icao.int/iwxxm/1.1/req/xsd-aerodrome-cloud-forecast/vertical-visibility}{http://icao.int/iwxxm/2.1/req/xsd-aerodrome-cloud-forecast/vertical-visibility}

When cloud of operational significance is forecast, then the XML element //iwxxm:AerodromeCloudForecast/iwxxm:verticalVisibility shall be used to report the vertical visual range.\strut
\end{minipage}\tabularnewline
\begin{minipage}[t]{0.47\columnwidth}\raggedright
Requirement\strut
\end{minipage} & \begin{minipage}[t]{0.47\columnwidth}\raggedright
\href{http://icao.int/iwxxm/1.1/req/xsd-aerodrome-cloud-forecast/vertical-visibility-unit-of-measure}{http://icao.int/iwxxm/2.1/req/xsd-aerodrome-cloud-forecast/vertical-visibility-unit-of-measure}

If the vertical visibility is reported, then the vertical distance shall be expressed in metres or feet. The unit of measure shall be indicated using the XML attribute\\
//iwxxm:AerodromeCloudForecast/iwxxm:verticalVisibility/@uom with value ``m'' (metres) or ``{[}ft\_i{]}'' (feet).\strut
\end{minipage}\tabularnewline
\begin{minipage}[t]{0.47\columnwidth}\raggedright
Requirement\strut
\end{minipage} & \begin{minipage}[t]{0.47\columnwidth}\raggedright
\href{http://icao.int/iwxxm/1.1/req/xsd-aerodrome-cloud-forecast/cloud-layers}{http://icao.int/iwxxm/2.1/req/xsd-aerodrome-cloud-forecast/cloud-layers}

When cloud of operational significance is forecast, then the XML element //iwxxm:AerodromeCloudForecast/iwxxm:layer, containing a valid child element //iwxxm:AerodromeCloudForecast/iwxxm:layer/iwxxm:CloudLayer, shall be used to describe each cloud layer.\strut
\end{minipage}\tabularnewline
\begin{minipage}[t]{0.47\columnwidth}\raggedright
Requirement\strut
\end{minipage} & \begin{minipage}[t]{0.47\columnwidth}\raggedright
\href{http://icao.int/iwxxm/1.1/req/xsd-aerodrome-cloud-forecast/number-of-cloud-layers}{http://icao.int/iwxxm/2.1/req/xsd-aerodrome-cloud-forecast/number-of-cloud-layers}

No more than four cloud layers shall be reported. If more than four significant cloud layers are forecast, then the four most significant cloud layers with respect to aviation operations shall be prioritized.\strut
\end{minipage}\tabularnewline
\bottomrule
\end{longtable}

Notes:

1. Cloud of operational significance includes cloud below 1~500 metres or the highest minimum sector altitude, whichever is greater, and cumulonimbus whenever present.

2. Vertical visibility is defined as the vertical visual range into an obscuring medium.

3. Units of measurement are specified in accordance with 1.9 above.

205-16.37 Requirements class: Aerodrome surface wind forecast

205-16.37.1 This requirements class is used to describe the surface wind conditions forecast at an aerodrome as appropriate for inclusion in an aerodrome forecast (TAF) report.

Note: The requirements for reporting the surface wind conditions within a TAF are specified in the \emph{Technical Regulations} (WMO-No.~49), Volume~II, Part~II, Appendix~5, 1.2.1.

205-16.37.2 XML elements describing surface wind conditions forecast shall conform to all requirements specified in Table~205-16.37.

205-16.37.3 XML elements describing surface wind conditions forecast shall conform to all requirements of all relevant dependencies specified in Table~205-16.37.

Table~205-16.37. Requirements class xsd-aerodrome-surface-wind-forecast

\begin{longtable}[]{@{}ll@{}}
\toprule
Requirements class &\tabularnewline
\midrule
\endhead
\href{http://icao.int/iwxxm/1.1/req/xsd-aerodrome-surface-wind-forecast}{http://icao.int/iwxxm/2.1/req/xsd-aerodrome-surface-wind-forecast} &\tabularnewline
Target type & Data instance\tabularnewline
Name & Aerodrome surface wind forecast\tabularnewline
Dependency & \href{http://icao.int/iwxxm/1.1/req/xsd-aerodrome-surface-wind-trend-forecast}{http://icao.int/iwxxm/2.1/req/xsd-aerodrome-surface-wind-trend-forecast}, 205-16.38\tabularnewline
\begin{minipage}[t]{0.47\columnwidth}\raggedright
Requirement\strut
\end{minipage} & \begin{minipage}[t]{0.47\columnwidth}\raggedright
http://icao.int/iwxxm/2.1/req/xsd-aerodrome-surface-wind-forecast/valid

The content model of this element shall have a value that matches the content model of iwxxm:AerodromeSurfaceWindForecast.\strut
\end{minipage}\tabularnewline
\begin{minipage}[t]{0.47\columnwidth}\raggedright
Requirement\strut
\end{minipage} & \begin{minipage}[t]{0.47\columnwidth}\raggedright
\href{http://icao.int/iwxxm/1.1/req/xsd-aerodrome-surface-wind-forecast/variable-wind-direction}{http://icao.int/iwxxm/2.1/req/xsd-aerodrome-surface-wind-forecast/variable-wind-direction}

If the wind direction is variable, then the XML attribute\\
//iwxxm:AerodromeSurfaceWindForecast/@variableWindDirection shall have the value ``true'' and XML element //iwxxm:AerodromeSurfaceWindForecast/iwxxm:meanWindDirection shall be absent.\strut
\end{minipage}\tabularnewline
\bottomrule
\end{longtable}

Note: Wind direction is reported as variable (VRB) if is not possible to forecast a prevailing surface wind direction due to expected variability, for example, during light wind conditions (less than 3~knots) or thunderstorms.

205-16.38 Requirements class: Aerodrome surface wind trend forecast

205-16.38.1 This requirements class is used to describe the surface wind conditions forecast at an aerodrome as appropriate for inclusion in a trend forecast of a routine or special meteorological aerodrome report.

Note: The requirements for reporting the surface wind conditions within a trend forecast are specified in the \emph{Technical Regulations} (WMO-No.~49), Volume~II, Part~II, Appendix~5, 2.2.2.

205-16.38.2 XML elements describing surface wind conditions within a trend forecast shall conform to all requirements specified in Table~205-16.38.

205-16.38.3 XML elements describing surface wind conditions within a trend forecast shall conform to all requirements of all relevant dependencies specified in Table~205-16.38.

Table~205-16.38. Requirements class xsd-aerodrome-surface-wind-trend-forecast

\begin{longtable}[]{@{}ll@{}}
\toprule
Requirements class &\tabularnewline
\midrule
\endhead
http://icao.int/iwxxm/2.1/req/xsd-aerodrome-surface-wind-trend-forecast &\tabularnewline
Target type & Data instance\tabularnewline
Name & Aerodrome surface wind trend forecast\tabularnewline
\begin{minipage}[t]{0.47\columnwidth}\raggedright
Requirement\strut
\end{minipage} & \begin{minipage}[t]{0.47\columnwidth}\raggedright
\href{http://icao.int/iwxxm/1.1/req/xsd-aerodrome-surface-wind-trend-forecast/valid}{http://icao.int/iwxxm/2.1/req/xsd-aerodrome-surface-wind-trend-forecast/valid}

The content model of this element shall have a value that matches the content model of iwxxm:AerodromeSurfaceWindTrendForecast.\strut
\end{minipage}\tabularnewline
\begin{minipage}[t]{0.47\columnwidth}\raggedright
Requirement\strut
\end{minipage} & \begin{minipage}[t]{0.47\columnwidth}\raggedright
\href{http://icao.int/iwxxm/1.1/req/xsd-aerodrome-surface-wind-trend-forecast/mean-wind-speed}{http://icao.int/iwxxm/2.1/req/xsd-aerodrome-surface-wind-trend-forecast/mean-wind-speed}

The forecast mean wind speed shall be stated using the XML element //iwxxm:AerodromeSurfaceWindTrendForecast/iwxxm:meanWindSpeed, with the unit of measure metres per second, knots or kilometres per hour. The unit of measure shall be indicated using the XML attribute @uom with value ``m/s'' (metres per second), ``{[}kn\_i{]}'' (knots) or ``km/h'' (kilometres per hour).\strut
\end{minipage}\tabularnewline
\begin{minipage}[t]{0.47\columnwidth}\raggedright
Requirement\strut
\end{minipage} & \begin{minipage}[t]{0.47\columnwidth}\raggedright
\href{http://icao.int/iwxxm/1.1/req/xsd-aerodrome-surface-wind-trend-forecast/wind-direction}{http://icao.int/iwxxm/2.1/req/xsd-aerodrome-surface-wind-trend-forecast/wind-direction}

If the forecast mean wind direction is reported, then the angle between true north and the mean direction from which the wind is forecast to be blowing shall be expressed using XML element //iwxxm:AerodromeSurfaceWindTrendForecast/iwxxm:meanWindDirection, with the unit of measure indicated using the XML attribute\\
@uom with value ``deg''.\strut
\end{minipage}\tabularnewline
\begin{minipage}[t]{0.47\columnwidth}\raggedright
Requirement\strut
\end{minipage} & \begin{minipage}[t]{0.47\columnwidth}\raggedright
\href{http://icao.int/iwxxm/1.1/req/xsd-aerodrome-surface-wind-trend-forecast/gust-speed}{http://icao.int/iwxxm/2.1/req/xsd-aerodrome-surface-wind-trend-forecast/gust-speed}

If reported, the forecast gust speed shall be stated using the XML element //iwxxm:AerodromeSurfaceWindTrendForecast/iwxxm:windGustSpeed and expressed in metres per second, knots or kilometres per hour.

The unit of measure shall be indicated using the XML attribute //iwxxm:AerodromeSurfaceWind/iwxxm:windGustSpeed/@uom with value ``m/s'' (metres per second), ``{[}kn\_i{]}'' (knots) or ``km/h'' (kilometres per hour).\strut
\end{minipage}\tabularnewline
\bottomrule
\end{longtable}

Notes:

1. Units of measurement are specified in accordance with 1.9 above.

2. The true north is the north point at which the meridian lines meet.

205-16.39 Requirements class: Cloud layer

205-16.39.1 This requirements class is used to describe the representation of a cloud layer. The class is targeted at providing a basic description of the cloud layer as required for international civil aviation purposes.

Notes:

1. Representations providing more detailed information may be used if required.

2. The requirements for reporting cloud information are specified in the \emph{Technical Regulations} (WMO-No.~49), Volume~II, Part~II, Appendix~3, 4.5 and Appendix~5, 2.2.5 and 1.2.4.

205-16.39.2 XML elements describing cloud layers shall conform to all requirements specified in Table~205-16.39.

205-16.39.3 XML elements describing cloud layers shall conform to all requirements of all relevant dependencies specified in Table~205-16.39.

Table~205-16.39. Requirements class: xsd-cloud-layer

\begin{longtable}[]{@{}ll@{}}
\toprule
Requirements class &\tabularnewline
\midrule
\endhead
\href{http://icao.int/iwxxm/1.1/req/xsd-cloud-layer}{http://icao.int/iwxxm/2.1/req/xsd-cloud-layer} &\tabularnewline
Target type & Data instance\tabularnewline
Name & Cloud layer\tabularnewline
\begin{minipage}[t]{0.47\columnwidth}\raggedright
Requirement\strut
\end{minipage} & \begin{minipage}[t]{0.47\columnwidth}\raggedright
\href{http://icao.int/iwxxm/1.1/req/xsd-cloud-layer/valid}{http://icao.int/iwxxm/2.1/req/xsd-cloud-layer/valid}

The content model of this element shall have a value that matches the content model of iwxxm:CloudLayer.\strut
\end{minipage}\tabularnewline
\begin{minipage}[t]{0.47\columnwidth}\raggedright
Requirement\strut
\end{minipage} & \begin{minipage}[t]{0.47\columnwidth}\raggedright
\href{http://icao.int/iwxxm/1.1/req/xsd-cloud-layer/cloud-amount}{http://icao.int/iwxxm/2.1/req/xsd-cloud-layer/cloud-amount}

The XML element //iwxxm:CloudLayer/iwxxm:amount shall be used to report an amount of cloud of operational significance.\strut
\end{minipage}\tabularnewline
\begin{minipage}[t]{0.47\columnwidth}\raggedright
Requirement\strut
\end{minipage} & \begin{minipage}[t]{0.47\columnwidth}\raggedright
\href{http://icao.int/iwxxm/1.1/req/xsd-cloud-layer/cloud-amount-code}{http://icao.int/iwxxm/2.1/req/xsd-cloud-layer/cloud-amount-code}

If cloud amount is reported, the value of XML attribute //iwxxm:CloudLayer/iwxxm:amount/@xlink:href shall be the URI of the valid term from Code table~D‑8: Cloud amount reported at aerodrome.\strut
\end{minipage}\tabularnewline
\begin{minipage}[t]{0.47\columnwidth}\raggedright
Requirement\strut
\end{minipage} & \begin{minipage}[t]{0.47\columnwidth}\raggedright
\href{http://icao.int/iwxxm/1.1/req/xsd-cloud-layer/cloud-base}{http://icao.int/iwxxm/2.1/req/xsd-cloud-layer/cloud-base}

The XML element //iwxxm:CloudLayer/iwxxm:base shall indicate the height of the lowest level in the atmosphere that contains a perceptible quantity of cloud particles or the reason for not reporting the cloud base shall be expressed using the XML attribute //iwxxm:CloudLayer/iwxxm:base/@nilReason to indicate the appropriate nil-reason code.

If a nil-reason code is provided, the XML attributes //iwxxm:CloudLayer/iwxxm:base/@xsi:nil and //iwxxm:CloudLayer/iwxxm:base/@uom shall have the values ``true'' and ``N/A'', respectively.\strut
\end{minipage}\tabularnewline
\begin{minipage}[t]{0.47\columnwidth}\raggedright
Requirement\strut
\end{minipage} & \begin{minipage}[t]{0.47\columnwidth}\raggedright
\href{http://icao.int/iwxxm/1.1/req/xsd-cloud-layer/cloud-base-unit-of-measure}{http://icao.int/iwxxm/2.1/req/xsd-cloud-layer/cloud-base-unit-of-measure}

If the cloud base is reported, then the vertical distance shall be expressed in metres or feet. The unit of measure shall be indicated using the XML attribute //iwxxm:CloudLayer/iwxxm:base/@uom with value ``m'' (metres) or ``{[}ft\_i{]}'' (feet).\strut
\end{minipage}\tabularnewline
\begin{minipage}[t]{0.47\columnwidth}\raggedright
Requirement\strut
\end{minipage} & \begin{minipage}[t]{0.47\columnwidth}\raggedright
\href{http://icao.int/iwxxm/1.1/req/xsd-cloud-layer/cloud-type-code}{http://icao.int/iwxxm/2.1/req/xsd-cloud-layer/cloud-type-code}

If cloud type is reported, the value of XML attribute //iwxxm:CloudLayer/iwxxm:cloudType/@xlink:href shall be the URI of the valid cloud type from Code table~D-9: Significant convective cloud type.\strut
\end{minipage}\tabularnewline
\begin{minipage}[t]{0.47\columnwidth}\raggedright
Recommendation\strut
\end{minipage} & \begin{minipage}[t]{0.47\columnwidth}\raggedright
\href{http://icao.int/iwxxm/1.1/req/xsd-cloud-layer/cloud-type}{http://icao.int/iwxxm/2.1/req/xsd-cloud-layer/cloud-type}

If reporting observed cloud, then the XML element //iwxxm:CloudLayer/iwxxm:cloudType should be used to report the most cloud of operational significance type in the layer of cloud.\strut
\end{minipage}\tabularnewline
\begin{minipage}[t]{0.47\columnwidth}\raggedright
Recommendation\strut
\end{minipage} & \begin{minipage}[t]{0.47\columnwidth}\raggedright
\href{http://icao.int/iwxxm/1.1/req/xsd-cloud-layer/nil-significant-cloud}{http://icao.int/iwxxm/2.1/req/xsd-cloud-layer/nil-significant-cloud}

If no cloud of operational significance is reported, then the value of XML attribute\\
//iwxxm:CloudLayer/iwxxm:amount/@nilReason should be set to \url{http://codes.wmo.int/common/nil/nothingOfOperationalSignificance}.

If reporting observed cloud, then the value of XML attribute //iwxxm:CloudLayer/iwxxm:cloudType/@nilReason should also be set to \url{http://codes.wmo.int/common/nil/nothingOfOperationalSignificance}.\strut
\end{minipage}\tabularnewline
\bottomrule
\end{longtable}

Notes:

1. Cloud of operational significance includes cloud below 1~500~metres or the highest minimum sector altitude, whichever is greater, and cumulonimbus whenever present.

2. Code table~D-1 provides a set of nil-reason codes and is published at \url{http://codes.wmo.int/common/nil}.

3. Units of measurement are specified in accordance with 1.9 above.

4. Code table~D-8 is published online at \url{http://codes.wmo.int/49-2/CloudAmountReportedAtAerodrome}.

5. Code table~D-9 is published online at \url{http://codes.wmo.int/49-2/SigConvectiveCloudType}.

205-16.40 Requirements class: Angle with nil reason

205-16.40.1 This requirements class is a nillable angle quantity. Unlike the base angle measure, references to this type may be nil and may include a nilReason.

205-16.40.2 XML elements describing an angle with nil reason shall conform to all requirements specified in Table~205-16.40.

205-16.40.3 XML elements describing an angle with nil reason shall conform to all requirements of all relevant dependencies specified in Table~205-16.40.

Table~205-16.40. Requirements class: xsd-angle-with-nil-reason

\begin{longtable}[]{@{}ll@{}}
\toprule
Requirements class &\tabularnewline
\midrule
\endhead
\url{http://icao.int/iwxxm/2.1/req/xsd-angle-with-nil-reason} &\tabularnewline
Target type & Data instance\tabularnewline
Name & Angle with nil reason\tabularnewline
\begin{minipage}[t]{0.47\columnwidth}\raggedright
Requirement\strut
\end{minipage} & \begin{minipage}[t]{0.47\columnwidth}\raggedright
\url{http://icao.int/iwxxm/2.1/req/xsd-angle-with-nil-reason/value}

The XML element iwxxm:AngleWithNilReason shall contain valid numerical content or shall have the ``nil'' attribute set to ``true'' and the ``nilReason'' attribute set with the reason for nil content.\strut
\end{minipage}\tabularnewline
\bottomrule
\end{longtable}

205-16.41 Requirements class: Distance with nil reason

205-16.41.1 This requirements class is a nillable distance quantity. Unlike the base distance measure, references to this type may be nil and may include a nilReason.

205-16.41.2 XML elements describing a distance with nil reason shall conform to all requirements specified in Table~205-16.41.

205-16.41.3 XML elements describing a distance with nil reason shall conform to all requirements of all relevant dependencies specified in Table~205-16.41.

Table~205-16.41. Requirements class: xsd-distance-with-nil-reason

\begin{longtable}[]{@{}ll@{}}
\toprule
Requirements class &\tabularnewline
\midrule
\endhead
\url{http://icao.int/iwxxm/2.1/req/xsd-distance-with-nil-reason} &\tabularnewline
Target type & Data instance\tabularnewline
Name & Distance with nil reason\tabularnewline
\begin{minipage}[t]{0.47\columnwidth}\raggedright
Requirement\strut
\end{minipage} & \begin{minipage}[t]{0.47\columnwidth}\raggedright
\url{http://icao.int/iwxxm/2.1/req/xsd-distance-with-nil-reason/value}

The XML element iwxxm:DistanceWithNilReason shall contain valid numerical content or shall have the ``nil'' attribute set to ``true'' and the ``nilReason'' attribute set with the reason for nil content.\strut
\end{minipage}\tabularnewline
\bottomrule
\end{longtable}

205-16.42 Requirements class: Length with nil reason

205-16.42.1 This requirements class is a nillable length quantity. Unlike the base length measure, references to this type may be nil and may include a nilReason.

205-16.42.2 XML elements describing a length with nil reason shall conform to all requirements specified in Table~205-16.42.

205-16.42.3 XML elements describing a length with nil reason shall conform to all requirements of all relevant dependencies specified in Table~205-16.42.

Table~205-16.42. Requirements class: xsd-length-with-nil-reason

\begin{longtable}[]{@{}ll@{}}
\toprule
Requirements class &\tabularnewline
\midrule
\endhead
\url{http://icao.int/iwxxm/2.1/req/xsd-length-with-nil-reason} &\tabularnewline
Target type & Data instance\tabularnewline
Name & Length with nil reason\tabularnewline
\begin{minipage}[t]{0.47\columnwidth}\raggedright
Requirement\strut
\end{minipage} & \begin{minipage}[t]{0.47\columnwidth}\raggedright
\url{http://icao.int/iwxxm/2.1/req/xsd-length-with-nil-reason/value}

The XML element iwxxm:LengthWithNilReason shall contain valid numerical content or shall have the ``nil'' attribute set to ``true'' and the ``nilReason'' attribute set with the reason for nil content.\strut
\end{minipage}\tabularnewline
\bottomrule
\end{longtable}

FM~221: TSML

FM 221-16 TSML-XML REPRESENTATION OF INFORMATION AS TIME SERIES

221-16.1 Scope

221-16.1.1 TSML-XML shall be used for the exchange in XML of time series information conforming to the ``Timeseries Profile of Observations and Measurements'' conceptual model. TSML-XML may be used directly to encode time series information or incorporated as components within other XML encodings.

221-16.1.2 The definition of the TSML-XML application schema is in documents published by the Open Geospatial Consortium as "OGC/IS 15-043r3 Timeseries Profile of Observations and Measurements" with the XML encoding described in "OGC/IS 15-042r3 TimeseriesML~1.0 -- XML Encoding of the Timeseries Profile of Observations and Measurements".

Notes:

1. Copies of the schema are also available at \href{http://schemas.wmo.int/tsml/1.0}{http://schemas.wmo.int/tsml/1.0.}

2. Copies of the definition documents are available at \url{http://schemas.wmo.int/tsml/1.0/documents/15-043r3_TimeseriesProfile.pdf} and \href{http://schemas.wmo.int/tsml/1.0/documents/15-042r3_TimeseriesXML.pdf}{http://schemas.wmo.int/tsml/1.0/documents/15-042r3\_TimeseriesXML.pdf.}

FM 231: WMLTS

FM 231-16 WMLTS-XML WATERML2 TIME SERIES OBSERVATIONS

231-16.1 Scope

WMLTS-XML shall be used for the exchange in XML of time series of hydrological information conforming to the ``WaterML2.0: Part~1 -- Timeseries'' conceptual model. WMLTS-XML may be used directly to represent time series information or incorporated as components within other XML encodings.

Notes:

1. WaterML2.0: Part~1 -- Timeseries was developed jointly by WMO and the Open Geospatial Consortium.

2. The WMLTS-XML application schema and XML encoding are both described in the document OGC/IS 10-126r4 WaterML~2.0: Part~1 -- Timeseries. A copy of that document is available at \url{http://wis.wmo.int/WMLTS} and the reference version of the associated schema is available at \url{http://schemas.opengis.net/waterml/2.0} (WMO retains a copy of the schema at \url{http://schemas.wmo.int/waterml/2.0}).

3. Further information on handling application schema and data modelling can be found in the \emph{Guidelines on Data Modelling for WMO Codes} (available in English only from \url{http://wis.wmo.int/metce-uml}).

4. Representation of non-hydrological information in time series should use FM-221 TSML.

FM 232: WaterML2

FM 232-16 WMLRGS --XML WATERML2 Ratings, gaugings and sections

232-16.1 Scope

WMLRGS-XML shall be used for the exchange in XML of hydrological information conforming to the ``WaterML2.0: Part~2 -- Ratings, Gaugings and Sections'' conceptual model. WMLRGS-XML may be used directly to encode ratings, gaugings and sections information or incorporated as components within other XML encodings.

Notes:

1. WaterML2.0: Part 2 -- Ratings, Gaugings and Sections was developed jointly by WMO and the Open Geospatial Consortium.

2. The WMLRGS-XML application schema and XML encoding are both described in the document 15-018r2 OGC WaterML2.0: Part 2 -- Ratings, Gaugings and Sections (Version 1.0). A copy of that document is available at \url{http://wis.wmo.int/WMLRGS} and the reference version of the associated schema is available at \url{http://schemas.opengis.net/waterml/part2/1.0} (WMO retains a copy of the schema at \url{http://schemas.wmo.int/waterml/part2/1.0}).

3. Further information on handling application schema and data modelling can be found in the \emph{Guidelines on Data Modelling for WMO Codes} (available in English only from \url{http://wis.wmo.int/metce-uml}).

FM 241: WMDR

FM 241-16 WMDR-XML WIGOS METADATA DATA REPRESENTATION

Table~241-16.1 lists the code tables and their locations that shall be used in WIGOS metadata records.

Table 241-16.1. Code tables used by the WIGOS metadata standard and WMDR-XML

\begin{longtable}[]{@{}lll@{}}
\toprule
WIGOS table reference & Description & Location of code table\tabularnewline
\midrule
\endhead
1-01 & Observed variable -- measurand & \href{http://codes.wmo.int/common/wmdrObservedVariable}{http://codes.wmo.int/wmdr/ObservedVariable}\tabularnewline
1-02 & Measurement unit & \url{http://codes.wmo.int/common/unit}\tabularnewline
1-05 & Representativeness & \href{http://codes.wmo.int/common/wmdsRepresentativeness}{http://codes.wmo.int/wmdr/Representativeness}\tabularnewline
2-01 & Application areas & \href{http://codes.wmo.int/common/wmdsApplicationArea}{http://codes.wmo.int/wmdr/ApplicationArea}\tabularnewline
2-02 & Programme/network affiliation & \href{http://codes.wmo.int/common/wmdsProgramAffiliation}{http://codes.wmo.int/wmdr/ProgramAffiliation}\tabularnewline
3-01 & Region of origin of data & \href{http://codes.wmo.int/common/wmdsWMORegion}{http://codes.wmo.int/wmdr/WMORegion}\tabularnewline
3-02 & Territory of origin of data & \href{http://codes.wmo.int/common/wmdsTerritoryName}{http://codes.wmo.int/wmdr/TerritoryName}\tabularnewline
3-04 & Station/platform type & \href{http://codes.wmo.int/common/wmdsFacilityType}{http://codes.wmo.int/wmdr/FacilityType}\tabularnewline
3-08 & Data communication method & \href{http://codes.wmo.int/common/wmdsDataCommunicationMethod}{http://codes.wmo.int/wmdr/DataCommunicationMethod}\tabularnewline
3-09 & Station/Platform operating status & \href{http://codes.wmo.int/common/wmdsReportingStatus}{http://codes.wmo.int/wmdr/ReportingStatus}\tabularnewline
4-01-01 & Surface cover types (IGBP) & \href{http://codes.wmo.int/common/wmdsSurfaceCoverIGBP}{http://codes.wmo.int/wmdr/SurfaceCoverIGBP}\tabularnewline
4-01-02 & Surface cover types (UMD) & \href{http://codes.wmo.int/common/wmdsSurfaceCoverUMD}{http://codes.wmo.int/wmdr/SurfaceCoverUMD}\tabularnewline
4-01-03 & Surface cover types (LAI/fPAR) & \href{http://codes.wmo.int/common/wmdsSurfaceCoverLAI}{http://codes.wmo.int/wmdr/SurfaceCoverLAI}\tabularnewline
4-01-04 & Surface cover types (NPP) & \href{http://codes.wmo.int/common/wmdsSurfaceCoverNPP}{http://codes.wmo.int/wmdr/SurfaceCoverNPP}\tabularnewline
4-01-05 & Surface cover types (PFT) & \href{http://codes.wmo.int/common/wmdsSurfaceCoverPFT}{http://codes.wmo.int/wmdr/SurfaceCoverPFT}\tabularnewline
4-01-06 & Surface cover types (LCCS) & \href{http://codes.wmo.int/common/wmdsSurfaceCoverLCCS}{http://codes.wmo.int/wmdr/SurfaceCoverLCCS}\tabularnewline
4-02 & Surface cover classification scheme & \href{http://codes.wmo.int/common/wmdsSurfaceCoverClassification}{http://codes.wmo.int/wmdr/SurfaceCoverClassification}\tabularnewline
4-03-01 & Local topography & \href{http://codes.wmo.int/common/wmdsLocalTopography}{http://codes.wmo.int/wmdr/LocalTopography}\tabularnewline
4-03-02 & Relative elevation & \href{http://codes.wmo.int/common/wmdsRelativeElevation}{http://codes.wmo.int/wmdr/RelativeElevation}\tabularnewline
4-03-03 & Topographic context & \href{http://codes.wmo.int/common/wmdsTopographicContext}{http://codes.wmo.int/wmdr/TopographicContext}\tabularnewline
4-03-04 & Altitude/depth & \href{http://codes.wmo.int/common/wmdsAltitudeOrDepth}{http://codes.wmo.int/wmdr/AltitudeOrDepth}\tabularnewline
4-04 & Events at station/platform & \href{http://codes.wmo.int/common/wmdsEventAtFacility}{http://codes.wmo.int/wmdr/EventAtFacility}\tabularnewline
4-06 & Surface roughness (Davenport roughness classification) & \href{http://codes.wmo.int/common/wmdsSurfaceRoughnessDavenport}{http://codes.wmo.int/wmdr/SurfaceRoughnessDavenport}\tabularnewline
4-07 & Climate zone & \href{http://codes.wmo.int/common/wmdsClimateZone}{http://codes.wmo.int/wmdr/ClimateZone}\tabularnewline
5-01 & Source of observation & \href{http://codes.wmo.int/common/wmdsSourceOfObservation}{http://codes.wmo.int/wmdr/SourceOfObservation}\tabularnewline
5-02 & Measurement/observing method & \href{http://codes.wmo.int/common/wmdsObservingMethod}{http://codes.wmo.int/wmdr/ObservingMethod}\tabularnewline
5-04 & Instrument operating status & \href{http://codes.wmo.int/common/wmdsInstrumentOperatingStatus}{http://codes.wmo.int/wmdr/InstrumentOperatingStatus}\tabularnewline
5-08-01 & Control standard type & \href{http://codes.wmo.int/common/wmdsControlStandardType}{http://codes.wmo.int/wmdr/ControlStandardType}\tabularnewline
5-08-02 & Control location & \href{http://codes.wmo.int/common/wmdsControlLocation}{http://codes.wmo.int/wmdr/ControlLocation}\tabularnewline
5-08-03 & Instrument control result & \href{http://codes.wmo.int/common/wmdsInstrumentControlResult}{http://codes.wmo.int/wmdr/InstrumentControlResult}\tabularnewline
5-14 & Status of observation & \href{http://codes.wmo.int/common/wmdr/ObservationOperatingStatus}{http://codes.wmo.int/wmdr/ObservationOperatingStatus}\tabularnewline
5-15 & Exposure of instrument & \href{http://codes.wmo.int/common/wmdsExposure}{http://codes.wmo.int/wmdr/Exposure}\tabularnewline
6-03 & Sampling strategy & \href{http://codes.wmo.int/common/wmdsSamplingStrategy}{http://codes.wmo.int/wmdr/SamplingStrategy}\tabularnewline
7-06 & Level of data & \href{http://codes.wmo.int/common/wmdsLevelOfData}{http://codes.wmo.int/wmdr/LevelOfData}\tabularnewline
7-07 & Data format & \href{http://codes.wmo.int/common/wmdsDataFormat}{http://codes.wmo.int/wmdr/DataFormat}\tabularnewline
7-10 & Reference time & \href{http://codes.wmo.int/common/wmdsReferenceTime}{http://codes.wmo.int/wmdr/ReferenceTime}\tabularnewline
8-03-01 & Quality flag (BUFR derived from CIMO guide) & \href{http://codes.wmo.int/common/wmdsQualityFlagCIMO}{http://codes.wmo.int/wmdr/QualityFlagCIMO}\tabularnewline
8-03-02 & Quality flag (From WaterML2) & \href{http://codes.wmo.int/common/wmdsQualityFlagOGC}{http://codes.wmo.int/wmdr/QualityFlagOGC}\tabularnewline
8-04 & Quality Flag System & \href{http://codes.wmo.int/common/wmdsQualityFlagSystem}{http://codes.wmo.int/wmdr/QualityFlagSystem}\tabularnewline
8-05 & Traceability & \href{http://codes.wmo.int/common/wmdsTraceability}{http://codes.wmo.int/wmdr/Traceability}\tabularnewline
9-02 & Data policy/use constraints & \href{http://codes.wmo.int/common/wmdsDataPolicy}{http://codes.wmo.int/wmdr/DataPolicy}\tabularnewline
11-01 & Coordinates source/service & \href{http://codes.wmo.int/common/wmdsGeopositioningMethod}{http://codes.wmo.int/wmdr/GeopositioningMethod}\tabularnewline
11-02 & Coordinates reference & \href{http://codes.wmo.int/common/wmdsCoordinateReferenceSystem}{http://codes.wmo.int/wmdr/CoordinateReferenceSystem}\tabularnewline
11-03 & Meaning of time stamp & \href{http://codes.wmo.int/common/wmdsTimeStampMeaning}{http://codes.wmo.int/wmdr/TimeStampMeaning}\tabularnewline
\bottomrule
\end{longtable}

Appendix A. Code Tables

CODE TABLE D-1: NIL REASONS

\emph{Nil-reason} terms are used to provide an explanation for recording a missing (or void) value within a data product. Terms are drawn from authorities in addition to WMO including ISO/TC~211 (from ISO~19136:2007, Geographic information -- Geography Markup Language, clause~8.2.3.1; published on behalf of ISO by the Open Geospatial Consortium). The \emph{code-space} indicates the authority under which the nil-reason terms are published. A URI for each nil-reason term is composed by appending the \emph{notation} to the \emph{code-space}. As an example, the URI of \emph{notDetectedByAutoSystem} is \url{http://codes.wmo.int/common/nil/notDetectedByAutoSystem}. The URI is also a URL providing additional information about the associated nil-reason term. This code table is published at \url{http://codes.wmo.int/common/nil}.

\begin{longtable}[]{@{}llll@{}}
\toprule
Label & Notation & Code-space & Description\tabularnewline
\midrule
\endhead
Above detection range & AboveDetectionRange & \url{http://www.opengis.net/def/nil/OGC/0/} & The value was above the detection range of the instrument used to estimate it.\tabularnewline
Below detection range & BelowDetectionRange & \url{http://www.opengis.net/def/nil/OGC/0/} & The value was below the detection range of the instrument used to estimate it.\tabularnewline
Inapplicable & inapplicable & \url{http://www.opengis.net/def/nil/OGC/0/} & There is no value.\tabularnewline
Missing & missing & \url{http://www.opengis.net/def/nil/OGC/0/} & The correct value is not readily available to the sender of these data. Furthermore, a correct value may not exist.\tabularnewline
No significant change (NOSIG) & noSignificantChange & \url{http://codes.wmo.int/common/nil/} & No significant change is expected to occur.\tabularnewline
Nothing detected by automated system & notDetectedByAutoSystem & \url{http://codes.wmo.int/common/nil/} & The automated observing system did not detect a value (for example, no cloud detected ``NCD'').\tabularnewline
Not observable & notObservable & \url{http://codes.wmo.int/common/nil/} & A system failure, sensor failure, or sensor obstruction prevented the intended observation of the value.\tabularnewline
Nothing of operational significance & nothingOfOperationalSignificance & \url{http://codes.wmo.int/common/nil/} & Nothing was observed or forecast of operational significance (for example, nil significant cloud ``NSC'', nil significant weather ``NSW'').\tabularnewline
Template & template & \url{http://www.opengis.net/def/nil/OGC/0/} & The value will be available later.\tabularnewline
Unknown & unknown & \url{http://www.opengis.net/def/nil/OGC/0/} & The correct value is not known to, and not computable by, the sender of the data. However, a correct value probably exists.\tabularnewline
Withheld & withheld & \url{http://www.opengis.net/def/nil/OGC/0/} & The value is not divulged.\tabularnewline
\bottomrule
\end{longtable}

CODE TABLE D-2: PHYSICAL QUANTITY KINDS

The uniform resource identifier (URI) of each physical quantity kind is composed by the prefix http://codes.wmo.int/common/quantity-kind/ and the notation. As an example, the URI of airTemperature is \url{http://codes.wmo.int/common/quantity-kind/airTemperature}. The URI can be used in the XML code format and is also a URL providing comprehensive information regarding the physical quantity kind.

Meteorological quantities

\begin{longtable}[]{@{}llll@{}}
\toprule
Label & Notation & Description & Dimensions\tabularnewline
\midrule
\endhead
Air temperature & \href{http://codes.wmo.int/common/quantity-kind/airTemperature}{airTemperature} & The temperature indicated by a thermometer exposed to the air in a place sheltered from direct solar radiation. & Θ\tabularnewline
Atmospheric pressure & \href{http://codes.wmo.int/common/quantity-kind/atmosphericPressure}{atmosphericPressure} & The atmospheric pressure on a given surface is the force per unit area exerted by virtue of the weight of the atmosphere above. The pressure is thus equal to the weight of a vertical column of air above a horizontal projection of the surface, extending to the outer limit of the atmosphere. & ML\textsuperscript{--1}T\textsuperscript{--2}\tabularnewline
Dewpoint temperature & \href{http://codes.wmo.int/common/quantity-kind/dewPointTemperature}{dewPointTemperature} & The temperature to which a given air parcel must be cooled at constant pressure and constant water vapour content in order for saturation to occur. & Θ\tabularnewline
Height of base of cloud & \href{http://codes.wmo.int/common/quantity-kind/heightOfBaseOfCloud}{heightOfBaseOfCloud} & For a given cloud or cloud layer, vertical distance (measured from local ground surface) of the lowest level in the atmosphere at which the air contains a perceptible quantity of cloud particles. & L\tabularnewline
Horizontal visibility & \href{http://codes.wmo.int/common/quantity-kind/horizontalVisibility}{horizontalVisibility} & The greatest distance determined in the horizontal plane at the ground surface that prominent objects can be seen and identified by unaided, normal eyes. & L\tabularnewline
Maximum wind gust speed & \href{http://codes.wmo.int/common/quantity-kind/maximumWindGustSpeed}{maximumWindGustSpeed} & Nominal maximum speed of wind during a given period; usually determined as a mean wind speed over a short duration (for example, 1~minute) within a longer period (for example, 10~minutes). & LT\textsuperscript{--1}\tabularnewline
Sea-surface temperature & \href{http://codes.wmo.int/common/quantity-kind/seaSurfaceTemperature}{seaSurfaceTemperature} & Temperature of the sea water at surface. & Θ\tabularnewline
Vertical visibility & \href{http://codes.wmo.int/common/quantity-kind/verticalVisibility}{verticalVisibility} & Maximum distance at which an observer can see and identify an object on the same vertical as himself or herself, above or below. & L\tabularnewline
\bottomrule
\end{longtable}

Oceanographic quantities

\begin{longtable}[]{@{}llll@{}}
\toprule
Label & Notation & Description & Dimensions\tabularnewline
\midrule
\endhead
Sea-surface temperature & \href{http://codes.wmo.int/common/quantity-kind/seaSurfaceTemperature}{seaSurfaceTemperature} & Temperature of the sea water at surface. & Θ\tabularnewline
\bottomrule
\end{longtable}

Aeronautical quantities

\begin{longtable}[]{@{}llll@{}}
\toprule
Label & Notation & Description & Dimensions\tabularnewline
\midrule
\endhead
Aerodrome maximum wind gust speed & \href{http://codes.wmo.int/common/quantity-kind/aerodromeMaximumWindGustSpeed}{aerodromeMaximumWindGustSpeed} & Maximum wind speed in the 10-minute period of observation. It is reported only if it exceeds the mean speed by 5~m~s\textsuperscript{--1} (10~knots). & LT\textsuperscript{--1}\tabularnewline
Aerodrome mean wind direction & \href{http://codes.wmo.int/common/quantity-kind/aerodromeMeanWindDirection}{aerodromeMeanWindDirection} & The mean true direction in degrees from which the wind is blowing over the 10-minute period immediately preceding the observation. When the 10-minute period includes a marked discontinuity in the wind characteristics (see Note), only data after the discontinuity shall be used for mean wind direction and variations of the wind direction, hence the time interval in these circumstances shall be correspondingly reduced. & dimensionless\tabularnewline
Aerodrome mean wind speed & \href{http://codes.wmo.int/common/quantity-kind/aerodromeMeanWindSpeed}{aerodromeMeanWindSpeed} & The mean speed of the wind over the 10-minute period immediately preceding the observation. When the 10-minute period includes a marked discontinuity in the wind characteristics (see Note), only data after the discontinuity shall be used for obtaining mean wind speed, hence the time interval in these circumstances shall be correspondingly reduced. & LT\textsuperscript{--1}\tabularnewline
Aerodrome minimum horizontal visibility & \href{http://codes.wmo.int/common/quantity-kind/aerodromeMinimumHorizontalVisibility}{aerodromeMinimumHorizontalVisibility} & The minimum horizontal visibility that is reported when the horizontal visibility is not the same in different directions and when the minimum visibility is different from the prevailing visibility, and less than 1~500~metres or less than 50\% of the prevailing visibility, and less than 5~000~metres. & L\tabularnewline
Aerodrome minimum visibility direction & \href{http://codes.wmo.int/common/quantity-kind/aerodromeMinimumVisibilityDirection}{aerodromeMinimumVisibilityDirection} & When the minimum horizontal visibility is reported, its general direction in relation to the aerodrome reference point has to be reported and indicated by reference to one of the eight points of the compass. If the minimum visibility is observed in more than one direction, the Dv shall represent the most operationally significant direction. & dimensionless\tabularnewline
Aeronautical prevailing horizontal visibility & \href{http://codes.wmo.int/common/quantity-kind/aeronauticalPrevailingHorizontalVisibility}{aeronauticalPrevailingHorizontalVisibility} & The greatest visibility value, observed in accordance with the definition of visibility, which is reached within at least half the horizon circle or within at least half of the surface of the aerodrome. These areas could comprise contiguous or non-contiguous sectors. & L\tabularnewline
\begin{minipage}[t]{0.22\columnwidth}\raggedright
Aeronautical visibility\strut
\end{minipage} & \begin{minipage}[t]{0.22\columnwidth}\raggedright
\href{http://codes.wmo.int/common/quantity-kind/aeronauticalVisibility}{aeronauticalVisibility}\strut
\end{minipage} & \begin{minipage}[t]{0.22\columnwidth}\raggedright
The greater of:

(a) The greatest distance at which a black object of suitable dimensions, situated near the ground, can be seen and recognized when observed against a bright background;

(b) The greatest distance at which lights in the vicinity of 1~000~candelas can be seen and identified against an unlit background.\strut
\end{minipage} & \begin{minipage}[t]{0.22\columnwidth}\raggedright
L\strut
\end{minipage}\tabularnewline
Altimeter setting (QNH) & \href{http://codes.wmo.int/common/quantity-kind/altimeterSettingQnh}{altimeterSettingQnh} & Altimeter setting (also known as QNH) is defined as barometric pressure adjusted to sea level. It is a pressure setting used by pilots, air traffic control (ATC) and low frequency weather beacons to refer to the barometric setting which, when set on an aircraft's altimeter, will cause the altimeter to read altitude above mean sea level within a certain defined region. & ML\textsuperscript{--1}T\textsuperscript{--2}\tabularnewline
Depth of runway deposit & \href{http://codes.wmo.int/common/quantity-kind/depthOfRunwayDeposit}{depthOfRunwayDeposit} & Depth of deposit on surface of runway. & L\tabularnewline
\begin{minipage}[t]{0.22\columnwidth}\raggedright
Runway contamination coverage\strut
\end{minipage} & \begin{minipage}[t]{0.22\columnwidth}\raggedright
\href{http://codes.wmo.int/common/quantity-kind/runwayContaminationCoverage}{runwayContaminationCoverage}\strut
\end{minipage} & \begin{minipage}[t]{0.22\columnwidth}\raggedright
Proportion of runway that is contaminated. A runway is considered to be contaminated when more than 25\% of the runway surface area (whether in isolated areas or not) within the required length and width being used is covered by the following:

(a) Surface water more than 3~mm deep, or by slush or loose snow equivalent to more than 3~mm of water;

(b) Snow which has been compressed into a solid mass which resists further compression and will hold together or break into lumps if picked up (compacted snow); or

(c) Ice, including wet ice.\strut
\end{minipage} & \begin{minipage}[t]{0.22\columnwidth}\raggedright
dimensionless\strut
\end{minipage}\tabularnewline
Runway friction coefficient & \href{http://codes.wmo.int/common/quantity-kind/runwayFrictionCoefficient}{runwayFrictionCoefficient} & Quantitative assessment of friction coefficient of runway surface. & dimensionless\tabularnewline
Runway visual range (RVR) & \href{http://codes.wmo.int/common/quantity-kind/runwayVisualRangeRvr}{runwayVisualRangeRvr} & The range over which the pilot of an aircraft on the centre line of a runway can see the runway surface markings or the lights delineating the runway or identifying its centre line. & L\tabularnewline
\bottomrule
\end{longtable}

Note: A marked discontinuity occurs when there is an abrupt and sustained change in wind direction of 30° or more, with a wind speed of 5~m~s\textsuperscript{--1} (10~KT) or more before or after the change, or a change in wind speed of 5~m~s\textsuperscript{--1} (10~KT) or more, lasting at least two minutes.

CODE TABLE: FM~201 -- COLLECT

There are no code tables that are specific to FM~201.

CODE TABLE: FM~202 -- METCE

CODE TABLE D-3: METCE OBSERVATION TYPES

The items within this code table are specialized observation or measurement types defined within the \emph{Modèle pour l'échange des informations sur le temps, le climat et l'eau} (METCE). Each observation or measurement type listed herein is specified as a class in METCE that derives from the OM\_Observation class (defined in ISO 19156, Geographic information -- Observations and measurements, clause~6.2) or subclass thereof. A URI for each observation type is composed by appending the \emph{notation} to the \emph{code-space}. As an example, the URI of ComplexSamplingMeasurement is \url{http://codes.wmo.int/common/observation-type/METCE/2013/ComplexSamplingMeasurement}. The URI is also a URL providing additional information about the associated observation type. This code table is published at \url{http://codes.wmo.int/common/observation-type/METCE/2013}.

\begin{longtable}[]{@{}llll@{}}
\toprule
Label & Notation & Code-space & Description\tabularnewline
\midrule
\endhead
\vtop{\hbox{\strut Complex sampling measurement}\hbox{\strut (Deprecated in}\hbox{\strut FM 202-16)}} & ComplexSamplingMeasurement & \href{http://codes.wmo.int/common/observation-type/METCE/2013/}{http://codes.wmo.int/common/ observation-type/METCE/2013/} & ComplexSamplingMeasurement (a subclass of OM\_ComplexObservation) is intended for use where the observation event is concerned with the evaluation of multiple measurands at a specified location and time instant or duration. The result of this observation type shall refer to an entity of metatype Record (from ISO~19103). ComplexSamplingMeasurement enforces the following additional constraints: ``featureOfInterest'' shall refer to an entity of type SF\_SpatialSamplingFeature (from ISO~19156), or subclass thereof; and ``procedure'' shall refer to an entity of type Process (from METCE), or subclass thereof. The OM\_ComplexObservation is used because the ``result'' of this class of observations is a group of measures, provided as a record. Again, this matches the WMO application domain wherein multiple phenomena are measured within a single ``observation event''. The term measurement is used in the name in an attempt to reduce confusion arising from the overloading of the term observation.\tabularnewline
\vtop{\hbox{\strut Sampling coverage measurement}\hbox{\strut (Deprecated in}\hbox{\strut FM 202-16)}} & SamplingCoverageMeasurement & \href{http://codes.wmo.int/common/observation-type/METCE/2013/}{http://codes.wmo.int/common/ observation-type/METCE/2013/} & SamplingCoverageMeasurement (a subclass of OM\_DiscreteCoverageObservation) is intended for use where the observation is concerned with the evaluation of measurands that vary with respect to space and/or time -- the result of this observation type shall refer to an entity of type CV\_DiscreteCoverage (from ISO~19123). SamplingCoverageMeasurement enforces the following additional constraints: ``featureOfInterest'' shall refer to an entity of type SF\_SpatialSamplingFeature (from ISO~19156), or subclass thereof; and ``procedure'' shall refer to an entity of type Process (from METCE), or subclass thereof. The class ``SamplingCoverageMeasurement'' is based on the SamplingCoverageObservation which is defined in an informative annex of ISO~19156. The term measurement is used in the name in an attempt to reduce confusion arising from the overloading of the term observation.\tabularnewline
\vtop{\hbox{\strut Sampling observation}\hbox{\strut (Deprecated in}\hbox{\strut FM~202-16)}} & SamplingObservation & \href{http://codes.wmo.int/common/observation-type/METCE/2013/}{http://codes.wmo.int/common/ observation-type/METCE/2013/} & SamplingObservation (a subclass of OM\_Observation) provides a general-purpose observation type. It enforces the following additional constraints: ``featureOfInterest'' shall refer to an entity of type SF\_SpatialSamplingFeature (from ISO~19156), or subclass thereof; and ``procedure'' shall refer to an entity of type Process (from METCE), or subclass thereof. SamplingObservation is intended for use where measurement of physical phenomena is not the goal of the procedure. For example, the procedure executed to define SIGMET reports results in the identification of areas of turbulence, icing or other meteorological phenomena.\tabularnewline
\begin{minipage}[t]{0.22\columnwidth}\raggedright
Category observation

(FM 202-16 onwards)\strut
\end{minipage} & \begin{minipage}[t]{0.22\columnwidth}\raggedright
OM\_CategoryObservation\strut
\end{minipage} & \begin{minipage}[t]{0.22\columnwidth}\raggedright
\href{http://www.opengis.net/def/observationType/OGC-OM/2.0/}{http://www.opengis.net/def/ observationType/OGC-OM/2.0/}\strut
\end{minipage} & \begin{minipage}[t]{0.22\columnwidth}\raggedright
Observation whose result is a ScopedName (as defined in ISO~19156:2011, clause~7.2.2).

For example: A category observation of the ``taxon'' (property-type) of ``specimen~123'' (feature-of-interest) by ``Amy Bachrach'' (procedure) had the result ``Eucalyptus caesia'' (from the \emph{Flora of Australia}).\strut
\end{minipage}\tabularnewline
\begin{minipage}[t]{0.22\columnwidth}\raggedright
Complex observation

(FM 202-16 onwards)\strut
\end{minipage} & \begin{minipage}[t]{0.22\columnwidth}\raggedright
OM\_ComplexObservation\strut
\end{minipage} & \begin{minipage}[t]{0.22\columnwidth}\raggedright
\href{http://www.opengis.net/def/observationType/OGC-OM/2.0/}{http://www.opengis.net/def/ observationType/OGC-OM/2.0/}\strut
\end{minipage} & \begin{minipage}[t]{0.22\columnwidth}\raggedright
Observation whose result is a Record (as defined in ISO~19156:2011, clause~7.2.2).

For example: A complex observation of ``major element composition'' (property-type) for ``specimen h8j'' (feature-of-interest) using the ``ICPMS'' (procedure) had the result ``(\ldots{} array of element proportions \ldots)''.

OM\_ComplexObservation is intended for use where the observation event is concerned with the evaluation of multiple properties at a specified location and time instant or duration. The `result' of this class of observations is a group of values, provided as a Record (as defined in ISO~19103:2005 Geographic information -- Conceptual schema language).\strut
\end{minipage}\tabularnewline
\begin{minipage}[t]{0.22\columnwidth}\raggedright
Count observation

(FM 202-16 onwards)\strut
\end{minipage} & \begin{minipage}[t]{0.22\columnwidth}\raggedright
OM\_CountObservation\strut
\end{minipage} & \begin{minipage}[t]{0.22\columnwidth}\raggedright
\href{http://www.opengis.net/def/observationType/OGC-OM/2.0/}{http://www.opengis.net/def/ observationType/OGC-OM/2.0/}\strut
\end{minipage} & \begin{minipage}[t]{0.22\columnwidth}\raggedright
Observation whose result is an Integer (as defined in ISO~19156:2011, clause~7.2.2).

For example: A count observation of ``the number of votes cast'' (property-type) at ``the municipal election'' (feature-of-interest) using ``electronic voting machine tally'' (procedure) had the result ``3542''.\strut
\end{minipage}\tabularnewline
\begin{minipage}[t]{0.22\columnwidth}\raggedright
Discrete coverage observation

(FM 202-16 onwards)\strut
\end{minipage} & \begin{minipage}[t]{0.22\columnwidth}\raggedright
OM\_DiscreteCoverageObservation\strut
\end{minipage} & \begin{minipage}[t]{0.22\columnwidth}\raggedright
\href{http://www.opengis.net/def/observationType/OGC-OM/2.0/}{http://www.opengis.net/def/ observationType/OGC-OM/2.0/}\strut
\end{minipage} & \begin{minipage}[t]{0.22\columnwidth}\raggedright
Observation whose result is a DiscreteCoverage (as defined in ISO 19156:2011, clause 7.2.2).

Examples include:

(i) The colour of a scene varies with position. The result of an observation of the property ``colour'' of the scene is a coverage. Each domain element is a pixel whose index allows the spatial location within the scene to be obtained.

(ii) Temperature may be sampled using an array of weather stations. The temperature field of the region covered by the array may be represented as a discrete point coverage, whose domain-elements correspond to the station locations.

OM\_DiscreteCoverageObservation is intended for use where the observation event is concerned with the evaluation of properties that vary with respect to space and/or time. The `result' of this class of observations is a discrete coverage (as defined in ISO 19123:2005 Geographic information -- Schema for coverage geometry and functions).\strut
\end{minipage}\tabularnewline
\begin{minipage}[t]{0.22\columnwidth}\raggedright
Geometry observation

(FM 202-16 onwards)\strut
\end{minipage} & \begin{minipage}[t]{0.22\columnwidth}\raggedright
OM\_GeometryObservation\strut
\end{minipage} & \begin{minipage}[t]{0.22\columnwidth}\raggedright
\href{http://www.opengis.net/def/observationType/OGC-OM/2.0/}{http://www.opengis.net/def/ observationType/OGC-OM/2.0/}\strut
\end{minipage} & \begin{minipage}[t]{0.22\columnwidth}\raggedright
Observation whose result is a geometry (as defined in ISO~19156:2011, clause 7.2.2).

For example: A geometry observation of ``perimeter'' (property-type) for ``plot 987'' (feature-of-interest) using ``field survey GHJ'' (procedure) had the result\\
``(\ldots{} description of polygon \ldots)''.\strut
\end{minipage}\tabularnewline
\begin{minipage}[t]{0.22\columnwidth}\raggedright
Measurement

(FM 202-16 onwards)\strut
\end{minipage} & \begin{minipage}[t]{0.22\columnwidth}\raggedright
OM\_Measurement\strut
\end{minipage} & \begin{minipage}[t]{0.22\columnwidth}\raggedright
\href{http://www.opengis.net/def/observationType/OGC-OM/2.0/}{http://www.opengis.net/def/ observationType/OGC-OM/2.0/}\strut
\end{minipage} & \begin{minipage}[t]{0.22\columnwidth}\raggedright
Observation whose result is a scaled number or Measure (as defined in ISO 19156:2011, clause 7.2.2).

For example: A measurement of ``mass'' (property-type) of ``the seventh banana'' (feature-of-interest) using ``kitchen scales'' (procedure) had the result ``150g''.\strut
\end{minipage}\tabularnewline
\begin{minipage}[t]{0.22\columnwidth}\raggedright
Observation

(FM 202-16 onwards)\strut
\end{minipage} & \begin{minipage}[t]{0.22\columnwidth}\raggedright
OM\_Observation\strut
\end{minipage} & \begin{minipage}[t]{0.22\columnwidth}\raggedright
\href{http://www.opengis.net/def/observationType/OGC-OM/2.0/}{http://www.opengis.net/def/ observationType/OGC-OM/2.0/}\strut
\end{minipage} & \begin{minipage}[t]{0.22\columnwidth}\raggedright
Observation whose result type is unconstrained (as defined in ISO 19156:2011, clause 7).

An observation is an act that results in the estimation of the value of a feature property, and involves application of a specified procedure, such as a sensor, instrument, algorithm or process chain. The procedure may be applied in situ, remotely, or ex situ with respect to the sampling location. Use of a common model allows observation data using different procedures to be combined unambiguously.

Observation details are also important for data discovery and for data quality estimation. Observation feature types are defined by the properties that support these applications.\strut
\end{minipage}\tabularnewline
\begin{minipage}[t]{0.22\columnwidth}\raggedright
Temporal observation

(FM 202-16 onwards)\strut
\end{minipage} & \begin{minipage}[t]{0.22\columnwidth}\raggedright
OM\_TemporalObservation\strut
\end{minipage} & \begin{minipage}[t]{0.22\columnwidth}\raggedright
\href{http://www.opengis.net/def/observationType/OGC-OM/2.0/}{http://www.opengis.net/def/ observationType/OGC-OM/2.0/}\strut
\end{minipage} & \begin{minipage}[t]{0.22\columnwidth}\raggedright
Observation whose result is a TM\_Object (as defined in ISO 19156:2011, clause 7.2.2).

For example: A temporal observation of ``duration'' (property-type) for ``Usain Bolt 100m dash'' (feature-of-interest) using ``stopwatch'' (procedure) had the result ``9.6s''.\strut
\end{minipage}\tabularnewline
\begin{minipage}[t]{0.22\columnwidth}\raggedright
Truth observation

(FM 202-16 onwards)\strut
\end{minipage} & \begin{minipage}[t]{0.22\columnwidth}\raggedright
OM\_TruthObservation\strut
\end{minipage} & \begin{minipage}[t]{0.22\columnwidth}\raggedright
\href{http://www.opengis.net/def/observationType/OGC-OM/2.0/}{http://www.opengis.net/def/ observationType/OGC-OM/2.0/}\strut
\end{minipage} & \begin{minipage}[t]{0.22\columnwidth}\raggedright
Observation whose result is a Boolean (as defined in ISO~19156:2011, clause 7.2.2).

For example: A truth observation of ``presence'' (property-type) of ``intruder'' (feature-of-interest) using ``CCTV'' (procedure) had the result ``false''.\strut
\end{minipage}\tabularnewline
\bottomrule
\end{longtable}

CODE TABLE: FM~203 -- OPM

There are no code tables that are specific to FM~203.

CODE TABLE: FM~204 -- SAF

There are no code tables that are specific to FM~204.

CODE TABLE: FM~205 -- IWXXM

CODE TABLE D-4: FM 205-15 EXT. -- IWXXM OBSERVATION TYPES

The items within this code table are specialized observation or measurement types defined within the ICAO Meteorological Information Exchange Model (IWXXM). Each observation or measurement type listed herein is specified as a class in IWXXM that derives from the OM\_Observation class (defined in ISO~19156, Geographic information -- Observations and measurements, clause~6.2) or subclass thereof. A URI for each observation type is composed by appending the \emph{notation} to the \emph{code-space}. As an example, the URI of MeteorologicalAerodromeForecast is \url{http://codes.wmo.int/49-2/observation-type/IWXXM/1.0/MeteorologicalAerodromeForecast}. The URI is also a URL providing additional information about the associated observation type. This code table is published at \url{http://codes.wmo.int/49-2/observation-type/IWXXM/1.0}.

\begin{longtable}[]{@{}llll@{}}
\toprule
Label & Notation & Code-space & Description\tabularnewline
\midrule
\endhead
Meteorological aerodrome forecast & MeteorologicalAerodromeForecast & \href{http://codes.wmo.int/49-2/observation-type/IWXXM/1.0/}{http://codes.wmo.int/49-2/ observation-type/IWXXM/1.0/} & MeteorologicalAerodromeForecast (a subclass of ComplexSamplingMeasurement from METCE) is intended for use when reporting an aggregate set of forecast meteorological conditions at an aerodrome. The result of this observation type shall refer to an entity of type MeteorologicalAerodromeForecastRecord. MeteorologicalAerodromeForecast enforces the following additional constraints: ``featureOfInterest'' shall refer to an entity of type SF\_SamplingPoint and the associated ``sampledFeature'' must be an aerodrome. This class is also related but not identical to MeteorologicalAerodromeTrendForecast which is reported on a METAR/SPECI -- conditions reported in trend forecasts in METAR/SPECI differ from forecast groups in a TAF. The TAF forecast group from/to variants (FM, TL, AT, etc.) are represented on the OM\_Observation validTime, which is always an instance of TM\_Period. When there is only an instant at which a condition occurs, the start and end times are the same.\tabularnewline
Meteorological aerodrome observation & MeteorologicalAerodromeObservation & \href{http://codes.wmo.int/49-2/observation-type/IWXXM/1.0/}{http://codes.wmo.int/49-2/ observation-type/IWXXM/1.0/} & MeteorologicalAerodromeObservation (a subclass of ComplexSamplingMeasurement from METCE) is intended for use when reporting an aggregate set of observed meteorological conditions at an aerodrome. The result of this observation type shall refer to an entity of type MeteorologicalAerodromeObservationRecord. MeteorologicalAerodromeObservation enforces the following additional constraints: ``featureOfInterest'' shall refer to an entity of type SF\_SamplingPoint and the associated ``sampledFeature'' must be an aerodrome. MeteorologicalAerodromeObservation has a peer class for forecast information at an aerodrome: MeteorologicalAerodromeTrendForecast.\tabularnewline
Meteorological aerodrome trend forecast & MeteorologicalAerodromeTrendForecast & \href{http://codes.wmo.int/49-2/observation-type/IWXXM/1.0/}{http://codes.wmo.int/49-2/ observation-type/IWXXM/1.0/} & MeteorologicalAerodromeTrendForecast (a subclass of ComplexSamplingMeasurement from METCE) is intended for use when reporting an aggregate set of forecast meteorological conditions at an aerodrome. The result of this observation type shall refer to an entity of type MeteorologicalAerodromeTrendForecastRecord. MeteorologicalAerodromeTrendForecast enforces the following additional constraints: ``featureOfInterest'' shall refer to an entity of type SF\_SamplingPoint and the associated ``sampledFeature'' must be an aerodrome. MeteorologicalAerodromeTrendForecasts are reported in surface observation reports such as SPECI and METAR. MeteorologicalAerodromeTrendForecast has a peer class for observation information at an aerodrome (MeteorologicalAerodromeObservation), which is also reported on a METAR and SPECI for observed phenomena. This class is also related but not identical to MeteorologicalAerodromeForecast which is reported on a TAF -- conditions reported in trend forecasts in METAR/SPECI differ from forecast groups in a TAF.\tabularnewline
SIGMET evolving condition analysis & SIGMETEvolvingConditionAnalysis & \href{http://codes.wmo.int/49-2/observation-type/IWXXM/1.0/}{http://codes.wmo.int/49-2/ observation-type/IWXXM/1.0/} & SIGMETEvolvingConditionAnalysis (a subclass of SamplingObservation from METCE) is intended for use when reporting an observed or forecast aggregate set of meteorological conditions hazardous to flight over a large airspace, including anticipated intensity change plus speed and direction of motion. The result of this observation type shall refer to a single EvolvingMeteorologicalCondition which represents a SIGMET observation or forecast of meteorological conditions. SIGMETEvolvingConditionAnalysis enforces the following additional constraints: ``featureOfInterest'' shall refer to an entity of type SF\_SamplingSurface and the associated ``sampledFeature'' must be an airspace.\tabularnewline
SIMGET position analysis & SIGMETPositionAnalysis & \href{http://codes.wmo.int/49-2/observation-type/IWXXM/1.0/}{http://codes.wmo.int/49-2/ observation-type/IWXXM/1.0/} & SIGMETPositionAnalysis (a subclass of SamplingObservation from METCE) is intended for use when reporting the forecast position of meteorological conditions hazardous to flight. The result of this observation type shall refer to one or more MeteorologicalPositions which represents the forecast positions of SIGMET phenomena. SIGMETPositionAnalysis enforces the following additional constraints: ``featureOfInterest'' shall refer to an entity of type SF\_SamplingSurface and the associated ``sampledFeature''must be an airspace.\tabularnewline
\bottomrule
\end{longtable}

CODE TABLE D-4: FM 205-16 -- IWXXM OBSERVATION TYPES

The items within this code table are specialized observation or measurement types defined within the ICAO Meteorological Information Exchange Model (IWXXM). Each observation or measurement type listed herein is specified as a class in IWXXM that derives from the OM\_Observation class (defined in ISO~19156, Geographic information -- Observations and measurements, clause~6.2) or subclass thereof. A URI for each observation type is composed by appending the \emph{notation} to the \emph{code-space}. As an example, the URI of MeteorologicalAerodromeForecast is \href{http://codes.wmo.int/49-2/observation-type/IWXXM/1.0/MeteorologicalAerodromeForecast}{http://codes.wmo.int/49-2/observation-type/IWXXM/2.1/MeteorologicalAerodromeForecast}. The URI is also a URL providing additional information about the associated observation type. This code table is published at \href{http://codes.wmo.int/49-2/observation-type/IWXXM/1.0}{http://codes.wmo.int/49-2/observation-type/IWXXM/2.1}.

\begin{longtable}[]{@{}llll@{}}
\toprule
Label & Notation & Code-space & Description\tabularnewline
\midrule
\endhead
Meteorological aerodrome forecast & MeteorologicalAerodromeForecast & \href{http://codes.wmo.int/49-2/observation-type/IWXXM/1.0/}{http://codes.wmo.int/49-2/observation-type/IWXXM/2.1/} & MeteorologicalAerodromeForecast (a subclass of ComplexSamplingMeasurement from METCE) is intended for use when reporting an aggregate set of forecast meteorological conditions at an aerodrome. The result of this observation type shall refer to an entity of type MeteorologicalAerodromeForecastRecord. MeteorologicalAerodromeForecast enforces the following additional constraints: ``featureOfInterest'' shall refer to an entity of type SF\_SamplingPoint and the associated ``sampledFeature'' must be an aerodrome. This class is also related but not identical to MeteorologicalAerodromeTrendForecast which is reported on a METAR/SPECI -- conditions reported in trend forecasts in METAR/SPECI differ from forecast groups in a TAF. The TAF forecast group from/to variants (FM, TL, AT, etc.) are represented on the OM\_Observation validTime, which is always an instance of TM\_Period. When there is only an instant at which a condition occurs, the start and end times are the same.\tabularnewline
Meteorological aerodrome observation & MeteorologicalAerodromeObservation & \href{http://codes.wmo.int/49-2/observation-type/IWXXM/1.0/}{http://codes.wmo.int/49-2/observation-type/IWXXM/2.1/} & MeteorologicalAerodromeObservation (a subclass of ComplexSamplingMeasurement from METCE) is intended for use when reporting an aggregate set of observed meteorological conditions at an aerodrome. The result of this observation type shall refer to an entity of type MeteorologicalAerodromeObservationRecord. MeteorologicalAerodromeObservation enforces the following additional constraints: ``featureOfInterest'' shall refer to an entity of type SF\_SamplingPoint and the associated ``sampledFeature'' must be an aerodrome. MeteorologicalAerodromeObservation has a peer class for forecast information at an aerodrome: MeteorologicalAerodromeTrendForecast.\tabularnewline
Meteorological aerodrome trend forecast & MeteorologicalAerodromeTrendForecast & \href{http://codes.wmo.int/49-2/observation-type/IWXXM/1.0/}{http://codes.wmo.int/49-2/observation-type/IWXXM/2.1/} & MeteorologicalAerodromeTrendForecast (a subclass of ComplexSamplingMeasurement from METCE) is intended for use when reporting an aggregate set of forecast meteorological conditions at an aerodrome. The result of this observation type shall refer to an entity of type MeteorologicalAerodromeTrendForecastRecord. MeteorologicalAerodromeTrendForecast enforces the following additional constraints: ``featureOfInterest'' shall refer to an entity of type SF\_SamplingPoint and the associated ``sampledFeature'' must be an aerodrome. MeteorologicalAerodromeTrendForecasts are reported in surface observation reports such as SPECI and METAR. MeteorologicalAerodromeTrendForecast has a peer class for observation information at an aerodrome (MeteorologicalAerodromeObservation), which is also reported on a METAR and SPECI for observed phenomena. This class is also related but not identical to MeteorologicalAerodromeForecast which is reported on a TAF -- conditions reported in trend forecasts in METAR/SPECI differ from forecast groups in a TAF.\tabularnewline
SIGMET evolving condition analysis & SIGMETEvolvingConditionAnalysis & \href{http://codes.wmo.int/49-2/observation-type/IWXXM/1.0/}{http://codes.wmo.int/49-2/observation-type/IWXXM/2.1/} & \vtop{\hbox{\strut SIGMETEvolvingConditionAnalysis (a subclass of SamplingObservation from METCE) is intended for use when reporting an observed or forecast aggregate set of meteorological conditions hazardous to flight over a large airspace, including anticipated intensity change plus speed and direction of motion. The result of this observation type shall refer to a single EvolvingMeteorologicalCondition which represents a SIGMET observation or forecast of meteorological conditions. SIGMETEvolvingConditionAnalysis enforces the following additional constraints: ``featureOfInterest'' shall refer to an entity of type}\hbox{\strut SF\_SamplingSurface and the associated ``sampledFeature'' must be an airspace.}}\tabularnewline
SIMGET position analysis & SIGMETPositionAnalysis & \href{http://codes.wmo.int/49-2/observation-type/IWXXM/1.0/}{http://codes.wmo.int/49-2/observation-type/IWXXM/2.1/} & SIGMETPositionAnalysis (a subclass of SamplingObservation from METCE) is intended for use when reporting the forecast position of meteorological conditions hazardous to flight. The result of this observation type shall refer to one or more~MeteorologicalPositions which represents the forecast positions of SIGMET phenomena. SIGMETPositionAnalysis enforces the following additional constraints: ``featureOfInterest'' shall refer to an entity of type SF\_SamplingSurface and the associated ``sampledFeature'' must be an airspace.\tabularnewline
\bottomrule
\end{longtable}

CODE TABLE D-5: IWXXM OBSERVABLE PROPERTIES

The items within this code table are composite observable properties that define the set of physical properties evaluated as a result of regulated procedures specified in the \emph{Technical Regulations} (WMO-No.~49), Volume~II -- Meteorological Service for International Air Navigation. These include aerodrome observation and forecast reports (for example, METAR, SPECI and TAF). A URI for each observable property is composed by appending the \emph{notation} to the \emph{code-space}. As an example, the URI of MeteorologicalAerodromeForecast is \url{http://codes.wmo.int/49-2/observable-property/MeteorologicalAerodromeForecast}. The URI is also a URL providing additional information about the associated observable property. This code table is published at \url{http://codes.wmo.int/49-2/observable-property}.

\begin{longtable}[]{@{}llll@{}}
\toprule
Label & Notation & Code-space & Description\tabularnewline
\midrule
\endhead
Meteorological aerodrome forecast & MeteorologicalAerodromeForecast & \url{http://codes.wmo.int/49-2/observable-property/} & The set of physical properties evaluated as a result of an aerodrome forecast (TAF), as specified in the \emph{Technical Regulations} (WMO-No.~49), Volume~II -- Meteorological Service for International Air Navigation.\tabularnewline
Meteorological aerodrome observation & MeteorologicalAerodromeObservation & \url{http://codes.wmo.int/49-2/observable-property/} & The set of physical properties evaluated as a result of the observation procedure of a routine or special aerodrome meteorological report (METAR or SPECI), as specified in the \emph{Technical Regulations} (WMO-No.~49), Volume~II -- Meteorological Service for International Air Navigation.\tabularnewline
Meteorological aerodrome trend forecast & MeteorologicalAerodromeTrendForecast & \url{http://codes.wmo.int/49-2/observable-property/} & The set of physical properties evaluated as a result of the trend forecast procedure of a routine or special aerodrome meteorological report (METAR or SPECI), as specified in the \emph{Technical Regulations} (WMO-No.~49), Volume~II -- Meteorological Service for International Air Navigation.\tabularnewline
\bottomrule
\end{longtable}

CODE TABLE D-6: AERODROME RECENT WEATHER

The items within this code table are the weather types that may be reported within a meteorological aerodrome observation report that have occurred during the period since the last issued routine report or last hour, whichever is shorter, but are not observed at the time of the observation. Requirements for reporting recent weather at an aerodrome are specified in the \emph{Technical Regulations} (WMO-No.~49), Volume~II, Part~II, Appendix~3, 4.8.1.1.

This code table contains the set of weather types that are permitted for reporting recent weather. These are a subset of the enumerated set of meteorologically valid combinations specified in Volume~I.1, Code table~4678 comprising the following elements: intensity or proximity; descriptor; precipitation; obscuration; and/or other.

Each weather type is uniquely identified using a URI. The URI is also a URL providing additional information about the associated weather type. This code table is published at \url{http://codes.wmo.int/49-2/AerodromeRecentWeather}.

\begin{longtable}[]{@{}lll@{}}
\toprule
Label & Notation & URI\tabularnewline
\midrule
\endhead
Blowing snow & REBLSN & \url{http://codes.wmo.int/306/4678/BLSN}\tabularnewline
Duststorm & REDS & \url{http://codes.wmo.int/306/4678/DS}\tabularnewline
Precipitation of drizzle & REDZ & \url{http://codes.wmo.int/306/4678/DZ}\tabularnewline
Funnel cloud(s) (tornado or waterspout) & REFC & \url{http://codes.wmo.int/306/4678/FC}\tabularnewline
Precipitation of freezing drizzle & REFZDZ & \url{http://codes.wmo.int/306/4678/FZDZ}\tabularnewline
Precipitation of freezing rain & REFZRA & \url{http://codes.wmo.int/306/4678/FZRA}\tabularnewline
Unidentified freezing precipitation & REFZUP & \url{http://codes.wmo.int/306/4678/FZUP}\tabularnewline
Precipitation of ice pellets & REPL & \url{http://codes.wmo.int/306/4678/PL}\tabularnewline
Precipitation of rain & RERA & \url{http://codes.wmo.int/306/4678/RA}\tabularnewline
Precipitation of snow grains & RESG & \url{http://codes.wmo.int/306/4678/SG}\tabularnewline
Showery precipitation of hail & RESHGR & \url{http://codes.wmo.int/306/4678/SHGR}\tabularnewline
Showery precipitation of snow pellets/small hail & RESHGS & \url{http://codes.wmo.int/306/4678/SHGS}\tabularnewline
Showery precipitation of rain & RESHRA & \url{http://codes.wmo.int/306/4678/SHRA}\tabularnewline
Showery precipitation of snow & RESHSN & \url{http://codes.wmo.int/306/4678/SHSN}\tabularnewline
Unidentified showery precipitation & RESHUP & \url{http://codes.wmo.int/306/4678/SHUP}\tabularnewline
Precipitation of snow & RESN & \url{http://codes.wmo.int/306/4678/SN}\tabularnewline
Sandstorm & RESS & \url{http://codes.wmo.int/306/4678/SS}\tabularnewline
Thunderstorm & RETS & \url{http://codes.wmo.int/306/4678/TS}\tabularnewline
Thunderstorm with precipitation of hail & RETSGR & \url{http://codes.wmo.int/306/4678/TSGR}\tabularnewline
Thunderstorm with precipitation of snow pellets/small hail & RETSGS & \url{http://codes.wmo.int/306/4678/TSGS}\tabularnewline
Thunderstorm with precipitation of rain & RETSRA & \url{http://codes.wmo.int/306/4678/TSRA}\tabularnewline
Thunderstorm with precipitation of snow & RETSSN & \url{http://codes.wmo.int/306/4678/TSSN}\tabularnewline
Thunderstorm with unidentified precipitation & RETSUP & \url{http://codes.wmo.int/306/4678/TSUP}\tabularnewline
Unidentified precipitation & REUP & \url{http://codes.wmo.int/306/4678/UP}\tabularnewline
Volcanic ash & REVA & \url{http://codes.wmo.int/306/4678/VA}\tabularnewline
\bottomrule
\end{longtable}

CODE TABLE D-7: AERODROME PRESENT OR FORECAST WEATHER

The items within this code table are the weather phenomena that may be reported as forecast to occur or have been observed at an aerodrome. Requirements for reporting present or forecast weather at an aerodrome are specified in the \emph{Technical Regulations} (WMO-No.~49), Volume~II, Part~II, Appendix~3, 4.4 (observation), and Appendix~5, 2.2.4 (trend forecast) and 1.2.3 (for TAF).

The weather phenomena listed here are a subset of the enumerated set of meteorologically valid combinations specified in Volume~I.1, Code table~4678 comprising the following elements: intensity or proximity; descriptor; precipitation; obscuration; and/or other.

Each weather type is uniquely identified using a URI. The URI is also a URL providing additional information about the associated weather type. This code table is published at \url{http://codes.wmo.int/49-2/AerodromePresentOrForecastWeather}.

\begin{longtable}[]{@{}lll@{}}
\toprule
Label & Notation & URI\tabularnewline
\midrule
\endhead
Light precipitation of drizzle & -DZ & \url{http://codes.wmo.int/306/4678/-DZ}\tabularnewline
Light precipitation of rain & -RA & \url{http://codes.wmo.int/306/4678/-RA}\tabularnewline
Light precipitation of snow & -SN & \url{http://codes.wmo.int/306/4678/-SN}\tabularnewline
Light precipitation of snow grains & -SG & \url{http://codes.wmo.int/306/4678/-SG}\tabularnewline
Light precipitation of ice pellets & -PL & \url{http://codes.wmo.int/306/4678/-PL}\tabularnewline
Light unidentified precipitation & -UP & \url{http://codes.wmo.int/306/4678/-UP}\tabularnewline
Light precipitation of drizzle and rain & -DZRA & \url{http://codes.wmo.int/306/4678/-DZRA}\tabularnewline
Light precipitation of rain and drizzle & -RADZ & \url{http://codes.wmo.int/306/4678/-RADZ}\tabularnewline
Light precipitation of snow and drizzle & -SNDZ & \url{http://codes.wmo.int/306/4678/-SNDZ}\tabularnewline
Light precipitation of snow grains and drizzle & -SGDZ & \url{http://codes.wmo.int/306/4678/-SGDZ}\tabularnewline
Light precipitation of ice pellets and drizzle & -PLDZ & \url{http://codes.wmo.int/306/4678/-PLDZ}\tabularnewline
Light precipitation of drizzle and snow & -DZSN & \url{http://codes.wmo.int/306/4678/-DZSN}\tabularnewline
Light precipitation of rain and snow & -RASN & \url{http://codes.wmo.int/306/4678/-RASN}\tabularnewline
Light precipitation of snow and rain & -SNRA & \url{http://codes.wmo.int/306/4678/-SNRA}\tabularnewline
Light precipitation of snow grains and rain & -SGRA & \url{http://codes.wmo.int/306/4678/-SGRA}\tabularnewline
Light precipitation of ice pellets and rain & -PLRA & \url{http://codes.wmo.int/306/4678/-PLRA}\tabularnewline
Light precipitation of drizzle and snow grains & -DZSG & \url{http://codes.wmo.int/306/4678/-DZSG}\tabularnewline
Light precipitation of rain and snow grains & -RASG & \url{http://codes.wmo.int/306/4678/-RASG}\tabularnewline
Light precipitation of snow and snow grains & -SNSG & \url{http://codes.wmo.int/306/4678/-SNSG}\tabularnewline
Light precipitation of snow grains and snow & -SGSN & \url{http://codes.wmo.int/306/4678/-SGSN}\tabularnewline
Light precipitation of ice pellets and snow & -PLSN & \url{http://codes.wmo.int/306/4678/-PLSN}\tabularnewline
Light precipitation of drizzle and ice pellets & -DZPL & \url{http://codes.wmo.int/306/4678/-DZPL}\tabularnewline
Light precipitation of rain and ice pellets & -RAPL & \url{http://codes.wmo.int/306/4678/-RAPL}\tabularnewline
Light precipitation of snow and ice pellets & -SNPL & \url{http://codes.wmo.int/306/4678/-SNPL}\tabularnewline
Light precipitation of snow grains and ice pellets & -SGPL & \url{http://codes.wmo.int/306/4678/-SGPL}\tabularnewline
Light precipitation of ice pellets and snow grains & -PLSG & \url{http://codes.wmo.int/306/4678/-PLSG}\tabularnewline
Light precipitation of drizzle, rain and snow & -DZRASN & \url{http://codes.wmo.int/306/4678/-DZRASN}\tabularnewline
Light precipitation of drizzle, snow and rain & -DZSNRA & \url{http://codes.wmo.int/306/4678/-DZSNRA}\tabularnewline
Light precipitation of rain, drizzle and snow & -RADZSN & \url{http://codes.wmo.int/306/4678/-RADZSN}\tabularnewline
Light precipitation of rain, snow and drizzle & -RASNDZ & \url{http://codes.wmo.int/306/4678/-RASNDZ}\tabularnewline
Light precipitation of snow, drizzle and rain & -SNDZRA & \url{http://codes.wmo.int/306/4678/-SNDZRA}\tabularnewline
Light precipitation of snow, rain and drizzle & -SNRADZ & \url{http://codes.wmo.int/306/4678/-SNRADZ}\tabularnewline
Light precipitation of drizzle, rain and snow grains & -DZRASG & \url{http://codes.wmo.int/306/4678/-DZRASG}\tabularnewline
Light precipitation of drizzle, snow grains and rain & -DZSGRA & \url{http://codes.wmo.int/306/4678/-DZSGRA}\tabularnewline
Light precipitation of rain, drizzle and snow grains & -RADZSG & \url{http://codes.wmo.int/306/4678/-RADZSG}\tabularnewline
Light precipitation of rain, snow grains and drizzle & -RASGDZ & \url{http://codes.wmo.int/306/4678/-RASGDZ}\tabularnewline
Light precipitation of snow grains, drizzle and rain & -SGDZRA & \url{http://codes.wmo.int/306/4678/-SGDZRA}\tabularnewline
Light precipitation of snow grains, rain and drizzle & -SGRADZ & \url{http://codes.wmo.int/306/4678/-SGRADZ}\tabularnewline
Light precipitation of drizzle, rain and ice pellets & -DZRAPL & \url{http://codes.wmo.int/306/4678/-DZRAPL}\tabularnewline
Light precipitation of drizzle, ice pellets and rain & -DZPLRA & \url{http://codes.wmo.int/306/4678/-DZPLRA}\tabularnewline
Light precipitation of rain, drizzle and ice pellets & -RADZPL & \url{http://codes.wmo.int/306/4678/-RADZPL}\tabularnewline
Light precipitation of rain, ice pellets and drizzle & -RAPLDZ & \url{http://codes.wmo.int/306/4678/-RAPLDZ}\tabularnewline
Light precipitation of ice pellets, drizzle and rain & -PLDZRA & \url{http://codes.wmo.int/306/4678/-PLDZRA}\tabularnewline
Light precipitation of ice pellets, rain and drizzle & -PLRADZ & \url{http://codes.wmo.int/306/4678/-PLRADZ}\tabularnewline
Light precipitation of rain, snow and snow grains & -RASNSG & \url{http://codes.wmo.int/306/4678/-RASNSG}\tabularnewline
Light precipitation of rain, snow grains and snow & -RASGSN & \url{http://codes.wmo.int/306/4678/-RASGSN}\tabularnewline
Light precipitation of snow, rain and snow grains & -SNRASG & \url{http://codes.wmo.int/306/4678/-SNRASG}\tabularnewline
Light precipitation of snow, snow grains and rain & -SNSGRA & \url{http://codes.wmo.int/306/4678/-SNSGRA}\tabularnewline
Light precipitation of snow grains, rain and snow & -SGRASN & \url{http://codes.wmo.int/306/4678/-SGRASN}\tabularnewline
Light precipitation of snow grains, snow and rain & -SGSNRA & \url{http://codes.wmo.int/306/4678/-SGSNRA}\tabularnewline
Light precipitation of rain, snow and ice pellets & -RASNPL & \url{http://codes.wmo.int/306/4678/-RASNPL}\tabularnewline
Light precipitation of rain, ice pellets and snow & -RAPLSN & \url{http://codes.wmo.int/306/4678/-RAPLSN}\tabularnewline
Light precipitation of snow, rain and ice pellets & -SNRAPL & \url{http://codes.wmo.int/306/4678/-SNRAPL}\tabularnewline
Light precipitation of snow, ice pellets and rain & -SNPLRA & \url{http://codes.wmo.int/306/4678/-SNPLRA}\tabularnewline
Light precipitation of ice pellets, rain and snow & -PLRASN & \url{http://codes.wmo.int/306/4678/-PLRASN}\tabularnewline
Light precipitation of ice pellets, snow and rain & -PLSNRA & \url{http://codes.wmo.int/306/4678/-PLSNRA}\tabularnewline
Light precipitation of ice pellets, snow and snow grains & -PLSNSG & \url{http://codes.wmo.int/306/4678/-PLSNSG}\tabularnewline
Light precipitation of ice pellets, snow grains and snow & -PLSGSN & \url{http://codes.wmo.int/306/4678/-PLSGSN}\tabularnewline
Light precipitation of snow, ice pellets and snow grains & -SNPLSG & \url{http://codes.wmo.int/306/4678/-SNPLSG}\tabularnewline
Light precipitation of snow, snow grains and ice pellets & -SNSGPL & \url{http://codes.wmo.int/306/4678/-SNSGPL}\tabularnewline
Light precipitation of snow grains, ice pellets and snow & -SGPLSN & \url{http://codes.wmo.int/306/4678/-SGPLSN}\tabularnewline
Light precipitation of snow grains, snow and ice pellets & -SGSNPL & \url{http://codes.wmo.int/306/4678/-SGSNPL}\tabularnewline
Precipitation of drizzle & DZ & \url{http://codes.wmo.int/306/4678/DZ}\tabularnewline
Precipitation of rain & RA & \url{http://codes.wmo.int/306/4678/RA}\tabularnewline
Precipitation of snow & SN & \url{http://codes.wmo.int/306/4678/SN}\tabularnewline
Precipitation of snow grains & SG & \url{http://codes.wmo.int/306/4678/SG}\tabularnewline
Precipitation of ice pellets & PL & \url{http://codes.wmo.int/306/4678/PL}\tabularnewline
Unidentified precipitation & UP & \url{http://codes.wmo.int/306/4678/UP}\tabularnewline
Precipitation of drizzle and rain & DZRA & \url{http://codes.wmo.int/306/4678/DZRA}\tabularnewline
Precipitation of rain and drizzle & RADZ & \url{http://codes.wmo.int/306/4678/RADZ}\tabularnewline
Precipitation of snow and drizzle & SNDZ & \url{http://codes.wmo.int/306/4678/SNDZ}\tabularnewline
Precipitation of snow grains and drizzle & SGDZ & \url{http://codes.wmo.int/306/4678/SGDZ}\tabularnewline
Precipitation of ice pellets and drizzle & PLDZ & \url{http://codes.wmo.int/306/4678/PLDZ}\tabularnewline
Precipitation of drizzle and snow & DZSN & \url{http://codes.wmo.int/306/4678/DZSN}\tabularnewline
Precipitation of rain and snow & RASN & \url{http://codes.wmo.int/306/4678/RASN}\tabularnewline
Precipitation of snow and rain & SNRA & \url{http://codes.wmo.int/306/4678/SNRA}\tabularnewline
Precipitation of snow grains and rain & SGRA & \url{http://codes.wmo.int/306/4678/SGRA}\tabularnewline
Precipitation of ice pellets and rain & PLRA & \url{http://codes.wmo.int/306/4678/PLRA}\tabularnewline
Precipitation of drizzle and snow grains & DZSG & \url{http://codes.wmo.int/306/4678/DZSG}\tabularnewline
Precipitation of rain and snow grains & RASG & \url{http://codes.wmo.int/306/4678/RASG}\tabularnewline
Precipitation of snow and snow grains & SNSG & \url{http://codes.wmo.int/306/4678/SNSG}\tabularnewline
Precipitation of snow grains and snow & SGSN & \url{http://codes.wmo.int/306/4678/SGSN}\tabularnewline
Precipitation of ice pellets and snow & PLSN & \url{http://codes.wmo.int/306/4678/PLSN}\tabularnewline
Precipitation of drizzle and ice pellets & DZPL & \url{http://codes.wmo.int/306/4678/DZPL}\tabularnewline
Precipitation of rain and ice pellets & RAPL & \url{http://codes.wmo.int/306/4678/RAPL}\tabularnewline
Precipitation of snow and ice pellets & SNPL & \url{http://codes.wmo.int/306/4678/SNPL}\tabularnewline
Precipitation of snow grains and ice pellets & SGPL & \url{http://codes.wmo.int/306/4678/SGPL}\tabularnewline
Precipitation of ice pellets and snow grains & PLSG & \url{http://codes.wmo.int/306/4678/PLSG}\tabularnewline
Precipitation of drizzle, rain and snow & DZRASN & \url{http://codes.wmo.int/306/4678/DZRASN}\tabularnewline
Precipitation of drizzle, snow and rain & DZSNRA & \url{http://codes.wmo.int/306/4678/DZSNRA}\tabularnewline
Precipitation of rain, drizzle and snow & RADZSN & \url{http://codes.wmo.int/306/4678/RADZSN}\tabularnewline
Precipitation of rain, snow and drizzle & RASNDZ & \url{http://codes.wmo.int/306/4678/RASNDZ}\tabularnewline
Precipitation of snow, drizzle and rain & SNDZRA & \url{http://codes.wmo.int/306/4678/SNDZRA}\tabularnewline
Precipitation of snow, rain and drizzle & SNRADZ & \url{http://codes.wmo.int/306/4678/SNRADZ}\tabularnewline
Precipitation of drizzle, rain and snow grains & DZRASG & \url{http://codes.wmo.int/306/4678/DZRASG}\tabularnewline
Precipitation of drizzle, snow grains and rain & DZSGRA & \url{http://codes.wmo.int/306/4678/DZSGRA}\tabularnewline
Precipitation of rain, drizzle and snow grains & RADZSG & \url{http://codes.wmo.int/306/4678/RADZSG}\tabularnewline
Precipitation of rain, snow grains and drizzle & RASGDZ & \url{http://codes.wmo.int/306/4678/RASGDZ}\tabularnewline
Precipitation of snow grains, drizzle and rain & SGDZRA & \url{http://codes.wmo.int/306/4678/SGDZRA}\tabularnewline
Precipitation of snow grains, rain and drizzle & SGRADZ & \url{http://codes.wmo.int/306/4678/SGRADZ}\tabularnewline
Precipitation of drizzle, rain and ice pellets & DZRAPL & \url{http://codes.wmo.int/306/4678/DZRAPL}\tabularnewline
Precipitation of drizzle, ice pellets and rain & DZPLRA & \url{http://codes.wmo.int/306/4678/DZPLRA}\tabularnewline
Precipitation of rain, drizzle and ice pellets & RADZPL & \url{http://codes.wmo.int/306/4678/RADZPL}\tabularnewline
Precipitation of rain, ice pellets and drizzle & RAPLDZ & \url{http://codes.wmo.int/306/4678/RAPLDZ}\tabularnewline
Precipitation of ice pellets, drizzle and rain & PLDZRA & \url{http://codes.wmo.int/306/4678/PLDZRA}\tabularnewline
Precipitation of ice pellets, rain and drizzle & PLRADZ & \url{http://codes.wmo.int/306/4678/PLRADZ}\tabularnewline
Precipitation of rain, snow and snow grains & RASNSG & \url{http://codes.wmo.int/306/4678/RASNSG}\tabularnewline
Precipitation of rain, snow grains and snow & RASGSN & \url{http://codes.wmo.int/306/4678/RASGSN}\tabularnewline
Precipitation of snow, rain and snow grains & SNRASG & \url{http://codes.wmo.int/306/4678/SNRASG}\tabularnewline
Precipitation of snow, snow grains and rain & SNSGRA & \url{http://codes.wmo.int/306/4678/SNSGRA}\tabularnewline
Precipitation of snow grains, rain and snow & SGRASN & \url{http://codes.wmo.int/306/4678/SGRASN}\tabularnewline
Precipitation of snow grains, snow and rain & SGSNRA & \url{http://codes.wmo.int/306/4678/SGSNRA}\tabularnewline
Precipitation of rain, snow and ice pellets & RASNPL & \url{http://codes.wmo.int/306/4678/RASNPL}\tabularnewline
Precipitation of rain, ice pellets and snow & RAPLSN & \url{http://codes.wmo.int/306/4678/RAPLSN}\tabularnewline
Precipitation of snow, rain and ice pellets & SNRAPL & \url{http://codes.wmo.int/306/4678/SNRAPL}\tabularnewline
Precipitation of snow, ice pellets and rain & SNPLRA & \url{http://codes.wmo.int/306/4678/SNPLRA}\tabularnewline
Precipitation of ice pellets, rain and snow & PLRASN & \url{http://codes.wmo.int/306/4678/PLRASN}\tabularnewline
Precipitation of ice pellets, snow and rain & PLSNRA & \url{http://codes.wmo.int/306/4678/PLSNRA}\tabularnewline
Precipitation of ice pellets, snow and snow grains & PLSNSG & \url{http://codes.wmo.int/306/4678/PLSNSG}\tabularnewline
Precipitation of ice pellets, snow grains and snow & PLSGSN & \url{http://codes.wmo.int/306/4678/PLSGSN}\tabularnewline
Precipitation of snow, ice pellets and snow grains & SNPLSG & \url{http://codes.wmo.int/306/4678/SNPLSG}\tabularnewline
Precipitation of snow, snow grains and ice pellets & SNSGPL & \url{http://codes.wmo.int/306/4678/SNSGPL}\tabularnewline
Precipitation of snow grains, ice pellets and snow & SGPLSN & \url{http://codes.wmo.int/306/4678/SGPLSN}\tabularnewline
Precipitation of snow grains, snow and ice pellets & SGSNPL & \url{http://codes.wmo.int/306/4678/SGSNPL}\tabularnewline
Heavy precipitation of drizzle & +DZ & \url{http://codes.wmo.int/306/4678/+DZ}\tabularnewline
Heavy precipitation of rain & +RA & \url{http://codes.wmo.int/306/4678/+RA}\tabularnewline
Heavy precipitation of snow & +SN & \url{http://codes.wmo.int/306/4678/+SN}\tabularnewline
Heavy precipitation of snow grains & +SG & \url{http://codes.wmo.int/306/4678/+SG}\tabularnewline
Heavy precipitation of ice pellets & +PL & \url{http://codes.wmo.int/306/4678/+PL}\tabularnewline
Heavy unidentified precipitation & +UP & \url{http://codes.wmo.int/306/4678/+UP}\tabularnewline
Heavy precipitation of drizzle and rain & +DZRA & \url{http://codes.wmo.int/306/4678/+DZRA}\tabularnewline
Heavy precipitation of rain and drizzle & +RADZ & \url{http://codes.wmo.int/306/4678/+RADZ}\tabularnewline
Heavy precipitation of snow and drizzle & +SNDZ & \url{http://codes.wmo.int/306/4678/+SNDZ}\tabularnewline
Heavy precipitation of snow grains and drizzle & +SGDZ & \url{http://codes.wmo.int/306/4678/+SGDZ}\tabularnewline
Heavy precipitation of ice pellets and drizzle & +PLDZ & \url{http://codes.wmo.int/306/4678/+PLDZ}\tabularnewline
Heavy precipitation of drizzle and snow & +DZSN & \url{http://codes.wmo.int/306/4678/+DZSN}\tabularnewline
Heavy precipitation of rain and snow & +RASN & \url{http://codes.wmo.int/306/4678/+RASN}\tabularnewline
Heavy precipitation of snow and rain & +SNRA & \url{http://codes.wmo.int/306/4678/+SNRA}\tabularnewline
Heavy precipitation of snow grains and rain & +SGRA & \url{http://codes.wmo.int/306/4678/+SGRA}\tabularnewline
Heavy precipitation of ice pellets and rain & +PLRA & \url{http://codes.wmo.int/306/4678/+PLRA}\tabularnewline
Heavy precipitation of drizzle and snow grains & +DZSG & \url{http://codes.wmo.int/306/4678/+DZSG}\tabularnewline
Heavy precipitation of rain and snow grains & +RASG & \url{http://codes.wmo.int/306/4678/+RASG}\tabularnewline
Heavy precipitation of snow and snow grains & +SNSG & \url{http://codes.wmo.int/306/4678/+SNSG}\tabularnewline
Heavy precipitation of snow grains and snow & +SGSN & \url{http://codes.wmo.int/306/4678/+SGSN}\tabularnewline
Heavy precipitation of ice pellets and snow & +PLSN & \url{http://codes.wmo.int/306/4678/+PLSN}\tabularnewline
Heavy precipitation of drizzle and ice pellets & +DZPL & \url{http://codes.wmo.int/306/4678/+DZPL}\tabularnewline
Heavy precipitation of rain and ice pellets & +RAPL & \url{http://codes.wmo.int/306/4678/+RAPL}\tabularnewline
Heavy precipitation of snow and ice pellets & +SNPL & \url{http://codes.wmo.int/306/4678/+SNPL}\tabularnewline
Heavy precipitation of snow grains and ice pellets & +SGPL & \url{http://codes.wmo.int/306/4678/+SGPL}\tabularnewline
Heavy precipitation of ice pellets and snow grains & +PLSG & \url{http://codes.wmo.int/306/4678/+PLSG}\tabularnewline
Heavy precipitation of drizzle, rain and snow & +DZRASN & \url{http://codes.wmo.int/306/4678/+DZRASN}\tabularnewline
Heavy precipitation of drizzle, snow and rain & +DZSNRA & \url{http://codes.wmo.int/306/4678/+DZSNRA}\tabularnewline
Heavy precipitation of rain, drizzle and snow & +RADZSN & \url{http://codes.wmo.int/306/4678/+RADZSN}\tabularnewline
Heavy precipitation of rain, snow and drizzle & +RASNDZ & \url{http://codes.wmo.int/306/4678/+RASNDZ}\tabularnewline
Heavy precipitation of snow, drizzle and rain & +SNDZRA & \url{http://codes.wmo.int/306/4678/+SNDZRA}\tabularnewline
Heavy precipitation of snow, rain and drizzle & +SNRADZ & \url{http://codes.wmo.int/306/4678/+SNRADZ}\tabularnewline
Heavy precipitation of drizzle, rain and snow grains & +DZRASG & \url{http://codes.wmo.int/306/4678/+DZRASG}\tabularnewline
Heavy precipitation of drizzle, snow grains and rain & +DZSGRA & \url{http://codes.wmo.int/306/4678/+DZSGRA}\tabularnewline
Heavy precipitation of rain, drizzle and snow grains & +RADZSG & \url{http://codes.wmo.int/306/4678/+RADZSG}\tabularnewline
Heavy precipitation of rain, snow grains and drizzle & +RASGDZ & \url{http://codes.wmo.int/306/4678/+RASGDZ}\tabularnewline
Heavy precipitation of snow grains, drizzle and rain & +SGDZRA & \url{http://codes.wmo.int/306/4678/+SGDZRA}\tabularnewline
Heavy precipitation of snow grains, rain and drizzle & +SGRADZ & \url{http://codes.wmo.int/306/4678/+SGRADZ}\tabularnewline
Heavy precipitation of drizzle, rain and ice pellets & +DZRAPL & \url{http://codes.wmo.int/306/4678/+DZRAPL}\tabularnewline
Heavy precipitation of drizzle, ice pellets and rain & +DZPLRA & \url{http://codes.wmo.int/306/4678/+DZPLRA}\tabularnewline
Heavy precipitation of rain, drizzle and ice pellets & +RADZPL & \url{http://codes.wmo.int/306/4678/+RADZPL}\tabularnewline
Heavy precipitation of rain, ice pellets and drizzle & +RAPLDZ & \url{http://codes.wmo.int/306/4678/+RAPLDZ}\tabularnewline
Heavy precipitation of ice pellets, drizzle and rain & +PLDZRA & \url{http://codes.wmo.int/306/4678/+PLDZRA}\tabularnewline
Heavy precipitation of ice pellets, rain and drizzle & +PLRADZ & \url{http://codes.wmo.int/306/4678/+PLRADZ}\tabularnewline
Heavy precipitation of rain, snow and snow grains & +RASNSG & \url{http://codes.wmo.int/306/4678/+RASNSG}\tabularnewline
Heavy precipitation of rain, snow grains and snow & +RASGSN & \url{http://codes.wmo.int/306/4678/+RASGSN}\tabularnewline
Heavy precipitation of snow, rain and snow grains & +SNRASG & \url{http://codes.wmo.int/306/4678/+SNRASG}\tabularnewline
Heavy precipitation of snow, snow grains and rain & +SNSGRA & \url{http://codes.wmo.int/306/4678/+SNSGRA}\tabularnewline
Heavy precipitation of snow grains, rain and snow & +SGRASN & \url{http://codes.wmo.int/306/4678/+SGRASN}\tabularnewline
Heavy precipitation of snow grains, snow and rain & +SGSNRA & \url{http://codes.wmo.int/306/4678/+SGSNRA}\tabularnewline
Heavy precipitation of rain, snow and ice pellets & +RASNPL & \url{http://codes.wmo.int/306/4678/+RASNPL}\tabularnewline
Heavy precipitation of rain, ice pellets and snow & +RAPLSN & \url{http://codes.wmo.int/306/4678/+RAPLSN}\tabularnewline
Heavy precipitation of snow, rain and ice pellets & +SNRAPL & \url{http://codes.wmo.int/306/4678/+SNRAPL}\tabularnewline
Heavy precipitation of snow, ice pellets and rain & +SNPLRA & \url{http://codes.wmo.int/306/4678/+SNPLRA}\tabularnewline
Heavy precipitation of ice pellets, rain and snow & +PLRASN & \url{http://codes.wmo.int/306/4678/+PLRASN}\tabularnewline
Heavy precipitation of ice pellets, snow and rain & +PLSNRA & \url{http://codes.wmo.int/306/4678/+PLSNRA}\tabularnewline
Heavy precipitation of ice pellets, snow and snow grains & +PLSNSG & \url{http://codes.wmo.int/306/4678/+PLSNSG}\tabularnewline
Heavy precipitation of ice pellets, snow grains and snow & +PLSGSN & \url{http://codes.wmo.int/306/4678/+PLSGSN}\tabularnewline
Heavy precipitation of snow, ice pellets and snow grains & +SNPLSG & \url{http://codes.wmo.int/306/4678/+SNPLSG}\tabularnewline
Heavy precipitation of snow, snow grains and ice pellets & +SNSGPL & \url{http://codes.wmo.int/306/4678/+SNSGPL}\tabularnewline
Heavy precipitation of snow grains, ice pellets and snow & +SGPLSN & \url{http://codes.wmo.int/306/4678/+SGPLSN}\tabularnewline
Heavy precipitation of snow grains, snow and ice pellets & +SGSNPL & \url{http://codes.wmo.int/306/4678/+SGSNPL}\tabularnewline
Light showery precipitation of rain & -SHRA & \url{http://codes.wmo.int/306/4678/-SHRA}\tabularnewline
Light showery precipitation of snow & -SHSN & \url{http://codes.wmo.int/306/4678/-SHSN}\tabularnewline
Light showery precipitation of hail & -SHGR & \url{http://codes.wmo.int/306/4678/-SHGR}\tabularnewline
Light showery precipitation of snow pellets/small hail & -SHGS & \url{http://codes.wmo.int/306/4678/-SHGS}\tabularnewline
Light unidentified showery precipitation & -SHUP & \url{http://codes.wmo.int/306/4678/-SHUP}\tabularnewline
Light showery precipitation of rain and snow & -SHRASN & \url{http://codes.wmo.int/306/4678/-SHRASN}\tabularnewline
Light showery precipitation of snow and rain & -SHSNRA & \url{http://codes.wmo.int/306/4678/-SHSNRA}\tabularnewline
Light showery precipitation of hail and rain & -SHGRRA & \url{http://codes.wmo.int/306/4678/-SHGRRA}\tabularnewline
Light showery precipitation of snow pellets/small hail and rain & -SHGSRA & \url{http://codes.wmo.int/306/4678/-SHGSRA}\tabularnewline
Light showery precipitation of rain and hail & -SHRAGR & \url{http://codes.wmo.int/306/4678/-SHRAGR}\tabularnewline
Light showery precipitation of snow and hail & -SHSNGR & \url{http://codes.wmo.int/306/4678/-SHSNGR}\tabularnewline
Light showery precipitation of hail and snow & -SHGRSN & \url{http://codes.wmo.int/306/4678/-SHGRSN}\tabularnewline
Light showery precipitation of snow pellets/small hail and snow & -SHGSSN & \url{http://codes.wmo.int/306/4678/-SHGSSN}\tabularnewline
Light showery precipitation of rain and snow pellets/small hail & -SHRAGS & \url{http://codes.wmo.int/306/4678/-SHRAGS}\tabularnewline
Light showery precipitation of snow and snow pellets/small hail & -SHSNGS & \url{http://codes.wmo.int/306/4678/-SHSNGS}\tabularnewline
Light showery precipitation of rain, snow and hail & -SHRASNGR & \url{http://codes.wmo.int/306/4678/-SHRASNGR}\tabularnewline
Light showery precipitation of rain, hail and snow & -SHRAGRSN & \url{http://codes.wmo.int/306/4678/-SHRAGRSN}\tabularnewline
Light showery precipitation of snow, rain and hail & -SHSNRAGR & \url{http://codes.wmo.int/306/4678/-SHSNRAGR}\tabularnewline
Light showery precipitation of snow, hail and rain & -SHSNGRRA & \url{http://codes.wmo.int/306/4678/-SHSNGRRA}\tabularnewline
Light showery precipitation of hail, rain and snow & -SHGRRASN & \url{http://codes.wmo.int/306/4678/-SHGRRASN}\tabularnewline
Light showery precipitation of hail, snow and rain & -SHGRSNRA & \url{http://codes.wmo.int/306/4678/-SHGRSNRA}\tabularnewline
Light showery precipitation of rain, snow and snow pellets/small hail & -SHRASNGS & \url{http://codes.wmo.int/306/4678/-SHRASNGS}\tabularnewline
Light showery precipitation of rain, snow pellets/small hail and snow & -SHRAGSSN & \url{http://codes.wmo.int/306/4678/-SHRAGSSN}\tabularnewline
Light showery precipitation of snow, rain and snow pellets/small hail & -SHSNRAGS & \url{http://codes.wmo.int/306/4678/-SHSNRAGS}\tabularnewline
Light showery precipitation of snow, snow pellets/small hail and rain & -SHSNGSRA & \url{http://codes.wmo.int/306/4678/-SHSNGSRA}\tabularnewline
Light showery precipitation of snow pellets/small hail, rain and snow & -SHGSRASN & \url{http://codes.wmo.int/306/4678/-SHGSRASN}\tabularnewline
Light showery precipitation of snow pellets/small hail, snow and rain & -SHGSSNRA & \url{http://codes.wmo.int/306/4678/-SHGSSNRA}\tabularnewline
Showery precipitation of rain & SHRA & \url{http://codes.wmo.int/306/4678/SHRA}\tabularnewline
Showery precipitation of snow & SHSN & \url{http://codes.wmo.int/306/4678/SHSN}\tabularnewline
Showery precipitation of hail & SHGR & \url{http://codes.wmo.int/306/4678/SHGR}\tabularnewline
Showery precipitation of snow pellets/small hail & SHGS & \url{http://codes.wmo.int/306/4678/SHGS}\tabularnewline
Unidentified showery precipitation & SHUP & \url{http://codes.wmo.int/306/4678/SHUP}\tabularnewline
Showery precipitation of rain and snow & SHRASN & \url{http://codes.wmo.int/306/4678/SHRASN}\tabularnewline
Showery precipitation of snow and rain & SHSNRA & \url{http://codes.wmo.int/306/4678/SHSNRA}\tabularnewline
Showery precipitation of hail and rain & SHGRRA & \url{http://codes.wmo.int/306/4678/SHGRRA}\tabularnewline
Showery precipitation of snow pellets/small hail and rain & SHGSRA & \url{http://codes.wmo.int/306/4678/SHGSRA}\tabularnewline
Showery precipitation of rain and hail & SHRAGR & \url{http://codes.wmo.int/306/4678/SHRAGR}\tabularnewline
Showery precipitation of snow and hail & SHSNGR & \url{http://codes.wmo.int/306/4678/SHSNGR}\tabularnewline
Showery precipitation of hail and snow & SHGRSN & \url{http://codes.wmo.int/306/4678/SHGRSN}\tabularnewline
Showery precipitation of snow pellets/small hail and snow & SHGSSN & \url{http://codes.wmo.int/306/4678/SHGSSN}\tabularnewline
Showery precipitation of rain and snow pellets/small hail & SHRAGS & \url{http://codes.wmo.int/306/4678/SHRAGS}\tabularnewline
Showery precipitation of snow and snow pellets/small hail & SHSNGS & \url{http://codes.wmo.int/306/4678/SHSNGS}\tabularnewline
Showery precipitation of rain, snow and hail & SHRASNGR & \url{http://codes.wmo.int/306/4678/SHRASNGR}\tabularnewline
Showery precipitation of rain, hail and snow & SHRAGRSN & \url{http://codes.wmo.int/306/4678/SHRAGRSN}\tabularnewline
Showery precipitation of snow, rain and hail & SHSNRAGR & \url{http://codes.wmo.int/306/4678/SHSNRAGR}\tabularnewline
Showery precipitation of snow, hail and rain & SHSNGRRA & \url{http://codes.wmo.int/306/4678/SHSNGRRA}\tabularnewline
Showery precipitation of hail, rain and snow & SHGRRASN & \url{http://codes.wmo.int/306/4678/SHGRRASN}\tabularnewline
Showery precipitation of hail, snow and rain & SHGRSNRA & \url{http://codes.wmo.int/306/4678/SHGRSNRA}\tabularnewline
Showery precipitation of rain, snow and snow pellets/small hail & SHRASNGS & \url{http://codes.wmo.int/306/4678/SHRASNGS}\tabularnewline
Showery precipitation of rain, snow pellets/small hail and snow & SHRAGSSN & \url{http://codes.wmo.int/306/4678/SHRAGSSN}\tabularnewline
Showery precipitation of snow, rain and snow pellets/small hail & SHSNRAGS & \url{http://codes.wmo.int/306/4678/SHSNRAGS}\tabularnewline
Showery precipitation of snow, snow pellets/small hail and rain & SHSNGSRA & \url{http://codes.wmo.int/306/4678/SHSNGSRA}\tabularnewline
Showery precipitation of snow pellets/small hail, rain and snow & SHGSRASN & \url{http://codes.wmo.int/306/4678/SHGSRASN}\tabularnewline
Showery precipitation of snow pellets/small hail, snow and rain & SHGSSNRA & \url{http://codes.wmo.int/306/4678/SHGSSNRA}\tabularnewline
Heavy showery precipitation of rain & +SHRA & \url{http://codes.wmo.int/306/4678/+SHRA}\tabularnewline
Heavy showery precipitation of snow & +SHSN & \url{http://codes.wmo.int/306/4678/+SHSN}\tabularnewline
Heavy showery precipitation of hail & +SHGR & \url{http://codes.wmo.int/306/4678/+SHGR}\tabularnewline
Heavy showery precipitation of snow pellets/small hail & +SHGS & \url{http://codes.wmo.int/306/4678/+SHGS}\tabularnewline
Heavy unidentified showery precipitation & +SHUP & \url{http://codes.wmo.int/306/4678/+SHUP}\tabularnewline
Heavy showery precipitation of rain and snow & +SHRASN & \url{http://codes.wmo.int/306/4678/+SHRASN}\tabularnewline
Heavy showery precipitation of snow and rain & +SHSNRA & \url{http://codes.wmo.int/306/4678/+SHSNRA}\tabularnewline
Heavy showery precipitation of hail and rain & +SHGRRA & \url{http://codes.wmo.int/306/4678/+SHGRRA}\tabularnewline
Heavy showery precipitation of snow pellets/small hail and rain & +SHGSRA & \url{http://codes.wmo.int/306/4678/+SHGSRA}\tabularnewline
Heavy showery precipitation of rain and hail & +SHRAGR & \url{http://codes.wmo.int/306/4678/+SHRAGR}\tabularnewline
Heavy showery precipitation of snow and hail & +SHSNGR & \url{http://codes.wmo.int/306/4678/+SHSNGR}\tabularnewline
Heavy showery precipitation of hail and snow & +SHGRSN & \url{http://codes.wmo.int/306/4678/+SHGRSN}\tabularnewline
Heavy showery precipitation of snow pellets/small hail and snow & +SHGSSN & \url{http://codes.wmo.int/306/4678/+SHGSSN}\tabularnewline
Heavy showery precipitation of rain and snow pellets/small hail & +SHRAGS & \url{http://codes.wmo.int/306/4678/+SHRAGS}\tabularnewline
Heavy showery precipitation of snow and snow pellets/small hail & +SHSNGS & \url{http://codes.wmo.int/306/4678/+SHSNGS}\tabularnewline
Heavy showery precipitation of rain, snow and hail & +SHRASNGR & \url{http://codes.wmo.int/306/4678/+SHRASNGR}\tabularnewline
Heavy showery precipitation of rain, hail and snow & +SHRAGRSN & \url{http://codes.wmo.int/306/4678/+SHRAGRSN}\tabularnewline
Heavy showery precipitation of snow, rain and hail & +SHSNRAGR & \url{http://codes.wmo.int/306/4678/+SHSNRAGR}\tabularnewline
Heavy showery precipitation of snow, hail and rain & +SHSNGRRA & \url{http://codes.wmo.int/306/4678/+SHSNGRRA}\tabularnewline
Heavy showery precipitation of hail, rain and snow & +SHGRRASN & \url{http://codes.wmo.int/306/4678/+SHGRRASN}\tabularnewline
Heavy showery precipitation of hail, snow and rain & +SHGRSNRA & \url{http://codes.wmo.int/306/4678/+SHGRSNRA}\tabularnewline
Heavy showery precipitation of rain, snow and snow pellets/small hail & +SHRASNGS & \url{http://codes.wmo.int/306/4678/+SHRASNGS}\tabularnewline
Heavy showery precipitation of rain, snow pellets/small hail and snow & +SHRAGSSN & \url{http://codes.wmo.int/306/4678/+SHRAGSSN}\tabularnewline
Heavy showery precipitation of snow, rain and snow pellets/small hail & +SHSNRAGS & \url{http://codes.wmo.int/306/4678/+SHSNRAGS}\tabularnewline
Heavy showery precipitation of snow, snow pellets/small hail and rain & +SHSNGSRA & \url{http://codes.wmo.int/306/4678/+SHSNGSRA}\tabularnewline
Heavy showery precipitation of snow pellets/small hail, rain and snow & +SHGSRASN & \url{http://codes.wmo.int/306/4678/+SHGSRASN}\tabularnewline
Heavy showery precipitation of snow pellets/small hail, snow and rain & +SHGSSNRA & \url{http://codes.wmo.int/306/4678/+SHGSSNRA}\tabularnewline
Thunderstorm with light precipitation of rain & -TSRA & \url{http://codes.wmo.int/306/4678/-TSRA}\tabularnewline
Thunderstorm with light precipitation of snow & -TSSN & \url{http://codes.wmo.int/306/4678/-TSSN}\tabularnewline
Thunderstorm with light precipitation of hail & -TSGR & \url{http://codes.wmo.int/306/4678/-TSGR}\tabularnewline
Thunderstorm with light precipitation of snow pellets/small hail & -TSGS & \url{http://codes.wmo.int/306/4678/-TSGS}\tabularnewline
Thunderstorm with light unidentified precipitation & -TSUP & \url{http://codes.wmo.int/306/4678/-TSUP}\tabularnewline
Thunderstorm with light precipitation of rain and snow & -TSRASN & \url{http://codes.wmo.int/306/4678/-TSRASN}\tabularnewline
Thunderstorm with light precipitation of snow and rain & -TSSNRA & \url{http://codes.wmo.int/306/4678/-TSSNRA}\tabularnewline
Thunderstorm with light precipitation of hail and rain & -TSGRRA & \url{http://codes.wmo.int/306/4678/-TSGRRA}\tabularnewline
Thunderstorm with light precipitation of snow pellets/small hail and rain & -TSGSRA & \url{http://codes.wmo.int/306/4678/-TSGSRA}\tabularnewline
Thunderstorm with light precipitation of rain and hail & -TSRAGR & \url{http://codes.wmo.int/306/4678/-TSRAGR}\tabularnewline
Thunderstorm with light precipitation of snow and hail & -TSSNGR & \url{http://codes.wmo.int/306/4678/-TSSNGR}\tabularnewline
Thunderstorm with light precipitation of hail and snow & -TSGRSN & \url{http://codes.wmo.int/306/4678/-TSGRSN}\tabularnewline
Thunderstorm with light precipitation of snow pellets/small hail and snow & -TSGSSN & \url{http://codes.wmo.int/306/4678/-TSGSSN}\tabularnewline
Thunderstorm with light precipitation of rain and snow pellets/small hail & -TSRAGS & \url{http://codes.wmo.int/306/4678/-TSRAGS}\tabularnewline
Thunderstorm with light precipitation of snow and snow pellets/small hail & -TSSNGS & \url{http://codes.wmo.int/306/4678/-TSSNGS}\tabularnewline
Thunderstorm with light precipitation of rain, snow and hail & -TSRASNGR & \url{http://codes.wmo.int/306/4678/-TSRASNGR}\tabularnewline
Thunderstorm with light precipitation of rain, hail and snow & -TSRAGRSN & \url{http://codes.wmo.int/306/4678/-TSRAGRSN}\tabularnewline
Thunderstorm with light precipitation of snow, rain and hail & -TSSNRAGR & \url{http://codes.wmo.int/306/4678/-TSSNRAGR}\tabularnewline
Thunderstorm with light precipitation of snow, hail and rain & -TSSNGRRA & \url{http://codes.wmo.int/306/4678/-TSSNGRRA}\tabularnewline
Thunderstorm with light precipitation of hail, rain and snow & -TSGRRASN & \url{http://codes.wmo.int/306/4678/-TSGRRASN}\tabularnewline
Thunderstorm with light precipitation of hail, snow and rain & -TSGRSNRA & \url{http://codes.wmo.int/306/4678/-TSGRSNRA}\tabularnewline
Thunderstorm with light precipitation of rain, snow and snow pellets/small hail & -TSRASNGS & \url{http://codes.wmo.int/306/4678/-TSRASNGS}\tabularnewline
Thunderstorm with light precipitation of rain, snow pellets/small hail and snow & -TSRAGSSN & \url{http://codes.wmo.int/306/4678/-TSRAGSSN}\tabularnewline
Thunderstorm with light precipitation of snow, rain and snow pellets/small hail & -TSSNRAGS & \url{http://codes.wmo.int/306/4678/-TSSNRAGS}\tabularnewline
Thunderstorm with light precipitation of snow, snow pellets/small hail and rain & -TSSNGSRA & \url{http://codes.wmo.int/306/4678/-TSSNGSRA}\tabularnewline
Thunderstorm with light precipitation of snow pellets/small hail, rain and snow & -TSGSRASN & \url{http://codes.wmo.int/306/4678/-TSGSRASN}\tabularnewline
Thunderstorm with light precipitation of snow pellets/small hail, snow and rain & -TSGSSNRA & \url{http://codes.wmo.int/306/4678/-TSGSSNRA}\tabularnewline
Thunderstorm with precipitation of rain & TSRA & \url{http://codes.wmo.int/306/4678/TSRA}\tabularnewline
Thunderstorm with precipitation of snow & TSSN & \url{http://codes.wmo.int/306/4678/TSSN}\tabularnewline
Thunderstorm with precipitation of hail & TSGR & \url{http://codes.wmo.int/306/4678/TSGR}\tabularnewline
Thunderstorm with precipitation of snow pellets/small hail & TSGS & \url{http://codes.wmo.int/306/4678/TSGS}\tabularnewline
Thunderstorm with unidentified precipitation & TSUP & \url{http://codes.wmo.int/306/4678/TSUP}\tabularnewline
Thunderstorm with precipitation of rain and snow & TSRASN & \url{http://codes.wmo.int/306/4678/TSRASN}\tabularnewline
Thunderstorm with precipitation of snow and rain & TSSNRA & \url{http://codes.wmo.int/306/4678/TSSNRA}\tabularnewline
Thunderstorm with precipitation of hail and rain & TSGRRA & \url{http://codes.wmo.int/306/4678/TSGRRA}\tabularnewline
Thunderstorm with precipitation of snow pellets/small hail and rain & TSGSRA & \url{http://codes.wmo.int/306/4678/TSGSRA}\tabularnewline
Thunderstorm with precipitation of rain and hail & TSRAGR & \url{http://codes.wmo.int/306/4678/TSRAGR}\tabularnewline
Thunderstorm with precipitation of snow and hail & TSSNGR & \url{http://codes.wmo.int/306/4678/TSSNGR}\tabularnewline
Thunderstorm with precipitation of hail and snow & TSGRSN & \url{http://codes.wmo.int/306/4678/TSGRSN}\tabularnewline
Thunderstorm with precipitation of snow pellets/small hail and snow & TSGSSN & \url{http://codes.wmo.int/306/4678/TSGSSN}\tabularnewline
Thunderstorm with precipitation of rain and snow pellets/small hail & TSRAGS & \url{http://codes.wmo.int/306/4678/TSRAGS}\tabularnewline
Thunderstorm with precipitation of snow and snow pellets/small hail & TSSNGS & \url{http://codes.wmo.int/306/4678/TSSNGS}\tabularnewline
Thunderstorm with precipitation of rain, snow and hail & TSRASNGR & \url{http://codes.wmo.int/306/4678/TSRASNGR}\tabularnewline
Thunderstorm with precipitation of rain, hail and snow & TSRAGRSN & \url{http://codes.wmo.int/306/4678/TSRAGRSN}\tabularnewline
Thunderstorm with precipitation of snow, rain and hail & TSSNRAGR & \url{http://codes.wmo.int/306/4678/TSSNRAGR}\tabularnewline
Thunderstorm with precipitation of snow, hail and rain & TSSNGRRA & \url{http://codes.wmo.int/306/4678/TSSNGRRA}\tabularnewline
Thunderstorm with precipitation of hail, rain and snow & TSGRRASN & \url{http://codes.wmo.int/306/4678/TSGRRASN}\tabularnewline
Thunderstorm with precipitation of hail, snow and rain & TSGRSNRA & \url{http://codes.wmo.int/306/4678/TSGRSNRA}\tabularnewline
Thunderstorm with precipitation of rain, snow and snow pellets/small hail & TSRASNGS & \url{http://codes.wmo.int/306/4678/TSRASNGS}\tabularnewline
Thunderstorm with precipitation of rain, snow pellets/small hail and snow & TSRAGSSN & \url{http://codes.wmo.int/306/4678/TSRAGSSN}\tabularnewline
Thunderstorm with precipitation of snow, rain and snow pellets/small hail & TSSNRAGS & \url{http://codes.wmo.int/306/4678/TSSNRAGS}\tabularnewline
Thunderstorm with precipitation of snow, snow pellets/small hail and rain & TSSNGSRA & \url{http://codes.wmo.int/306/4678/TSSNGSRA}\tabularnewline
Thunderstorm with precipitation of snow pellets/small hail, rain and snow & TSGSRASN & \url{http://codes.wmo.int/306/4678/TSGSRASN}\tabularnewline
Thunderstorm with precipitation of snow pellets/small hail, snow and rain & TSGSSNRA & \url{http://codes.wmo.int/306/4678/TSGSSNRA}\tabularnewline
Thunderstorm with heavy precipitation of rain & +TSRA & \url{http://codes.wmo.int/306/4678/+TSRA}\tabularnewline
Thunderstorm with heavy precipitation of snow & +TSSN & \url{http://codes.wmo.int/306/4678/+TSSN}\tabularnewline
Thunderstorm with heavy precipitation of hail & +TSGR & \url{http://codes.wmo.int/306/4678/+TSGR}\tabularnewline
Thunderstorm with heavy precipitation of snow pellets/small hail & +TSGS & \url{http://codes.wmo.int/306/4678/+TSGS}\tabularnewline
Thunderstorm with heavy precipitation of unidentified precipitation & +TSUP & \url{http://codes.wmo.int/306/4678/+TSUP}\tabularnewline
Thunderstorm with heavy precipitation of rain and snow & +TSRASN & \url{http://codes.wmo.int/306/4678/+TSRASN}\tabularnewline
Thunderstorm with heavy precipitation of snow and rain & +TSSNRA & \url{http://codes.wmo.int/306/4678/+TSSNRA}\tabularnewline
Thunderstorm with heavy precipitation of hail and rain & +TSGRRA & \url{http://codes.wmo.int/306/4678/+TSGRRA}\tabularnewline
Thunderstorm with heavy precipitation of snow pellets/small hail and rain & +TSGSRA & \url{http://codes.wmo.int/306/4678/+TSGSRA}\tabularnewline
Thunderstorm with heavy precipitation of rain and hail & +TSRAGR & \url{http://codes.wmo.int/306/4678/+TSRAGR}\tabularnewline
Thunderstorm with heavy precipitation of snow and hail & +TSSNGR & \url{http://codes.wmo.int/306/4678/+TSSNGR}\tabularnewline
Thunderstorm with heavy precipitation of hail and snow & +TSGRSN & \url{http://codes.wmo.int/306/4678/+TSGRSN}\tabularnewline
Thunderstorm with heavy precipitation of snow pellets/small hail and snow & +TSGSSN & \url{http://codes.wmo.int/306/4678/+TSGSSN}\tabularnewline
Thunderstorm with heavy precipitation of rain and snow pellets/small hail & +TSRAGS & \url{http://codes.wmo.int/306/4678/+TSRAGS}\tabularnewline
Thunderstorm with heavy precipitation of snow and snow pellets/small hail & +TSSNGS & \url{http://codes.wmo.int/306/4678/+TSSNGS}\tabularnewline
Thunderstorm with heavy precipitation of rain, snow and hail & +TSRASNGR & \url{http://codes.wmo.int/306/4678/+TSRASNGR}\tabularnewline
Thunderstorm with heavy precipitation of rain, hail and snow & +TSRAGRSN & \url{http://codes.wmo.int/306/4678/+TSRAGRSN}\tabularnewline
Thunderstorm with heavy precipitation of snow, rain and hail & +TSSNRAGR & \url{http://codes.wmo.int/306/4678/+TSSNRAGR}\tabularnewline
Thunderstorm with heavy precipitation of snow, hail and rain & +TSSNGRRA & \url{http://codes.wmo.int/306/4678/+TSSNGRRA}\tabularnewline
Thunderstorm with heavy precipitation of hail, rain and snow & +TSGRRASN & \url{http://codes.wmo.int/306/4678/+TSGRRASN}\tabularnewline
Thunderstorm with heavy precipitation of hail, snow and rain & +TSGRSNRA & \url{http://codes.wmo.int/306/4678/+TSGRSNRA}\tabularnewline
Thunderstorm with heavy precipitation of rain, snow and snow pellets/small hail & +TSRASNGS & \url{http://codes.wmo.int/306/4678/+TSRASNGS}\tabularnewline
Thunderstorm with heavy precipitation of rain, snow pellets/small hail and snow & +TSRAGSSN & \url{http://codes.wmo.int/306/4678/+TSRAGSSN}\tabularnewline
Thunderstorm with heavy precipitation of snow, rain and snow pellets/small hail & +TSSNRAGS & \url{http://codes.wmo.int/306/4678/+TSSNRAGS}\tabularnewline
Thunderstorm with heavy precipitation of snow, snow pellets/small hail and rain & +TSSNGSRA & \url{http://codes.wmo.int/306/4678/+TSSNGSRA}\tabularnewline
Thunderstorm with heavy precipitation of snow pellets/small hail, rain and snow & +TSGSRASN & \url{http://codes.wmo.int/306/4678/+TSGSRASN}\tabularnewline
Thunderstorm with heavy precipitation of snow pellets/small hail, snow and rain & +TSGSSNRA & \url{http://codes.wmo.int/306/4678/+TSGSSNRA}\tabularnewline
Light precipitation of freezing drizzle & -FZDZ & \url{http://codes.wmo.int/306/4678/-FZDZ}\tabularnewline
Light precipitation of freezing rain & -FZRA & \url{http://codes.wmo.int/306/4678/-FZRA}\tabularnewline
Light unidentified freezing precipitation & -FZUP & \url{http://codes.wmo.int/306/4678/-FZUP}\tabularnewline
Light precipitation of freezing drizzle and rain & -FZDZRA & \url{http://codes.wmo.int/306/4678/-FZDZRA}\tabularnewline
Light precipitation of freezing rain and drizzle & -FZRADZ & \url{http://codes.wmo.int/306/4678/-FZRADZ}\tabularnewline
Precipitation of freezing drizzle & FZDZ & \url{http://codes.wmo.int/306/4678/FZDZ}\tabularnewline
Precipitation of freezing rain & FZRA & \url{http://codes.wmo.int/306/4678/FZRA}\tabularnewline
Unidentified freezing precipitation & FZUP & \url{http://codes.wmo.int/306/4678/FZUP}\tabularnewline
Precipitation of freezing drizzle and rain & FZDZRA & \url{http://codes.wmo.int/306/4678/FZDZRA}\tabularnewline
Precipitation of freezing rain and drizzle & FZRADZ & \url{http://codes.wmo.int/306/4678/FZRADZ}\tabularnewline
Heavy precipitation of freezing drizzle & +FZDZ & \url{http://codes.wmo.int/306/4678/+FZDZ}\tabularnewline
Heavy precipitation of freezing rain & +FZRA & \url{http://codes.wmo.int/306/4678/+FZRA}\tabularnewline
Heavy unidentified freezing precipitation & +FZUP & \url{http://codes.wmo.int/306/4678/+FZUP}\tabularnewline
Heavy precipitation of freezing drizzle and rain & +FZDZRA & \url{http://codes.wmo.int/306/4678/+FZDZRA}\tabularnewline
Heavy precipitation of freezing rain and drizzle & +FZRADZ & \url{http://codes.wmo.int/306/4678/+FZRADZ}\tabularnewline
Duststorm & DS & \url{http://codes.wmo.int/306/4678/DS}\tabularnewline
Heavy duststorm & +DS & \url{http://codes.wmo.int/306/4678/+DS}\tabularnewline
Duststorm in the vicinity & VCDS & \url{http://codes.wmo.int/306/4678/VCDS}\tabularnewline
Sandstorm & SS & \url{http://codes.wmo.int/306/4678/SS}\tabularnewline
Heavy sandstorm & +SS & \url{http://codes.wmo.int/306/4678/+SS}\tabularnewline
Sandstorm in the vicinity & VCSS & \url{http://codes.wmo.int/306/4678/VCSS}\tabularnewline
Fog & FG & \url{http://codes.wmo.int/306/4678/FG}\tabularnewline
Funnel cloud(s) (tornado or waterspout) & FC & \url{http://codes.wmo.int/306/4678/FC}\tabularnewline
Well-developed funnel cloud(s) (tornado or waterspout) & +FC & \url{http://codes.wmo.int/306/4678/+FC}\tabularnewline
Dust/sand whirls (dust devils) & PO & \url{http://codes.wmo.int/306/4678/PO}\tabularnewline
Volcanic ash & VA & \url{http://codes.wmo.int/306/4678/VA}\tabularnewline
Fog in the vicinity & VCFG & \url{http://codes.wmo.int/306/4678/VCFG}\tabularnewline
Funnel cloud(s) (tornado or waterspout) in the vicinity & VCFC & \url{http://codes.wmo.int/306/4678/VCFC}\tabularnewline
Dust/sand whirls (dust devils) in the vicinity & VCPO & \url{http://codes.wmo.int/306/4678/VCPO}\tabularnewline
Volcanic ash in the vicinity & VCVA & \url{http://codes.wmo.int/306/4678/VCVA}\tabularnewline
Thunderstorm & TS & \url{http://codes.wmo.int/306/4678/TS}\tabularnewline
Thunderstorm in the vicinity & VCTS & \url{http://codes.wmo.int/306/4678/VCTS}\tabularnewline
Shower(s) in the vicinity & VCSH & \url{http://codes.wmo.int/306/4678/VCSH}\tabularnewline
Blowing sand in the vicinity & VCBLSA & \url{http://codes.wmo.int/306/4678/VCBLSA}\tabularnewline
Blowing dust in the vicinity & VCBLDU & \url{http://codes.wmo.int/306/4678/VCBLDU}\tabularnewline
Blowing snow in the vicinity & VCBLSN & \url{http://codes.wmo.int/306/4678/VCBLSN}\tabularnewline
Blowing sand & BLSA & \url{http://codes.wmo.int/306/4678/BLSA}\tabularnewline
Blowing dust & BLDU & \url{http://codes.wmo.int/306/4678/BLDU}\tabularnewline
Blowing snow & BLSN & \url{http://codes.wmo.int/306/4678/BLSN}\tabularnewline
Low drifting sand & DRSA & \url{http://codes.wmo.int/306/4678/DRSA}\tabularnewline
Low drifting dust & DRDU & \url{http://codes.wmo.int/306/4678/DRDU}\tabularnewline
Low drifting snow & DRSN & \url{http://codes.wmo.int/306/4678/DRSN}\tabularnewline
Sand & SA & \url{http://codes.wmo.int/306/4678/SA}\tabularnewline
Dust & DU & \url{http://codes.wmo.int/306/4678/DU}\tabularnewline
Shallow fog & MIFG & \url{http://codes.wmo.int/306/4678/MIFG}\tabularnewline
Partial fog (covering part of the aerodrome) & PRFG & \url{http://codes.wmo.int/306/4678/PRFG}\tabularnewline
Patches of fog & BCFG & \url{http://codes.wmo.int/306/4678/BCFG}\tabularnewline
Freezing fog & FZFG & \url{http://codes.wmo.int/306/4678/FZFG}\tabularnewline
Mist & BR & \url{http://codes.wmo.int/306/4678/BR}\tabularnewline
Haze & HZ & \url{http://codes.wmo.int/306/4678/HZ}\tabularnewline
Smoke & FU & \url{http://codes.wmo.int/306/4678/FU}\tabularnewline
Squalls & SQ & \url{http://codes.wmo.int/306/4678/SQ}\tabularnewline
\bottomrule
\end{longtable}

CODE TABLE D-8: CLOUD AMOUNT REPORTED AT AERODROME

The items within this code table are the cloud amount categories of operational significance for aviation as specified in the \emph{Technical Regulations} (WMO-No~49), Volume~II -- Meteorological Service for International Air Navigation.

This code table contains a subset of the cloud amount categories defined in Volume~I.2, FM~94 BUFR, Code table~0~20~008. Each code item is uniquely identified using a URI. The URI is also a URL providing additional information about the associated cloud amount category. This code table is published at \url{http://codes.wmo.int/49-2/CloudAmountReportedAtAerodrome}.

\begin{longtable}[]{@{}llll@{}}
\toprule
Label & Notation & URI & Description\tabularnewline
\midrule
\endhead
Broken & BKN & \url{http://codes.wmo.int/bufr4/codeflag/0-20-008/3} & Broken (5--7 oktas).\tabularnewline
Embedded & EMBD & \url{http://codes.wmo.int/bufr4/codeflag/0-20-008/16} & Embedded. Applicable only to cumulonimbus (CB).\tabularnewline
Few & FEW & \url{http://codes.wmo.int/bufr4/codeflag/0-20-008/1} & Few (1--2 oktas).\tabularnewline
Frequent & FRQ & \url{http://codes.wmo.int/bufr4/codeflag/0-20-008/12} & Frequent. Applicable only to cumulonimbus (CB).\tabularnewline
Isolated & ISOL & \url{http://codes.wmo.int/bufr4/codeflag/0-20-008/8} & Isolated. Applicable only to cumulonimbus (CB).\tabularnewline
Layers & LYR & \url{http://codes.wmo.int/bufr4/codeflag/0-20-008/14} & Layers. Applicable only to cumulonimbus (CB).\tabularnewline
Occasional & OCNL & \url{http://codes.wmo.int/bufr4/codeflag/0-20-008/10} & Occasional. Applicable only to cumulonimbus (CB).\tabularnewline
Overcast & OVC & \url{http://codes.wmo.int/bufr4/codeflag/0-20-008/4} & Overcast (8 oktas).\tabularnewline
Scattered & SCT & \url{http://codes.wmo.int/bufr4/codeflag/0-20-008/2} & Scattered (3--4 oktas).\tabularnewline
Sky clear & SKC & \url{http://codes.wmo.int/bufr4/codeflag/0-20-008/0} & Sky clear (0 oktas).\tabularnewline
\bottomrule
\end{longtable}

CODE TABLE D-9: SIGNIFICANT CONVECTIVE CLOUD TYPE

The items within this code table are the cloud types of operational significance for aviation as specified in the \emph{Technical Regulations} (WMO-No.~49), Volume II -- Meteorological Service for International Air Navigation. This code table contains a subset of the cloud types defined in Volume~I.2, FM~94 BUFR, Code table~0~20~012. Each cloud type is uniquely identified using a URI. The URI is also a URL providing additional information about the associated cloud type. This code table is published at \url{http://codes.wmo.int/49-2/SigConvectiveCloudType}.

\begin{longtable}[]{@{}llll@{}}
\toprule
Label & Notation & URI & Description\tabularnewline
\midrule
\endhead
Cumulonimbus & CB & \url{http://codes.wmo.int/bufr4/codeflag/0-20-012/9} & A principal cloud type, exceptionally dense and vertically developed, occurring either as isolated clouds or as a line or wall of clouds with separated upper portions.\tabularnewline
Towering cumulus & TCU & \url{http://codes.wmo.int/bufr4/codeflag/0-20-012/32} & Cumulus mediocris or congestus, towering cumulus (TCU), with or without cumulus of species fractus or humilis or stratocumulus, all having their bases at the same level.\tabularnewline
\bottomrule
\end{longtable}

CODE TABLE D-10: SIGNIFICANT WEATHER PHENOMENA

The items within this code table are the types of weather phenomena of significance to aeronautical operations -- as used in SIGMET and AIRMET reports and specified in the \emph{Technical Regulations} (WMO-No.~49), Volume~II, Part~II, Appendix~6, 1.1.4. Each weather phenomenon type is uniquely identified using a URI. The URI is also a URL providing additional information about the associated weather phenomena type. This code table is published at \url{http://codes.wmo.int/49-2/SigWxPhenomena}.

\begin{longtable}[]{@{}llll@{}}
\toprule
Label & Notation & URI & Description\tabularnewline
\midrule
\endhead
Embedded thunderstorm & EMBD\_TS & \url{http://codes.wmo.int/49-2/SigWxPhenomena/EMBD_TS} &\tabularnewline
Embedded thunderstorm with hail & EMBD\_TSGR & \url{http://codes.wmo.int/49-2/SigWxPhenomena/EMBD_TSGR} &\tabularnewline
Frequent thunderstorm & FRQ\_TS & \url{http://codes.wmo.int/49-2/SigWxPhenomena/FRQ_TS} &\tabularnewline
Frequent thunderstorm with hail & FRQ\_TSGR & \url{http://codes.wmo.int/49-2/SigWxPhenomena/FRQ_TSGR} &\tabularnewline
Heavy duststorm & HVY\_DS & \url{http://codes.wmo.int/49-2/SigWxPhenomena/HVY_DS} &\tabularnewline
Heavy sandstorm & HVY\_SS & \url{http://codes.wmo.int/49-2/SigWxPhenomena/HVY_SS} &\tabularnewline
Obscured thunderstorm & OBSC\_TS & \url{http://codes.wmo.int/49-2/SigWxPhenomena/OBSC_TS} &\tabularnewline
Obscured thunderstorm with hail & OBSC\_TSGR & \url{http://codes.wmo.int/49-2/SigWxPhenomena/OBSC_TSGR} &\tabularnewline
Radioactive cloud & RDOACT\_CLD & \url{http://codes.wmo.int/49-2/SigWxPhenomena/RDOACT_CLD} &\tabularnewline
Severe airframe icing & SEV\_ICE & \url{http://codes.wmo.int/49-2/SigWxPhenomena/SEV_ICE} &\tabularnewline
Severe airframe icing from freezing rain & SEV\_ICE\_FZRA & \url{http://codes.wmo.int/49-2/SigWxPhenomena/SEV_ICE_FZRA} &\tabularnewline
Severe mountain wave & SEV\_MTW & \url{http://codes.wmo.int/49-2/SigWxPhenomena/SEV_MTW} &\tabularnewline
Severe turbulence & SEV\_TURB & \url{http://codes.wmo.int/49-2/SigWxPhenomena/SEV_TURB} &\tabularnewline
Squall line & SQL\_TS & \url{http://codes.wmo.int/49-2/SigWxPhenomena/SQL_TS} &\tabularnewline
Squall line with hail & SQL\_TSGR & \url{http://codes.wmo.int/49-2/SigWxPhenomena/SQL_TSGR} &\tabularnewline
Tropical cyclone & TC & \url{http://codes.wmo.int/49-2/SigWxPhenomena/TC} &\tabularnewline
Volcanic ash & VA & \url{http://codes.wmo.int/49-2/SigWxPhenomena/VA} &\tabularnewline
\bottomrule
\end{longtable}

Appendix B. Definition of Schemas

\hypertarget{collectxml}{%
\section{1. COLLECT‑XML}\label{collectxml}}

FM 201-15 EXT. COLLECT-XML (1.1)

\url{http://schemas.wmo.int/collect/1.1/collect.xsd}

FM 201-16 COLLECT-XML (1.2)

\url{http://schemas.wmo.int/collect/1.2/collect.xsd}

\hypertarget{metce-xml-moduxe8le-pour-luxe9change-des-informations-sur-le-temps-le-climat-et-leau}{%
\section{\texorpdfstring{2. METCE-XML \emph{\textbf{(MODÈLE POUR L'éCHANGE DES INFORMATIONS SUR LE TEMPS, LE CLIMAT ET L'EAU)}}}{2. METCE-XML (MODÈLE POUR L'éCHANGE DES INFORMATIONS SUR LE TEMPS, LE CLIMAT ET L'EAU)}}\label{metce-xml-moduxe8le-pour-luxe9change-des-informations-sur-le-temps-le-climat-et-leau}}

FM 202-15 EXT. METCE-XML (1.1)

\url{http://schemas.wmo.int/metce/1.1/metce.xsd}

\url{http://schemas.wmo.int/metce/1.1/procedure.xsd}

\url{http://schemas.wmo.int/metce/1.1/phenomena.xsd}

FM 202-16 METCE-XML (1.2)

\url{http://schemas.wmo.int/metce/1.2/metce.xsd}

\url{http://schemas.wmo.int/metce/1.2/procedure.xsd}

\url{http://schemas.wmo.int/metce/1.2/phenomena.xsd}

\hypertarget{opmxml-observable-property-model}{%
\section{3. OPM‑XML (OBSERVABLE PROPERTY MODEL)}\label{opmxml-observable-property-model}}

\url{http://schemas.wmo.int/opm/1.1/opm.xsd}

\url{http://schemas.wmo.int/opm/1.1/observable-property.xsd}

\hypertarget{safxml-simple-aeronautical-features}{%
\section{4. SAF‑XML (Simple Aeronautical Features)}\label{safxml-simple-aeronautical-features}}

http://schemas.wmo.int/saf/1.1/saf.xsd

\url{http://schemas.wmo.int/saf/1.1/dataTypes.xsd}

\url{http://schemas.wmo.int/saf/1.1/features.xsd}

\url{http://schemas.wmo.int/saf/1.1/measures.xsd}

\hypertarget{iwxxmxml-icao-meteorological-information-exchange-model}{%
\section{5. IWXXM‑XML (ICAO METEOROLOGICAL INFORMATION EXCHANGE MODEL)}\label{iwxxmxml-icao-meteorological-information-exchange-model}}

FM 205-15 EXT. IWXXM (1.1)

\url{http://schemas.wmo.int/iwxxm/1.1/iwxxm.xsd}

\url{http://schemas.wmo.int/iwxxm/1.1/common.xsd}

\url{http://schemas.wmo.int/iwxxm/1.1/metarSpeci.xsd}

\url{http://schemas.wmo.int/iwxxm/1.1/taf.xsd}

\url{http://schemas.wmo.int/iwxxm/1.1/sigmet.xsd}

FM 205-16 IWXXM (2.1)

\url{http://schemas.wmo.int/iwxxm/2.1/iwxxm.xsd}

\url{http://schemas.wmo.int/iwxxm/2.1/common.xsd}

\url{http://schemas.wmo.int/iwxxm/2.1/metarSpeci.xsd}

\url{http://schemas.wmo.int/iwxxm/2.1/taf.xsd}

\url{http://schemas.wmo.int/iwxxm/2.1/sigmet.xsd}

\url{http://schemas.wmo.int/iwxxm/2.1/measures.xsd}

\url{http://schemas.wmo.int/iwxxm/2.1/tropicalCycloneAdvisory.xsd}

\url{http://schemas.wmo.int/iwxxm/2.1/airmet.xsd}

\url{http://schemas.wmo.int/iwxxm/2.1/iwxxm-collect.xsd}

\hypertarget{tsmlxml-representation-of-information-as-time-series}{%
\section{6. TSML‑XML (REPRESENTATION OF INFORMATION AS TIME SERIES)}\label{tsmlxml-representation-of-information-as-time-series}}

\url{http://schemas.opengis.net/tsml/1.0/timeseriesML.xsd}
